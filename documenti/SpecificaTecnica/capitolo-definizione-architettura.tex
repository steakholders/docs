\section{Descrizione architettura}

% Nell'indice proposto da Tullio c'è:
% a. Metodo e formalismo di specifica
% b. Presentazione dell'architettura generale del sistema e identificazione dei componenti architetturali di alto livello

\subsection{Formalismo di specifica}

I diagrammi dei package e delle classi presentati di seguito utilizzano la specifica \glossario{UML} 2.0.

\subsection{Architettura generale}

L'architettura scelta per lo sviluppo di \ProjectName{} servendosi del design pattern \glossario{MVC} (Model-View-Controller). Nel diagramma in figura \ref{architetturaGeneralePackage} viene presentata la struttura ad alto livello evidenziando le componenti che ricoprono i ruoli di model e controller per il \glossario{back-end}, e controller per il \glossario{front-end}. Il \glossario{back-end} non presenta la view poiché, utilizzando il formato \glossario{JSON}, la conversione dalla rappresentazione interna alla presentazione testuale (JSON) è automatica e diretta.
Il \glossario{front-end} è gestito dal \glossario{framework} \glossario{AngularJS}, la cui architettura è stata per diversi anni vicina al modello \glossario{MVC}. Nel tempo si è avvicinato più al modello \glossario{MVVM}. Ora \glossario{AngularJS} si dichiara \glossario{MVW} ossia Model-View-Whatever\footnote{Si rimanda al sito ufficiale di AngularJS: \url{http://angularjs.org/}}, per la caratteristica di non corrispondere esattamente ad uno dei modelli classici. Nell'architettura scelta per \ProjectName, si è scelto di descrivere soltanto i \glossario{package} che compongono il controller perché l'aggiornamento della view e del model viene gestito da \glossario{AngularJS} internamente.

\begin{figure}[H]
\centering
\includegraphics[width=\textwidth]{uml/architettura-generale-package.png}
\caption{Architettura generale dei package}
\label{architetturaGeneralePackage}
\end{figure}

\subsection{REST}

\glossario{REST}, ovvero Representational State Transfer è un tipo di \glossario{RPC}. Si basa su un protocollo di comunicazione \glossario{stateless} di tipo client-server, e solitamente tale protocollo è \glossario{HTTP}, che è stato scelto anche per \ProjectName.

I motivi che hanno spinto alla scelta di REST sono:
\begin{itemize}
\item Semplicità di utilizzo;
\item Indipendenza dal sistema operativo utilizzato dal \glossario{client};
\item Indipendenza dai linguaggi di programmazione utilizzati;
%\item In coppia con HTTPS permette una certa sicurezza delle comunicazioni.
\end{itemize}

\glossario{REST} utilizza il concetto di risorsa, ovvero un aggregato di dati con un nome (\glossario{URI}) e una rappresentazione, su cui è possibile invocare le operazioni \glossario{CRUD} tramite il protocollo sopracitato.

Nell'\glossario{URI} inviato sono presenti il nome della risorsa e la sua rappresentazione, identificata dall'estensione del file scelto, e per \ProjectName{} è stato scelto il formato \glossario{JSON}, in quanto il suo \glossario{parsing} in \glossario{JavaScript} è più semplice rispetto, ad esempio, quello di \glossario{XML} o \glossario{CSV}.

Un'altra caratteristica di \glossario{REST} è che essendo \glossario{stateless} ogni richiesta dovrà contenere tutte le informazioni necessarie e non dipende dallo stato del sistema.