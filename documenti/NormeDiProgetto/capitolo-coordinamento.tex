\section{Coordinamento}


\subsection{Comunicazioni}

\subsubsection{Interne}
\label{Comunicazioniinterne}

Tutte le comunicazioni ad uso interno del gruppo avvengono tramite il servizio di messaggistica offerto da \emph{TeamworkPM}, altre vie di comunicazione sono sconsigliate.

Nel caso sia necessario comunicare in modo rapido possono essere utilizzati strumenti di messaggistica istantanea come Hangouts o Skype. L'utilizzo di \glossario{SMS} e di chiamate telefoniche è riservato alle situazioni di urgenza.

\paragraph{Messaggi}
\label{Comunicazioniinternemessaggi}

\begin{itemize}
\item L'\textbf{oggetto} deve essere sintetico e coerente con il contenuto del messaggio;
\item Il \textbf{contenuto} deve includere i dettagli necessari per una corretta comprensione del messaggio e non deve essere prolisso. Il mittente può servirsi del supporto al linguaggio di \glossario{Markdown} disponibile sulla piattaforma per rendere più chiaro il suo messaggio;
\item La \textbf{categoria} deve essere coerente con l'argomento trattato. Può essere creata una categoria nuova se ritenuto necessario;
\item La \textbf{notifica} via mail deve includere gli interessati al messaggio;
\item Il livello di \textbf{privacy} deve sempre essere "\textit{Everybody on project}" in modo da permettere a tutti i componenti del gruppo di intervenire;
\item Data l'esigua dimensione dello \glossario{storage} che offre la piattaforma nella versione \textit{free} utilizzata dal gruppo e l'assenza di visualizzatori online integrati, il numero e la dimensione degli \textbf{allegati} devono essere ridotti quanto più possibile.
\end{itemize}

\paragraph{Commenti a tasks o sub-tasks}

Per tali comunicazioni valgono le regole del paragrafo \ref{Comunicazioniinternemessaggi}. Il contenuto del commento deve riguardare la task che riferisce. Se la discussione si sviluppa in argomenti non più coerenti, i componenti del gruppo \GroupName{} devono terminare la discussione, aprire un messaggio secondo le norme descritte al paragrafo \ref{Comunicazioniinternemessaggi} e inserire come nuovo commento alla discussione interrotta il link al messaggio creato con l'aggiunta della segnatura ``[OT]''. Questo non vincola il proseguimento della discussione sulla task secondo le norme.

\subsubsection{Esterne}
\label{email}

Le comunicazioni esterne sono gestite esclusivamente dal Responsabile di Progetto. A tal fine è stato creato l'indirizzo di posta elettronica
\begin{center} \GroupEmail{} \end{center}

Il Responsabile di Progetto si fa dunque carico di notificare ai restanti membri del gruppo eventuali corrispondenze
intrattenute con committenti e proponenti, applicando le norme stabilite al paragrafo \ref{Comunicazioniinterne}.


\subsection{Riunioni}

Qualora fosse necessaria una riunione di tutti o alcuni membri del gruppo sarà compito del Responsabile di Progetto avvisare gli interessati, tramite le procedure stabilite al paragrafo \ref{Comunicazioniinterne}.
Il Responsabile di Progetto decide inoltre luogo, date e ora della riunione in base al calendario a sua disposizione. Nel caso in cui qualche membro non risponda entro 24 ore il Responsabile di Progetto dovrà accertarsi con mezzi di comunicazione adeguati che tutti siano stati informati.

Ad ogni riunione verrà prodotto un verbale redatto da un segretario, ruolo svolto a turno da ogni membro del gruppo e deciso di volta in volta dal Responsabile di Progetto. Tale verbale, descritto nel paragrafo \ref{verbale} dovrà essere reso disponibile per la consultazione a tutti i membri del gruppo.

È inoltre presente un \glossario{facilitatore}, ruolo svolto a turno da ogni membro del gruppo e deciso di volta in volta dal Responsabile di Progetto, che aiuterà a rispettare l'ordine del giorno senza prolungare eccessivamente la riunione.


\subsection{Repository e condivisione file}

Come repository è stato scelto Git, in particolare il servizio \glossario{GitHub} come indicato nel capitolo \ref{github}.

\subsubsection{Repository dei documenti}

\paragraph{Struttura del repository}

La struttura del repository, la cui cartella principale chiameremo \file{root}, è così composta:
\begin{itemize}
 \item \textbf{\file{root/ufficiali/}} \\
	Contiene i documenti ufficiali approvati dal Responsabile di progetto.

 \item \textbf{\file{root/modello/}} \\
	Contiene i file comuni a due o più documenti.

 \item \textbf{\file{root/documenti/\{NomeDelDocumento\}/}} \\
	Ciascuna cartella descritta da questo percorso contiene i file che vengono utilizzati dal documento \file{\{NomeDelDocumento\}}. In particolare conterrà il file \file{\{NomeDelDocumento\}.pdf} e il file \file{diario\_ modifiche.tex}, contenente il diario delle modifiche del documento.
	
	Il nome della cartella \file{\{NomeDelDocumento\}} e il file \file{\{NomeDelDocumento\}.pdf} devono rispettare la convenzione \glossario{CamelCase}.

 \item \textbf{\file{root/script/}} \\
	Contiene gli script di supporto alle operazioni di verifica e compilazione dei documenti.
\end{itemize}

È raccomandato che tutti i file e le cartelle non contengano spazi nel loro nome. Non devono mai esserci due file o cartelle il cui percorso differisca soltanto per maiuscole/minuscole. Non bisogna inoltre rinominare file o cartelle modificandone soltanto il \glossario{case} di alcuni caratteri del nome.

\paragraph{Branch}

Il nome utilizzato per le \glossario{branch} del repository dovrà essere nella forma \textbf{nome-branch}, ossia con tutte le lettere minuscole e le parole separate da un trattino.

\paragraph{Script di pre-commit}

Con l'intenzione di ridurre al minimo la presenza di errori nei documenti caricati sul repository verrà utilizzato uno script di \code{pre-commit}. Lo script, attivato direttamente dal programma \code{git} nel momento in cui viene eseguito un commit, esegue in automatico i comandi \code{make test} e \code{make documents} descritti nella sezione \ref{makefile}. Se uno dei comandi fallisce il commit viene annullato e all'utente viene mostrato un messaggio di errore.

Per abilitare tale script di \code{pre-commit} è necessario creare un \glossario{link simbolico} con il seguente comando:
\begin{lstlisting}
ln -s {percorso assoluto del repository}/script/pre-commit {percorso assoluto del repository}/.git/hooks/pre-commit
\end{lstlisting}

\subsubsection{Repository del codice}

Non è stato ancora definito.

\subsubsection{Condivisione file}

Per condividere internamente al gruppo dei file che non necessitano di versionamento verranno utilizzate le piattaforme:
\begin{itemize}
 \item \textbf{Google Drive} (\url{http://drive.google.com}), con l'account descritto nella sezione \ref{email};
 \item \textbf{TeamworkPM}, descritta nella sezione \ref{teamworkpm}.
\end{itemize}
