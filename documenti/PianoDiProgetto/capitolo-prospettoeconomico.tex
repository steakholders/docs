\definecolor{amministratore}{RGB}{51,102,204}
\definecolor{analista}{RGB}{255,153,0}
\definecolor{progettista}{RGB}{153,0,153}
\definecolor{programmatore}{RGB}{7,55,99}
\definecolor{responsabile}{RGB}{220,57,18}
\definecolor{verificatore}{RGB}{16,150,24}

\section{Prospetto economico}

In questa sezione è presentato il prospetto economico del progetto \ProjectName{}, suddiviso per fasi. Per ogni periodo sono indicate le ore preventivate per ogni ruolo impiegato.
Il costo è calcolato utilizzando i dati della tabella al paragrafo \ref{tabellacostiruolo}.

\subsection{Analisi}

A scopo di trasparenza viene redatto il prospetto economico riguardante il periodo di Analisi dei requisiti, ma si precisa che le ore spese in questo periodo sono a carico del fornitore e non del proponente.

\begin{table}[H]
	\centering
	\begin{tabular}{ l c c }
	\textbf{Ruolo} & \textbf{Ore} & \textbf{Costo} \\
	\hline
		Amministratore & 30 & 600 € \\
	Analista & 58 & 1450 € \\
	Progettista & 9 & 198 € \\
	Programmatore & 0 & 0 € \\
	Responsabile & 14 & 420 € \\
	Verificatore & 31 & 465 € \\
\hline
	Totale & 142 & 3133 € \\
\hline

	\end{tabular}
	\caption{Ore e costo per ruolo, periodo di Analisi}
	\end{table}

I seguenti grafici mostrano il peso orario e di costo di ogni ruolo in questo periodo.

\begin{figure}[H]
\begin{tikzpicture}

	\pie[text=legend, color={amministratore, analista, progettista, programmatore, responsabile, verificatore}]{21.1/Amministratore, 40.8/Analista, 6.3/Progettista, ./Programmatore, 9.9/Responsabile, 21.8/Verificatore}


\end{tikzpicture}
\caption{Ore per ruolo, periodo di Analisi}
\end{figure}

\begin{figure}[H]
\begin{tikzpicture}

	\pie[text=legend, color={amministratore, analista, progettista, programmatore, responsabile, verificatore}]{19.2/Amministratore, 46.3/Analista, 6.3/Progettista, ./Programmatore, 13.4/Responsabile, 14.8/Verificatore}


\end{tikzpicture}
\caption{Costo per ruolo, periodo di Analisi}
\end{figure}

\subsection{Progettazione architetturale}

Nel periodo di Progettazione architetturale le ore per ogni ruolo sono state cosi suddivise:

\begin{table}[H]
	\centering
	\begin{tabular}{ l c c }
	\textbf{Ruolo} & \textbf{Ore} & \textbf{Costo} \\
	\hline
	
			Amministratore & 48 & 960 € \\
	Analista & 24 & 600 € \\
	Progettista & 108 & 2376 € \\
	Programmatore & 0 & 0 € \\
	Responsabile & 26 & 780 € \\
	Verificatore & 72 & 1080 € \\
\hline
	Totale & 278 & 5796 € \\
\hline

	
	\end{tabular}
	\caption{Ore e costo per ruolo, periodo di Progettazione architetturale}
	\end{table}
	
I seguenti grafici mostrano il peso orario e di costo di ogni ruolo in questo periodo.

\begin{figure}[H]
\begin{tikzpicture}

	\pie[sum=auto, text=legend, color={amministratore, analista, progettista, programmatore, responsabile, verificatore}]{48/Amministratore, 24/Analista, 70/Progettista, ./Programmatore, 19.0/Responsabile, 42/Verificatore}


\end{tikzpicture}
\caption{Ore per ruolo, periodo di Progettazione architetturale}
\end{figure}

\begin{figure}[H]
\begin{tikzpicture}

	\pie[text=legend, color={amministratore, analista, progettista, programmatore, responsabile, verificatore}]{14.7/Amministratore, 10.5/Analista, 41.5/Progettista, ./Programmatore, 13.6/Responsabile, 19.7/Verificatore}


\end{tikzpicture}
\caption{Costo per ruolo, periodo di Progettazione architetturale}
\end{figure}

\subsection{Progettazione di dettaglio e codifica}

Nel periodo di Progettazione di dettaglio e codifica le ore per ogni ruolo sono state cosi suddivise:

\begin{table}[H]
	\centering
	\begin{tabular}{ l c c }
	\textbf{Ruolo} & \textbf{Ore} & \textbf{Costo} \\
	\hline
	
			Amministratore & 38 & 760 € \\
	Analista & 8 & 200 € \\
	Progettista & 120 & 2640 € \\
	Programmatore & 132 & 1980 € \\
	Responsabile & 0 & 0 € \\
	Verificatore & 90 & 1350 € \\
\hline
	Totale & 388 & 6930 € \\
\hline

	
	\end{tabular}
	\caption{Ore e costo per ruolo, periodo di Progettazione di dettaglio e codifica}
	\end{table}

I seguenti grafici mostrano il peso orario e di costo di ogni ruolo in questo periodo.

\begin{figure}[H]
\begin{tikzpicture}

	\pie[text=legend, color={amministratore, analista, progettista, programmatore, responsabile, verificatore}]{10.3/Amministratore, 2.9/Analista, 25.0/Progettista, 41.2/Programmatore, ./Responsabile, 20.6/Verificatore}


\end{tikzpicture}
\caption{Ore per ruolo, periodo di Progettazione di dettaglio e codifica}
\end{figure}

\begin{figure}[H]
\begin{tikzpicture}

	\pie[text=legend, color={amministratore, analista, progettista, programmatore, responsabile, verificatore}]{15.2/Amministratore, 3.8/Analista, 29.2/Progettista, 39.3/Programmatore, ./Responsabile, 12.5/Verificatore}


\end{tikzpicture}
\caption{Costo per ruolo, periodo di Progettazione di dettaglio e codifica}
\end{figure}

\subsection{Validazione}

Nel periodo di Validazione le ore per ogni ruolo sono state cosi suddivise:

\begin{table}[H]
	\centering
	\begin{tabular}{ l c c }
	\textbf{Ruolo} & \textbf{Ore} & \textbf{Costo} \\
	
			Amministratore & 16 & 320 € \\
	Analista & 0 & 0 € \\
	Progettista & 8 & 176 € \\
	Programmatore & 0 & 0 € \\
	Responsabile & 8 & 240 € \\
	Verificatore & 84 & 1260 € \\
\hline
	Totale & 116 & 1996 € \\
\hline

	
	\end{tabular}
	\caption{Ore e costo per ruolo, periodo di Validazione}
	\end{table}

I seguenti grafici mostrano il peso orario e di costo di ogni ruolo in questa periodo.

\begin{figure}[H]
\begin{tikzpicture}

	\pie[text=legend, color={amministratore, analista, progettista, programmatore, responsabile, verificatore}]{11.6/Amministratore, ./Analista, 5.8/Progettista, ./Programmatore, 5.8/Responsabile, 76.8/Verificatore}


\end{tikzpicture}\caption{Ore per ruolo, periodo di Validazione}
\end{figure}

\begin{figure}[H]
\begin{tikzpicture}

	\pie[text=legend, color={amministratore, analista, progettista, programmatore, responsabile, verificatore}]{7.3/Amministratore, ./Analista, 16.1/Progettista, ./Programmatore, 5.5/Responsabile, 71.2/Verificatore}


\end{tikzpicture}
\caption{Costo per ruolo, periodo di Validazione}
\end{figure}

\subsection{Totale}

In totale le ore per ogni ruolo sono state cosi suddivise:

\begin{table}[H]
	\centering
	\begin{tabular}{ l c c }
	\textbf{Ruolo} & \textbf{Ore} & \textbf{Costo} \\
	\hline
	
			Amministratore & 95 & 1900 € \\
	Analista & 32 & 800 € \\
	Progettista & 190 & 4180 € \\
	Programmatore & 126 & 1890 € \\
	Responsabile & 34 & 1020 € \\
	Verificatore & 229 & 3435 € \\
\hline
	Totale & 706 & 13225 € \\
\hline

	
	\end{tabular}
	\caption{Ore e costo per ruolo, riassunto progetto}
	\end{table}

Il seguente grafico mostra il costo di ogni ruolo durante tutto lo svolgimento del progetto, escluso il periodo di Analisi dei requisiti.

\begin{figure}[H]
\begin{tikzpicture}

	\pie[text=legend, color={amministratore, analista, progettista, programmatore, responsabile, verificatore}]{16.2/Amministratore, 5.7/Analista, 30.5/Progettista, 14.4/Programmatore, 8.0/Responsabile, 25.2/Verificatore}


\end{tikzpicture}
\caption{Costo per ruolo}
\end{figure}