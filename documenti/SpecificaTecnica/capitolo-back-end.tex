\section{Back-end}


\subsection{Interfaccia REST}

Ad ogni richiesta il server può rispondere con un messaggio di errore nel formato \glossario{JSON} e inviato con un codice \glossario{HTTP} della tipologia 4xx o 5xx. 
Il formato \glossario{JSON} del messaggio di errore sarà:
\begin{lstlisting}
{ 
  "code": [codice numerico dell'errore],
  "message": [descrizione testuale dell'errore]
}
\end{lstlisting}
I tipi di permessi possibili sono: 
\begin{itemize}
\item \textbf{Utente}: questa risorsa può essere richiesta da qualsiasi tipo di utente;
\item \textbf{Utente Autenticato}: questa risorsa può essere richiesta solo dagli utenti autenticati a \glossario{MaaP};
\item \textbf{Admin}: tale risorsa può essere richiesta solo da utenti con livello Admin.
\end{itemize}

\begin{center}
	\def\arraystretch{1}
	\bgroup
	\begin{longtable}{ p{4.5cm}| p{6cm}| p{3cm} }
	\hline 
	\textbf{Risorsa} & \textbf{Metodo } & \textbf{Permessi} \\ \hline
	
\multirow{4}{*}{\emph{/login}} 
 	& \multirow{2}{*}{POST} \vspace{0.2cm} \\ 
		& Crea una nuova sessione associata all'utente, corrisponde al login. 
		& Utente \\\hline 
 		
\multirow{4}{*}{ \emph{/logout} }
 	& \multirow{2}{*}{DELETE} \vspace{0.2cm} \\  
 		& Elimina la sessione utente, corrisponde al logout. 
 		& Utente \\ \hline 
 		
\multirow{4}{*}{ \emph{/profile} }
 	& \multirow{2}{*}{GET} \vspace{0.2cm} \\ 
 		& Restituisce i dati relativi all'utente. 
 		& Utente Autenticato \\ 
 	\cmidrule{2-3}
 	& \multirow{2}{*}{PUT} \vspace{0.2cm} \\ 
 		& Modifica i dati utente. 
 		& Utente Autenticato \\ \hline 

\multirow{2}{*}{ \emph{/password/lost} }
 	& \multirow{2}{*}{POST} \vspace{0.2cm} \\ 
 		& Effettua la richiesta di recupero password.
 		& Utente \\ \hline
 		
\multirow{2}{*}{ \emph{/password/reset} }
 	& \multirow{2}{*}{PUT} \vspace{0.2cm} \\ 
 		& Effettua la richiesta di modifica della password utente.
 		& Utente \\ %\tabucline[2pt]{-}
 		\hline 
 		
\multirow{4}{*}{ \emph{/users} } 
 	& \multirow{2}{*}{GET} \vspace{0.2cm} \\ 
 		& Restituisce la lista di tutti gli utenti.
 		& Admin \\ 
 	\cmidrule{2-3}
 	& \multirow{2}{*}{POST} \vspace{0.2cm} \\ 
 		& Effettua la richiesta di creazione di un nuovo utente.
 		& Admin \\ \hline  
 		
\multirow{6}{*}{ \emph{/users/$\{$user id$\}$} } 
 	& \multirow{2}{*}{GET} \vspace{0.2cm} \\ 
 		& Restituisce i dati corrispondenti all'utente con id $\{$user id$\}$.
 		& Admin \\ 
 	\cmidrule{2-3}
 	& \multirow{2}{*}{PUT} \vspace{0.2cm} \\ 
 		& Effettua la richiesta di modifica dei dati utente con id $\{$user id$\}$.
 		& Admin \\
 	\cmidrule{2-3}
 	& \multirow{2}{*}{DELETE} \vspace{0.2cm} \\ 
 		& Elimina l'utente con id $\{$user id$\}$.
 		& Admin \\ \hline 
 		
\multirow{2}{*}{ \emph{/collection/} } 
 	& \multirow{2}{*}{GET} \vspace{0.2cm} \\ 
 		& Restituisce la lista delle collection
 		& Utente Autenticato \\
 	\hline 

\multirow{2}{*}{ \emph{/collection/$\{$collection name$\}$} } 
 	& \multirow{2}{*}{GET} \vspace{0.2cm} \\ 
 		& Restituisce la lista di document della collection $\{$collection name$\}$
 		& Utente Autenticato \\ 
 	\hline 

\multirow{6}{*}{ \emph{/collection/$\{$collection name$\}$/$\{$document id$\}$} } 
 	& \multirow{2}{*}{GET} \vspace{0.2cm} \\ 
 		& Restituisce la lista di attributi del Document $\{$document id$\}$ appartenente alla collection $\{$collection name$\}$
 		& Utente Autenticato \\ 
 	\cmidrule{2-3}
 	& \multirow{2}{*}{PUT} \vspace{0.2cm} \\ 
 		& Modifica il document $\{$document id$\}$
 		& Admin \\ 
 	\cmidrule{2-3}
 	& \multirow{2}{*}{DELETE} \vspace{0.2cm} \\ 
 		& Elimina il document con id $\{$document id$\}$  
 		& Admin \\
	\hline

\multirow{2}{*}{ \emph{/action/$\{$action name$\}$/$\{$collection name$\}$} } 
 	& \multirow{2}{*}{GET} \vspace{0.2cm} \\ 
 		& Esegue l'azione $\{$action name$\}$ sulla Collection $\{$collection name$\}$
 		& Utente Autenticato \\
 	\hline 

\multirow{2}{*}{ \emph{/action/$\{$action name$\}$/$\{$collection name$\}$/$\{$document id$\}$} } 
 	& \multirow{2}{*}{GET} \vspace{0.2cm} \\ 
 		& Esegue l'azione $\{$action name$\}$ sul Document $\{$document name$\}$ della Collection $\{$collection name$\}$
 		& Utente Autenticato \\
 	\hline 

	\end{longtable}
	  \egroup
\end{center} 


\subsection{Diagramma dei package}

\subsection{Diagramma delle classi}