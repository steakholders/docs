\section*{Registro delle modifiche}

\small{
\begin{tabularx}{\textwidth}{|c|c|P{3cm}|X|}
 \hline \textbf{Versione} & \textbf{Data} & \textbf{Persone coinvolte} & \textbf{Descrizione} \\

% VA FATTO IN ORIDNE DECRESCENTE!!! 

\hline 3.0.0 & 2014-02-05 & %TODO
 \linebreak (Responsabile) & Approvazione \\

\hline 2.2.0 & 2014-02-04 & Enrico Rotundo \linebreak (Analista) & Verifica \\

\hline 2.1.1 & 2014-01-22 & Gianluca Donato \linebreak (Analista) & Incremento \\

\hline 2.1.1 & 2014-01-22 & Serena Girardi \linebreak (Analista) & Incremento \\

\hline 2.1.0 & 2014-01-20 & Federico Poli \linebreak (Verificatore) & Verifica dettagliata \\

\hline 2.0.2 & 2014-01-15 & Nicolò Tresoldi \linebreak (Analista) & Apportate correzioni \\

\hline 2.0.1 & 2014-01-12 & Gianluca Donato \linebreak (Analista) & Aggiunta di nuovi casi d'uso. \\ 

\hline 2.0.0 & 2014-01-13 & Serena Girardi \linebreak (Analista) & Aggiornamento sistema di versionamento \\

\hline 1.3.1 & 2013-12-20 &  Nicolò Tresoldi \linebreak (Responsabile) & Approvazione documento \\

\hline 1.2.4 & 2013-12-19 & Gianluca Donato \linebreak (Verificatore) & Verifica requisiti opzionali \\

\hline 1.2.3 & 2013-12-18 & Federico Poli \linebreak (Analista) \linebreak Enrico Rotundo \linebreak (Analista) & Stesura requisiti opzionali \\ 
 
 \hline 1.2.2 & 2013-12-17 & Federico Poli \linebreak (Analista) \linebreak Enrico Rotundo \linebreak (Analista) & Stesura casi d'uso opzionali \\

 \hline 1.2.1 & 2013-12-16 & Federico Poli \linebreak (Verificatore) & Verifica requisiti fondamentali e desiderabili \\

 \hline 1.1.3 & 2013-12-14 & Luca De Franceschi \linebreak (Analista) \linebreak Serena Girardi \linebreak (Analista) & Stesura requisiti fondamentali e desiderabili \\

 \hline 1.1.2 & 2013-12-12 & Luca De Franceschi \linebreak (Analista) \linebreak Gianluca Donato \linebreak (Analista) \linebreak Serena Girardi \linebreak (Analista) & Stesura casi d'uso fondamentali e desiderabili \\

 \hline 1.1.1 & 2013-12-10 & Luca De Franceschi \linebreak (Analista) & Stesura introduzione e descrizione generale \\
\hline
\end{tabularx}
}
