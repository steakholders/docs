\section{Preventivo a finire}
\label{consuntivo}

In questa sezione verrà presentato il preventivo a finire ovvero il preventivo al netto del consuntivo, sia per le ore che per i costi sostenuti. In base alle ore di differenza tra il preventivo e il consuntivo avremo che il bilancio è:
\begin{itemize}
\item \textbf{Positivo} Si sono risparmiate ore, che potranno essere ricollocate in un periodo successivo;
\item \textbf{Negativo} Si sono consumate più ore di quante preventivate, per non fare deficit si dovranno rivedere le ore future;
\item \textbf{In pari} Sono state consumate esattamente le ore preventivate.
\end{itemize}

\subsection{Analisi}
\begin{table}[H]
\begin{tabular}{lccccccc}
\toprule
    \textbf{Nome}  & \multicolumn{6}{c}{\textbf{Ore per ruolo}} & \textbf{Ore totali} \\
     & Amm. & Ana. & Pgt. & Pgr. & Res. & Ver. & \\
    \midrule
    
    		Enrico Rotundo & 0 (+0) & 12 (-12) & 9 (-9) & 0 (+0) & 0 (+0) & 2 (-2) & 23 (-23) \\
	Federico Poli & 0 (+0) & 9 (-9) & 0 (+0) & 0 (+0) & 4 (-4) & 9 (-9) & 22 (-22) \\
	Giacomo Fornari & 9 (-9) & 2 (-2) & 0 (+0) & 0 (+0) & 5 (-5) & 4 (-4) & 20 (-20) \\
	Gianluca Donato & 0 (+0) & 10 (-10) & 0 (+0) & 0 (+0) & 0 (+0) & 11 (-11) & 21 (-21) \\
	Luca De Franceschi & 6 (-6) & 11 (-11) & 0 (+0) & 0 (+0) & 0 (+0) & 0 (+0) & 17 (-17) \\
	Nicolò Tresoldi & 15 (-15) & 0 (+0) & 0 (+0) & 0 (+0) & 5 (-5) & 0 (+0) & 20 (-20) \\
	Serena Girardi & 0 (+0) & 14 (-14) & 0 (+0) & 0 (+0) & 0 (+0) & 5 (-5) & 19 (-19) \\

    
    \bottomrule
\end{tabular}
\caption{Ore per componente, periodo di Progettazione Architetturale}
\end{table}

\subsection{Progettazione architetturale}

\begin{table}[H]
\begin{tabular}{lccccccc}
\toprule
    \textbf{Nome}  & \multicolumn{6}{c}{\textbf{Ore per ruolo}} & \textbf{Ore totali} \\
     & Amm. & Ana. & Pgt. & Pgr. & Res. & Ver. & \\
    \midrule
    
    		Enrico Rotundo & 0 (+0) & 4 (+0) & 16 (+2) & 0 (+0) & 0 (+0) & 16 (+0) & 36 (+2) \\
	Federico Poli & 4 (+0) & 0 (+0) & 28 (+2) & 0 (+0) & 4 (+0) & 4 (+0) & 40 (+2) \\
	Giacomo Fornari & 8 (+0) & 2 (-0) & 8 (+2) & 0 (+0) & 4 (-2) & 22 (+1) & 44 (+1) \\
	Gianluca Donato & 6 (+0) & 8 (+0) & 14 (+2) & 0 (+0) & 4 (+0) & 4 (-1) & 36 (+1) \\
	Luca De Franceschi & 6 (-4) & 0 (+0) & 16 (+2) & 0 (+0) & 6 (-1) & 10 (-4) & 38 (-7) \\
	Nicolò Tresoldi & 10 (+0) & 0 (+0) & 8 (+2) & 0 (+0) & 8 (+0) & 5 (+1) & 31 (+3) \\
	Serena Girardi & 8 (+0) & 10 (-2) & 18 (+0) & 0 (+0) & 0 (+0) & 14 (-2) & 50 (-4) \\

    
    \bottomrule
\end{tabular}
\caption{Ore per componente, periodo di Progettazione Architetturale}
\end{table}

%\subsection{Progettazione di dettaglio e codifica}