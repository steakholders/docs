\section{Resoconto delle attività di verifica}

	\subsection{Riassunto delle attività di verifica}
	\label{RiassuntoAttivitaVerifica}
	
	 	\subsubsection{Revisione dei Requisiti}
	 	L'attività di verifica svolta dai \emph{Verificatori} è avvenuta come determinato dal \PianoDiProgetto{} al termine della stesura di ogni documento previsto. La verifica svolta sui documenti è avvenuta seguendo le indicazioni delle \NormeDiProgetto{} e misurando le metriche indicate in \ref{metrichedocumenti}. L'attività di \emph{walkthrough} ha evidenziato una serie di anomalie, in questo modo è stato possibile stilare la lista di anomalie frequenti (vedi \NormeDiProgetto{}) che si potranno controllare tramite \emph{Inspection}. Successivamente si è proceduto con le misurazioni delle metriche relative ai documenti.
In questa revisione non è stato possibile valutare i processi poiché lo stato embrionale del team e   impegni universitari sovrapposti non hanno permesso il rilevamento accurato di tutti i parametri necessari. Il gruppo ha in programma di colmare tale mancanza per la revisione successiva.

		\paragraph{Miglioramenti}
		A seguito delle attività di verifica e controllo è stato sottoposto un questionario ad ogni membro del gruppo che ha contribuito ad identificare le problematiche relative ai processi e a formulare proposte risolutive:
		
	
		\begin{table}[h]
		\centering
	\begin{tabularx}{\textwidth}{|X|X|X|}
\hline
Problema & Possibile soluzione & Stato \\
\hline
\hline
Il dizionario personale di \glossario{Aspell}, essendo un file collaborativo compilato in automatico da tale \glossario{tool}, impone molto spesso attività manuali extra di gestione del \glossario{repository}, in particolare vanno risolti molti conflitti. & Uno script che ordina il file in questione dovrebbe diminuire i conflitti. & Da eseguire. \\ \hline
 	    Contrassegnare le parole di glossario con il relativo \glossario{tag} è un attività fortemente propensa a dimenticanze ed errori. & Uno script potrebbe contrassegnare le parole di glossario presenti nei documenti in automatico. & Da eseguire.   \\ \hline
 	    Scarsa frammentazione dei \glossario{task} & Incremento dell'utilizzo dello strumento di gestione dei processi. & Completato.  \\ \hline
 	    Mantenere la corrispondenza tra casi d'uso e relativi \glossario{url} dei diagrammi è un operazione lunga e manuale. & Uno script può automatizzare tale attività. & Completato.   \\ \hline
 	    Difficoltà nell'uso dello strumento di controllo di versione. & I membri del gruppo devono eseguire attività extra di autoformazione sulla base del materiale messo a disposizione da alcuni membri. & Completato.  \\ \hline
\end{tabularx}
	\caption{Miglioramenti post RR}
	\end{table}
	
	 
	 \subsection{Dettaglio delle verifiche tramite analisi}
	 \label{DettaglioVerificheAnalisi}
	 	\subsubsection{Analisi}
	 	\paragraph{Documenti}
	 	Vengono qui riportati i valori dell’indice Gulpease per ogni documento durante l'analisi e relativo esito basato sui range stabiliti in \ref{gulpease}. Questa tabella verrà aggiornata in modo incrementale. 	
	
	\begin{table}[h]
	\centering
	\begin{tabular}{ | c | c | c | }
    \hline
    Documento & Valore indice & Esito \\ \hline
    \AnalisiDeiRequisiti{} & 45 &  superato \\ \hline
    \Glossario{} & 63 &  superato \\ \hline
    \NormeDiProgetto{} & 47 &  superato \\ \hline
    \PianoDiProgetto{} & 51 &  superato \\ \hline
    \PianoDiQualifica{} & 53 &  superato \\ \hline
    \StudioDiFattibilita{} & 43 &  superato \\ \hline
    \end{tabular}
	\caption{Esiti verifica documenti, Analisi}
	\end{table}
	
	Il file \AnalisiDeiRequisiti{} contiene numerose tabelle che possono facilmente falsare l'indice rilevato, il gruppo si riserva di analizzare tale problematica e aggiornare i relativi strumenti entro la prossima revisione.
	
	
	

	 	
%	 		\paragraph{Processi}
%	 		\paragraph{Documenti}
	 	
%	 \subsection{Dettaglio delle verifiche tramite prove}
%		\subsubsection{Analisi}
%			\paragraph{Processi}
%	 		\paragraph{Documenti}
	 
	 
%	\subsection{Dettaglio dell'esito delle revisioni}


	
