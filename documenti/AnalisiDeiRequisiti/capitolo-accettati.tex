\section{Requisiti accettati }
I requisiti di qualità, di vincolo e i requisiti funzionali obbligatori sono stati accettati e \textbf{{\color{green} implementati}}. \\
I requisiti opzionali nel ramo servizio presenti nel capitolato e riguardanti il sistema \glossario{MaaS} non verranno implementati. \\
Nella tabella seguente vengono elencati i requisiti desiderabili e facoltativi con conseguente descrizione della loro implementazione o non implementazione nel prodotto finale.

      \bgroup
      \def\arraystretch{1.8}
      \begin{longtable}{ | l | p{2cm} | p{5cm} | p{4cm}| }
    
      \cellcolor[gray]{0.9} \textbf{Requisito} & \cellcolor[gray]{0.9} \textbf{Tipologia} 
      & \cellcolor[gray]{0.9} \textbf{Descrizione}  & \cellcolor[gray]{0.9} \textbf{Stato} \\ \hline
      
        
        RA1D 3 & Funzionale \newline  Desiderabile  & Il sistema deve mettere a disposizione dell'utente autenticato la visualizzazione della dashboard. & \textbf{{\color{green}implementato}} \newline \\ \hline      
            
        RA1D 4.1.2 & Funzionale \newline  Desiderabile  & L’applicazione può mettere a disposizione dell’admin un link di selezione rapida per l’eliminazione di un documento. \newline & \textbf{{\color{green} implementato}} \newline \\ \hline      
        RA1D 4.1.3 & Funzionale \newline  Desiderabile  & L’applicazione deve permette all’admin la modifica di un document presente nella collection-index. & \textbf{{\color{green} implementato}} \newline\\ \hline      
        RA1D 4.2 & Funzionale \newline  Desiderabile  & L’applicazione dà la possibilità di impostare dei filtri personalizzati secondo determinati attributi per visualizzare un sottoinsieme di Document.  & \textbf{{\color{red}non implementato}} \newline \\ \hline      
        RA1F 4.3 & Funzionale \newline  Facoltativo  & L’applicazione deve permettere all’amministratore di creare un nuovo Document all’interno della base di dati. & \textbf{{\color{red} non implementato}} \newline \\ \hline      
        
        RA1F 5.2 & Funzionale \newline  Facoltativo  & L’applicazione deve permettere all’utente di poter eseguire un’azione personalizzata tramite l’esecuzione di un pulsante. & \textbf{{\color{red} non implementato}} \newline  \\ \hline      
             
           
        RF1F 8.3 & Funzionale \newline  Facoltativo  & Il framework MaaP deve permettere allo sviluppatore di definire un namespace per l’applicazione generata. & \textbf{{\color{red} non implementato}} \newline  \\ \hline      
        
          
        RF1F 9.2.5 & Funzionale \newline  Facoltativo  & Lo sviluppatore deve poter personalizzare la show-page definendo delle operazioni personalizzate che l’utente potrà utilizzare tramite appositi pulsanti. & \textbf{{\color{red} non implementato}} \newline  \\ \hline      
          
             
        RA1D 11 & Funzionale \newline  Desiderabile  & Deve essere possibile da parte di un utente la registrazione all’Applicazione MaaP. & \textbf{{\color{green} implementato}} \newline  \\ \hline      
        RA1D 12 & Funzionale \newline  Desiderabile  & L’utente autenticato nell’applicazione deve poter eseguire il logout. & \textbf{{\color{green} implementato}} \newline  \\ \hline      
        RA1D 13 & Funzionale \newline  Desiderabile  & L'utente deve poter modificare le proprie credenziali d'accesso all'interno della propria pagina profilo. & \textbf{{\color{green} implementato}} \newline  \\ \hline      
        RA1D 13.1 & Funzionale \newline  Desiderabile  & L’utente deve poter modificare la password di accesso all’applicazione. & \textbf{{\color{green} implementato}} \newline  \\ \hline      
        
        RF1F 14.3 & Funzionale \newline  Facoltativo  & Il framework MaaP deve rendere possibile la selezione di un name-space per un database se la funzione di namespace è abilitata. & \textbf{{\color{red} non implementato}} \newline  \\ \hline      
        RA1F 15 & Funzionale \newline  Facoltativo  & L’applicazione MaaP deve mettere a disposizione dell’admin una pagina di gestione degli indici. & \textbf{{\color{red} non implementato}} \newline  \\ \hline      
        RA1F 15.1 & Funzionale \newline  Facoltativo  & L’applicazione MaaP deve mettere a disposizione dell’admin la visualizzazione degli indici in base alle query più richieste dall’applicazione. & \textbf{{\color{red} non implementato}} \newline  \\ \hline      
        RA1F 15.2 & Funzionale \newline  Facoltativo  & L’applicazione MaaP deve permettere all’admin di aggiungere gli indici in base ai suggerimenti forniti. & \textbf{{\color{red} non implementato}} \newline  \\ \hline      
        RA1F 15.3 & Funzionale \newline  Facoltativo  & L’applicazione MaaP deve permettere all’admin di rimuovere gli indici in base ai suggerimenti forniti. & \textbf{{\color{red} non implementato}} \newline  \\ \hline      
        RF1F 16 & Funzionale \newline  Facoltativo  & Il framework MaaP deve permettere allo sviluppatore di abilitare i namespace per l’applicazione creata. & \textbf{{\color{red} non implementato}} \newline  \\ \hline      
       
      \caption{Requisiti funzionali desiderabili e facoltativi implementati}
      \end{longtable}
      \egroup 
\clearpage