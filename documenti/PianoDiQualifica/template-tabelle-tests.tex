\pagebreak


% TS / req
	
	\begin{table}[h]
		\begin{tabularx}{\textwidth}{ | c | P{3cm} | c | X | }
 			\hline \textbf{Test} & \textbf{Descrizione} & \textbf{Stato} & \textbf{Requisito} \\ % INTESTAZIONE TABELLA
 				\hline col1 & col2 & col3 & col4 \\ % <------ CONTENUTO
 			\hline % obbligatorio alla fine !
		\end{tabularx}
		\caption{Associazione test di sistema con relativo requisito.}
	\end{table}
	
	
	
% TI / componente

	\begin{table}[h]
		\begin{tabularx}{\textwidth}{ | c | P{3cm} | c | X | }
 			\hline \textbf{Test} & \textbf{Descrizione} & \textbf{Componente} & \textbf{Stato} \\ % INTESTAZIONE TABELLA
 				\hline col1 & col2 & col3 & col4 \\ % <------ CONTENUTO
 			\hline % obbligatorio alla fine !
		\end{tabularx}
		\caption{Associazione test di integrazione con relativo componente.}
	\end{table}

% componente / TI
	\begin{table}[h]
		\begin{tabularx}{\textwidth}{ | X | X | }
 			\hline \textbf{Componente} & \textbf{Test}  \\ % INTESTAZIONE TABELLA
 				\hline col1 & col2 \\ % <------ CONTENUTO
 			\hline % obbligatorio alla fine !
		\end{tabularx}
		\caption{Tracciamento componenti test di integrazione.}
	\end{table}

% TV



\textbf{NOME TEST PRIMO LIVELLO} % PRIMO LVL = TVx
\begin{center}
      \bgroup
      \def\arraystretch{1.3}
	\begin{longtable}{ p{2.3cm} p{10cm} }
		\textbf{Scopo:} & TESTO SCOPO \\ % TESTO SCOPO TV primo livello
		\textbf{Dipendenze:} & TESTO DIPENDENZE \\ % TESTO DIP. TV primo livevllo
		\textbf{All'utente è richiesto di:} &
			\begin{itemize} % ELENCO PUNTATO DEI TEST FIGLI IN MODO RICORSIVO cioè per ogni livello va un sotto-elenco puntato.
				\item TESTO DESCRIZIONE TV DI LIVELLO 2  (TV x.x) % TESTO DESCR + CODICE TV secondo livello!
			\end{itemize}
	\end{longtable}
      \egroup
\end{center}  



