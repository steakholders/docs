\clearpage
\section{Gestione amministrativa della revisione}
% qui bisognerebbe prendere ampiamente spunto Sommerville e SWEBOK, viene spiegate le tipologie di anomalie
	\subsection{Comunicazione e risoluzione anomalie}
	Il processo di \emph{Software Quality Managment} è finalizzato alla ricerca dei difetti. L'identificazione delle anomalie ne permette la correzione e informa il \emph{Responsabile di Progetto} sullo stato del prodotto. Distinguere e catalogare le anomalie è utile per discutere, durante revisioni e riunioni, su che correzioni attuare e con quale priorità. Di seguito vengono elencate le definizioni di anomalie (IEEE610.12-90) adottate dal gruppo:
	\begin{itemize}
		\item \textbf{Error:} differenza riscontrata tra risultato di una computazione e valore teorico atteso, e.g. uscita dal range di accettazione degli indici di misurazione.
		\item \textbf{Fault:} un passo, processo o dato definito in modo erroneo; e.g. violazioni di norme tipografiche da parte di un documento. Corrisponde a quanto viene definito come \glossario{\emph{Bug}}.
		\item \textbf{Failure:} il risultato erroneo di un \emph{fault}; e.g. incongruenza del prodotto con funzionalità indicate nell'analisi dei requisiti, incongruenza del codice con il design del prodotto.
		\item \textbf{Mistake:} azione umana che produce un risultato errato; e.g. anomalie nel repository.
	\end{itemize}
	La catalogazione delle anomalie permette l'impostazione di metriche in grado di valutarne l'andamento e in alcuni casi di predirlo, in particolare è stata scelta la metrica che conteggia il \emph{numero di bugs per lines of code}. Il \glossario{\emph{SCR - software change request}} utilizzato dal gruppo è il sistema di task presente in  \glossario{\emph{TeamworkPM}} secondo le modalità descritte in \emph{Norme di Progetto}.


	\subsection{Procedure di controllo qualità di processo} % potrebbe evolvere in un Appendice escluisvamente dedicata alla qualità 
	La qualità dei processi tratta quei controlli e miglioramenti volti a migliorare la qualità del prodotto e/o di diminuire i costi e tempi di sviluppo. Esistono due approcci principali:
	\begin{itemize}
		\item A maturità di processo: il livello di maturità riflette le buone pratiche tecniche e di management. L'obiettivo primario è la qualità del prodotto e la prevedibilità dei processi.
		\item Agile: sviluppo iterativo senza l'overhead della documentazione e di tutti gli aspetti predeterminabili. Ha come caratteristica la responsività ai cambiamenti dei requisti cliente e un rapido sviluppo.
	\end{itemize}
	Verrà utilizzato il primo approccio in quanto più adatto ad un team senza esperienze di lavoro di gruppo. Con una visione proattiva si cerca di avere maggior controllo e previsione sulle attività da svolgere. Questa viene anche indicata come \glossario{\emph{best practise}} per gruppi inesperti.\\
	Il processo con maggiore influenza sulla qualità del sistema non è quello di sviluppo ma è la progettazione, è qui che le capacità e le esperienze dei singoli danno un contributo decisivo.
	Il miglioramento dei processi è un processo ciclico composto da tre sotto processi:
	
	\begin{itemize}
		\item Misurazione del processo: misura gli attributi del progetto, punta ad allineare gli obiettivi con le misurazioni effettuate. Questo forma una \glossario{baseline} che aiuta a capire se i miglioramenti hanno avuto effetto.
		\item Analisi del processo: vengono identificate le problematiche ed i colli di bottiglia dei processi.
		\item Modifiche del processo: i cambiamenti vengono proposti in risposta alle problematiche riscontrate.
	\end{itemize}
	
	Il gruppo procederà nel seguente modo: nella sezione \emph{Dettaglio delle verifiche tramite analisi} (\ref{DettaglioVerificheAnalisi}) del capitolo \emph{Resoconto delle attività di verifica} di questo documento verranno inserite le misurazioni rilevate secondo le metriche descritte in \emph{Misure e Metriche} (\ref{MisureMetriche}). L'analisi viene effettuata i giorni precedenti alle consegne previste dal committente; il \emph{Riassunto delle attività di verifica} (\label{RiassuntoAttivitaVerifica}) contiene l'analisi del processo, le relative considerazioni  comprendenti le problematiche riscontrate.
	Le modifiche al processo vengono attuate all'inizio del processo incrementale successivo. \\
	Queste attività sono programmate nel \emph{Piano di Progetto}.
	
