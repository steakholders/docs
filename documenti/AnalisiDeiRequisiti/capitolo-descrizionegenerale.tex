\section{Descrizione generale}

	\subsection{Contesto d'uso del prodotto}
		
		La tematica che \ProjectName{} affronta riguarda la necessità di creare un'interfaccia di amministrazione di dati \glossario{business} adatta ad un utente finale non esperto di informatica ma del dominio di attività. La creazione di ambienti di sviluppo per la modellazione di queste interfacce risulta spesso onerosa se non si utilizzano strutture di supporto rese disponibili dal mercato, ed è per questo che \ProjectName{} vuole essere	un ambiente di sviluppo innovativo che consenta agli sviluppatori di fare in pochi passaggi un lavoro che normalmente richiederebbe loro diverse ore.\\
		\ProjectName{} é quindi sostanzialmente un \glossario{framework} che permette allo sviluppatore di fornire in maniera rapida pagine web per l'amministrazione di dati, al fine di offrire strumenti adeguati all'esperto di \glossario{business}. Le pagine create vengono rese disponibili agli utenti, permettendo loro (a seconda del tipo di permessi) di poter modificarne i campi e visualizzarne contenuti in maniera chiara e ordinata.
		
I dati verranno prelevati da \glossario{MongoDB}, un sistema gestionale di base di dati di tipo \glossario{NoSQL}. La realizzazione della componente server di \ProjectName{} include l'utilizzo di \textit{\glossario{Node.js}}, un \glossario{framework} \glossario{event-driven} per il motore \glossario{Javascript} V8. Per la realizzazione dell'infrastruttura della \glossario{web application} generata dallo sviluppatore \ProjectName{} include \textit{\glossario{Express}}, un framework basato su \glossario{Node.js} che mette a disposizione una serie di caratteristiche robuste per la creazione di pagine web. L'interfacciamento tra \glossario{MongoDB} e \glossario{Node.js} è supportato da \glossario{Mongoose.js}, che si occupa della modellazione dei dati dell'applicazione e risolve il problema dell'assenza di vincoli di integrità referenziale di \glossario{MongoDB}.

	\subsection{Funzioni del Prodotto}
	Il prodotto \glossario{Framework MaaP} mette a disposizione un linguaggio astratto \glossario{DSL} (Domain Specific Language) con il quale lo sviluppatore potrà:
	\begin{itemize}
		\item Creare un nuovo progetto comprensivo di:
			\begin{itemize}
			\item Directory di descrizione delle pagine web;
			\item File di configurazione;
			\item Librerie di sistema necessarie;
			\item Sistema di autenticazione. 
			\end{itemize}
		\item Configurare la connessione al database delle credenziali utente e al database delle \glossario{Collection};
		\item Registrare nuove \glossario{Collection} con la possibilità di configurare:
		\begin{itemize}
		\item Il nome della \glossario{Collection};
		\item Un eventuale limite di elementi da visualizzare;
		\item La relativa \glossario{index-page};
		\item La relativa \glossario{show-page};
		\end{itemize}
	\end{itemize}
	
	Le pagine web generate da \glossario{Framework MaaP} compongono \glossario{Applicazione MaaP}, la quale mette a disposizione dell'utente la possibilità di:
	\begin{itemize}
	\item Visualizzare la index-page di una \glossario{Collection} all'interno della quale sarà possibile:
	\begin{itemize}
		\item Visualizzare tutti i \glossario{Document} relativi alla \glossario{Collection}
		\item Definire la funzione \textit{\glossario{populate}} di \glossario{MongoDB} necessaria per risolvere i riferimenti ad altri \glossario{Document}. 
		\item Creare un nuovo \glossario{Document} nel caso in cui l'utente abbia permessi di scrittura;
		\item Selezionare per ogni \glossario{Document} una chiave che rimanda alla corrispondente show-page.
	\end{itemize}
	
	\item Visualizzare la show-page di un \glossario{Document} nella quale verranno mostrati tutti i suoi attributi. Nel caso l'utente abbia i permessi di scrittura, può modificare gli attributi del \glossario{Document} e 
	\end{itemize}
		
	\subsection{Caratteristiche degli utenti}
	L'utente principale che interagisce con il \glossario{Framework MaaP} è lo
	\begin{itemize}
		\item \textbf{"\textit{Sviluppatore}"}, il quale, tramite un linguaggio \glossario{DSL} fornito da \ProjectName{}, sarà in grado di generare l'applicazione web rivolta all'utente esperto di business.
	\end{itemize}		
	 
	L'\glossario{Applicazione MaaP} invece potrà essere utilizzata da tre tipologie di utenti:
	\begin{itemize}
		\item \textbf{"\textit{Utente}"}, il quale gode di permessi di sola lettura delle pagine;
		\item \textbf{"\textit{Admin}"}, il quale ha gli stessi privilegi di uno "\textit{Utente}" e inoltre può:
		\begin{itemize}
			\item creare o rimuovere utenti;
			\item elevare o declassare utenti al/dal livello di "\textit{admin}";
			\item scrivere sul \glossario{database} creando o modificando indici.
		\end{itemize}	
		\item \textbf{"\textit{Super Admin}"}, il quale eredita i privilegi dell'admin, non può essere declassato e non se ne possono modificare i dati.
	\end{itemize}

Infine il servizio \glossario{MaaS} potrà essere utilizzato da
	\begin{itemize}
		\item \textbf{"\textit{Utente MaaS}"}, il quale usufruirà del servizio per la creazione di pagine web.
	\end{itemize}
		
	\subsection{Vincoli generali}
	Lo sviluppatore per utilizzare \ProjectName{} deve aver installato correttamente il \glossario{framework}, il quale richiede a sua volta l'installazione di tutto lo \glossario{stack tecnologico} sui cui esso si basa:
	\begin{itemize}
		\item \textbf{\glossario{MongoDB}};
		\item \textbf{\glossario{Node.js}};
		\item \textbf{\glossario{Express}};
		\item \textbf{\glossario{Mongoose.js}}.
	\end{itemize}
	Le applicazioni web create dallo sviluppatore richiedono che l'esperto \glossario{business} disponga anzitutto di una connessione internet nel caso in cui l'applicazione sia resa disponibile on-line, e in secondo luogo di un browser. L'applicazione deve essere inoltre compatibile con la versione 30.0x o superiore di \glossario{Chrome} e la versione 24.x o superiore di \glossario{Firefox}.