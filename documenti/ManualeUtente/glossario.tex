\clearpage
\section{Glossario}

\letteraGlossario{B}
	\definizione{Browser}
	Un web browser, o navigatore, è un programma che consente di usufruire dei servizi di connettività in Internet, o di una rete di computer, e di navigare sul \glossario{World Wide Web}.

\letteraGlossario{C}

	\definizione{Collection}
	Insieme di \glossario{Document}.

	\definizione{Collection index}
	Insieme di pagine generate da MaaP contenenti la visualizzazione in forma tabellare dell'elenco di tutti i documenti della Collection MongoDB.

\letteraGlossario{D}

	\definizione{dashboard}
	Pagina web che rappresenta lo stato corrente di un'applicazione con un interfaccia semplice ed immediata.

	\definizione{document}
	\`E l'unita dato basilare ovvero una riga di una tabella di un insieme di dati.

\letteraGlossario{F}

	\definizione{Footer}
	Traducibile con piè di pagina è una sezione di pagina posizionata sotto il testo vero e proprio

	\definizione{form}
	Indica l'interfaccia di un'applicazione che consente all'utente client di inserire e inviare al web server uno o più dati liberamente digitati dallo stesso.

\letteraGlossario{G}

	\definizione{GitHub}
	Servizio web di hosting per lo sviluppo di progetti software che usa il sistema di controllo di versione Git.

\letteraGlossario{I}

	\definizione{Issue}
	Questioni da discutere per al fine di apportare possibili modifiche sul prodotto.
	
\letteraGlossario{M}

	\definizione{MaaP}
	\glossario{Framework} per generare interfacce web di amministrazione dei \glossario{dati di business} basato su stack \glossario{Node.js} e \glossario{MongoDB}.


	\definizione{MongoDB}
	Sistema gestionale di basi di dati non relazionale, orientato ai documenti, di tipo \glossario{NoSQL}. Il linguaggio utilizzato per la gestione dei dati è \glossario{JavaScript}, del quale sfrutta in particolare la notazione BSON.

\letteraGlossario{N}

	\definizione{Node.js}
	Piattaforma software utilizzata per creare applicazioni distribuite facilmente scalabili.
	Node.js utilizza \glossario{JavaScript} come linguaggio di scripting e gestisce le attese I/O in modo asincrono.

\letteraGlossario{S}

	\definizione{showpage}
	Pagina generata da MaaP che visualizza tutti gli attributi di un documento.

\letteraGlossario{U}

	\definizione{URI}
	Acronimo di Uniform Resource Identifier, è una stringa che identifica univocamente una risorsa generica che può essere un indirizzo Web, un documento, un'immagine, un file, un servizio, ecc. e la rende disponibile tramite protocolli quali HTTP, FTP, ecc.

	\definizione{URL}
	Acronimo di Uniform Resource Locator,  è una sequenza di caratteri che identifica univocamente l'indirizzo di una risorsa in Internet, si tratta di un termine più specifico di \glossario{URI}.

\letteraGlossario{W}

	\definizione{webmaster}
	Persona che amministra e gestisce il servizio web.
	
	\definizione{World Wide Web}
	Il World Wide Web è un servizio Internet che permette di navigare ed usufruire di un insieme vastissimo di contenuti (multimediali e non) collegati tra loro attraverso legami (link), e di ulteriori servizi accessibili a tutti o ad una parte selezionata degli utenti di Internet.

\end{document}
