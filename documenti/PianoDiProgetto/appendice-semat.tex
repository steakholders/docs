\section{Report SEMAT}
Vengono qui riportati i report redatti dal \textit{Responsabile} di progetto che mette in relazione lo stato del gruppo con quelli identificati nel modello SEMAT.

\subsection{Report Progettazione architetturale}

	\subparagraph{Opportunity}
	Una soluzione che rispetti i vincoli è stata individuata, alcuni rischi previsti si sono verificati e sono stati gestiti con successo secondo quanto pianificato nell'analisi dei rischi.

	\subparagraph{\glossario{Stakeholders}}
	I dubbi nati durante lo sviluppo sono stati discussi con gli \glossario{stakeholder} raggiungendo una visione condivisa.

	\subparagraph{Requirements}
	I requisiti descritti costituiscono una soluzione accettabile per gli \glossario{stakeholder}. La frequenza di cambiamento dei requisiti è progressivamente diminuita con la definizione di essi.

	In ogni caso, non si esclude la possibilità che questi debbano essere rivisti in seguito ad una valutazione degli \glossario{stakeholder}.

	\subparagraph{Software System}
	L'architettura ad alto livello del prodotto software è stata individuata. Le tecnologie e gli strumenti necessari per lo sviluppo sono stati individuati. Sono stati scelti i componenti e gli strumenti riusabili, e sviluppati quelli per cui non è stato possibile individuarne di adeguati per il progetto.

	\subparagraph{Team}
	Sono state superate le poche difficoltà iniziali di relazione con i componenti del gruppo. Il team lavora più coordinato e collabora per il conseguimento degli obiettivi che ha definito.

	\subparagraph{Work}
	Il progresso del lavoro è monitorato grazie alle tecniche, agli strumenti utilizzati e alla frammentazione decisa dal \textit{Responsabile} di progetto. Il gruppo opera coerentemente con la suddivisione del lavoro.

	\subparagraph{Way of Working}
	Il gruppo procede con delle variazioni sulle scadenze pianificate. Le pratiche di lavoro stabilite recentemente non sono totalmente assimilate dal gruppo, mentre per quelle discusse in precedenza non si notano evidenti violazioni delle norme. L'utilizzo dei nuovi strumenti necessari per lo sviluppo del prodotto necessitano un periodo di addestramento dei componenti del gruppo.
	
\subsection{Pianificazione Progettazione di dettaglio e codifica}
Di seguito viene pianificato il raggiungimento degli obiettivi sul modello \glossario{SEMAT}. 

\begin{itemize}
	\item Opportunity: confermiamo la pianificazione originaria al livello di \emph{Addressed} in quanto stimiamo di giungere alla milestone con un sistema globalmente funzionante e che soddisfi gli \glossario{stakeholders};
	\item Stakeholders: dato le particolarità dello stack tecnologico adottato, la progettazione si prospetta più ardua del previsto, ecco perché ripianifichiamo di giungere alla milestone soddisfando lo stato \emph{in Agreement};
	\item Requirements: la pianificazione originaria è stata troppo ottimistica, non tutti i requisiti iniziali verranno implementati ma stimiamo di sviluppare quei requisiti che rendono il sistema accettabile, collocandoci quindi sullo stato \emph{Addressed} del \glossario{SEMAT};
	\item Software System: il sistema software prodotto sarà usabile e soddisferà le caratteristiche di qualità fissate in \PianoDiQualifica{}, parte dei test saranno stati svolti e il livello di difetti sarà accettabile. Confermiamo la pianificazione di raggiungere lo stato \emph{Usable};
	\item Team: viste le difficoltà nella collaborazione ripianifichiamo lo stato di avanzamento in \emph{Collaborating};
	\item Work: confermiamo l'avanzamento a \emph{Under Control} poiché l'attività vengono attualmente gestite dal gruppo in modo soddisfacente;
	\item Way-of-Working: pianifichiamo di giungere allo stato \emph{In Place}, retrocedendo di uno stato rispetto a quanto inizialmente pianificato.
\end{itemize}


\subsection{Report Progettazione di dettaglio e codifica}
	
	\newpage