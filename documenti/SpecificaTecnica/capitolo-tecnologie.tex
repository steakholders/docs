\section{Tecnologie utilizzate}


L'architettura è stata progettata utilizzando diverse tecnologie, alcune delle quali espressamente richieste nel capitolato d'appalto. Vengono di seguito elencate e descritte le principali tecnologie impiegate e le motivazioni del loro utilizzo.

\begin{itemize}
	\item \textbf{Node.js}: piattaforma per il \glossario{back-end};
	\item \textbf{Express.js}: framework per la realizzazione dell’applicazione web in \glossario{Node.js};
	\item \textbf{MongoDB}: database di tipo \glossario{NoSQL} per la parte di recupero e salvataggio dei dati;
	\item \textbf{Mongoose}: libreria per interfacciarsi con il driver di \textbf{MongoDB};
	\item \textbf{Angular.js}: framework \glossario{JavaScript} la realizzazione del \glossario{front-end}.
\end{itemize}

\subsection{Node.js}
\textbf{Node.js} è una \glossario{piattaforma} software costruita sul motore \glossario{JavaScript} di \glossario{Chrome} che permette di realizzare facilmente applicazioni di rete scalabili e veloci. \glossario{\textbf{Node.js}} utilizza \glossario{\textbf{JavaScript}} come linguaggio di programmazione, e grazie al suo modello \glossario{event-driver} con chiamate di \glossario{I/O} non bloccanti risulta essere leggero e efficiente.

I principali vantaggi dell'utilizzo di \glossario{Node.js} sono:
\begin{itemize}
	\item \textbf{Approccio asincrono}: \glossario{Node.js} permette di accedere alle risorse del sistema operativo in modalità \glossario{event-driven} e non sfruttando il classico modello basato su processi o thread concorrenti utilizzato dai classici web server. Ciò garantisce una maggiore efficienza in termini di prestazioni, poiché durante le attese il runtime può gestire qualcos'altro in maniera asincrona.
	\item \textbf{Architettura modulare}: Lavorando con \glossario{Node.js} è molto facile organizzare il lavoro in librerie, importare i \glossario{moduli} e combinarli fra loro. Questo è reso molto comodo attraverso il \glossario{\emph{node package manager}} (\textbf{npm}) attraverso il quale lo sviluppatore può contribuire e accedere ai \glossario{package} messi a disposizione dalla community.
\end{itemize}

\subsection{Express.js}
\textbf{Express.js} è un \glossario{framework} minimale per creare applicazioni web con \glossario{Node.js}. Richiede \glossario{moduli} Node di terze parti per applicazioni che prevedono l'interazione con le \glossario{basi di dati}. \\
È stato utilizzato il \glossario{framework} \glossario{Express.js} per supportare lo sviluppo dell'application server grazie alle utili e robuste features da esso offerte, le quali sono pensate per non oscurare le funzionalità fornite da \glossario{Node.js} aprendo così le porte all'utilizzo di moduli per \glossario{Node.js} atti a supportare specifiche funzionalità.

\subsection{MongoDB}
\glossario{\textbf{MongoDB}} è un database \glossario{NoSQL} \glossario{open source} scalabile e altamente performante di tipo document-oriented, in cui i dati sono archiviati sotto forma di documenti in stile \glossario{JSON} con schemi dinamici, secondo una struttura semplice e potente.

I principali vantaggi derivati dal suo utilizzo sono:
\begin{itemize}
	\item \textbf{Alte performance}: non ci sono \emph{join} che possono rallentare le operazioni di lettura o scrittura. L'indicizzazione include gli indici di chiave anche sui documenti innestati e sugli array, permettendo una rapida interrogazione al database;
	\item \textbf{Affidabilità}: alto meccanismo di replicazione su server;
	\item \textbf{Schemaless}: non esiste nessuno \glossario{schema}, è più flessibile e può essere facilmente trasposto in un modello ad oggetti;
	\item Permette di definire query complesse utilizzando un linguaggio che non è \glossario{SQL};
	\item Permette di processare parallelamente i dati (\glossario{Map-Reduce});
	\item Tipi di dato più flessibili.
\end{itemize}

\subsection{Mongoose}
\textbf{Mongoose} è una libreria per interfacciarsi a \glossario{MongoDB} che permette di definire degli schemi per modellare i dati del database, imponendo una certa struttura per la creazione di nuovi \glossario{Document}. Inoltre fornisce molti strumenti utili per la validazione dei dati, per la definizione di query e per il cast dei tipi predefiniti.

Per interfacciare l'application server con \glossario{MongoDB} sono disponibili diversi progetti \glossario{open source}. Per questo progetto è stato scelto di utilizzare \glossario{Mongoose.js}, attualmente il più diffuso.