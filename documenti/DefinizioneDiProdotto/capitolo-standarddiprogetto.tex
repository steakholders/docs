\section{Standard di progetto}

\subsection{Standard di progettazione architetturale}
Gli standard di progettazione sono definiti nella \SpecificaTecnica{}.

Per chiarezza, % TODO (Federico)

\subsection{Standard di documentazione del codice}
Gli standard per la scrittura della documentazione del codice sono definiti nelle \NormeDiProgetto{}.

\subsection{Standard di denominazione di entità e relazioni}
Tutti gli elementi definiti come \glossario{package}, classi, metodi o attributi, devono avere denominazioni chiare ed esplicative. Il nome deve avere una lunghezza tale da non pregiudicarne la leggibilità e chiarezza. \`E preferibile utilizzare dei sostantivi per le entità e dei verbi per le relazioni. Le abbreviazioni sono ammesse se:
\begin{itemize}
	\item immediatamente comprensibili; 
	\item non ambigue;
	\item sufficientemente contestualizzate.
\end{itemize}

Le regole tipografiche relative ai nomi delle entità sono definite nelle \NormeDiProgetto{}.

\subsection{Standard di programmazione}
Gli standard di programmazione sono definiti e descritti nelle \NormeDiProgetto{}.

\subsection{Strumenti di lavoro}
Per gli strumenti di lavoro da utilizzare durante la codifica e le procedure per il loro corretto funzionamento e coordinamento si rimanda al documento \NormeDiProgetto{}.