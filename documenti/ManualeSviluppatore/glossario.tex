\section{Glossario}

\letteraGlossario{A}

\textbf{API}: Application programming interface, sono un insieme di procedure che un sistema software rende accessibile a terzi per interfacciarsi ad esso.

\letteraGlossario{B}

\textbf{Business}: 
Inteso come dominio di Business, l'insieme di tutti i dati che riguardano un dato campo.

\letteraGlossario{C}

\textbf{Collection}: 
In \glossario{MongoDB} è un insieme di \glossario{documents}.

\letteraGlossario{D}

\textbf{Dashboard}: Pagina web che rappresenta lo stato corrente di un'applicazione con un interfaccia semplice ed immediata.

\definizione{Database}
Archivio di dati in cui le informazioni contenute sono strutturate seguendo una logica, può essere relazionale o non-relazionale.

\definizione{Documents}
\glossario{MongoDB} è un \glossario{database} a documenti. I dati al suo interno sono dunque inseriti in documenti, che possono essere paragonati alle tabelle nel mondo dei \glossario{database} relazionali, hanno però una struttura meno rigida e sono codificati in BSON.

\letteraGlossario{F}

\textbf{Framework}:
Struttura di supporto su cui un applicativo può essere progettato.
Un framework comprende librerie di codice, convenzioni di sviluppo e una serie di strumenti di supporto allo sviluppo.

\letteraGlossario{M}

\textbf{MaaP}:
\glossario{Framework} per generare interfacce web di amministrazione dei dati di \glossario{business} basato su stack \glossario{Node.js} e \glossario{MongoDB}.

\textbf{MongoDB}: Sistema gestionale di basi di dati non relazionale, orientato ai documenti, di tipo \glossario{NoSQL}. Il linguaggio utilizzato per la gestione dei dati è JavaScript, del quale sfrutta in particolare la notazione BSON.

\letteraGlossario{N}

\textbf{Node.js}:
Piattaforma software utilizzata per creare applicazioni distribuite facilmente scalabili.
Node.js utilizza JavaScript come linguaggio di scripting e gestisce le attese I/O in modo asincrono.

\textbf{NoSQL}: È un movimento che promuove sistemi software dove la persistenza dei dati è caratterizzata dal fatto di non utilizzare il modello relazionale, tipicamente usato dai database tradizionali. L'espressione NoSQL fa riferimento al linguaggio SQL, che è il più comune linguaggio di interrogazione dei dati nei database relazionali, qui preso a simbolo dell'intero paradigma relazionale. Questi archivi di dati tipicamente non richiedono uno schema fisso, evitano spesso le operazioni di unione e puntano a scalare orizzontalmente.

\letteraGlossario{P}

\textbf{Parsing}: Traducibile con analisi sintattica, è il processo atto ad analizzare uno stream continuo in input in modo da determinare la sua struttura grammaticale grazie ad una data grammatica formale.

\textbf{Populate}:
Inserimento di dati all'interno del \glossario{database}. In questo caso all'interno dei documenti di \glossario{MongoDB}.

\letteraGlossario{S}

\textbf{Stack}: 
Il termine stack o pila indica un tipo di dato astratto che viene usato in diversi contesti per riferirsi a strutture dati, le cui modalità d'accesso ai dati in essa contenuti seguono una modalità LIFO (Last In First Out), tale per cui i dati vengono estratti in ordine rigorosamente inverso rispetto a quello in cui sono stati inseriti.

\letteraGlossario{U}

\textbf{URI}: Acronimo di Uniform Resource Identifier, è una stringa che identifica univocamente una risorsa generica che può essere un indirizzo Web, un documento, un'immagine, un file, un servizio, ecc. e la rende disponibile tramite protocolli quali HTTP, FTP, ecc.