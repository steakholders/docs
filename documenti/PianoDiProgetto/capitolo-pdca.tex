\section{PDCA}
In questo capitolo verrà descritto come è stato applicato il modello \glossario{PDCA} descritto nel \PianoDiQualifica.

\subsection{Revisione dei requisiti}

In questo periodo è stata svolta un attività di \emph{walkthrough} non avendo i dati necessari per effettuare attività di \emph{inspection}, come descritto nel \PianoDiQualifica. Gli errori frequenti rilevati sono visionabili nelle \NormeDiProgetto.

Non è stato possibile eseguire nessun ciclo \glossario{PDCA} in mancanza di misurazioni sui processi, non avendo quindi modo di pianificare processi per la qualità, ma è stato studiato e descritto nel \PianoDiQualifica e verrà attuato dalla prossima \glossario{milestone}.

\subsection{Revisione di progettazione}

Al fine di valutare su quali processi pianificare dei processi di miglioramento sono state eseguite diverse misurazioni utilizzando le metriche per i processi descritti nel \PianoDiQualifica.

I risultati ottenuti sono riportati nella seguente tabella:

\begin{tabular}{ | c | c | c | }
\hline
Documento & Valore indice & Esito \\
Produttività di documentazione & 199 \\
\hline
Impegno & 0,71 & \\
\hline
Modifiche & 

\end{tabular}

Analizzando tali dati si è deciso di pianificare i seguenti processi per il miglioramento della qualità:


  

%\subsection{Revisione di qualifica}

%\subsection{Revisione di accettazione}