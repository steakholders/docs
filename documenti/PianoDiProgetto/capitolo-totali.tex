\section{Totali}


	\subsection{Suddivisione del lavoro}

	Il totale delle ore, comprensive delle ore di Analisi dei requisiti che saranno fornite da ciascun membro del gruppo nel corso dell'intero progetto sono le seguenti:

	\begin{table}[H]
	\centering
	\begin{tabular}{lccccccc}
	\toprule 
	    \textbf{Nome}  & \multicolumn{6}{c}{\textbf{Ore per ruolo}} & \textbf{Ore totali}\\
	     & Amm. & Ana. & Pgt. & Pgr. & Res. & Ver. \\
	    \midrule
			Enrico Rotundo & 14 & 16 & 39 & 18 & 2 & 34 & 123 \\
	Federico Poli & 14 & 13 & 28 & 18 & 10 & 41 & 124 \\
	Giacomo Fornari & 25 & 4 & 22 & 18 & 9 & 46 & 124 \\
	Gianluca Donato & 20 & 18 & 32 & 18 & 4 & 31 & 123 \\
	Luca De Franceschi & 12 & 11 & 28 & 18 & 8 & 40 & 117 \\
	Nicolò Tresoldi & 29 & 4 & 22 & 18 & 13 & 36 & 122 \\
	Serena Girardi & 8 & 24 & 28 & 18 & 2 & 39 & 119 \\

	    \bottomrule
	\end{tabular}
	\caption{Ore per componente totali, incluso periodo di Analisi}
	\end{table}


	\begin{figure}[H]
	\begin{tikzpicture}
	\begin{axis}
		\addplot+[color=Amministratore] plotcoordinates{(Enrico Rotundo,14)(Federico Poli,17.0)(Giacomo Fornari,25)(Gianluca Donato,21.0)(Luca De Franceschi,12)(Nicolò Tresoldi,26.0)(Serena Girardi,8)};
\addplot+[color=Analista] plotcoordinates{(Enrico Rotundo,16)(Federico Poli,13)(Giacomo Fornari,4)(Gianluca Donato,18)(Luca De Franceschi,11)(Nicolò Tresoldi,4)(Serena Girardi,24)};
\addplot+[color=Progettista] plotcoordinates{(Enrico Rotundo,39)(Federico Poli,28)(Giacomo Fornari,22)(Gianluca Donato,29.0)(Luca De Franceschi,28)(Nicolò Tresoldi,22)(Serena Girardi,28)};
\addplot+[color=Programmatore] plotcoordinates{(Enrico Rotundo,18)(Federico Poli,18)(Giacomo Fornari,18)(Gianluca Donato,18)(Luca De Franceschi,31.0)(Nicolò Tresoldi,18)(Serena Girardi,32.0)};
\addplot+[color=Responsabile] plotcoordinates{(Enrico Rotundo,2)(Federico Poli,10)(Giacomo Fornari,9)(Gianluca Donato,4)(Luca De Franceschi,8)(Nicolò Tresoldi,13)(Serena Girardi,2)};
\addplot+[color=Verificatore] plotcoordinates{(Enrico Rotundo,38.0)(Federico Poli,40.0)(Giacomo Fornari,46.0)(Gianluca Donato,35.0)(Luca De Franceschi,31.0)(Nicolò Tresoldi,41.0)(Serena Girardi,29)};

		\legend{amministratore, analista, progettista, programmatore, responsabile, verificatore}
	\end{axis}
	\end{tikzpicture}
	\caption{Ore per componente totali, incluso periodo di Analisi}
	\end{figure}


	\subsection{Prospetto orario}

	La tabella sottostante riporta le ore preventivate e le ore effettivamente impiegate (riportate tra parentesi) per ciascun componente del gruppo \GroupName{} comprensive del periodo di \textit{Analisi}.

	\begin{center}
	\begin{table}[H]
	\begin{tabular}{lccccccc}
	\toprule
	    \textbf{Nome}  & \multicolumn{6}{c}{\textbf{Ore per ruolo}} & \textbf{Ore totali} \\
	     & Amm. & Ana. & Pgt. & Pgr. & Res. & Ver. & \\
	    \midrule
	    	Enrico Rotundo & 10 (+0) & 16 (+2) & 39 (+3) & 18 (+0) & 2 (+0) & 20 (+0) & 105 (+5) \\
	Federico Poli & 14 (+0) & 13 (+2) & 28 (+2) & 18 (+0) & 8 (-1) & 27 (-4) & 108 (-1) \\
	Giacomo Fornari & 21 (+0) & 4 (-0) & 22 (+2) & 18 (+0) & 9 (-2) & 30 (+1) & 104 (+1) \\
	Gianluca Donato & 20 (+0) & 18 (+2) & 24 (+2) & 18 (+0) & 4 (+0) & 21 (-8) & 105 (-4) \\
	Luca De Franceschi & 12 (-4) & 11 (+2) & 20 (+2) & 18 (+0) & 8 (-1) & 26 (-4) & 95 (-5) \\
	Nicolò Tresoldi & 28 (+0) & 4 (+0) & 22 (+2) & 18 (+0) & 13 (+0) & 18 (+0) & 103 (+2) \\
	Serena Girardi & 8 (+0) & 24 (-2) & 28 (+0) & 18 (+0) & 0 (+0) & 21 (-2) & 99 (-4) \\

	    \bottomrule
	\end{tabular}
	\caption{Differenza preventivo consuntivo per componente, totale corrente con Analisi}
	\end{table}
	\end{center}

	Nella seguente tabella sono invece riportate le ore fornite da ciascun componente, escluse quelle rientranti nel periodo di Analisi dei requisiti. 
	Le ore totali preventivabili devono essere comprese tra la soglia minima di 85 ore e quella massima di 105.

	\begin{table}[H]
	\centering
	\begin{tabular}{lccccccc}
	\toprule 
	    \textbf{Nome}  & \multicolumn{6}{c}{\textbf{Ore per ruolo}} & \textbf{Ore totali}\\
	     & Amm. & Ana. & Pgt. & Pgr. & Res. & Ver. \\
	    \midrule
	     & 240 & 0 & 528 & 90 & 0 & 270 & 1128 \\
 & 200 & 0 & 132 & 120 & 150 & 450 & 1052 \\
 & 440 & 0 & 220 & 90 & 120 & 360 & 1230 \\
 & 200 & 200 & 308 & 210 & 120 & 240 & 1278 \\
 & 80 & 0 & 308 & 120 & 120 & 450 & 1078 \\
 & 560 & 100 & 88 & 90 & 0 & 240 & 1078 \\
 & 160 & 250 & 176 & 120 & 0 & 390 & 1096 \\

	    \bottomrule
	\end{tabular}
	\caption{Ore per componente totali, rendicontate}
	\end{table}


	\begin{figure}[H]
	\begin{tikzpicture}
	\begin{axis}
		\addplot+[color=Amministratore] plotcoordinates{(Enrico Rotundo,22)(Federico Poli,14)(Giacomo Fornari,32)(Gianluca Donato,14)(Luca De Franceschi,22)(Nicolò Tresoldi,44.0)(Serena Girardi,16)};
\addplot+[color=Analista] plotcoordinates{(Enrico Rotundo,4)(Federico Poli,4)(Giacomo Fornari,4)(Gianluca Donato,8)(Luca De Franceschi,0)(Nicolò Tresoldi,2)(Serena Girardi,10)};
\addplot+[color=Progettista] plotcoordinates{(Enrico Rotundo,18)(Federico Poli,30)(Giacomo Fornari,32)(Gianluca Donato,34)(Luca De Franceschi,34)(Nicolò Tresoldi,22)(Serena Girardi,26)};
\addplot+[color=Programmatore] plotcoordinates{(Enrico Rotundo,20)(Federico Poli,16)(Giacomo Fornari,16)(Gianluca Donato,28)(Luca De Franceschi,18)(Nicolò Tresoldi,20)(Serena Girardi,20)};
\addplot+[color=Responsabile] plotcoordinates{(Enrico Rotundo,0)(Federico Poli,9.0)(Giacomo Fornari,4)(Gianluca Donato,4)(Luca De Franceschi,6)(Nicolò Tresoldi,0)(Serena Girardi,4)};
\addplot+[color=Verificatore] plotcoordinates{(Enrico Rotundo,22)(Federico Poli,28)(Giacomo Fornari,30)(Gianluca Donato,20)(Luca De Franceschi,32)(Nicolò Tresoldi,34)(Serena Girardi,32)};

		\legend{amministratore, analista, progettista, programmatore, responsabile, verificatore}
	\end{axis}
	\end{tikzpicture}
	\caption{Ore per componente totali, rendicontate}
	\end{figure}

	La tabella sottostante riporta le ore preventivate e le ore effettivamente impiegate (riportate tra parentesi) per ciascun componente del gruppo \GroupName{} non comprensive del periodo di \textit{Analisi}.

	\begin{center}
	\begin{table}[H]
	\begin{tabular}{lccccccc}
	\toprule
	    \textbf{Nome}  & \multicolumn{6}{c}{\textbf{Ore per ruolo}} & \textbf{Ore totali} \\
	     & Amm. & Ana. & Pgt. & Pgr. & Res. & Ver. & \\
	    \midrule
	    	Enrico Rotundo & 10 (+0) & 4 (+0) & 30 (+2) & 18 (+0) & 2 (+0) & 18 (+0) & 82 (+2) \\
	Federico Poli & 14 (+0) & 4 (+0) & 28 (+2) & 18 (+0) & 4 (+0) & 18 (+0) & 86 (+2) \\
	Giacomo Fornari & 12 (+0) & 2 (-0) & 22 (+2) & 18 (+0) & 4 (-2) & 24 (+1) & 82 (+1) \\
	Gianluca Donato & 20 (+0) & 8 (+0) & 24 (+2) & 18 (+0) & 4 (+0) & 7 (-1) & 81 (+1) \\
	Luca De Franceschi & 6 (-4) & 0 (+0) & 20 (+2) & 18 (+0) & 8 (-1) & 26 (-1) & 78 (-4) \\
	Nicolò Tresoldi & 15 (+0) & 4 (+0) & 22 (+2) & 18 (+0) & 8 (+0) & 15 (+2) & 82 (+4) \\
	Serena Girardi & 8 (+0) & 10 (-2) & 28 (+0) & 18 (+0) & 0 (+0) & 16 (-2) & 80 (-4) \\

	    \bottomrule
	\end{tabular}
	\caption{Differenza preventivo consuntivo per componente, corrente totale}
	\end{table}
	\end{center}


	\subsection{Prospetto economico}

	In totale le ore per ogni ruolo sono state cosi suddivise:

	\begin{table}[H]
	\centering
	\begin{tabular}{ l c c }
		\textbf{Ruolo} & \textbf{Ore} & \textbf{Costo} \\
		\hline
			Amministratore & 95 & 1900 € \\
	Analista & 32 & 800 € \\
	Progettista & 190 & 4180 € \\
	Programmatore & 126 & 1890 € \\
	Responsabile & 34 & 1020 € \\
	Verificatore & 229 & 3435 € \\
\hline
	Totale & 706 & 13225 € \\
\hline

	\end{tabular}
	\caption{Ore e costo per ruolo, riassunto progetto}
	\end{table}

	Il seguente grafico mostra il costo di ogni ruolo durante tutto lo svolgimento del progetto, escluso il periodo di Analisi dei requisiti.

	\begin{figure}[H]
	\begin{tikzpicture}
		\pie[text=legend, color={amministratore, analista, progettista, programmatore, responsabile, verificatore}]{16.2/Amministratore, 5.7/Analista, 30.5/Progettista, 14.4/Programmatore, 8.0/Responsabile, 25.2/Verificatore}

	\end{tikzpicture}
	\caption{Costo per ruolo}
	\end{figure}

	La tabella sottostante riporta le ore preventivate e le ore effettivamente impiegate (riportate tra parentesi) per ciascun ruolo presente nel periodo totale corrente alla consegna del presente documento, escluso il periodo di Analisi dei requisiti.

	\begin{table}[H]
	\centering
	\begin{tabular}{lccccccc}
	\toprule
	    \textbf{Ruolo}  & \textbf{Ore} & \textbf{Costo} \\
	    \midrule
	    	Amministratore & 83 (+4) & 1660 (+77) € \\
	Analista & 32 (+2) & 800 (+54) € \\
	Progettista & 174 (-12) & 3828 (-268) € \\
	Programmatore & 126 (+0) & 1890 (+0) € \\
	Responsabile & 30 (+3) & 900 (+90) € \\
	Verificatore & 132 (+8) & 1980 (+124) € \\
\hline
\textbf{Totale consuntivo} & +572 & +10981 € \\
\textbf{Totale preventivo} & +577 & +11058 € \\
\textbf{Differenza dei totali} & +5 & +77 € \\

	    \bottomrule
	\end{tabular}
	\caption{Differenza preventivo consuntivo per ruolo, totale corrente}
	\end{table}

	\begin{figure}[H]
	\centering
	\begin{tikzpicture}
	\begin{axis}[stileRuoli]
		\addplot+[color=Pianificate] plotcoordinates {(Amministratore,83.0)(Analista,32)(Progettista,174)(Programmatore,126)(Responsabile,30)(Verificatore,132)};
\addplot+[color=Consumate] plotcoordinates {(Amministratore,79.1666666667)(Analista,29.8333333333)(Progettista,186.166666667)(Programmatore,126)(Responsabile,27.0)(Verificatore,123.75)};

		\legend{preventivate,consumate}
	\end{axis}
	\end{tikzpicture}
	\caption{Differenza preventivo consuntivo per ruolo, totale corrente}
	\end{figure}