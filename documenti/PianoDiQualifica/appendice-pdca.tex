\section{PDCA}
In questo capitolo verrà descritto come è stato applicato il modello \glossario{PDCA} descritto nel \PianoDiQualifica.

\subsection{Revisione dei requisiti}

In questo periodo è stata svolta un attività di \emph{walkthrough} non avendo i dati necessari per effettuare attività di \emph{inspection}, come descritto nel \PianoDiQualifica. Gli errori frequenti rilevati sono visionabili nelle \NormeDiProgetto.

Non è stato possibile eseguire nessun ciclo \glossario{PDCA} in mancanza di misurazioni sui processi, non avendo quindi modo di pianificare processi per la qualità, ma è stato studiato e descritto nel \PianoDiQualifica e verrà attuato dalla prossima \glossario{milestone}.

\subsection{Revisione di progettazione}

\subparagraph{PLAN}

Al fine di valutare su quali processi pianificare dei processi di miglioramento sono state eseguite diverse misurazioni utilizzando le metriche per i processi descritti nel \PianoDiQualifica.

I risultati ottenuti sono riportati nella seguente tabella:

\begin{tabular}{ | c | c | c | }
\hline
Documento & Valore indice & Esito \\
\hline
Produttività di documentazione & 199 & Superato \\
\hline
Impegno & 0,71 & Superato \\
\hline
Modifiche & 23 & Non superato \\
\hline
\end{tabular}

Analizzando tali dati si è deciso di pianificare i seguenti processi per il miglioramento della qualità:

Un numero troppo elevato di modifiche incide pesantemente sulla produttività. È necessario decrementare tale valore al fine di aumentare la produttività e di conseguenza diminuire i costi. Questo primo ciclo \glossario{PDCA} si prefigge dunque l'obiettivo di portare entro un range di accettazione\footnote{Vedi \PianoDiQualifica} il numero di modifiche approvate.

Probabilmente un numero elevato di modifiche è causato dall'inesperienza del gruppo nel primo periodo, e ragionevolmente con l'aumentare delle conoscenze il numero di modifiche andrà calando di conseguenza. 

In ogni caso si pianifica di:
\begin{itemize}
\item Frammentare maggiormente i task assegnati in sotto-task;
\item Specificare in modo esteso cosa prevede ogni singolo sotto-task, escludendo quindi dubbi che poi porteranno a successive richieste di modifica;
\item Creare la label ``Domanda'' nella sezione \glossario{issue} di \glossario{GitHub}, per permettere la richiesta di delucidazioni sullo svolgimento di \glossario{task} assegnati.
\end{itemize}
  
\subparagraph{CHECK}

Al fine di valutare se le azioni pianificate hanno portato ad un miglioramento dei processi sono state eseguite le necessarie misurazioni.

I risultati ottenuti sono riportati nella seguente tabella:

\begin{tabular}{ | c | c | c | }
\hline
Documento & Valore indice & Esito \\
\hline
Modifiche & 18 & Superato \\
\hline
\end{tabular}

Gli obiettivi posti nello stadio di pianificazione sono stati soddisfatti, si passerà dunque allo stadio di standardizzazione delle soluzioni applicate. 

\subsection{Revisione di qualifica}

\subparagraph{PLAN}

Al fine di valutare su quali processi pianificare dei processi di miglioramento sono state eseguite diverse misurazioni utilizzando le metriche per i processi descritti nel \PianoDiQualifica.

I risultati ottenuti sono riportati nella seguente tabella:

%TODO Completare tabella 

\begin{tabular}{ | c | c | c | }
\hline
Documento & Valore indice & Esito \\
\hline
Produttività di documentazione & ?? & ?? \\
\hline
Impegno & ?? & ?? \\
\hline
Modifiche & ?? & ?? \\
\hline
\end{tabular}

%\subsection{Revisione di accettazione}