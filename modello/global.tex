%%%%%%%%%%%%%%
%  COSTANTI  %
%%%%%%%%%%%%%%

% In questa prima parte vanno definite le 'costanti' utilizzate da due o più documenti.

\newcommand{\GroupName}{\emph{SteakHolders}}
\newcommand{\ProjectName}{\emph{MaaP}}

\newcommand{\Proponente}{CoffeeStrap}
\newcommand{\Committente}{Prof. Tullio Vardanega}
\newcommand{\Responsabile}{}

% La versione dei documenti deve essere definita qui in global, perchè serve anche agli altri documenti
%TODO: Aggiungere tutte le versioni
\newcommand{\VersioneG}{??.??.??}
\newcommand{\VersionePQ}{??.??.??}
\newcommand{\VersioneNP}{1.1.5}
\newcommand{\VersionePP}{??.??.??}
\newcommand{\VersioneV}{ - } %Il vebale non ha versionamento.
\newcommand{\VersioneAR}{1.1.1}
\newcommand{\VersioneSF}{??.??.??}

% Quando serve riferirsi a ``Nome del Documento + ultima versione x.y.z'' usiamo queste costanti:
\newcommand{\Glossario}{\emph{Glossario v\VersioneG{}}}
\newcommand{\PianoDiQualifica}{\emph{Piano di Qualifica v\VersionePQ{}}}
\newcommand{\NormeDiProgetto}{\emph{Norme di Progetto v\VersioneNP}}
\newcommand{\PianoDiProgetto}{\emph{Piano di Progetto v\VersionePP}}
\newcommand{\StudioDiFattibilita}{\emph{Studio di Fattibilità v\VersioneSF}}

\newcommand{\ScopoDelProdotto}{
	Lo scopo del progetto è la realizzazione di un \glossario{framework} per generare interfacce web di amministrazione dei dati di \glossario{business} basato su \glossario{stack} \glossario{Node.js} e \glossario{MondoDB}. L'obbiettivo è quello di semplificare il processo di implementazione di tali interfacce che lo sviluppatore, appoggiandosi alla produttività del framework MaaP, potrà generare in maniera semplice e veloce ottenendo quindi un considerevole risparmio di tempo e di sforzo. Il fruitore finale delle pagine generate sarà infine l'esperto di business che potrà visualizzare, gestire e modificare le varie entità e dati residenti in \glossario{MongoDB}.
	Il prodotto atteso si chiama \glossario{MaaP} ossia \emph{MongoDB as an admin Platform}.
}

%%%%%%%%%%%%%%
%  FUNZIONI  %
%%%%%%%%%%%%%%

% In questa seconda parte vanno definite le 'funzioni' utilizzate da due o più documenti.

% Serve a dare la giusta formattazione alle parole presenti nel glossario
% il nome del comando \glossary è già usato da LaTeX
\newcommand{\glossario}[1]{\mbox{\textit{#1}\ped{\ped{G}}}}

% Serve a dare la giusta formattazione al codice inline
\newcommand{\code}[1]{\texttt{#1}}

% Serve a dare la giusta formattazione a tutte le path presenti nei documenti
\newcommand{\file}[1]{\texttt{#1}}

% Permette di andare a capo all'interno di una cella in una tabella
\newcommand{\multiLineCell}[2][c]{\begin{tabular}[#1]{@{}l@{}}#2\end{tabular}}

% Genera automaticamente la pagina di copertina
\newcommand{\makeFrontPage}{
  \begin{titlepage}
  \begin{center}

  \begin{center}
  \includegraphics[width=10cm]{../../modello/steakman.png}
  \end{center}
  
  \vspace{33pt}

  \begin{Huge}
  \textbf{\DocTitle{}}
  \end{Huge}
  
  \textbf{\emph{Gruppo} \GroupName{} \, \texttwelveudash \, \emph{Progetto} \ProjectName{}}
  
  \vspace{11pt}

  \bgroup
  \def\arraystretch{2}
  \begin{tabular}{ r|l }
    \multicolumn{2}{c}{\textbf{Informazioni sul documento}} \\
    \hline
    \textbf{Versione} & \DocVersion{} \\
    \textbf{Redazione} & \multiLineCell[t]{\DocRedazione{}} \\
    \textbf{Verifica} & \multiLineCell[t]{\DocVerifica{}} \\
    \textbf{Approvazione} & \multiLineCell[t]{\DocApprovazione{}} \\
    \textbf{Uso} & \DocUso{} \\
    \textbf{Distribuzione} & \multiLineCell[t]{\DocDistribuzione{}} \\
  \end{tabular}
  \egroup

  \vspace{22pt}

  \textbf{Descrizione} \\
  \DocDescription{}

  \end{center}
  \end{titlepage}
}
