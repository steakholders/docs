\section{Report SEMAT}
Vengono qui riportati i report redatti dal \textit{Responsabile} di progetto che mette in relazione lo stato del gruppo con quelli identificati nel modello SEMAT.

\subsection{Report Progettazione architetturale}

	\subparagraph{Opportunity}
	Una soluzione che rispetti i vincoli è stata individuata, alcuni rischi previsti si sono verificati e sono stati gestiti con successo secondo quanto pianificato nell'analisi dei rischi.

	\subparagraph{\glossario{Stakeholders}}
	I dubbi nati durante lo sviluppo sono stati discussi con gli \glossario{stakeholder} raggiungendo una visione condivisa.

	\subparagraph{Requirements}
	I requisiti descritti costituiscono una soluzione accettabile per gli \glossario{stakeholder}. La frequenza di cambiamento dei requisiti è progressivamente diminuita con la definizione di essi.

	In ogni caso, non si esclude la possibilità che questi debbano essere rivisti in seguito ad una valutazione degli \glossario{stakeholder}.

	\subparagraph{Software System}
	L'architettura ad alto livello del prodotto software è stata individuata. Le tecnologie e gli strumenti necessari per lo sviluppo sono stati individuati. Sono stati scelti i componenti e gli strumenti riusabili, e sviluppati quelli per cui non è stato possibile individuarne di adeguati per il progetto.

	\subparagraph{Team}
	Sono state superate le poche difficoltà iniziali di relazione con i componenti del gruppo. Il team lavora più coordinato e collabora per il conseguimento degli obiettivi che ha definito.

	\subparagraph{Work}
	Il progresso del lavoro è monitorato grazie alle tecniche, agli strumenti utilizzati e alla frammentazione decisa dal \textit{Responsabile} di progetto. Il gruppo opera coerentemente con la suddivisione del lavoro.

	\subparagraph{Way of Working}
	Il gruppo procede con delle variazioni sulle scadenze pianificate. Le pratiche di lavoro stabilite recentemente non sono totalmente assimilate dal gruppo, mentre per quelle discusse in precedenza non si notano evidenti violazioni delle norme. L'utilizzo dei nuovi strumenti necessari per lo sviluppo del prodotto necessitano un periodo di addestramento dei componenti del gruppo.
	
\subsection{Pianificazione Progettazione di dettaglio e codifica}
Di seguito viene pianificato il raggiungimento degli obiettivi sul modello \glossario{SEMAT}. 

	\subparagraph{Opportunity}
	Confermiamo la pianificazione originaria al livello di \emph{Addressed} in quanto stimiamo di giungere alla milestone con un sistema globalmente funzionante e che soddisfi gli \glossario{stakeholders}.

	\subparagraph{\glossario{Stakeholders}}
	Date le particolarità dello stack tecnologico adottato, la progettazione si prospetta più ardua del previsto, ecco perché ripianifichiamo di giungere alla milestone soddisfando lo stato \emph{In Agreement}.
	
	\subparagraph{Requirements}
	La pianificazione originaria è stata troppo ottimistica, non tutti i requisiti iniziali verranno implementati ma stimiamo di sviluppare quei requisiti che rendono il sistema accettabile, collocandoci quindi sullo stato \emph{Addressed} del \glossario{SEMAT}.
	
	\subparagraph{Software System}
	Il sistema software prodotto sarà usabile e soddisferà le caratteristiche di qualità fissate in \PianoDiQualifica{}, parte dei test saranno stati svolti e il livello di difetti sarà accettabile. Confermiamo la pianificazione di raggiungere lo stato \emph{Usable}.
	
	\subparagraph{Team}
	Viste le difficoltà nella collaborazione ripianifichiamo lo stato di avanzamento in \emph{Collaborating}.
	
	\subparagraph{Work}
	Confermiamo l'avanzamento a \emph{Under Control} poiché le attività vengono attualmente gestite dal gruppo in modo soddisfacente.
	
	\subparagraph{Way of Working}
	Confermiamo la pianificazione a \emph{Working Well} poiché il gruppo non ha ancora raggiunto tale stato.

\subsection{Report Progettazione di dettaglio e codifica}
	
	\subparagraph{Opportunity}
	L'obiettivo di raggiungere lo stato \emph{Addressed} è stato raggiunto. \`E disponibile un sistema funzionante e gli \glossario{stakeholders} concordano con le soluzioni adottate.

	\subparagraph{\glossario{Stakeholders}}
	Il gruppo è riuscire a giungere in uno stato intermedio tra \emph{In Agreement} e \emph{Satisfied for Deployment}. Gli \glossario{stakeholders} hanno fornito dei feedback positivi in merito allo stato del sistema. %TODO aggiungere ref al verbale del 5/3

	\subparagraph{Requirements}
	Il gruppo è giunto allo stato \emph{Addressed}. Gran parte dei requisiti sono stati sviluppati al fine di rendere il sistema accettabile. Gli \glossario{stakeholders} concordando con quanto sviluppato.

	\subparagraph{Software System}
	Il gruppo è giunto allo stato \emph{Usable}. \`E disponibile un sistema funzionante sia per la parte di Frontend che per quella di Backend, seppur con qualche anomalia. \`E disponibile documentazione fruibile dagli utenti del sistema. I test pianificati sono stati eseguiti.
 
	\subparagraph{Team}
	Il gruppo è stabile allo stato \emph{Collaborating}. La comunicazione nel gruppo è facilitata grazie al maturamento dei rapporti interpersonali. 

	\subparagraph{Work}
	Il gruppo è giunto allo stato \emph{Under Control}. In particolare, i rischi e il lavoro pianificato vengono gestiti in modo soddisfacente. Le metriche sono utilizzate per monitorare l'andamento del lavoro.

	\subparagraph{Way of Working}
	Il gruppo è si colloca in uno stato intermedio tra \emph{In Place} e \emph{Working Well}. Alcuni strumenti per lo sviluppo del prodotto sono utilizzati e le procedure sono state comprese. L'applicazione di queste ultime manca ancora di sistematicità.

\subsection{Pianificazione Validazione}
Di seguito viene pianificato il raggiungimento degli obiettivi sul modello \glossario{SEMAT}. 

	\subparagraph{Opportunity}
	Pianifichiamo di giungere allo stato \emph{Benefit Accrued} presentando il prodotto finale e che verrà valutato dai committenti.

	\subparagraph{\glossario{Stakeholders}}
	Pianifichiamo di giungere allo stato \emph{Satisfied for Deployment}, comunicando con gli \emph{stakeholders} per giungere ad una visione condivisa soddisfacente.
	
	\subparagraph{Requirements}
	Pianifichiamo di giungere allo stato \emph{Fulfilled} sviluppando i requisiti stabiliti e accettati.
	
	\subparagraph{Software System}
	Pianifichiamo di giungere allo stato \emph{Ready} consegnando la documentazione per l'utente e comunicando con gli \emph{stakeholders} per l'accettazione del sistema.
	
	\subparagraph{Team}
	Pianifichiamo di giungere allo stato \emph{Adjourned} raggiungendo la conclusione del progetto presentando il prodotto finale.
	
	\subparagraph{Work}
	Pianifichiamo di giungere allo stato \emph{Concluded/Closed} concludendo ufficialmente il lavoro.
	
	\subparagraph{Way of Working}
	Pianifichiamo di giungere allo stato \emph{Retired} concludendo ufficialmente il lavoro.

	\newpage