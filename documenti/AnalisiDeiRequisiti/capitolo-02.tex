\section{Descrizione generale}

	\subsection{Contesto d'uso del prodotto}
		
		La tematica che \ProjectName affronta è il bisogno di creare un'interfaccia di amministrazione dei dati di business adatta
		ad un utente finale non esperto di informatica ma del dominio di business.\\
		La creazione di ambienti di sviluppo per la modellazione di queste interfacce che si realizzano spesso in pagine web risulta
		onerosa se non si utilizzano stack tecnologici che il mercato rende disponibili, ed è questo che \ProjectName vuole essere,
		interfacciandosi per persistere i suoi dati a MongoDB ed utilizzando Node.js, dando vita ad un ambiente di sviluppo
		innovativo per le tecnologie adottate e per non essere presente, con tale stack tecnologico, nella panoramica di quelli già
		esistenti. \\
		 \ProjectName é quindi un framework che permette allo sviluppatore di fornire in maniera rapida pagine web per
		 l'amministrazione di dati, al fine di offrire strumenti adeguati all'esperto di business.
		Le pagine create vengono rese disponibili agli utenti, permettendo loro, a seconda del tipo di permessi, di poter modificarne
		i campi e visualizzarne contenuti dando quindi un interfacciamento ai dati semplice e personalizzabile.

	\subsection{Funzioni del Prodotto}
	
		
	\subsection{Caratteristiche degli utenti}
	Gli utenti che interagiscono con il prodotto sono:
	\begin{itemize}
	 \item \emph{Sviluppatore} \\
	        Lo sviluppatore potrà generare pagine web attraverso l'uso di un DSL/API?, potendo creare due tipi di pagine con
	        opportuni parametri/funzioni?.
	  \item \emph{Utente} \\
	  		L'utente, dopo essersi autenticato, potrà accedere alle pagine.\\
	  		Sono presenti due categorie di utenti: 		
	  		\begin{itemize}
	  		\item[-] \emph{User} : questa categoria di utenti ha i permessi di sola lettura e visualizzazione dei dati.
	  		\item[-] \emph{Admin} : gli admin hanno potere di creazione e rimozione degli utenti ed di poterli elevare o declassare
	  		                        alla categoria ''admin''.
	  								Possiedono inoltre permessi di scrittura ai dati.
	  		\end{itemize}
	\end{itemize}
	
	\subsection{Vincoli generali}
	Per l'accesso alle pagine web, l'utente deve predisporre di un collegamento internet e di un browser Chrome di versione 30.0.x o
	superiore o Firefox versione 24.x o superiore.
	\\ -Serve installare tutto lo stack tecnologico per quanto riguarda lo sviluppatore?

	\subsection{Assunzioni e dipendenze}

