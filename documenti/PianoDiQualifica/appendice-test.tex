\pagebreak
\section{Pianificazione dei test}
Si vuole adottare una strategia di verifica del \glossario{software} tramite test opportunamente organizzati e determinati precedentemente all attività di codifica. I test che si andranno ad applicare sono di quattro tipi, riservando la specifica dell'ultima tipologia alla prossima revisione. Ogni requisito verrà testato.

\begin{enumerate}
	\item Test di Validazione (TV): viene verificato che il prodotto soddisfi quanto richiesto dal \glossario{proponente}.
	\item Test di Sistema (TS): vengono verificate che le singole funzionalità del sistema funzioni come previsto.
	\item Test di Integrazione (TI): vengono verificate le componenti del sistema specificate nella \SpecificaTecnica{}.
	\item Test di Unità (TU): verrà descritta ed approfondita entro la prossima revisione.
\end{enumerate}

	
	\subsection{Test di sistema}
	Vengono qui descritti i test di sistema che andranno a verificare il funzionamento complessivo delle componenti.
	
	%TODO qui va la tabella descrzione dei TS / requisiti
	
	\subsection{Test di integrazione}
	I test di integrazione vanno a controllare il corretto funzionamento delle componenti descritti dalla progettazione ad alto livello. 
	
	%TODO tabella TI: test / descrizione / componente / stato
	%TODO tabellla: componente / test

	\subsection{Test di validazione}
	In questa sezione vengono elencati i test di validazione per verificare che il prodotto sia conforme alle attese. I test si svolgono seguendo e verificano tutti passi di cui si compongono. 
	
	%TODO elenco puntato dei TV
	
	
	
	
