\section{Use Case Points}

La metrica Use Case Points è stata applicata al fine di valutare l'attuabilità dello sviluppo di quali e quanti requisiti prima dell'accettazione.

Segue un breve studio di fattibilità basato sui risultati derivanti dall'applicazione degli Use Case Points.


\subsection{Fattori tecnici dell'implementazione}

I 13 fattori tecnici valutati sono:

\begin{enumerate}
	\item \textbf{Distributed System Required}: l'architettura della soluzione può essere centralizzata o single-tenant, o può essere distribuita (come una soluzione n-tier) o multi-tenant. Numeri più alti rappresentano un'architettura più complessa;
	\item \textbf{Response Time Is Important}: la rapidità di risposta per gli utenti è un fattore importante (e non banale), tranne nel caso in cui il carico del server dovesse essere molto basso. Numeri alti rappresentano un'alta importanza del tempo di risposta;
	\item \textbf{End User Efficiency}: l'applicazione è sviluppata per migliorare l'efficienza d'uso, o semplicemente la capacità? Numeri alti rappresentano progetti che si basano più pesantemente sull'applicazione per migliorare l'efficienza d'uso;
	\item \textbf{Complex Internal Processing Required}: ci sono molti algoritmi complessi da implementare e testare? Algoritmi complessi hanno numeri alti. Delle semplici query al database dovrebbero avere numeri bassi;
	\item \textbf{Reusable Code Must Be a Focus}: il riuso massivo del codice è un obiettivo? Maggiore è il riuso del codice, minore è il valore relativo;
	\item \textbf{Installation Ease}: un'installazione agevole è un fattore chiave? Maggiore è il livello di competenza degli utenti, minore è il valore relativo;
	\item \textbf{Usability}: la facilità di utilizzo è un criterio primario per l'accettazione? Maggiore è l'importanza dell'usabilità, maggiore è il valore relativo;
	\item \textbf{Cross-Platform Support}: è richiesto il supporto multi-piattaforma? Più piattaforme devono essere supportate, più alto è il valore relativo;
	\item \textbf{Easy To Change}: il fornitore richiede di poter cambiare o personalizzare l'applicazione in futuro? Più richieste sono mosse in questa direzione, maggiore è il valore relativo;
	\item \textbf{Highly Concurrent}: bisogna preoccuparsi di questioni di concorrenza? Più attenzione bisogna prestare a risolvere conflitti su dati o sull'applicazione, maggiore è il valore relativo;
	\item \textbf{Custom Security}: è possibile utilizzare soluzioni già esistenti riguardanti la sicurezza o è necessario svilupparne di personali? Maggiori soluzioni personalizzate si devono implementare, più alto è il valore relativo;
	\item \textbf{Dependence on Third Party Code}: l'applicazione necessita dell'uso di librerie o altro software di terze parti? Più codice di terze parti (e più affidabile è), minore è li valore relativo;
	\item \textbf{User Training}: quanta formazione dell'utente è richiesta? L'applicazione è complessa o supporta attività complesse? Più tempo l'utente impiega nella formazione, maggiore è il valore relativo. 
\end{enumerate}

Per ogni fattore è necessario attribuire un valore da 0 a 5 che costituisce l'importanza relativa.

Il Technical Complexity Factor è calcolato sommando le importanze relative moltiplicate per il peso corrispondente, diviso per 100 e sommato a 0,6.

\begin{table}[H]
\begin{adjustwidth}{-4cm}{-4cm}
	\begin{center}
		\begin{tabular}{|c|c|c|c|c|}
			\hline
			\multicolumn{ 2}{|c|}{\raisebox{-1\height}{\textbf{Fattori tecnici}}} & \multicolumn{ 1}{c|}{\raisebox{-1\height}{\textbf{Peso}}} & \multicolumn{ 2}{c|}{\textbf{Importanza relativa}} \\ \cline{ 4- 5}
			\multicolumn{ 2}{|c|}{} & \multicolumn{ 1}{c|}{} & \textbf{Con tutti i requisiti} & \textbf{Solo requisiti acc.} \\ \hline
			1 & Distributed System Required & 2 & \textbf{4} & \textbf{0} \\ \hline
			2 & Response Time Is Important & 1 & \textbf{1} & \textbf{1} \\ \hline
			3 & End User Efficiency & 1 & \textbf{3} & \textbf{3} \\ \hline
			4 & Complex Internal Processing Required & 1 & \textbf{1} & \textbf{1} \\ \hline
			5 & Reusable Code Must Be A Focus & 1 & \textbf{2} & \textbf{2} \\ \hline
			6 & Installation Ease & 0.5 & \textbf{1} & \textbf{1} \\ \hline
			7 & Usability & 0.5 & \textbf{3} & \textbf{3} \\ \hline
			8 & Cross-Platform Support & 2 & \textbf{4} & \textbf{4} \\ \hline
			9 & Easy To Change & 1 & \textbf{1} & \textbf{1} \\ \hline
			10 & Highly Concurrent & 1 & \textbf{1} & \textbf{1} \\ \hline
			11 & Custom Security & 1 & \textbf{2} & \textbf{2} \\ \hline
			12 & Dependence On Third-Party Code & 1 & \textbf{2} & \textbf{2} \\ \hline
			13 & User Training & 1 & \textbf{3} & \textbf{2} \\ \hline
			\multicolumn{ 3}{|c|}{\textbf{Technical Complexity Factor}} & \textbf{0.94} & \textbf{0.85} \\ \hline
		\end{tabular}
	\end{center}
\caption{UCP - Fattori tecnici}
\end{adjustwidth}
\end{table}

I fattori che variano sono \emph{Distributed System Required} poiché non verrà implementato \glossario{MaaS} e \emph{User Training} in quanto la formazione degli utenti dovrà trattare di una minore mole di materiale.


\subsection{Fattori ambientali}

Gli 8 fattori ambientali valutati sono:

\begin{enumerate}
	\item \textbf{Familiarity With The Project}: quanta esperienza ha maturato il team nel dominio di sviluppo? Un alto livello di esperienza corrisponde ad un numero alto;
	\item \textbf{Application Experience}: quanta esperienza ha maturato il team con l'applicazione? Questo è rilevante solo quando si attuano cambiamenti a applicazioni già esistenti. Un valore alto corrisponde a molta esperienza;
	\item \textbf{Object Oriented Programming Experience}: quanta esperienza ha maturato il team verso l'Object Oriented? Un valore alto corrisponde a molta esperienza nell'Object Oriented;
	\item \textbf{Lead Analyst Capability}: quanto è esperto e abile la persona responsabile dei requisiti? Un valore alto rappresenta abilità e esperienza;
	\item \textbf{Motivation}: quanto è motivato il team? Un valore alto corrisponde a molta motivazione;
	\item \textbf{Stable Requirements}: cambiamenti nei requisiti comportano un incremento della mole di lavoro. Un valore alto corrisponde a molti cambiamenti;
	\item \textbf{Part Time Staff}: un valore alto riflette un team part time, consulenti esterni e sviluppatori che si occupano contemporaneamente di più progetti. Un cambiamento di contesto è causa di inefficienza;
	\item \textbf{Difficult Programming Language}: linguaggi di programmazione complessi corrispondono a numeri elevati. Questo valore è da attribuire in relazione alle capacità del team e non in senso assoluto.
\end{enumerate}

Per ogni fattore è necessario attribuire un valore da 0 a 5 che costituisce l'importanza relativa.

Ponendo $S$ la somma delle importanze relative moltiplicate per il peso corrispondente, l'Environmental Factor viene così calcolato:

\[
EF=1.4-(0.03 \cdot S)
\]

\begin{table}[H]
\begin{adjustwidth}{-4cm}{-4cm}
	\begin{center}
		\begin{tabular}{|c|c|c|c|c|}
			\hline
			\multicolumn{ 2}{|c|}{\raisebox{-1\height}{\textbf{Fattori ambientali}}} & \multicolumn{ 1}{c|}{\raisebox{-1\height}{\textbf{Peso}}} & \multicolumn{ 2}{c|}{\textbf{Importanza relativa}} \\ \cline{ 4- 5}
			\multicolumn{ 2}{|c|}{\textbf{}} & \multicolumn{ 1}{c|}{} & \textbf{Con tutti i requisiti} & \textbf{Solo requisiti acc.} \\ \hline
			1 & Familiarity With The Project & 1.5 & \textbf{0} & \textbf{0} \\ \hline
			2 & Application Experience & 0.5 & \textbf{0} & \textbf{0} \\ \hline
			3 & OO Programming Experience & 1 & \textbf{1} & \textbf{1} \\ \hline
			4 & Lead Analyst Capability & 0.5 & \textbf{1} & \textbf{1} \\ \hline
			5 & Motivation & 1 & \textbf{5} & \textbf{5} \\ \hline
			6 & Stable Requirements & 2 & \textbf{2} & \textbf{3} \\ \hline
			7 & Part Time Staff & -1 & \textbf{3} & \textbf{3} \\ \hline
			8 & Difficult Programming Language & -1 & \textbf{5} & \textbf{5} \\ \hline
			\multicolumn{ 3}{|c|}{\textbf{Environmental Factor}} & \textbf{1.325} & \textbf{1.265} \\ \hline
		\end{tabular}
	\end{center}
\caption{UCP - Fattori ambientali}
\end{adjustwidth}
\end{table}

L'unico fattore che varia è \emph{Stable Requirements} poiché un maggior numero di requisiti porta ad una maggiore probabilità di cambiamento degli stessi.


\subsection{Quantità e complessità degli use case}

Ad ogni use case valutato viene attribuito un numero di \emph{use case points} in base al suo grado di complessità, calcolato sul numero di transazioni, inteso come scambio, una risposta del sistema ad un'azione dell'utente:

\begin{itemize}
	\item \textbf{Simple}: fino a 3 transazioni;
	\item \textbf{Average}: da 4 a 7 transazioni;
	\item \textbf{Complex}: più di 7 transazioni.
\end{itemize}

Il calcolo degli Unadjusted Use Case Points è uguale alla somma delle importanze relative moltiplicate per il peso corrispondente.

\begin{table}[H]
\begin{adjustwidth}{-4cm}{-4cm}
	\begin{center}
		\begin{tabular}{|c|c|c|c|c|}
			\hline
			\multicolumn{ 2}{|c|}{\raisebox{-1\height}{\textbf{Use Case Points grezzi}}} & \multicolumn{ 1}{c|}{\raisebox{-1\height}{\textbf{Peso}}} & \multicolumn{ 2}{c|}{\textbf{Numero di use case}} \\ \cline{ 4- 5}
			\multicolumn{ 2}{|c|}{\textbf{}} & \multicolumn{ 1}{c|}{} & \textbf{Con tutti i requisiti} & \textbf{Solo requisiti acc.} \\ \hline
			1 & Simple & 5 & \textbf{21} & \textbf{15} \\ \hline
			2 & Average & 10 & \textbf{1} & \textbf{1} \\ \hline
			3 & Complex & 15 & \textbf{0} & \textbf{0} \\ \hline
			\multicolumn{ 3}{|c|}{\textbf{Unadjusted Use Case Points}} & \textbf{115} & \textbf{85} \\ \hline
		\end{tabular}
	\end{center}
\caption{UCP - Quantità e complessità degli use case}
\end{adjustwidth}
\end{table}

Il tracciamento dei requisiti riportato in \AnalisiDeiRequisiti{} permette di non considerare i casi d'uso che sono legati a requisiti non accettati.


\subsection{Quantità e complessità degli attori}

Ad ogni attore valutato viene attribuito un peso in base al suo grado di complessità:

\begin{itemize}
	\item \textbf{Simple}: sono altri sistemi che comunicano con il proprio software attraverso delle \glossario{API} predefinite. L'elemento chiave è che si sta esponendo un'interazione con il proprio software attraverso un meccanismo specifico e ben definito;
	\item \textbf{Average}: possono essere sia uomini che interagiscono con un protocollo ben definito, sia sistemi che utilizzano \glossario{API} più complesse o flessibili;
	\item \textbf{Complex}: utenti che interagiscono con il sistema in modo imprevedibile anche attraverso interfacce grafiche.
\end{itemize}

Il calcolo dell'Actor Complexity Factor è uguale alla somma delle importanze relative moltiplicate per il peso corrispondente.

\begin{table}[H]
\begin{adjustwidth}{-4cm}{-4cm}
	\begin{center}
		\begin{tabular}{|c|c|c|c|c|}
			\hline
			\multicolumn{ 2}{|c|}{\raisebox{-1\height}{\textbf{Attori}}} & \multicolumn{ 1}{c|}{\raisebox{-1\height}{\textbf{Peso}}} & \multicolumn{ 2}{c|}{\textbf{Number of Actors}} \\ \cline{ 4- 5}
			\multicolumn{ 2}{|c|}{\textbf{}} & \multicolumn{ 1}{c|}{} & \textbf{Con tutti i requisiti} & \textbf{Solo requisiti acc.} \\ \hline
			1 & Simple & 1 & \textbf{0} & \textbf{0} \\ \hline
			2 & Average & 2 & \textbf{1} & \textbf{1} \\ \hline
			3 & Complex & 3 & \textbf{6} & \textbf{4} \\ \hline
			\multicolumn{ 3}{|c|}{\textbf{Actor Complexity Factor}} & \textbf{20} & \textbf{14} \\ \hline
		\end{tabular}
	\end{center}
\caption{UCP - Quantità e complessità degli attori}
\end{adjustwidth}
\end{table}

Gli attori implicati nell'interazione con il sistema \glossario{MaaS} non vengono calcolati. Il fattore di complessità risulta diminuire sensibilmente considerando solo i requisiti accettati poiché gli utenti \glossario{MaaS} sono ritenuti \emph{complessi}.


\subsection{Risultati e conclusioni}

Ponendo

\begin{itemize}
	\item Technical Complexity Factor = $TCF$;
	\item Environmental Factor = $EF$;
	\item Unadjusted Use Case Points = $UUCP$;
	\item Actor Complexity Factor = $ACF$;
	\item Use Case Points = $UCP$;
	\item Ore di impegno per Use Case Point = $HUCP$;
\end{itemize}

il calcolo delle ore di impegno sarà

\[
Ore = [ (UUCP+ACF) \cdot TCF \cdot EF ] \cdot HUCP
\]

\begin{table}[H]
\begin{adjustwidth}{-4cm}{-4cm}
	\begin{center}
		\begin{tabular}{|r|r|r|r|}
			\hline
			\multicolumn{ 4}{|c|}{\textbf{Fattori di complessità}} \\ \hline
			TCF & Technical Complexity Factor & 0.94 & 0.85 \\ \hline
			EF & Environmental Factor & 1.325 & 1.265 \\ \hline
			UUCP & Unadjusted Use Case Points & 115 & 85 \\ \hline
			ACF & Actor Complexity Factor & 20 & 14 \\ \hline
			\multicolumn{ 4}{|c|}{\textbf{Calcolo Use Case Points}} \\ \hline
			UCP & Use Case Points & 168.1 & 106.4 \\ \hline
			\multicolumn{ 4}{|c|}{\textbf{Calcolo ore di sviluppo}} \\ \hline
			Rapporto & Ore di impegno per Use Case Point & 7 & 7 \\ \hline
			\multicolumn{ 4}{|c|}{} \\ \hline
			\multicolumn{ 2}{|r|}{\textbf{Ore di impegno}} & \textbf{ 1,177 } & \textbf{ 745 } \\ \hline
		\end{tabular}
	\end{center}
\caption{UCP - Ore di impegno stimate}
\label{tab:UCP-ore-di-impegno-stimate}
\end{adjustwidth}
\end{table}

I risultati riportati in tabella \ref{tab:UCP-ore-di-impegno-stimate} mettono in evidenza come le ore necessarie per lo sviluppo di tutti i requisiti richieda uno sforzo quantificato in ore non sostenibile per il gruppo, e che eccede eccessivamente il limite superiore delle 105 ore produttive per ciascun componente del gruppo. Considerando solo i requisiti accettati (vedi \AnalisiDeiRequisiti), si nota come le ore di impegno scendano ad un livello molto vicino alle attese. Dal momento che la quantificazione risultante dalla tecnica utilizzata è una stima, è plausibile ritenere accettabile tale livello e procedere verso la direzione scelta con l'accettazione dei requisiti.