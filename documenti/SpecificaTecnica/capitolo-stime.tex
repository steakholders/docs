\section{Stime di fattibilità e di bisogno di risorse}

Durante l'analisi dell'architettura progettata oltre alle tecnologie e librerie consigliate e richieste nei requisiti, ne sono state ricercate altre per poter avere funzionalità già pedisposte e da integrare, garantendo una maggiore fattibilità nel ricoprire le esigenze progettuali. \\
Gli strumenti e le tecnologie integrate a quelle richieste dal capitolato sono : 
\begin{itemize}
\item 
\item 
\end{itemize}

Le tecnologie adottate sono attualmente molto diffuse: si trovano innumerevoli esempi, progetti, librerie, tutorial al riguardo. Da un lato alcune tecnologie non sono del tutto mature, visto che la gran parte dei progetti basati su di esse non raggiungono la versione ``stabile''. Dall'altro lato, però, il supporto della comunità è una grande risorsa: per ogni tipo di problema tecnico è molto facile trovare qualcuno che spieghi come risolverlo. \\
Questo viene in aiuto ai membri del gruppo, la cui maggioranza non aveva sufficienti conoscenze degli strumenti utilizzati per la realizzazione del progetto, conoscenze che sono state approfondite grazie anche alla realizzazione di diversi prototipi interni, relativi all'applicazione front-end, alla realizzazione dell'interfaccia REST e alla realizzazione del parser del linguaggio \glossario{DSL}.
Tali conoscenze continueranno ad essere sviluppate da ognuno dei componenti di \GroupName{}. \\

Gli strumenti definiti durante la progettazione sono ritenuti adeguati per garantire una soddisfabilità delle necessità progettuali, inoltre sono \glossario{open source} e quindi di facile reperimento rendendo il bisogno di risorse non problematico.



%Le tecnologie e le librerie utilizzate sono attualmente molto diffuse: si trovano innumerevoli esempi, progetti, librerie, tutorial al riguardo. Da un lato la tecnologia non è del tutto matura, visto che la gran parte dei progetti basati su di essa non raggiunge la versione ``stabile''. Dall'altro lato, però, il supporto della comunità è una grande risorsa: per ogni tipo di problema tecnico è molto facile trovare qualcuno che spieghi come risolverlo.
%nel corso della progettazione sono state realizzati diversi prototipi interni, relativi all'applicazione front-end, alla realizzazione dell'interfaccia REST, alla realizzazione del parser del linguaggio DSL. 
