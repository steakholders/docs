\clearpage
\section{Glossario}

\letteraGlossario{C}

\definizione{Collection}
In \glossario{MongoDB} è un insieme di \glossario{documents}.

\definizione{Collection-Index}
Insieme di pagine generate da \ProjectName{} contenenti la visualizzazione in forma tabellare dell'elenco di tutti i documenti della \glossario{Collection} \glossario{MongoDB}.

\definizione{Collection-Show}
Insieme di pagine generate da \ProjectName{} contenenti tutte le coppie chiavi-valore.

\letteraGlossario{D}

\definizione{Dashboard}
Pagina web che rappresenta lo stato corrente di un'applicazione con un interfaccia semplice ed immediata.

\definizione{Index-page}
Vedi Collection Index.

\letteraGlossario{F}
\definizione{Footer}
Traducibile con piè di pagina è una sezione di pagina posizionata sotto il testo vero e proprio

\letteraGlossario{M}

\definizione{MaaP}
\glossario{Framework} per generare interfacce web di amministrazione dei \glossario{dati di business} basato su stack \glossario{Node.js} e \glossario{MongoDB}.


\definizione{MongoDB}
Sistema gestionale di basi di dati non relazionale, orientato ai documenti, di tipo \glossario{NoSQL}. Il linguaggio utilizzato per la gestione dei dati è \glossario{JavaScript}, del quale sfrutta in particolare la notazione BSON.

\letteraGlossario{N}

\definizione{Node.js}
Piattaforma software utilizzata per creare applicazioni distribuite facilmente scalabili.
Node.js utilizza \glossario{JavaScript} come linguaggio di scripting e gestisce le attese I/O in modo asincrono.


\letteraGlossario{S}

\definizione{Show-page}
Vedi Collection-show

\definizione{Stack}
Il termine stack o pila indica un tipo di dato astratto che viene usato in diversi contesti per riferirsi a strutture dati, le cui modalità d'accesso ai dati in essa contenuti seguono una modalità LIFO (Last In First Out), tale per cui i dati vengono estratti in ordine rigorosamente inverso rispetto a quello in cui sono stati inseriti.