\section{Back-end}


\subsection{Interfaccia REST}

Ad ogni richiesta il server può rispondere con un messaggio di errore nel formato \glossario{JSON} e inviato con un codice \glossario{HTTP} della tipologia 4xx o 5xx. 
Il formato \glossario{JSON} del messaggio di errore sarà:
\begin{lstlisting}
{ 
  "code": [codice numerico dell'errore],
  "message": [descrizione testuale dell'errore]
}
\end{lstlisting}
Di seguito sono elencate le risorse REST associate al tipo di metodo che è possibile richiedere su esse e i permessi richiesti per poter effettuare la richiesta. \\
I tipi di permessi possibili sono: 
\begin{itemize}
\item \textbf{Utente}: questa risorsa può essere richiesta da qualsiasi tipo di utente;
\item \textbf{Utente Autenticato}: questa risorsa può essere richiesta solo dagli utenti autenticati a \glossario{MaaP};
\item \textbf{Admin}: tale risorsa può essere richiesta solo da utenti con livello Admin.
\end{itemize}

\begin{center}
	\def\arraystretch{1.5}
	\bgroup
	\begin{longtable}{| p{9cm} | p{1.5cm} | p{4cm} |}
	\hline 
	\textbf{\emph{/login}} & \textbf{POST} & \textbf{Utente} \\ \hline
	\multicolumn{3}{|c|} {Crea una nuova sessione associata all'utente, corrisponde al login.} \\ \specialrule{1pt}{1pt}{1pt}
	
	\multicolumn{3}{c} {} \\ \hline
	
	%\specialrule{1pt}{1pt}{1pt} 
	
	\textbf{\emph{/logout}} & \textbf{DELETE} & \textbf{Utente Autenticato} \\ \hline
	\multicolumn{3}{|c|} {Elimina la sessione utente, corrisponde al logout. }  \\ \specialrule{1pt}{1pt}{1pt}
	
	\multicolumn{3}{c} {} \\ \hline
	
	\textbf{\emph{/profile}} & \textbf{GET} & \textbf{Utente Autenticato} \\ \hline
	\multicolumn{3}{|c|} {Restituisce i dati relativi all'utente. }  \\ \hline
	\textbf{\emph{/profile}} & \textbf{PUT} & \textbf{Utente Autenticato} \\ \hline
	\multicolumn{3}{|c|} { Modifica i dati utente. }  \\ \specialrule{1pt}{1pt}{1pt}
	
	\multicolumn{3}{c} {} \\ \hline
	
	\textbf{\emph{/password/lost}} & \textbf{POST} & \textbf{Utente} \\ \hline
	\multicolumn{3}{|c|} {Effettua la richiesta di recupero password. }  \\ \specialrule{1pt}{1pt}{1pt}
	
	\multicolumn{3}{c} {} \\ \hline
	
	\textbf{\emph{/password/reset}} & \textbf{PUT} & \textbf{Utente} \\ \hline
	\multicolumn{3}{|c|} {Effettua la richiesta di modifica della password utente. }  \\ \specialrule{1pt}{1pt}{1pt}
	
	\multicolumn{3}{c} {} \\ \hline
	
	\textbf{\emph{/users}} & \textbf{GET} & \textbf{Admin} \\ \hline
	\multicolumn{3}{|c|} {Restituisce la lista di tutti gli utenti. }  \\ \hline
	\textbf{\emph{/users}} & \textbf{POST} & \textbf{Admin} \\ \hline
	\multicolumn{3}{|c|} {Effettua la richiesta di creazione di un nuovo utente. }  \\ \specialrule{1pt}{1pt}{1pt}
	
	\multicolumn{3}{c} {} \\ \hline
	
	\textbf{\emph{/users/$\{$user id$\}$}} & \textbf{GET} & \textbf{Admin} \\ \hline
	\multicolumn{3}{|c|} {Restituisce i dati corrispondenti all'utente con id $\{$user id$\}$. }  \\ \hline
	\textbf{\emph{/users/$\{$user id$\}$}} & \textbf{PUT} & \textbf{Admin} \\ \hline
	\multicolumn{3}{|c|} {Modifica i dati dell'utente con id $\{$user id$\}$. }  \\ \hline
	\textbf{\emph{/users/$\{$user id$\}$}} & \textbf{DELETE} & \textbf{Admin} \\ \hline
	\multicolumn{3}{|c|} {Elimina l'utente con id $\{$user id$\}$. }  \\ \specialrule{1pt}{1pt}{1pt}
	
	\multicolumn{3}{c} {} \\ \hline
	
	\textbf{\emph{/collection}} & \textbf{GET} & \textbf{Utente Autenticato} \\ \hline
	\multicolumn{3}{|c|} {Restituisce la lista delle collection. }  \\ \specialrule{1pt}{1pt}{1pt}
	
	\multicolumn{3}{c} {} \\ \hline
	
	\textbf{\emph{/collection/$\{$collection name$\}$} } & \textbf{GET} & \textbf{Utente Autenticato} \\ \hline
	\multicolumn{3}{|c|} {Restituisce la lista di document della collection $\{$collection name$\}$.}  \\ \specialrule{1pt}{1pt}{1pt}
	
	\multicolumn{3}{c} {} \\ \hline
	
	\textbf{\emph{/collection/$\{$collection name$\}$/$\{$document id$\}$}  } & \textbf{GET} & \textbf{Utente Autenticato} \\ \hline
	\multicolumn{3}{|c|} {Restituisce la lista di attributi del Document $\{$document id$\}$ appartenente alla collection $\{$collection name$\}$}  \\ \hline
	\textbf{\emph{/collection/$\{$collection name$\}$/$\{$document id$\}$} } & \textbf{PUT} & \textbf{Admin} \\ \hline
	\multicolumn{3}{|c|} {Modifica il document $\{$document id$\}$. }  \\ \hline
	\textbf{\emph{\emph{/collection/$\{$collection name$\}$/$\{$document id$\}$} }} & \textbf{GET} & \textbf{Utente Autenticato} \\
	\hline
	\multicolumn{3}{|c|} {Elimina il document con id $\{$document id$\}$. }  \\ \specialrule{1pt}{1pt}{1pt}
	
	\multicolumn{3}{c} {} \\ \hline
	
	\textbf{\emph{/action/$\{$action name$\}$/$\{$collection name$\}$}} & \textbf{GET} & \textbf{Utente Autenticato} \\ \hline
	\multicolumn{3}{|c|} {Esegue l'azione $\{$action name$\}$ sulla Collection $\{$collection name$\}$.}  \\ 
	\specialrule{1pt}{1pt}{1pt}
	
	\multicolumn{3}{c} {} \\ \hline
	
	\textbf{\emph{/action/$\{$action name$\}$/$\{$collection name$\}$/$\{$document id$\}$}} & \textbf{GET} & 
	\textbf{Utente Autenticato} \\ \hline
	\multicolumn{3}{|c|} {Esegue l'azione $\{$action name$\}$ sul Document $\{$document name$\}$ della Collection 
	$\{$collection name$\}$.}  \\ 
	\specialrule{1pt}{1pt}{1pt}

	
\end{longtable}
	  \egroup
\end{center}

\subsection{Diagramma dei package}

\subsection{Diagramma delle classi}