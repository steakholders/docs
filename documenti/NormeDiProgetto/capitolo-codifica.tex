\section{Codifica}

\subsection{Intestazione}

\begin{lstlisting}
/*
* Name: {Nome del file}
* Module: {modulo di appartenenza}
* Location: {/path/della/cartella/}
*
* History:
* Version     Date            Programmer    
* ================================================
* 1.1.1       AAAA-MM-GG      {Nome Cognome      }
* ------------------------------------------------
* ...
* {Description}
* ...
* ================================================
* 1.1.2       AAAA-MM-GG      {Nome Cognome      }
* ------------------------------------------------
* ...
*
*/
\end{lstlisting}

\begin{itemize}
 \item \textbf{Name} è il nome del file, estensione compresa.
 \item \textbf{Module} è il nome del modulo di cui il file fa parte.
 \item \textbf{Location} è il percorso del file, a partire dalla cartella principale del progetto ``/'' fino alla cartella che contiene il file. Deve iniziare e terminare con uno slash ``/".
 \item \textbf{History} è il diario delle modifiche del file. Ogni modifica è composta dai seguenti campi:
 
	\begin{itemize}
	 \item \textbf{Version} è la versione del file.
	 \item \textbf{Date} è la data della modifica.
	 \item \textbf{Programmer} è il nome e cognome dell'autore della modifica. Al massimo può essere lungo 20 caratteri.
	 \item \textbf{Description} è la spiegazione delle modifiche fatte e del motivo per cui sono state fatte.	 
	\end{itemize}
\end{itemize}

\subsection{Formattazione}

La formattazione del codice sorgente deve essere definita in modo rigoroso e consistente, così che tutto il codice del progetto sembri che sia stato scritto da un'unica persona.

Per il codice \emph{Javascript} si è scelto di adottare le linee guida utilizzate dal progetto \emph{jQuery}, sia per l'effettiva facilità di lettura sia per la possibilità di automatizzare la formattazione del codice tramite il programma \emph{JSHint}. La pagina di riferimento è \url{http://contribute.jquery.org/style-guide/js/}.

%\subsubsection{Indentazione}
%\subsection{Norme di validazione}
%\subsection{Rendiconto ore}
%\subsubsection{Nomi}
%\subsubsection{Disciplina}
%\subsection{Protocolli vari}
