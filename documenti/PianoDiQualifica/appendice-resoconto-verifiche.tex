\section{Resoconto delle attività di verifica}

	\subsection{Riassunto delle attività di verifica}
	\label{RiassuntoAttivitaVerifica}
	
	 	\subsubsection{Revisione dei Requisiti}
	 	L'attività di verifica svolta dai \emph{Verificatori} è avvenuta come determinato dal \PianoDiProgetto{} al termine della stesura di ogni documento previsto. La verifica svolta sui documenti è avvenuta seguendo le indicazioni delle \NormeDiProgetto{} e misurando le metriche indicate in \ref{metrichedocumenti}. L'attività di \emph{walkthrough} ha evidenziato una serie di anomalie, in questo modo è stato possibile stilare la lista di anomalie frequenti (vedi \NormeDiProgetto{}) che si potranno controllare tramite \emph{Inspection}. Successivamente si è proceduto con le misurazioni delle metriche relative ai documenti.
In questa revisione non è stato possibile valutare i processi poiché lo stato embrionale del team e   impegni universitari sovrapposti non hanno permesso il rilevamento accurato di tutti i parametri necessari. Il gruppo ha in programma di colmare tale mancanza per la revisione successiva.

		\paragraph{Miglioramenti}
		A seguito delle attività di verifica e controllo è stato sottoposto un questionario ad ogni membro del gruppo che ha contribuito ad identificare le problematiche relative ai processi e a formulare proposte risolutive. Da queste idee sono nate diverse modifiche e miglioramenti ai documenti e in generale al nostro modo di lavorare. Seguendo questa linea abbiamo applicato coerentemente la politica di \textit{plan-do-check-act}, utilissima per il miglioramento della qualità: \\
			
	\begin{table}[h]
    \begin{tabular}{ | p{5cm} | p{5cm} | p{2cm} | }
	\hline
	Problema & Possibile soluzione & Stato \\ \hline
    Il dizionario personale di \glossario{Aspell}, essendo un file collaborativo compilato in automatico da tale \glossario{tool}, impone molto spesso attività manuali extra di gestione del \glossario{repository}, in particolare vanno risolti molti conflitti. & Uno script che ordina il file in questione dovrebbe diminuire i conflitti. & Da eseguire. \\ \hline
	 Contrassegnare le parole di glossario con il relativo \glossario{tag} è un attività fortemente propensa a dimenticanze ed errori. & Uno script potrebbe contrassegnare le parole di glossario presenti nei documenti in automatico. & Da eseguire.  \\ \hline
	Scarsa frammentazione dei \glossario{task} & Incremento dell'utilizzo dello strumento di gestione dei processi. & Completato.   \\ \hline
	Mantenere la corrispondenza tra casi d'uso e relativi \glossario{url} dei diagrammi è un operazione lunga e manuale. & Uno script può automatizzare tale attività. & Completato. \\ \hline
	Difficoltà nell'uso dello strumento di controllo di versione. & I membri del gruppo devono eseguire attività extra di autoformazione sulla base del materiale messo a disposizione da alcuni membri. & Completato. \\	
	\hline
    \end{tabular}
    	\caption{Problemi pre RR individuati e relative soluzioni.}
\end{table}
	
	 \pagebreak
	 \subsection{Dettaglio delle verifiche tramite analisi}
	 \label{DettaglioVerificheAnalisi}
	 	\subsubsection{Analisi}
	 	\paragraph{Documenti}
	 	Vengono qui riportati i valori dell’indice Gulpease per ogni documento durante l'analisi e relativo esito basato sui range stabiliti in \ref{gulpease}.
	
	\begin{table}[H]
	\centering
	\begin{tabular}{ | c | c | c | }
    \hline
    Documento & Valore indice & Esito \\ \hline
     \emph{Analisi dei Requisiti v1.3.1} & 45 &  superato \\ \hline
     \emph{Glossario v1.3.1} & 63 &  superato \\ \hline
     \emph{Norme di Progetto v1.3.1} & 47 &  superato \\ \hline
     \emph{Piano di Progetto v1.3.1} & 51 &  superato \\ \hline
     \emph{Piano di Qualifica v1.3.1} & 53 &  superato \\ \hline
     \emph{Studio di Fattibilità v1.3.1} & 43 &  superato \\ \hline
    \end{tabular}
	\caption{Esiti verifica documenti, Analisi}
	\end{table}
	
	%Il file \AnalisiDeiRequisiti{} contiene numerose tabelle che possono facilmente falsare l'indice rilevato. Il gruppo si riserva di analizzare tale problematica e aggiornare i relativi strumenti entro la prossima revisione.
	
	\subsubsection{Progettazione Architetturale}
	\paragraph{Documenti}
	 Vengono qui riportati i valori dell’indice Gulpease per ogni documento durante la progettazione architetturale e relativo esito basato sui range stabiliti in \ref{gulpease}.
	
	\begin{table}[H]
	\centering
	\begin{tabular}{ | c | c | c | }
    \hline
    Documento & Valore indice & Esito \\ \hline
    \AnalisiDeiRequisiti{} & 51 &  superato \\ \hline
    \Glossario{} & 49 &  superato \\ \hline
    \NormeDiProgetto{} & 47 &  superato \\ \hline
    \PianoDiProgetto{} & 73 &  superato \\ \hline
    \PianoDiQualifica{} & 50 &  superato \\ \hline
    \StudioDiFattibilita{} & 44 &  superato \\ \hline
    \SpecificaTecnica{} & 58 & superato \\ \hline
    \end{tabular}
	\caption{Esiti verifica documenti, Progettazione Architetturale}
	\end{table}
	

	 	
%	 		\paragraph{Processi}
%	 		\paragraph{Documenti}
	 	
%	 \subsection{Dettaglio delle verifiche tramite prove}
%		\subsubsection{Analisi}
%			\paragraph{Processi}
%	 		\paragraph{Documenti}
	 
	 
	\subsection{Dettaglio dell'esito delle revisioni}
	Lo sviluppo di questo progetto didattico si basa sull'attraversamento di quattro revisioni presiedute dal committente. Tre delle quattro revisioni produrranno delle segnalazioni degli errori riscontrati da parte del committente, deve seguire un report di come sono state risolte in ogni documento.
		
		\subsubsection{Revisione dei Requisiti}
		Per la revisione dei requisiti le segnalazioni da parte del committente sono state corrette:
		
		\begin{itemize}
			\item Norme di Progetto: il documento è stato riorganizzato per processi, attività, procedure, strumenti. Sono state aggiunte indicazioni sugli strumenti per la gestione del \glossario{repository} e le regole per la rotazione dei ruoli sono state definite in modo dettagliato;
			\item Analisi dei Requisiti: le \NormeDiProgetto{} descrivono la modalità di consegna che è stata ben definita che include la generazione dei nomi dei documenti con la relativa versione. Inoltre sono stati rivisti tutti i requisiti e casi d'uso segnalati dal committente;
			\item Piano di Progetto: l'Organigramma è stato spostato in appendice e sono stati rimossi i costi orari dei ruoli. Sono state ripartizionate le ore considerando attività di analisi successive al 2013-12-20 e la percentuale di ore di verifica è almeno del 30\% del totale;
			\item Piano di Qualifica: la trattazione del \glossario{SEMAT} è stata spostata ed approfondita nel piano di progetto e le tecniche adottate sono state spostate nelle norme di progetto;
			\item Glossario: è stato creato l'indice del documento e ogni gruppo di lettera inizia su una pagina nuova.
		\end{itemize}
		
%		\subsubsection{Revisione di Progettazione}
%		\subsubsection{Revisione di Qualifica}
			


	
