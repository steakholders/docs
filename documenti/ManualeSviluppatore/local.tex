%%%%%%%%%%%%%%
%  COSTANTI  %
%%%%%%%%%%%%%%

% In questa prima parte vanno definite le 'costanti' utilizzate soltanto da questo documento.

\newcommand{\DocTitle}{Manuale Sviluppatore}
\newcommand{\DocVersion}{\VersioneMS}
\newcommand{\DocRedazione}{Enrico Rotundo, Gianluca Donato, Luca De Franceschi}
\newcommand{\DocVerifica}{Federico Poli}
\newcommand{\DocApprovazione}{Serena Girardi}
\newcommand{\DocUso}{Esterno}
\newcommand{\DocDistribuzione}{
	\Committente{} \\
	Gruppo \GroupName{} \\
	\Proponente{}
}

% La descrizione del documento
\newcommand{\DocDescription}{
Il presente documento é il manuale per lo sviluppatore del framework MaaP.
}

\definecolor{lightgray}{rgb}{.9,.9,.9}
\definecolor{darkgray}{rgb}{.4,.4,.4}
\definecolor{purple}{rgb}{0.65, 0.12, 0.82}

\lstdefinelanguage{JavaScript}{
  keywords={typeof, new, true, false, catch, function, return, null, catch, switch, var, if, in, while, do, else, case, break, collection, index, show, column, row},
  keywordstyle=\color{blue}\bfseries,
  ndkeywords={name, label, query, populate, sortable, selectable, transformation, sortby, order, perpage, id, weight},
  ndkeywordstyle=\color{darkgray}\bfseries,
  identifierstyle=\color{black},
  sensitive=false,
  comment=[l]{//},
  morecomment=[s]{/*}{*/},
  commentstyle=\color{purple}\ttfamily,
  stringstyle=\color{red}\ttfamily,
  morestring=[b]',
  morestring=[b]"
}

\lstset{
   language=JavaScript,
   extendedchars=true,
   %basicstyle=\footnotesize\ttfamily,
   showstringspaces=false,
   showspaces=false,
   numbers=left,
   numberstyle=\footnotesize,
   numbersep=20pt,
   tabsize=2,
   breaklines=true,
   showtabs=false,
   captionpos=b
}

%%%%%%%%%%%%%%
%  FUNZIONI  %
%%%%%%%%%%%%%%

% In questa seconda parte vanno definite le 'funzioni' utilizzate soltanto da questo documento.

\newcommand{\letteraGlossario}[1] { 
  % \newpage
  % \cleardoublepage
  \phantomsection
  % \addcontentsline{toc}{section}{#1}
  \vspace{11pt}
  \textbf{\huge{#1} } % Lettera grande 
  \\
  \rule[0.3pt]{\linewidth}{0.4pt} \\ % Linea orizzontale
} 

\newcommand{\definizione}[1] {\textbf{#1}:}
\newcommand{\nyi} {\subparagraph{NOTE} Questa funzionalità non è ancora stata implementata.}
