\section{Analisi dei requisiti}

Dal capitolato e dagli incontri con il proponente gli Analisti dovranno estrarre i requisiti del progetto, producendo l'Analisi dei Requisiti.

Tutti i requisiti che si possono evincere dal capitolato o ad un incontro con il proponente vanno specificati nell'Analisi dei Requisiti.

Per agevolare l'analisi dei requisiti viene utilizzata la tecnica dei casi d'uso.

	\subsection{Tracciamento requisiti}
	 I requisiti vengono tracciati mediante il software \emph{Requisteak} appositamente creato dal gruppo \GroupName . Tale strumento è raggiungibile all'indirizzo \url{http://steakholders.herokuapp.com}.
	 
\subsection{Casi d'Uso}

Ogni caso d'uso dovrà presentare i seguenti campi:
\begin{itemize}
 \item Codice identificativo
 \item Titolo
 \item Attori coinvolti
 \item Diagramma UML
 \item Scopo e descrizione
 \item Precondizione
 \item Postcondizione
 \item Flusso principale degli eventi
 \item Scenario alternativo
\end{itemize}

\subsection{Codice identificativo}

Ogni caso d'uso è identificato da un codice, che segue il seguente formalismo:
\begin{center}
	\code{UC\{X\} \{Gerarchia\}}
\end{center}

Dove:
\begin{itemize}
 \item \textbf{X} corrisponde all'ambito di riferimento e può assumere i seguenti valori:
	\begin{itemize}
	 \item[] \textbf{U} = Ambito \glossario{Utente};
	 \item[] \textbf{S} = Ambito \glossario{Sviluppatore}.
	\end{itemize}

	 \item \textbf{Gerarchia} identifica la relazione gerarchica che c'è tra i casi d'uso di uno stesso ambito. C'è quindi una struttura gerarchica per ogni ambito dei casi d'uso.
\end{itemize}

\subsection{Requisiti}

Ogni requisito dovrà presentare i seguenti campi:
\begin{itemize}
 \item Codice identificativo
 \item Descrizione
 \item Fonti
\end{itemize}

\subsection{Codice identificativo}

Ogni requisito è identificato da un codice, che segue il seguente formalismo:
\begin{center}
	\code{R\{X\}\{Y\}\{Z\} \{Gerarchia\}}
\end{center}

Dove:
\begin{itemize}
 \item \textbf{X} corrisponde al sistema di riferimento e può assumere i seguenti valori:
	\begin{itemize}
	 \item[] \textbf{A} = Applicazione \glossario{Maap};
	 \item[] \textbf{F} = \glossario{Framework} di \glossario{Maap};
	 \item[] \textbf{S} = \glossario{Maas}.
	\end{itemize}

 \item \textbf{Y} corrisponde alla tipologia del requisito e può assumere i seguenti valori:
	\begin{itemize}
	 \item[] \textbf{1} = Funzionale;
	 \item[] \textbf{2} = Di prestazione;
	 \item[] \textbf{3} = Di qualità;
	 \item[] \textbf{4} = Vincolo.
	\end{itemize}

 \item \textbf{Z} corrisponde alla priorità del requisito e può assumere i seguenti valori:
	\begin{itemize}
	 \item[] \textbf{O} = Obbligatorio
	 \item[] \textbf{D} = Desiderabile
	 \item[] \textbf{F} = Facoltativo o opzionale
	\end{itemize}

 \item \textbf{Gerarchia} identifica la relazione gerarchica che c'è tra i requisiti di uno stesso tipo. C'è quindi una struttura gerarchica per ogni tipologia di requisito.
\end{itemize}

\subsection{UML}

Per i diagrammi deve essere utilizzata il linguaggio \emph{UML versione 2.0}.