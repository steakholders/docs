
	\begin{center}
	\def\arraystretch{1.5}
	\bgroup
		\begin{longtable}{| p{3cm} | p{6cm} | p{1.5cm} | p{2cm} | }
		\hline 
		 \textbf{Test di Validazione} & \textbf{Descrizione} & \textbf{Stato} & \textbf{Requisito} \\ \hline
				TV-RA1O 1 & 
				L'utente non autenticato intende accedere all'applicazione, per farlo deve inserire le proprie credenziali composte da una email ed una password.
All'utente è richiesto di:
\begin{itemize}
\item Raggiungere la pagina di autenticazione;
\item Inserire la mail nel campo apposito;
\item Inserire la password;
\item Procedere con l'autenticazione.
\end{itemize}
 & N.E & RA1O 1\newline  \\ \hline 
				TV-RA1O 2 & 
				L'utente intende recuperare la password d'accesso all'applicazione.
All'utente è richiesto di:
\begin{itemize}
\item Essere autenticato;
\item Raggiungere la pagina per il reset della password;
\item Richiedere il reset;
\item Raggiungere la casella email collegata all'account del sistema;
\item Seguire il link contento nella mail:
\item Compilare il form richiedente la nuova password;
\item Eseguire il Logout e autenticarsi con la nuova password.
\end{itemize} & N.E & RA1O 2\newline  \\ \hline 
				TV-RA1D 3 & 
				L'utente autenticato può visualizzare la pagina di Dashboard nella quale potrà aver accesso ad esempio alla lista delle collection presenti e ad altre funzionalità disponibili.
All'utente è richiesto di:
\begin{itemize}
\item Accertarsi di essere autenticato;
\item Accedere alla pagina Dashboard tramite il menu di navigazione;
\end{itemize} & N.E & RA1D 3\newline  \\ \hline 
				TV-RA1O 4 & 
				L'utente autenticato, selezionata una \glossario{Collection}, ne visualizza in forma tabellare tutti i documenti che contiene. \newline Di questa collection può filtrarne i risultati visualizzabili, può eseguire tramite bottoni predisposti nella pagina azioni personalizzati e per ogni \glossario{Document}, selezionarlo e visualizzarne la show-page corrispondente. \newline L'Admin ha i permessi per modificare un documento o eliminare un \glossario{Document}. \newline
All'utente è richiesto di:
\begin{itemize}
\item Essere autenticato;
\item Aprire la show-page relativa ad un Document;
\item Usare i filtri per filtrare la Collection
\item Eseguire un azione personalizzata, laddove presente;
\item Se admin, modificare un Document;
\item Se admin, eliminare un Document;
\end{itemize} & N.E & RA1O 4\newline  \\ \hline 
				TV-RA1O 5 & 
				L'utente visualizza la pagina show-page corrispondente ad un \glossario{Document} selezionato visualizzandone gli attributi in forma tabellare. \newline In questa pagina può aprire la show-page o l'index-page dell'array di \glossario{Document} degli attributi innestati se presenti, eseguire un'operazione personalizzata se disponibile. \newline L'Admin può eliminare il \glossario{Document} a cui la show-page corrisponde o modificarlo. 
All'utente è richiesto di:
\begin{item}
\item Essere autenticato;
\item Aprire la show-page degli attributi innestati;
\item Aprire l'index-page dell'arra di Document;
\item Eseguire, se presente, un operazione personalizzata;
\item Se admin, modificare il Document;
\item Se admin, eliminare il Document.
\end{item} & N.E & RA1O 5\newline  \\ \hline 
				TV-RA1O 6 & 
				L'Admin entra nella sua pagina di amministrazione nella quale visualizza una \glossario{Collection-Index} di tutti gli utenti registrati al sistema.
All'utente è richiesto di:
\begin{itemize}
\item Essere autenticato come admin;
\item Accedere alla pagina di creazione nuovi utenti;
\item Creare un nuovo utente;
\item Accedere alla pagina degli utenti registrati al sistema;
\item Visualizzare la pagina Collection-Show di un utente;
\end{itemize}
 & N.E & RA1O 6\newline  \\ \hline 
				TV-RF1O 8 & 
				Lo sviluppatore deve poter creare un nuovo progetto tramite linea di comando.
\newline
Allo sviluppatore è richiesto di:
\begin{itemize}
\item Richiamare il comando di creazione di un nuovo progetto;
\item Passare come parametro il nome della directory che conterrà il progetto;
\item Verificare che siano state importate le librerie necessarie al corretto funzionamento del sistema;
\item Verificare che sia stato creato il file di configurazione di default dell’applicazione generata;
\item Verificare che sia stato creato il sistema di autenticazione per l’applicazione generata;
\item Verificare che siano state create le directory di descrizione delle pagine web;
\item Verificare che sia stato creato un account admin di default.
\end{itemize} & N.E & RF1O 8\newline  \\ \hline 
				TV-RF1O 9 & 
				Lo sviluppatore deve poter configurare le Collection tramite il DSL di Maap Framework.
All'utente è richiesto di:
\begin{itemize}
\item creare una Collection-index tramite DSL;
\item creare una Collection-show tramite DSL;
\item modificare il nome della Collection;
\item modificare l'ordine di visualizzazione della Collection.
\end{itemize} & N.E & RF1O 9\newline  \\ \hline 
				TV-RS1F 10 & 
				L'utente autenticato verifica che il \glossario{framework} MaaP sia messo a disposizione dal sistema \glossario{MaaS} come servizio Web.
\newline
All'utente è richiesto di:
\begin{itemize}
\item Accedere alla pagina di modifica del proprio profilo;
\item Modificare i dati associati al proprio profilo;
\item Verificare che i dati siano stati aggiornati;
\item Gestire i file di configurazione;
\item Eliminare il proprio account;
\item Verificare l'inaccessibilità al servizio tramite l'autenticazione con le credenziali associate all'account eliminato.
\end{itemize} & N.E & RS1F 10* \newline  \\ \hline 
				TV-RA1D 11 & 
				L'utente non autenticato deve potersi registrare all'applicazione MaaP.
\newline
All'utente è richiesto di:
\begin{itemize}
\item Inserire la mail nell'apposito campo di testo;
\item Inserire la password nell'apposito campo di testo;
\item Verificare che l'account sia stato registrato tramite l'autenticazione all'applicazione.
\end{itemize} & N.E & RA1D 11\newline  \\ \hline 
				TV-RA1D 12 & 
				L'utente autenticato deve poter eseguire il logout dall'applicazione.
\newline
All'utente è richiesto di:
\begin{itemize}
\item Selezionare l'apposita opzione di logout;
\item Verificare di non essere più autenticato.
\end{itemize} & N.E & RA1D 12\newline  \\ \hline 
				TV-RA1D 13 & 
				L'utente autenticato deve poter modificare le proprie credenziali d'accesso all'interno della propria pagina profilo.
\newline
All'utente viene richiesto di:
\begin{itemize}
\item Accedere alla propria pagina profilo;
\item Modificare la propria mail;
\item Modificare la propria password;
\item Eseguire il logout;
\item Autenticarsi con le nuove credenziali.
\end{itemize} & N.E & RA1D 13\newline  \\ \hline 
				TV-RF1O 14 & 
				Lo sviluppatore deve poter configurare i database che compongono il sistema MaaP.
\newline
Allo sviluppatore è richiesto di:
\begin{itemize}
\item Configurare la connessione al database delle credenziali degli utenti;
\item Configurare il \glossario{namespace} corrispondente, se la funzione di \glossario{namespace} è abilitata;
\item Configurare la connessione al database delle \glossario{Collection};
\item Configurare il \glossario{namespace} corrispondente, se la funzione di \glossario{namespace} è abilitata;
\item Selezionare un \glossario{namespace} per il database da configurare, se la funzione di \glossario{namespace} è abilitata.
\end{itemize} & N.E & RF1O 14\newline  \\ \hline 
				TV-RA1F 15 & 
				L'admin deve poter gestire gli indici da un'apposita pagina.
\newline
All'admin è richiesto di:
\begin{itemize}
\item Accedere alla pagina di gestione degli indici;
\item Visualizzare i suggerimenti per la creazione degli indici;
\item Creare un indice;
\item Creare un indice da quelli suggeriti;
\item Eliminare un indice;
\item Eliminare un indice da quelli suggeriti.
\end{itemize} & N.E & RA1F 15* \newline  \\ \hline 
				TV-RF1F 16 & 
				Lo sviluppatore deve poter abilitare i \glossario{namespace} per l’applicazione creata.
\newline
Allo sviluppatore è richiesto di:
\begin{itemize}
\item Attivare il \glossario{namespace}.
\end{itemize} & N.E & RF1F 16* \newline  \\ \hline 
		\caption{Tracciamento Test di Validazione - Requisiti}
		\end{longtable}
	 \egroup
\end{center}