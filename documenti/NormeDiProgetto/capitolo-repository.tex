\section{Repository e condivisione file}

\subsection{Repository dei documenti}

\subsubsection{Struttura del repository}

La struttura del repository, la cui cartella principale chiameremo \file{root}, è così composta:
\begin{itemize}
 \item \textbf{\file{root/ufficiali/}} \\
	Contiene i documenti ufficiali approvati dal Responsabile di progetto.

 \item \textbf{\file{root/modello/}} \\
	Contiene i file comuni a due o più documenti.

 \item \textbf{\file{root/documenti/\{NomeDelDocumento\}/}} \\
	Ciascuna cartella descritta da questo percorso contiene i file che vengono utilizzati dal documento \file{\{NomeDelDocumento\}}. In particolare conterrà il file \file{\{NomeDelDocumento\}.pdf} e il file \file{diario\_ modifiche.tex}, contenente il diario delle modifiche del documento.
	
	Il nome della cartella \file{\{NomeDelDocumento\}} e il file \file{\{NomeDelDocumento\}.pdf} devono rispettare la convenzione \glossario{CamelCase}.

 \item \textbf{\file{root/script/}} \\
	Contiene gli script di supporto alle operazioni di verifica e compilazione dei documenti.
\end{itemize}

È raccomandato che tutti i file e le cartelle non contengano spazi nel loro nome.

\subsubsection{Branch}

Il nome utilizzato per le \glossario{branch} del repository dovrà essere nella forma \textbf{nome-branch}, ossia con tutte le lettere minuscole e le parole separate da un trattino.

\subsubsection{Script di pre-commit}

Con l'intenzione di ridurre al minimo la presenza di errori nei documenti caricati sul repository verrà utilizzato uno script di \code{pre-commit}. Lo script, attivato direttamente dal programma \code{git} nel momento in cui viene eseguito un commit, esegue in automatico i comandi \code{make test} e \code{make documents} descritti nella sezione \ref{makefile}. Se uno comandi fallisce il commit viene annullato e all'utente viene mostrato un messaggio di errore.

Per abilitare tale script di \code{pre-commit} è necessario creare un \glossario{link simbolico} con il seguente comando:
\begin{lstlisting}
ln -s {percorso assoluto del repository}/script/pre-commit {percorso assoluto del repository}/.git/hooks/pre-commit
\end{lstlisting}

\subsection{Repository del codice}

Non è stato ancora definito.

\subsection{Condivisione file}

Per condividere internamente al gruppo dei file che non necessitano di versionamento verranno utilizzate le piattaforme:
\begin{itemize}
 \item \textbf{Google Drive} (\url{http://drive.google.com}), con l'account descritto nella sezione \ref{email};
 \item \textbf{TeamworkPM}, descritta nella sezione \ref{teamworkpm}.
\end{itemize}
