tex\section{Definizione del prodotto}

% Nell'indice proposto da Tullio c'è:
% a. Metodo e formalismo di specifica
% b. Presentazione dell'architettura generale del sistema e identificazione dei componenti architetturali di alto livello
\subsection{Formalismo di specifica}
\subsection{Architettura di sistema}
\subsubsection{REST}
REST, ovvero Representational State Transfer è un tipo di \glossario{RPC}. Si basa su un protocollo di comunicazione \glossario{stateless} di tipo client-server, e solitamente tale protocollo è \glossario{HTTP}, per \ProjectName è invece stato scelto \glossario{HTTPS}.

I motivi che hanno spinto alla scelta di REST sono:
\begin{itemize}
\item Semplicità di utilizzo;
\item Indipendenza dal sistema operativo utilizzato dal client;
\item Indipendente dai linguaggi di programmazione utilizzati;
\item In coppia con HTTPS permette una certa sicurezza delle comunicazioni.
\end{itemize}

REST utilizza il concetto di risorsa, ovvero un aggregato di dati con un nome e una rappresentazione, su cui è possibile invocare le operazioni \glossario{CRUD} tramite il protocollo sopracitato.

Nell'URL inviato sono presenti il nome della risorsa e la sua rappresentazione, identificata dall'estensione del file scelta, e per \ProjectName è stato scelto il formato \glossario{JSON}, in quanto il suo parsing in \glossario{JavaScript} è più semplice rispetto, ad esempio, quello di \glossario{XML} o \glossario{CSV}.

Un'altra caratteristica di REST è che essendo \glossario{stateless} ogni richiesta dovrà contenere tutte le informazioni necessarie, e non dipende dallo stato del sistema.