\section{Introduzione}

	\subsection{Scopo del documento}

Questo documento ha l'obiettivo di identificare e dettagliare la pianificazione del gruppo \GroupName{} relativa allo sviluppo del progetto \ProjectName{}. La ripartizione del carico di lavoro e di responsabilità tra i componenti del gruppo, e il conto economico preventivo sono oggetto di primo piano in tale documento.

	\subsection{Scopo del prodotto}

\ScopoDelProdotto

	\subsection{Glossario}

Al fine di evitare ogni ambiguità relativa al linguaggio impiegato nei documenti viene fornito il \Glossario{} , contenente la definizione dei termini marcati con una G pedice.
	
	\subsection{Riferimenti}
	
		\subsubsection{Normativi}
		
		\begin{itemize}
		\item \NormeDiProgetto
		\item Capitolato d'appalto C1: MaaP: MongoDB as an admin Platform:\\
			\url{http://www.math.unipd.it/~tullio/IS-1/2013/Progetto/C1.pdf}
		\item Vincoli sull'organigramma del gruppo e sull'offerta tecnico-economica:\\
			\url{http://www.math.unipd.it/~tullio/IS-1/2013/Progetto/PD01b.html}
        \end{itemize}
        
		\subsubsection{Informativi}
		
		\begin{itemize}
		\item IAN SOMMERVILLE, \textit{Software Engineering}, Part 4: Software Management, 9th edition, Boston, Pearson Education, 2011.
		\end{itemize}
		
	\pagebreak
	\subsection{Ciclo di vita}
	L'interesse del committente è limitato al segmento di ciclo di vita che va dall'analisi dei requisiti al rilascio del prodotto, escludendo dunque la successiva manutenzione ed il ritiro.
	Il modello di ciclo di vita scelto è il modello incrementale, ritenuto preferibile in quanto permette di scomporre in sottosistemi il problema principale, riducendo i rischi derivati dalla scarsa conoscenza da parte del gruppo delle tecnologie necessarie, come illustrato nello \textit{Studio di fattibilità}.
	Questo modello permette inoltre di:
	\begin{itemize}
	\item Soddisfare primariamente i requisiti principali, e dedicarsi successivamente a quelli opzionali, potendo però offrire al proponente un sistema funzionante;
	\item Minimizzare i rischi di ritardo rispetto ai tempi stabiliti in quanto i cicli hanno durata breve e sono precedentemente pianificati;
	\item Rendere più semplice la verifica.
	\end{itemize}
	\subsection{Scadenze}
	Di seguito sono presentate le scadenze che il gruppo ha deciso di rispettare e sulle quali si baserà la pianificazione del progetto:
	\begin{itemize}
	\item Revisione dei Requisiti (RR): 2014-01-08;
	\item Revisione di Progetto (RP): 2014-02-07;
	\item Revisione di Qualifica (RQ): 2014-03-07;
	\item Revisione di Accettazione (RA): n.d.
	\end{itemize}
	Si precisa inoltre che il gruppo intende presentare alla Revisione di Progetto la \textit{Specifica Tecnica} e non la \textit{Definizione di prodotto}.

	\subsection{Ruoli e costi}
	
	Durante lo sviluppo del progetto vi sono diversi ruoli, che ogni membro del gruppo \GroupName{} è tenuto a ricoprire almeno una volta, evitando conflitti d'interesse al momento della verifica. Nelle \NormeDiProgetto{} sono descritte le responsabilità che competono ogni ruolo. I ruoli che ogni componente del gruppo ricoprirà in tempi diversi sono: \textit{Amministratore}, \textit{Analista}, \textit{Progettista}, \textit{Programmatore}, \textit{Responsabile} e \textit{Verificatore}. \\
	Ciascun ruolo ha il proprio costo orario, riportato nella seguente tabella.
	\label{tabellacostiruolo}
	\begin{table}[h]
	\centering
	\begin{tabular}{ l c l }
	\hline
	\textbf{Ruolo} & \textbf{Costo} \\
	\hline
	Amministratore & 20 € \\
	Analista & 25 € \\
	Progettista & 22 € \\
	Programmatore & 15 € \\
	Responsabile & 30 € \\
	Verificatore & 15 €\\
	\hline
	\end{tabular}
	\caption{Costi per ruolo}
	\end{table}


