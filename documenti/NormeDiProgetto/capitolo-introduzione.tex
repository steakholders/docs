\section{Introduzione}

\subsection{Scopo del documento}

Questo documento ha come obiettivo quello di definire le regole, gli strumenti e le procedure che tutti i membri del team dovranno adottare per l'intero svolgimento del progetto. Tutti i componenti del gruppo sono obbligati a visionare tale documento e ad applicare quanto scritto al fine di mantenere omogeneità e coesione in ogni aspetto del progetto.

Qualora vengano apportate modifiche o aggiunte al presente documento sarà necessario informare tempestivamente ogni membro del gruppo.

\subsection{Ambiguità}

Al fine di evitare ogni ambiguità relativa al linguaggio impiegato nei documenti viene fornito il \Glossario{}, contenente la definizione dei termini marcati con una G pedice.

In questo documento, analogamente allo standard ?? di riferimento: %TODO cerca lo standard di riferimento
\begin{itemize}
 \item I termini \textbf{``deve''} e sinonimi stretti indicano una norma che deve essere sempre rispettata;
 \item I termini \textbf{``non deve''} e sinonimi stretti indicano una norma che proibisce di fare qualcosa. Deve essere sempre rispettata;
 \item I termini \textbf{``si raccomanda'', ``è preferibile''} e sinonimi stretti indicano una norma che può essere ignorata solo se ci sono \emph{validi motivi} per cui in un particolare contesto può essere preferibile fare diversamente;
 \item I termini \textbf{``si raccomanda di non'', ``è preferibile che non''} e sinonimi stretti indicano una norma che proibisce di fare qualcosa. Può essere ignorata solo se ci sono \emph{validi motivi} per cui in un particolare contesto può essere preferibile fare diversamente.
\end{itemize}
