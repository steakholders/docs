%\documentclass[a4paper, oneside, openany]{book}
\documentclass[a4paper]{article}

%**************************************************************
% Importazione package
%************************************************************** 

% modifica i margini
%\usepackage[top=3.1cm, bottom=3.1cm, left=2.2cm, right=2.2cm]{geometry}

% specifica con quale codifica bisogna leggere i file
\usepackage[utf8]{inputenc}

% per scrivere in italiano e in inglese;
% l'ultima lingua (l'italiano) risulta predefinita
\usepackage[english, italian]{babel}

% imposta lo stile italiano per i paragrafi
\usepackage{parskip}

% numera anche i paragrafi
\setcounter{secnumdepth}{4}

% elenca anche i paragrafi nell'indice
\setcounter{tocdepth}{4}

% permetti di definire dei colori
\usepackage[usenames,dvipsnames]{color}

% permette di usare il comando "paragraph" come subsubsubsection!
\usepackage{titlesec}
%  set the secnumdepth counter to four to obtain numbering for the paragraphs:
\setcounter{secnumdepth}{4}

% permette di inserire le immagini/tabelle esattamente dove viene usato il
% comando \begin{figure}[H] ... \end{figure}
% evitando che venga spostato in automatico
\usepackage{float}

% permette l'inserimento di url e di altri tipi di collegamento
\usepackage[colorlinks=true]{hyperref}

\hypersetup{
    colorlinks=true, % false: boxed links; true: colored links
    citecolor=black,
    filecolor=black,
    linkcolor=black, % color of internal links
    urlcolor=Maroon  % color of external links
}

% permette al comando \url{...} di andare a capo a metà di un link
\usepackage{breakurl}

% immagini
\usepackage{graphicx}

% permette di riferirsi all'ultima pagina con \pageref{LastPage}
\usepackage{lastpage}

% tabelle su più pagine
\usepackage{longtable}

% per avere dei comandi in più da poter usare sulle tabelle
\usepackage{booktabs}

% tabelle con il campo X per riempire lo spazio rimanente sulla riga
\usepackage{tabularx}

% multirow per tabelle
\usepackage{multirow}

% colore di sfondo per le celle
\usepackage[table]{xcolor}

% permette di fare longtable larghe tutta la pagina (parametro x)
% su Ubuntu non si può installare il pacchetto, deve essere in modello/
\usepackage{../../modello/tabu}

% imposta lo spazio tra le righe di una tabella
\setlength{\tabulinesep}{6pt}

% personalizza l'intestazione e piè di pagina
\usepackage{fancyhdr}

% permette di inserire caratteri speciali
\usepackage{textcomp}

% permette di includere i diagrammi Gantt
% su Ubuntu non si può installare il pacchetto, deve essere in modello/
\usepackage{../../modello/pgfgantt}

% permette i path delle immagini con gli spazi
\usepackage{grffile}

% ruota le immagini
\usepackage{rotating}

\fancypagestyle{plain}{
	% cancella tutti i campi di intestazione e piè di pagina
	\fancyhf{}

	\lhead{\GroupName{} \texttwelveudash \ProjectName{}}
	\chead{}
	\rhead{\slshape \leftmark}

	\lfoot{\DocTitle \\ v\DocVersion}
	\rfoot{\thepage\ di \pageref{LastPage}}

	% Visualizza una linea orizzontale in cima e in fondo alla pagina
	\renewcommand{\headrulewidth}{0.3pt}
	\renewcommand{\footrulewidth}{0.3pt}
}
\pagestyle{plain}

% Per inserire del codice sorgente formattato
\usepackage{listings}

\lstset{
  extendedchars=true,          % lets you use non-ASCII characters
  inputencoding=utf8,   % converte i caratteri utf8 in latin1, richiede \usepackage{listingsutf8} anzichè listings
  basicstyle=\ttfamily,        % the size of the fonts that are used for the code
  breakatwhitespace=false,     % sets if automatic breaks should only happen at whitespace
  breaklines=true,             % sets automatic line breaking
  captionpos=t,                % sets the caption-position to top
  commentstyle=\color{mygreen},   % comment style
  frame=none,               % adds a frame around the code
  keepspaces=true,            % keeps spaces in text, useful for keeping indentation of code (possibly needs columns=flexible)
  keywordstyle=\bfseries,     % keyword style
  numbers=none,               % where to put the line-numbers; possible values are (none, left, right)
  numbersep=5pt,              % how far the line-numbers are from the code
  numberstyle=\color{mygray}, % the style that is used for the line-numbers
  rulecolor=\color{black},    % if not set, the frame-color may be changed on line-breaks within not-black text (e.g. comments (green here))
  showspaces=false,           % show spaces everywhere adding particular underscores; it overrides 'showstringspaces'
  showstringspaces=false,     % underline spaces within strings only
  showtabs=false,             % show tabs within strings adding particular underscores
  stepnumber=5,               % the step between two line-numbers. If it's 1, each line will be numbered
  stringstyle=\color{red},    % string literal style
  tabsize=4,                  % sets default tabsize
  firstnumber=1      % visualizza i numeri dalla prima linea
}

% Permetti di utilizzare il grassetto per i caratteri Typewriter (per es. il font di \code{...} e \file{...})
\usepackage[T1]{fontenc}
\usepackage{lmodern}

%%%%%%%%%%%%%%
%  COSTANTI  %
%%%%%%%%%%%%%%

% In questa prima parte vanno definite le 'costanti' utilizzate da due o più documenti.

% Meglio non mettere gli \emph dentro le costanti, in certi casi creano problemi
\newcommand{\GroupName}{SteakHolders}
\newcommand{\GroupEmail}{steakholders.group@gmail.com}
\newcommand{\ProjectName}{MaaP}

\newcommand{\Proponente}{CoffeeStrap}
\newcommand{\Committente}{Prof. Tullio Vardanega \\ Prof. Riccardo Cardin}
\newcommand{\Responsabile}{Luca De Franceschi}

% La versione dei documenti deve essere definita qui in global, perchè serve anche agli altri documenti
\newcommand{\VersioneG}{2.0.2}
\newcommand{\VersionePQ}{2.0.1}
\newcommand{\VersioneNP}{2.0.3}
\newcommand{\VersionePP}{2.0.0}
\newcommand{\VersioneAR}{2.0.0}
\newcommand{\VersioneSF}{2.0.0}
\newcommand{\VersioneST}{2.0.0}
% Il verbale non ha versionamento.
% Lasciare vuoto, non mettere trattini o puntini
% Non sono permessi numeri nel nome di un comando :(
\newcommand{\VersioneVprimo}{} 
\newcommand{\VersioneVsecondo}{}

% Quando serve riferirsi a ``Nome del Documento + ultima versione x.y.z'' usiamo queste costanti:
\newcommand{\Glossario}{\emph{Glossario v\VersioneG{}}}
\newcommand{\PianoDiQualifica}{\emph{Piano di Qualifica v\VersionePQ{}}}
\newcommand{\NormeDiProgetto}{\emph{Norme di Progetto v\VersioneNP{}}}
\newcommand{\PianoDiProgetto}{\emph{Piano di Progetto v\VersionePP{}}}
\newcommand{\StudioDiFattibilita}{\emph{Studio di Fattibilità v\VersioneSF{}}}
\newcommand{\AnalisiDeiRequisiti}{\emph{Analisi dei Requisiti v\VersioneAR{}}}
\newcommand{\SpecificaTecnica}{\emph{Specifica Tecnica v\VersioneST{}}}

\newcommand{\ScopoDelProdotto}{
	Lo scopo del progetto è la realizzazione di un \glossario{framework} per generare interfacce web di amministrazione dei dati di \glossario{business} basato su \glossario{stack} \glossario{Node.js} e \glossario{MongoDB}. L'obiettivo è quello di semplificare il processo di implementazione di tali interfacce che lo sviluppatore, appoggiandosi alla produttività del framework MaaP, potrà generare in maniera semplice e veloce ottenendo quindi un considerevole risparmio di tempo e di sforzo. Il fruitore finale delle pagine generate sarà infine l'esperto di business che potrà visualizzare, gestire e modificare le varie entità e dati residenti in \glossario{MongoDB}.
	Il prodotto atteso si chiama \glossario{MaaP} ossia \emph{MongoDB as an admin Platform}.
}

%%%%%%%%%%%%%%
%  FUNZIONI  %
%%%%%%%%%%%%%%

% In questa seconda parte vanno definite le 'funzioni' utilizzate da due o più documenti.

% Serve a dare la giusta formattazione alle parole presenti nel glossario
% il nome del comando \glossary è già usato da LaTeX
\newcommand{\glossario}[1]{\textit{#1\ped{\ped{G}}}}

% Serve a dare la giusta formattazione al codice inline
\newcommand{\code}[1]{\texttt{#1}}

% Serve a dare la giusta formattazione a tutte le path presenti nei documenti
\newcommand{\file}[1]{\texttt{#1}}

% Permette di andare a capo all'interno di una cella in una tabella
\newcommand{\multiLineCell}[2][c]{\begin{tabular}[#1]{@{}l@{}}#2\end{tabular}}

% Genera automaticamente la pagina di copertina
\newcommand{\makeFrontPage}{
  % Declare new goemetry for the title page only.
  \newgeometry{top=3.5cm}
  
  \begin{titlepage}
  \begin{center}

  \begin{center}
  \includegraphics[width=10cm]{../../modello/steakman.png}
  \end{center}
  
  \vspace{1cm}

  \begin{Huge}
  \textbf{\DocTitle{}}
  \end{Huge}
  
  \textbf{\emph{Gruppo \GroupName{} \, \texttwelveudash{} \, Progetto \ProjectName{}}}
  
  \vspace{11pt}

  \bgroup
  \def\arraystretch{1.3}
  \begin{tabular}{ r|l }
    \multicolumn{2}{c}{\textbf{Informazioni sul documento}} \\
    \hline
		% differenzia a seconda che \DocVersion{} stampi testo o no
		\setbox0=\hbox{\DocVersion{}\unskip}\ifdim\wd0=0pt
			% nulla (non ho trovato come togliere l'a capo)
			\\
		\else
			\textbf{Versione} & \DocVersion{} \\
		\fi
    \textbf{Redazione} & \multiLineCell[t]{\DocRedazione{}} \\
    \textbf{Verifica} & \multiLineCell[t]{\DocVerifica{}} \\
    \textbf{Approvazione} & \multiLineCell[t]{\DocApprovazione{}} \\
    \textbf{Uso} & \DocUso{} \\
    \textbf{Distribuzione} & \multiLineCell[t]{\DocDistribuzione{}} \\
  \end{tabular}
  \egroup

  \vspace{22pt}

  \textbf{Descrizione} \\
  \DocDescription{}

  \end{center}
  \end{titlepage}
  
  % Ends the declared geometry for the titlepage
  \restoregeometry
}

%%%%%%%%%%%%%%
%  COSTANTI  %
%%%%%%%%%%%%%%

% In questa prima parte vanno definite le 'costanti' utilizzate soltanto da questo documento.

\newcommand{\DocTitle}{Manuale Admin}
\newcommand{\DocVersion}{\VersioneMA{}}
\newcommand{\DocRedazione}{Nicolò Tresoldi, Gianluca Donato, Enrico Rotundo}
\newcommand{\DocVerifica}{Federico Poli}
\newcommand{\DocApprovazione}{Luca De Franceschi}
\newcommand{\DocUso}{Esterno}
\newcommand{\DocDistribuzione}{
	\Committente{} \\
	Gruppo \GroupName{} \\
	\Proponente{}
}

% La descrizione del documento
\newcommand{\DocDescription}{Il presente documento è il manuale per l'utente che amministra il sistema \ProjectName{}, di seguito denominato Admin o Administrator. L'utente in questione ha il compito di gestire l'applicazione web e renderla fruibile all'utenza del servizio.}


%%%%%%%%%%%%%%
%  FUNZIONI  %
%%%%%%%%%%%%%%

% In questa seconda parte vanno definite le 'funzioni' utilizzate soltanto da questo documento.

\newcommand{\letteraGlossario}[1] { 
  % \newpage
  % \cleardoublepage
  \phantomsection
  \addcontentsline{toc}{section}{#1}
  \vspace{11pt}
  \textbf{\huge{#1} } % Lettera grande 
  \\
  \rule[0.3pt]{\linewidth}{0.4pt} \\ % Linea orizzontale
} 

\newcommand{\definizione}[1] {\textbf{#1}:}
\newcommand{\nyi} {\subparagraph{NOTE} Questa funzionalità non è ancora stata implementata.}

\begin{document}

\makeFrontPage

\section*{Registro delle modifiche}

\small{
\begin{tabularx}{\textwidth}{|c|c|P{3cm}|X|}
 \hline \textbf{Versione} & \textbf{Data} & \textbf{Persone coinvolte} & \textbf{Descrizione} \\

 % IN ORDINE DALLA MODIFICA PIÙ RECENTE ALLA PIÙ VECCHIA
 
 % TODO approvazione
 
 \hline 1.2.1 & 2013-12-4 & Serena Girardi \linebreak (Verificatore) &
 Stesura sezione ``Verifica''. \\

 \hline 1.1.7 & 2013-12-3 & Federico Poli \linebreak (Amministratore) &
 Stesura sezione ``Procedure''. \\

 \hline 1.1.6 & 2013-12-3 & Luca De Franceschi \linebreak (Amministratore) &
 Stesura sezioni ``Analisi'', ``Progettazione''. \\

 \hline 1.1.5 & 2013-12-2 & Luca De Franceschi \linebreak (Amministratore) &
 Stesura sezione ``Repository'', ``Documenti''. \\

 \hline 1.1.4 & 2013-12-2 & Giacomo Fornari \linebreak (Amministratore) &
 Stesura sezione ``Codifica''. \\

 \hline 1.1.3 & 2013-12-1 & Nicolò Tresoldi \linebreak (Responsabile) &
 Stesura sezioni ``Introduzione'', ``Comunicazioni'', ``Glossario''. \\

 \hline 1.1.2 & 2013-12-1 & Federico Poli \linebreak (Amministratore) &
 Stesura sezione ``Ambiente di lavoro''. \\

 \hline 1.1.1 & 2013-12-1 & Nicolò Tresoldi \linebreak (Amministratore) &
 Stesura indice delle sezioni. \\

 \hline
\end{tabularx}
}


\clearpage

\letteraGlossario{A}

\definizione{Abstract syntax tree}
Rappresentazione ad albero della struttura sintattica astratta del codice sorgente. Ogni nodo dell'albero denota un costrutto nel codice.

\definizione{Accidentali}
Problematiche non intrinseche alla produzione, ma che ne sono direttamente collegate.

\definizione{ActiveAdmin}
\glossario{Ruby on Rails} plugin per la generazione di interfacce di amministrazione dati in modo rapido.

\definizione{Agile}
Modello di ciclo di sviluppo software nato alla fine degli anni '90, si basa su quattro principi:
\begin{itemize} 
 \item L'eccessiva rigidità ostacola l'emergere del valore;
 \item Concentrarsi più sul software che sulla documentazione;
 \item La collaborazione con gli \glossario{stakeholder} è fondamentale;
 \item La capacità di adattamento al cambiare delle situazioni è importante.
\end{itemize} 

\definizione{Amazon}
Società di commercio elettronico statunitense che offre tra le altre cose una piattaforma di cloud computing.

\definizione{Amazon AWS}
Amazon Web Server è un servizio di Amazon che mette a disposizione agli utenti un web server il cui piano tariffario prevede 750 ore mensili gratuite.

\definizione{Angular.js}
Angular.js è un \glossario{framework} \glossario{open source} \glossario{Javascript}, mantenuto da Google, utilizzato per creare le componenti \glossario{front-end} di MaaP.

\definizione{API}
Application programming interface, sono un insieme di procedure che un sistema software rende accessibile a terzi per interfacciarsi ad esso.

\definizione{Autoformazione}
Si tratta di ore spese dai componenti del gruppo per apprendere le tecnologie utilizzate, tali ore non sono a carico del proponente.

\letteraGlossario{B}

\definizione{Baseline}
Uno stato dell'insieme di documenti e del software in cui ogni elemento è verificato ed approvato.

\definizione{Base di dati}
Archivio di dati in cui le informazioni contenute sono strutturate seguendo una logica, in questo caso non-relazionale.

\definizione{Best practice}
Insieme di tecniche, pattern e paradigmi che formano il miglior modo per procedere.

\definizione{Branch}
Un Branch in Git è un puntatore che punta ad un commit, e ad alto livello permette di lavorare su versioni diverse di uno stesso file.

\definizione{Broken windows theory}
Teoria sociale per cui non vengono tollerate le piccole trasgressioni che, se trascurate, potrebbero generare fenomeni di emulazione. In questo contesto ci si riferisce all'inserimento di \glossario{pollution} all'interno della \glossario{repository}. 

\definizione{Bug}
Errore nel codice di un software.

\definizione{Business}
Inteso come dominio di Business, l'insieme di tutti i dati che riguardano un dato campo.

\definizione{By correction}
Ottenere la correttezza di un software procedendo per correzioni, ovvero applicando un metodo iterativo. Si tratta di un metodo errato in quanto fa perdere molto tempo e non garantisce a priori la correttezza finale.

\letteraGlossario{C}

\definizione{C++}
Linguaggio di programmazione orientato agli oggetti, con tipizzazione statica, sviluppato nel 1983 come un miglioramento del linguaggio C.

\definizione{Callback}
Funzione, o blocco di codice che viene passata come parametro ad un'altra funzione.

\definizione{CamelCase}
Metodo per scrivere parole composte o frasi unendo tutte le parole tra loro, ma lasciando le loro iniziali maiuscole.

\definizione{Chrome}
Browser basato su WebKit sviluppato da Google. La versione più recente alla stesura di questo documento è la 31.x

\definizione{Collection}
In \glossario{MongoDB} è un insieme di \glossario{documents}.

\definizione{Collection-Index}
Insieme di pagine generate da \ProjectName{} contenenti la visualizzazione in forma tabellare dell'elenco di tutti i documenti della \glossario{Collection} \glossario{MongoDB}.

\definizione{Collection-Show}
Insieme di pagine generate da \ProjectName{} contenenti tutte le coppie chiavi-valore

\definizione{Committente}
Il committente è la figura che commissiona un lavoro, in questo caso il progetto \ProjectName{}.

\definizione{Consuntivo}
Rendiconto  dei risultati di un dato periodo di attività.

\letteraGlossario{D}

\definizione{Database}
Vedi Base di dati.

\definizione{Dati di business}
Vedi Business.

\definizione{Design pattern}
Soluzione progettuale generale per la risoluzione di un problema ricorrente.

\definizione{Documents}
\glossario{MongoDB} è un database a documenti. I dati al suo interno sono dunque inseriti in documenti, che possono essere paragonati alle tabelle nel mondo dei database relazionali, hanno però una struttura meno rigida e sono codificati in BSON.

\definizione{DSL}
Domain Specific Language è un linguaggio di programmazione o un linguaggio di specifica dedicato a particolari problemi di un dominio o a una particolare tecnica di rappresentazione.

\letteraGlossario{E}

\definizione{Ember.js}
Ember.js è un \glossario{framework} \glossario{open source} \glossario{Javascript}, basato sul pattern architetturale MVC, viene utilizzato dal gruppo per creare le componenti \glossario{front-end} di \ProjectName{}.

\definizione{Esperti di business}
Persone esperte del dominio di business.

\definizione{Event-driven}
Paradigma di programmazione, nei programmi scritti utilizzando la tecnica a eventi il flusso del programma è largamente determinato dal verificarsi di eventi esterni. 

\definizione{Express}
\glossario{Framework} per \glossario{node.js}.

\letteraGlossario{F}

\definizione{Facilitatore}
È qualcuno che aiuta un gruppo di persone a capire i loro obbiettivi comuni e che le assiste nel raggiungerli senza prendere posizione nella discussione.

\definizione{Firefox}
Browser \glossario{open source} prodotto da Mozilla Foundation.

\definizione{Framework}
Struttura di supporto su cui un applicativo può essere progettato.
Un framework comprende librerie di codice, convenzioni di sviluppo e una serie di strumenti di supporto allo sviluppo.

\definizione{Front-end}
Parte di un sistema software che gestisce l'interazione con l'utente o con sistemi esterni che producono dati di ingresso.

\definizione{FunGoStudios}
\glossario{StartUp} che si occupa di mobile games.

\definizione{Funzionale}
Vedi linguaggi funzionali.

\letteraGlossario{G}

\definizione{Gantt}
Diagramma di Gantt, è uno strumento di supporto alla gestione dei progetti, prende il nome da Henry Laurence Gantt.
Permette di distribuire le attività pianificate nell'arco temporale stabilito.

\definizione{GanttProject}
\glossario{Tool} online che permette la creazione di diagrammi di \glossario{Gantt} in modo cooperativo.

\definizione{GitHub}
Servizio web di hosting per lo sviluppo di progetti software che usa il sistema di controllo di versione Git.

\definizione{GUI}
Graphical User Interface, è un tipo di interfaccia che consente all'utente di interagire con la macchina manipolando oggetti grafici convenzionali.

\definizione{Gulpease}
L'indice Gulpease è un indice di leggibilità di un testo tarato sulla lingua italiana. Rispetto ad altri ha il vantaggio di utilizzare la lunghezza delle parole in lettere anziché in sillabe, semplificandone il calcolo automatico.

\letteraGlossario{H}

\definizione{HTML/CSS/JS}
Linguaggi comunemente utilizzati nell'ambito web.

\letteraGlossario{I}

\definizione{IDE}
Integrated development environment,  è un software che, in fase di programmazione, aiuta i programmatori nello sviluppo del codice sorgente di un programma.

\definizione{I/O}
Input/Output.

\letteraGlossario{J}

\definizione{Java}
Linguaggio di programmazione orientato agli oggetti, creato da Sun Microsystems.

\definizione{Javascript}
Linguaggio di scripting orientato agli oggetti comunemente usato nella programmazione Web.

\letteraGlossario{L}

\definizione{LAN}
Local Area Network, è una rete informatica di collegamento tra più computer, che copre un'area limitata, come un'abitazione, una scuola o un'azienda.

\definizione{Linea di comando}
Tipologia di interfaccia utente caratterizzata da un'interazione di tipo testuale tra utente ed elaboratore: l'utente impartisce comandi testuali in input mediante tastiera e riceve risposte testuali in output dall'elaboratore mediante display.

\definizione{Linguaggi funzionali}
Paradigma di programmazione in cui il flusso di esecuzione del programma assume la forma di una serie di valutazioni di funzioni matematiche. Il punto di forza principale di questo paradigma è la mancanza di effetti collaterali delle funzioni, il che comporta una più facile verifica della correttezza e della mancanza di bug del programma e la possibilità di una maggiore ottimizzazione dello stesso.

\definizione{Link simbolico}
Particolare tipo di file contenente un percorso relativo od assoluto ad un file o directory a cui fa riferimento.

\definizione{Log}
File in cui si tengono registrate le attività compiute per esempio da un'applicazione, da un server, o da un interprete di comandi.

\definizione{Logger}
In informatica è un componente non intrusivo usato nelle fasi di test di un prodotto software con lo scopo di registrare dei dati sull'esecuzione per favorire l'analisi dei risultati.

\letteraGlossario{M}

\definizione{MaaP}
\glossario{Framework} per generare interfacce web di amministrazione dei dati di business basato su stack \glossario{Node.js} e \glossario{MongoDB}.

\definizione{MaaS}
\ProjectName{} come servizio web.

\definizione{Makefile}
File in cui vengono elencati una serie di istruzioni che verranno eseguiti tramite il comando \code{make}.

\definizione{Markdown}
Linguaggi per la formattazione del testo. Vengono implementati da software utilizzati dal gruppo come \glossario{GitHub} e \glossario{TeamworkPM}.

\definizione{Microsoft Project}
Software di pianificazione sviluppato e venduto da Microsoft. È  uno strumento per assistere i project manager nella pianificazione, nell'assegnazione delle risorse, nella verifica del rispetto dei tempi, nella gestione dei budget e nell'analisi dei carichi di lavoro.

\definizione{Milestone}
Termine inglese che letteralmente significa pietra miliare. Viene utilizzato nella pianificazione e gestione di progetti per indicare il raggiungimento di obiettivi stabiliti in fase di definizione del progetto. 
Le milestone indicano quindi importanti traguardi intermedi nello svolgimento del progetto.

\definizione{Moduli}
Unità di lavoro di programmazione affidabile ad una singola persona. Tali moduli dovrebbero avere dimensione predefinita per evitare carichi di lavoro troppo grandi. Si può lavorare a più moduli in parallelo.

\definizione{MongoDB}
Sistema gestionale di basi di dati non relazionale, orientato ai documenti, di tipo \glossario{NoSQL}. Il linguaggio utilizzato per la gestione dei dati è \glossario{JavaScript}, del quale sfrutta in particolare la notazione BSON.

\definizione{Mongoose.js}
Utilizzato per creare una struttura logica nei documenti di \glossario{MongoDB}.

\letteraGlossario{N}

\definizione{Node.js}
Piattaforma software utilizzata per creare applicazioni distribuite facilmente scalabili.
Node.js utilizza \glossario{JavaScript} come linguaggio di scripting e gestisce le attese I/O in modo asincrono.

\definizione{NoSQL}
NoSQL è un movimento che promuove sistemi software dove la persistenza dei dati caratterizzata dal fatto di non utilizzare il modello relazionale, tipicamente usato dai database tradizionali. L'espressione NoSQL fa riferimento al linguaggio SQL, che è il più comune linguaggio di interrogazione dei dati nei database relazionali, qui preso a simbolo dell'intero paradigma relazionale.
Questi archivi di dati tipicamente non richiedono uno schema fisso, evitano spesso le operazioni di unione e puntano a scalare orizzontalmente.

\letteraGlossario{O}

\definizione{Open Source}
Software i cui autori ne permettono e favoriscono il libero studio e l'apporto di modifiche da parte di altri programmatori indipendenti. Questo è realizzato mediante l'applicazione di apposite licenze d'uso.

\letteraGlossario{P}

\definizione{PDCA}
PDCA (plan-do-check-act) è un metodo iterativo a quattro stadi usato per il controllo e il continuo miglioramento dei processi e dei prodotti.

\definizione{PDF}
Il \textbf{Portable Document Format}, comunemente abbreviato PDF, è un formato di file basato su un linguaggio di descrizione di pagina sviluppato da Adobe Systems nel 1993 per rappresentare documenti in modo indipendente dall'hardware e dal software utilizzati per generarli o per visualizzarli. 

\definizione{Package}
In alcuni linguaggi orientati agli oggetti, tra cui \glossario{Java}, è un meccanismo che permette di organizzare un insieme di classi tra loro correlate che concorrono allo stesso fine.

\definizione{Packet manager}
Collezione di strumenti presenti in un sistema operativo che automatizzano il processo di installazione, aggiornamento, configurazione e rimozione dei pacchetti software in un computer.

\definizione{Pollution}
Con tale termine ci si riferisce a tutti i file che non devono entrare nella \glossario{repository}.

\definizione{Populate}
Inserimento di dati all'interno del database. In questo caso all'interno dei documenti di \glossario{MongoDB}.

\definizione{Pre-commit}
Script che verifica i file inseriti prima di fare il commit, per evitare \glossario{pollution} dentro la \glossario{repository}.

\definizione{Proponente}
Colui che presenta una proposta, in questo caso il capitolato riguardante il progetto.
Nello specifico ci si riferisce a \Proponente{}.

\definizione{Prototipi}
Un modello approssimato o parziale del sistema che vogliamo sviluppare che simula o esegue alcune funzioni del sistema finale, realizzato allo scopo di valutarne le caratteristiche, se interno o per mostrarlo all'utente se esterno.

\letteraGlossario{Q}

\definizione{Qualità}
Insieme di caratteristiche di un'entità che ne determinano la capacità di soddisfare esigenze espresse o implicite.

\letteraGlossario{R}

\definizione{Report}
Resoconto strutturato in modo schematico.

\definizione{Repository}
Ambiente di un sistema informativo, in cui vengono gestiti i metadati, attraverso tabelle relazionali.
Nel nostro caso il sistema informativo è gestito con \glossario{GitHub}.

\definizione{Ruby on Rails}
\glossario{Framework} \glossario{open source} per applicazioni web scritto in Ruby, la cui architettura è fortemente ispirata al paradigma Model-View-Controller (MVC). I suoi obiettivi sono la semplicità e la possibilità di sviluppare applicazioni di concreto interesse con meno codice rispetto ad altri \glossario{framework}. Il tutto con necessità di configurazione minimale.

\letteraGlossario{S}

\definizione{Scala}
Linguaggio di programmazione di tipo general-purpose multi-paradigma studiato per integrare le caratteristiche e funzionalità dei linguaggi orientati agli oggetti e dei linguaggi funzionali. La compilazione di codice sorgente Scala produce Java bytecode per l'esecuzione su una JVM.

\definizione{Scalabilità}
Capacità di un sistema di ``crescere'' o ``decrescere'' (aumentare o diminuire di scala) in funzione delle necessità e delle disponibilità.

\definizione{Schema}
Lo schema di un \glossario{database} è la struttura o organizzazione logica dei dati in esso contenuti descritta mediante l'uso di un linguaggio formale supportato da un database management system.

\definizione{SEMAT}
Software Engineering Method and Theory è un'iniziativa nata con lo scopo di rimodellare il mondo dell'Ingegneria del Software imponendo metodi i cui fini sono la qualità e la disciplina.

\definizione{Server}
Componente che fornisce, attraverso una rete, un qualunque tipo di servizio o risorsa ad altri componenti tipicamente detti client.

\definizione{Slack time}
Durante la pianificazione delle attività di progetto, periodo di tempo lasciato libero da ogni attività per ammortizzare eventuali ritardi, in particolare se in presenza di forti dipendenze tra le attività.

\definizione{SMS}
Short Message Service, comunemente usato per indicare un breve messaggio di testo inviato da un telefono cellulare ad un altro.

\definizione{SPICE}
ISO/IEC 15504 \emph{Information technology - Process assessment}, conosciuto anche come SPICE (Software Process Improvement and Capability Determination), è un insieme di documenti di standard tecnici che fornisce informazioni generali sui concetti di valutazione dei processi e dei suoi usi nei due contesti di miglioramento dei processi e valutazione della maturità dei processi.

Riguardano i processo di sviluppo di software e le relative funzioni gestionali di azienda. In sostanza, spiega come poterne fare una valutazione.

\definizione{Software Quality Managment Techinques}
Tecniche utili alla gestione della qualità mediante valutazione della qualità dei prodotti.

\definizione{SCR - software change request}
Software che permette di tracciare le anomalie riscontrate nel software e nei documenti riguardanti il prodotto.

\definizione{Stack tecnologico}
L'insieme delle tecnologie utilizzate dal software.

\definizione{Stakeholders}
Portatori di interesse, l'insieme di persone coinvolte nel ciclo di vita del software.

\definizione{Stand-alone}
Software in grado di funzionare da solo o in maniera indipendente da altri software, con cui può però interagire.

\definizione{Startup}
Impresa recentemente creata in cerca di un mercato. Spesso ci si riferisce ad imprese che operano nel settore informatico.

\definizione{Storage}
Si tratta di dispositivi hardware, supporti per la memorizzazione, infrastrutture e software dedicati alla memorizzazione non volatile di grandi quantità di informazioni in formato elettronico.

\definizione{Stub}
Porzione di codice che, dati certi input, fornisce sempre gli stessi output prestabiliti, al fine di verificare se la funzione chiamante fornisce i risultati attesi dal test.

\definizione{Sviluppatore}
Chi produce applicazioni o sistemi software.

\letteraGlossario{T}

\definizione{TeamworkPM}
Teamwork Project Manager è una \glossario{web application} che aiuta le persone coinvolte in un progetto, indipendentemente dal ruolo ricoperto a gestire e organizzare tale progetto.

\definizione{Test driver}
Clone di prova che simula la classe contenente il main. È complementare allo \glossario{Stub}.

\definizione{Tools}
Applicazione che svolge un determinato compito di utilità.

\letteraGlossario{U}

\definizione{UML}
Unified Modeling Language, è un linguaggio di modellazione e specifica basato sul paradigma object-oriented. 
È utilizzato per descrivere soluzioni analitiche e progettuali in modo sintetico e comprensibile a un vasto pubblico.
Ad oggi si è giunti alla versione 2.0.

\definizione{Unità}
La più piccola parte di lavoro assegnabile ad un programmatore e che è utile verificare singolarmente. Un'unità raggruppa più \glossario{moduli}.

\definizione{User-friendly}
Usabilità, il grado di facilità e soddisfazione con cui si compie l'interazione tra l'uomo e uno strumento.

\definizione{Utente}
Fruitore del servizio \ProjectName{}.

\definizione{UTF-8}
Unicode Transformation Format-8 bit, è una codifica dei caratteri Unicode. 

\letteraGlossario{V}

\definizione{Validazione}
Controllo effettuato sul software, per controllare se tutti i requisiti previsti sono stati coperti.

\definizione{Verifica}
Ricerca la consistenza, correttezza e completezza

\letteraGlossario{W}

\definizione{Web application}
Applicazione accessibile via web, che offre determinati servizi all'utente.

%\appendix
%\include{appendice-A.tex}

\end{document}
