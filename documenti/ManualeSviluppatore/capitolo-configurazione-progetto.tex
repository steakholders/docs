\section{Configurazione nuovo progetto}

\subsection{Generazione}

Vengono riportate in questa sezione i passi necessari per la creazione di una nuova applicazione MaaP, utilizzando il modello ``scaffold'' fornito dal framework. I comandi descritti devono essere eseguiti da un terminale da cui sia possibile eseguire il Node Packet Manager ed è richiesta una connessione ad Internet.

\begin{enumerate}
 \item Installare il framework MaaP tramite il Node Packet Manager: \\
 \centerline{ \code{ \$> npm install -g maap } } \\
 oppure, se il framework viene fornito tramite repository git \\
 \centerline{ \code{ \$> npm install -g maap } }

 \item Creare un'applicazione utilizzando il modello fornito dal framework MaaP: posizionarsi all'interno della cartella nella quale si vorrà creare l'applicazione ed eseguire il comando \\
 \centerline{ \code{ \$> maap create <ProjectName> } }
 Dove \texttt{<ProjectName>} corrisponde al nome del progetto che si vuole creare.

 \item Installare le dipendenze dell'applicazione: posizionarsi all'interno della cartella dell'applicazione ed eseguire il comando \\
 \centerline{ \code{ \$> npm install } }

 \item Configurare l'applicazione: aprire con un editor di testo il file \code{config.js} generato assieme all'applicazione e configurarlo riferendosi alla sezione \ref{config} di questo manuale.\\

 \item Configurare le collection dell'applicazione, riferendosi alla sezione \ref{collection} di questo manuale.\\

 \item Eseguire il server dell'applicazione: posizionarsi all'interno della cartella dell'applicazione ed eseguire il comando \\
 \centerline{ \code{ \$> npm start } }
\end{enumerate}

\subsection{Configurazione}
\label{config}

Per configurare l'applicazione generata dal framework bisogna modificare il file \code{config.js} presente nella cartella principale dell'applicazione. All'interno del file sono presenti due sezioni: ``development'' e ``production'', l'applicazione sceglie quale utilizzare in funzione del valore della variabile di ambiente \code{NODE\_ENV}. In ciascuna di esse è possibile impostare i seguenti parametri:

\begin{itemize}
\item \textbf{webServer}: un oggetto contenente
		\begin{itemize}
		\item \textbf{port}: la porta su cui il server si metterà in ascolto (Integer);
		\item \textbf{static}: la path assoluta della cartella contenente i file statici che il server dovrà servire al client (String);
		\end{itemize}
		
\item \textbf{userDB}: un oggetto contenente
		\begin{itemize}
		\item \textbf{uri}: l'indirizzo del database che l'applicazione deve usare come database degli utenti;
		\end{itemize}
		
\item \textbf{dataDB}: un oggetto contenente
		\begin{itemize}
		\item \textbf{uri}: l'indirizzo del database che l'applicazione deve usare come database dei dati;
		\end{itemize}
		
\item \textbf{collectionPath}: la path assoluta della cartella contenente i file di configurazione dsl che il server dovrà utilizzare come configurazione (String);
		
\item \textbf{smtp}: l' oggetto contenente la configurazione del servizio smtp che verrà utilizzato dalla libreria Nodemailer. Per l'elenco completo di parametri configurabili riferirsi a \url{https://github.com/andris9/Nodemailer}.
\end{itemize}

\subsection{Struttura applicazione}

L'applicazione generata dal framework presenta i seguenti file e cartelle: 
\begin{itemize}
 \item \textbf{\file{\texttt{ProjectName}/}} \\
	Contiene i file che verranno utilizzati come base di partenza (scaffold) per il progetto dello sviluppatore che utilizzerà il prodotto.

 \item \textbf{\file{\texttt{ProjectName}/collections/}} \\
	Contiene i file di configurazione delle collection utilizzate dall'applicazione scaffold.

 \item \textbf{\file{\texttt{ProjectName}/app/}} \\
	Contiene i file relativi al Frontend dell'applicazione scaffold.

 \item \textbf{\file{\texttt{ProjectName}/app/view/}} \\
	Contiene i file statici html usati dal Frontend dell'applicazione scaffold.

 \item \textbf{\file{\texttt{ProjectName}/app/style/}} \\
	Contiene i file statici di stile usati dal Frontend dell'applicazione scaffold.

 \item \textbf{\file{\texttt{ProjectName}/app/scripts/}} \\
	Contiene i file dell'applicazione AngularJS usata nel Frontend.

 \item \textbf{\file{\texttt{ProjectName}/app/scripts/services/}} \\
	Contiene i file dei service di AngularJS usata nel Frontend.

 \item \textbf{\file{\texttt{ProjectName}/app/scripts/controllers/}} \\
	Contiene i file dei controller di AngularJS usata nel Frontend.

 \item \textbf{\file{\texttt{ProjectName}/app/bower\_components/}} \\
	Contiene i file delle librerie utilizzate dal Frontend dell'applicazione scaffold.
	
 \item \textbf{\file{\texttt{ProjectName}/node\_modules/}} \\
	Contiene i file delle librerie utilizzate dal Backend dell'applicazione scaffold.
\end{itemize}
