\section{Ruoli di progetto}
\subsection{Rotazione dei ruoli}
\label{rotazioneruoli}
Durante lo sviluppo del progetto vi sono diversi ruoli, che ogni membro del gruppo \GroupName{} è tenuto a ricoprire almeno una volta. Per evitare possibili conflitti causati dalla rotazione dei ruoli, le attività principali assegnabili a specifici ruoli sono pianificate nel \PianoDiProgetto. Ogni componente del gruppo è tenuto a svolgere le attività assegnategli e a rispettare il ruolo che ne consegue. Il \textit{Responsabile} di progetto ha il compito di fare rispettare i ruoli assegnati durante le attività, mentre il \textit{Verificatore} deve individuare le possibili incongruenze tra i ruoli e le modifiche registrate nei diari delle modifiche.
Le incongruenze possono essere di due tipi:
\begin{itemize}
\item Il compito svolto non fa parte dei compiti propri di un dato ruolo;
\item La stessa persona verifica ciò che ha precedentemente prodotto.
\end{itemize}

Inoltre, si raccomanda che:
\begin{itemize}
\item Una persona non impieghi più del 50\% delle ore di lavoro in un unico ruolo;
\item Sia assegnato un solo task al giorno per persona.
\item I ruoli vengano ruotati ogni settimana, ma si lascia libertà al \textit{Responsabile} di ruotare in base alle esigenze.
\end{itemize}

Si predilige dunque una maggior frequenza di rotazione a discapito dell'efficienza.

Il responsabile quando assegna un ticket, come indicato nella sezione \ref{teamwork} dovrà indicare il ruolo che il destinatario ricopre nell'adempiere tale task. Questo è necessario per il diario delle modifiche dei singoli documenti e per il rendiconto economico. I ruoli sono assegnati basandosi sul \PianoDiProgetto. 

Deve inoltre tener conto delle non disponibilità a lavorare come indicato nella sezione \ref{Calendario condiviso}.

	\subsection{Amministratore}
	
	L'\textit{Amministratore} equipaggia, organizza e gestisce l'ambiente di lavoro e di produzione. Collabora con il \textit{Responsabile} alla stesura del \textit{Piano di Progetto} e redige le \textit{Norme di Progetto}.
	Le responsabilità assunte dall'\textit{Amministratore} sono:

	\begin{itemize}

		\item Attuare le scelte tecnologiche concordate con il \textit{Responsabile} di progetto;
		\item Gestire le liste di distribuzione e assicurarne il rispetto;
		\item Controllare versioni e configurazioni del prodotto;
		\item Risolvere i problemi legati alla gestione dei processi.

	\end{itemize}

	\subsection{Analista}

	L'\textit{Analista} è responsabile dell'attività di analisi, pertanto deve comprendere appieno il dominio applicativo. Redige lo \textit{Studio di Fattibilità} e l'\textit{Analisi dei Requisiti} ovvero una specifica di progetto ad alto livello con vincoli e rischi tecnologici, affrontabile dal proponente, dal committente e dal \textit{Progettista}.


	\subsection{Progettista}

	Il \textit{Progettista} è responsabile dell'attività di progettazione, ha una profonda conoscenza dello stack tecnologico utilizzato e competenze tecniche aggiornate. Grazie a tali caratteristiche, ha una forte influenza sugli aspetti tecnici e tecnologici del progetto, e spesso ne assume responsabilità di scelta e gestione.


	\subsection{Programmatore}

	Il \textit{Programmatore} ha responsabilità circoscritte, si occupa dell'attività di codifica, nel rispetto delle \textit{Norme di Progetto}, miranti alla realizzazione del prodotto e delle componenti di ausilio necessarie per l'esecuzione delle prove di verifica e validazione.


	\subsection{Responsabile}
	
	Il \textit{Responsabile} ha l'ultima voce in capitolo per quanto concerne le decisioni sul progetto, è il responsabile ultimo dei risultati, infatti approva l'emissione dei documenti. Inoltre, redige il \textit{Piano di Progetto} assieme all'\textit{Amministratore}. 
	Le responsabilità assunte dal \textit{Responsabile} sono:

	\begin{itemize}

		\item Pianificazione e organizzazione dello sviluppo del progetto, stima tempi e costi, e assegnazione delle attività ai componenti del gruppo;
		\item Riportare lo stato del progetto ai committenti;
		\item Analizzare i rischi che posso incorrere, monitorarli e prendere provvedimenti a riguardo;
		\item Stabilire una \textit{ways of working} per ogni componente del gruppo ai fini di un'influenza positiva delle performance del gruppo.
	
	\end{itemize}

	
	\subsection{Verificatore}

	Il \textit{Verificatore} organizza ed attua le attività di verifica e controlla che le attività siano conformi alle norme. Redige la parte del \textit{Piano di Qualifica} che documenta le attività svolte e i risultati ottenuti, confrontandoli con le metriche espresse nel \PianoDiQualifica. 