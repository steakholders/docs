\section{Pianificazione dei test}

Si vuole adottare una strategia di verifica del software tramite test opportunamente predeterminati e che garantiscano almeno un test per ogni requisito. I test sono l'applicazione delle tecniche di verifica dinamica introdotte nelle \NormeDiProgetto{}; tali attività, oltre a richiedere l'esecuzione del programma, devono poter essere ripetibili, ossia tramite delle specifiche su come riprodurre i test vogliamo che il loro output sia deterministico. \`E importante che i test di unità vengano svolti in parallelo, dando precedenza alle unità che producono risultati utili alla comprensione del loro funzionamento integrato, l'ambiente di testing deve soddisfare tale obiettivo. \\
L'attività di test deve produrre un \glossario{log} che specifica quando e chi ha eseguito il test e con quali input; l'insorgenza di \glossario{failure} deve essere tracciata e catalogata.

	\subsection{Livelli di testing}
	Il testing del software viene suddiviso in livelli differenti e si concretizzano in un esecuzione bottom-up che avanza sequenzialmente alle attività di codifica e  di validazione. 
	I test che si andranno ad applicare sono di cinque tipi, riservando la specifica delle ultime due tipologie alla prossima revisione:

\begin{enumerate}
	\item Test di Validazione (TV): viene verificato che il prodotto soddisfi quanto richiesto dal \glossario{proponente} individuando delle macro azioni da eseguire sul sistema che un normale utente svolge comunemente;
	\item Test di Sistema (TS): sono test relativi al comportamento dell'intero sistema ossia viene verificato che la sua architettura generale funziona complessivamente bene;
	\item Test di Integrazione (TI): vengono verificate le componenti del sistema contenute nella \SpecificaTecnica{}, ossia viene verificato che i \glossario{package} siano funzionanti e in grado di funzionare nel loro insieme; %PACKAGE
	\item Test di Unità (TU): viene testata ogni unità, ossia la più piccola parte di lavoro assegnabile ad un programmatore. In questo progetto una unità dovrebbe corrispondere ad una \code{function} o a un \code{method};  %FUNCION METODI
	\item Test di Regressione (TR): possono essere test di tutte le tipologie succitate che devono mostrare il funzionamento del prodotto a seguito di una modifica.
\end{enumerate}

	La figura \ref{fig:V-Model} illustra come i test elencati vengono distribuiti durante in ciclo di vita del prodotto.

	\begin{figure}[H]
	\centering \includegraphics[width=1\textwidth]{V-Model.png}
	\caption{V-Model per il testing software}
	\label{fig:V-Model}
	\end{figure}

	\subsection{Test di sistema}
	Vengono qui descritti i test di sistema che andranno a verificare il funzionamento complessivo delle componenti.
	Nella seguente tabella, lo stato di ogni test è definito da N.E per non eseguito.
	
	

	\begin{center}
	\def\arraystretch{1.5}
	\bgroup
		\begin{longtable}{| p{3cm} | p{6cm} | p{1.5cm} | p{2cm} | }
		\hline 
		 \textbf{Test Sistema} & \textbf{Descrizione} & \textbf{Stato} & \textbf{Requisito} \\ \hline
				TS-RA1O 1.1 & 
				Verificare che durante la verifica delle credenziali l'indirizzo email venga immesso tramite un campo di testo apposito. & N.E & RA1O 1.1 \newline  \\ \hline 
				TS-RA1O 1.2 & 
				Verificare che durante la verifica delle credenziali, la password venga immessa tramite un capo di testo apposito.
 & N.E & RA1O 1.2 \newline  \\ \hline 
				TS-RA1O 1.3 & 
				Verificare che il sistema verifichi le credenziali di un utente tramite un database indipendente da quello che contiene la Collection. & N.E & RA1O 1.3 \newline  \\ \hline 
				TS-RA1O 1.3.1 & 
				Verificare che, in caso di fallimento dell'autenticazione di un utente, il sistema visualizzi una pagina di errore. & N.E & RA1O 1.3.1 \newline  \\ \hline 
				TS-RA1O 1.3.2 & 
				Verificare che in caso in cui autenticazione vada a buon fine, l'utente venga reindirizzato automaticamente sulla dashboard dell'applicazione.  & N.E & RA1O 1.3.2 \newline  \\ \hline 
				TS-RA1O 2.1 & 
				Verificare che il sistema permetta il recupero password attraverso l'inserimento dell'email.
 & N.E & RA1O 2.1 \newline  \\ \hline 
				TS-RA1O 2.2 & 
				Verificare che un utente non autenticato che richiede il reset della propria password riceva un email con un link per il reset. & N.E & RA1O 2.2 \newline  \\ \hline 
				TS-RA1O 2.3 & 
				Verificare che un utente non autenticato possa resettare la propria password tramite l'inserimento di una nuova password.
 & N.E & RA1O 2.3 \newline  \\ \hline 
				TS-RA1O 4.1 & 
				Verificare che la visualizzazione di una Collection-index consista in una tabella le cui righe corrispondono ai document presenti nel database e le cui colonne siano i relativi attributi. & N.E & RA1O 4.1 \newline  \\ \hline 
				TS-RA1O 4.1.1 & 
				Verificare che ogni riga della tabella corrispondente ad un Document abbia una chiave selezionabile che rimanda alla corrispondente pagina show.
 & N.E & RA1O 4.1.1 \newline  \\ \hline 
				TS-RA1D 4.1.2 & 
				Verificare che l'admin possa eliminare un documento tramite un link rapido. & N.E & RA1D 4.1.2 \newline  \\ \hline 
				TS-RA1D 4.1.3 & 
				Verificare che l'admin possa modificare un document della collection-index.
 & N.E & RA1D 4.1.3 \newline  \\ \hline 
				TS-RA1D 4.2 & 
				Verificare che sia possibile visualizzare un sottoinsieme di Document tramite dei filtri personalizzati sugli attributi. & N.E & RA1D 4.2*  \newline  \\ \hline 
				TS-RA1F 4.3 & 
				Verificare che l'amministratore possa creare un nuovo Document nella base di dati. & N.E & RA1F 4.3*  \newline  \\ \hline 
				TS-RA1O 5.1 & 
				Verificare che l'admin possa editare ogni singolo attributo modificabile del documento della pagina show. & N.E & RA1O 5.1 \newline  \\ \hline 
				TS-RA1F 5.2 & 
				Verificare che l'utente possa eseguire un'azione personalizzata tramite l'esecuzione di un pulsante. & N.E & RA1F 5.2*  \newline  \\ \hline 
				TS-RA1O 5.3 & 
				Viene verificato che l'utente possa eliminare il Document selezionato nella show-page. & N.E & RA1O 5.3 \newline  \\ \hline 
				TS-RA1O 6.1 & 
				Viene verificato che l'admin possa creare un nuovo utente dalla pagina di amministrazione. & N.E & RA1O 6.1 \newline  \\ \hline 
				TS-RA1O 6.1.1 & 
				Viene verificato che l'admin disponga di una pagina di creazione di un nuovo utente. & N.E & RA1O 6.1.1 \newline  \\ \hline 
				TS-RA1O 6.1.1.1 & 
				Verificare che l'admin possa inserire l'indirizzo email del nuovo utente in un apposito campo di testo presente all'interno della pagina di creazione di un nuovo utente. & N.E & RA1O 6.1.1.1 \newline  \\ \hline 
				TS-RA1O 6.1.1.2 & 
				Viene verificato che l'admin possa inserire la password del nuovo utente in un apposito campo di testo presente all'interno della pagina di creazione di un nuovo utente. & N.E & RA1O 6.1.1.2 \newline  \\ \hline 
				TS-RA1O 6.1.1.3 & 
				Verificare che l'admin possa inserire il ``livello utente'' del nuovo utente tramite una combo-box presente all'interno della pagina di creazione di un nuovo utente. & N.E & RA1O 6.1.1.3 \newline  \\ \hline 
				TS-RA1O 6.1.2 & 
				Viene verificato che l'applicazione prelevi tutti i dati inseriti dall'admin nella pagina di creazione di un nuovo utente e li invii al database delle credenziali, il quale provvederà all'inserimento del nuovo record. & N.E & RA1O 6.1.2 \newline  \\ \hline 
				TS-RA1O 6.1.3 & 
				Verificare che venga visualizzato un messaggio d'errore nel caso in cui l'admin non abbia compilato correttamente i campi presenti all'interno della pagina di creazione di un nuovo utente. & N.E & RA1O 6.1.3 \newline  \\ \hline 
				TS-RA1O 6.2 & 
				Viene verificato che l'admin abbia la possibilità di selezionare un utente dalla index-page e visualizzare la sua relativa show-page. & N.E & RA1O 6.2 \newline  \\ \hline 
				TS-RA1O 6.2.1 & 
				Verificare che l'admin possa elevare l'utente normale selezionato al livello ``admin'' dalla show-page relativa. & N.E & RA1O 6.2.1 \newline  \\ \hline 
				TS-RA1O 6.2.2 & 
				Verificare che l'admin possa declassare l'admin selezionato a livello di utente normale dalla show-page relativa. & N.E & RA1O 6.2.2 \newline  \\ \hline 
				TS-RA1O 6.2.3 & 
				Viene verificato che l'admin possa modificare l'attributo email dell'utente selezionato dalla relativa show-page. & N.E & RA1O 6.2.3 \newline  \\ \hline 
				TS-RA1O 6.2.4 & 
				Verificare che l'admin possa modificare l'attributo password dell'utente selezionato dalla relativa show-page. & N.E & RA1O 6.2.4 \newline  \\ \hline 
				TS-RA1O 6.2.5 & 
				Viene verificato che l'admin possa eliminare l'utente visualizzato nella \glossario{show-page}. & N.E & RA1O 6.2.5 \newline  \\ \hline 
				TS-RF1O 7 & 
				Verificare che il linguaggio DSL all'interno di MaaP Framework sia stato implementato e sia funzionante. & N.E & RF1O 7 \newline  \\ \hline 
				TS-RF1O 8.1  & 
				Verificare che Maap Framework generi automaticamente lo scheletro dell’applicazione creata dallo sviluppatore. & N.E & RF1O 8.1  \newline  \\ \hline 
				TS-RF1O 8.1.1 & 
				Verificare che Maap Framework importi automaticamente in un'apposita directory del progetto tutte le librerie necessarie al corretto funzionamento del sistema. Librerie necessarie: \begin{itemize} \item Express v-3.4.8 \item MongoDB v-1.3.23 \item Mongoose v-3.8.4 \end{itemize} & N.E & RF1O 8.1.1 \newline  \\ \hline 
				TS-RF1O 8.1.2 & 
				Verificare che Maap Framework crei automaticamente in un’apposita directory il file di configurazione di default dell’applicazione generata. & N.E & RF1O 8.1.2 \newline  \\ \hline 
				TS-RF1O 8.1.3 & 
				Viene verificato che Maap Framework crei automaticamente il sistema di autenticazione per l’applicazione generata. & N.E & RF1O 8.1.3 \newline  \\ \hline 
				TS-RF1O 8.1.4 & 
				Verificare che Maap Framework crei automaticamente le directory di descrizione delle pagine web. & N.E & RF1O 8.1.4 \newline  \\ \hline 
				TS-RF1O 8.2 & 
				Verificare che Maap Framework crei automaticamente un account admin di default. & N.E & RF1O 8.2 \newline  \\ \hline 
				TS-RF1F 8.3 & 
				Verificare che il framework MaaP permetta allo sviluppatore di definire un namespace per l’applicazione generata. & N.E & RF1F 8.3*  \newline  \\ \hline 
				TS-RF1O 9.1 & 
				Verificare che il DSL permetta allo sviluppatore di creare una pagina Collection-index. & N.E & RF1O 9.1 \newline  \\ \hline 
				TS-RF1O 9.1.1 & 
				Verificare che il DSL deve permetta allo sviluppatore di poter definire una serie di attributi da visualizzare all’interno della pagina Collection-index. & N.E & RF1O 9.1.1 \newline  \\ \hline 
				TS-RF1O 9.1.2 & 
				Viene verificato che il DSL permetta allo sviluppatore di poter definire un ordinamento di default (ordine alfanumerico) di visualizzazione dei document all'interno della pagina Collection-index. & N.E & RF1O 9.1.2 \newline  \\ \hline 
				TS-RF1O 9.1.3 & 
				Verificare che il DSL permetta allo sviluppatore di poter definire un eventuale limite di elementi da visualizzare all’interno della pagina Collection-index. & N.E & RF1O 9.1.3 \newline  \\ \hline 
				TS-RF1O 9.1.4 & 
				Viene verificato che il DSL permetta allo sviluppatore di poter definire quali attributi sono ordinabili all’interno della pagina Collection-index. & N.E & RF1O 9.1.4 \newline  \\ \hline 
				TS-RF1O 9.1.5 & 
				Verificare che il DSL permetta allo sviluppatore di definire la funzione populate per far si che una chiave riferisca ad un documento esterno. & N.E & RF1O 9.1.5 \newline  \\ \hline 
				TS-RF1O 9.1.6 & 
				Verificare che il DSL permetta allo sviluppatore di definire delle query per creare la pagina Collection-index in base al risultato della loro estrazione. & N.E & RF1O 9.1.6 \newline  \\ \hline 
				TS-RF1O 9.1.7 & 
				Viene verificato che il DSL permetta allo sviluppatore di definire delle trasformazioni sugli attributi da visualizzare. & N.E & RF1O 9.1.7 \newline  \\ \hline 
				TS-RF1O 9.2 & 
				Viene verificato che il DSL permetta allo sviluppatore di creare una pagina Collection-show. & N.E & RF1O 9.2 \newline  \\ \hline 
				TS-RF1O 9.2.1 & 
				Verificare che il DSL permetta allo sviluppatore di definire una serie di attributi visualizzabili all’interno della pagina Collection-show. & N.E & RF1O 9.2.1 \newline  \\ \hline 
				TS-RF1O 9.2.2 & 
				Verificare che il DSL permetta allo sviluppatore la definizione degli attributi del Document come attributi innestati o array di Document tramite la funzione populate. & N.E & RF1O 9.2.2 \newline  \\ \hline 
				TS-RF1O 9.2.3 & 
				Verificare che lo sviluppatore abbia la possibilità di personalizzare la show page definendone l’ordinamento degli attributi. & N.E & RF1O 9.2.3 \newline  \\ \hline 
				TS-RF1O 9.2.4 & 
				Viene verificato che lo sviluppatore possa definire trasformazioni agli attributi per poi visualizzarli nella show-page. & N.E & RF1O 9.2.4 \newline  \\ \hline 
				TS-RF1F 9.2.5 & 
				Verificare che lo sviluppatore possa personalizzare la show-page definendo delle operazioni personalizzate che l’utente potrà utilizzare tramite appositi pulsanti. & N.E & RF1F 9.2.5*  \newline  \\ \hline 
				TS-RF1O 9.3 & 
				Viene verificato che il framework MaaP permetta allo sviluppatore di cambiare il nome della Collection da visualizzare nel menu di navigazione. & N.E & RF1O 9.3 \newline  \\ \hline 
				TS-RF1O 9.4 & 
				Verificare che il framework MaaP permetta allo sviluppatore di modificare l’ordine di visualizzazione della Collection nel menu di navigazione. & N.E & RF1O 9.4 \newline  \\ \hline 
				TS-RS1F 10.1 & 
				Verificare che il sistema MaaS permetta allo sviluppatore di scrivere una Collection tramite editor di testo presente nella pagina web. & N.E & RS1F 10.1*  \newline  \\ \hline 
				TS-RS1F 10.2 & 
				Verificare che il sistema MaaS permetta all'utente di poter scrivere una Collection caricando un file prodotto dal framework MaaP. & N.E & RS1F 10.2*  \newline  \\ \hline 
				TS-RS1F 10.3 & 
				Verificare che il sistema MaaS permetta ad un utente non registrato di registrarsi al suo servizio. & N.E & RS1F 10.3*  \newline  \\ \hline 
				TS-RS1F 10.4 & 
				Verificare che il sistema MaaS assegni automaticamente un \glossario{namespace} sul sistema al nuovo utente registrato. & N.E & RS1F 10.4*  \newline  \\ \hline 
				TS-RS1F 10.5 & 
				Verificare che il servizio MaaS visualizzi un messaggio d’errore nel caso in cui la registrazione fallisca a causa di credenziali già esistenti. & N.E & RS1F 10.5*  \newline  \\ \hline 
				TS-RS1F 10.6 & 
				Verificare che il servizio MaaS metta a disposizione di un utente non autenticato la possibilità di effettuare il login al sistema. & N.E & RS1F 10.6*  \newline  \\ \hline 
				TS-RS1F 10.7 & 
				Verificare che il servizio MaaS visualizzi un messaggio d’errore nel caso in cui l’utente non autenticato abbia inserito credenziali errate nel sistema di login. & N.E & RS1F 10.7*  \newline  \\ \hline 
				TS-RS1F 10.8 & 
				Verificare che il sistema MaaS permetta ad un utente non autenticato di modificare il proprio profilo. & N.E & RS1F 10.8*  \newline  \\ \hline 
				TS-RS1F 10.9 & 
				Verificare che il sistema MaaS permetta ad un utente non autenticato di eliminare il proprio account dal sistema. & N.E & RS1F 10.9*  \newline  \\ \hline 
				TS-RS1F 10.9.1 & 
				Verificare che il sistema MaaS provveda all'eliminazione dei file di configurazione associati all'utente rimosso dal sistema. & N.E & RS1F 10.9.1*  \newline  \\ \hline 
				TS-RS1F 10.10 & 
				Verificare che il sistema MaaS permetta allo sviluppatore di eliminare una Collection esistente. & N.E & RS1F 10.10*  \newline  \\ \hline 
				TS-RA1D 13.1 & 
				Verificare che l’utente possa modificare la password di accesso all'applicazione. & N.E & RA1D 13.1 \newline  \\ \hline 
				TS-RF1O 14.1 & 
				Verificare che il framework MaaP renda possibile la configurazione dei database delle credenziali. & N.E & RF1O 14.1 \newline  \\ \hline 
				TS-RF1O 14.2 & 
				Verificare che il framework MaaP renda possibile la configurazione dei database delle Collection. & N.E & RF1O 14.2 \newline  \\ \hline 
				TS-RF1F 14.3 & 
				Verificare che il framework MaaP renda possibile la selezione di un name-space per un database se la funzione di \glossario{namespace} è abilitata. & N.E & RF1F 14.3*  \newline  \\ \hline 
				TS-RA1F 15.1 & 
				Verificare che l’applicazione MaaP metta a disposizione dell’admin la visualizzazione degli indici in base alle query più richieste dall’applicazione. & N.E & RA1F 15.1*  \newline  \\ \hline 
				TS-RA1F 15.2 & 
				Verificare che l’applicazione MaaP permetta all’admin di aggiungere gli indici in base ai suggerimenti forniti. & N.E & RA1F 15.2*  \newline  \\ \hline 
				TS-RA1F 15.3 & 
				Verificare che l’applicazione MaaP permetta all’admin di rimuovere gli indici in base ai suggerimenti forniti. & N.E & RA1F 15.3*  \newline  \\ \hline 
				TS-RS1F 17 & 
				Verificare che Il sistema MaaS si accerti che documenti creati rispettano i vincoli del database. & N.E & RS1F 17*  \newline  \\ \hline 
				TS-RA1O 18 & 
				Verificare che il sistema metta a disposizione un validatore del codice DSL e visualizzi gli eventuali errori logici o di sintassi in un'apposita pagina. & N.E & RA1O 18 \newline  \\ \hline 
				TS-RS1F 19 & 
				Verificare che il sistema MaaS salvi le pagine definite dagli utenti nel database e non su disco. & N.E & RS1F 19*  \newline  \\ \hline 
		\caption{Tracciamento Test di Sistema - Requisiti}
		\end{longtable}
	 \egroup
\end{center}
	
	\subsection{Test di integrazione}
	I test di integrazione vanno a controllare il corretto funzionamento delle componenti descritti dalla progettazione ad alto livello. Si è scelto di utilizzare un approccio \glossario{top-down} ad eccezione del test TI 10 che viene eseguito con la metodologia \glossario{bottom-up}. Di seguito viene riportato un diagramma informale per chiarire l'albero dei test di integrazione.

	\begin{figure}[H]
	\centering \includegraphics[width=1\textwidth]{sequenza-di-integrazione.png}
	\caption{Sequenza d'integrazione delle componenti}
	\label{fig:sequenza-di-integrazione}
	\end{figure}

	Con la tecnica \glossario{top-down} le componenti di più alto livello sono testate non appena sono implementate. Le componenti del sottosistema che non sono ancora state sviluppate, vengono simulate dagli \glossario{stub}. Man mano che si procede con la codifica delle componenti di più basso livello, queste vengono integrate e viene eseguito il relativo test. Grazie all'integrazione incrementale delle componenti del sistema, è più semplice determinare quale componente crea problemi e le funzioni di più alto livello sono testate prima.

	\begin{figure}[H]
	\centering \includegraphics[width=0.57\textwidth]{sequenza-dei-test.png}
	\caption{Diagramma di attività dei test}
	\label{fig:sequenza-dei-test}
	\end{figure}
	
	%TODO tabella: componente / test
	\bgroup
	\begin{longtable}[H]{|P{1cm}|P{5cm}|P{4.5cm}|P{2cm}|}
		\hline \textbf{Test} & \textbf{Descrizione} & \textbf{Componenti aggiunte} & \textbf{Stato} \\
		
		\hline TI 1 & Si verifica che l'applicazione Web carichi correttamente le librerie JavaScript utilizzate. & Front-end & N.E. \\
		\hline TI 2 & Si verifica che i controller si integrino correttamente nell'applicazione Web. & Front-end::Controller & N.E. \\
		\hline TI 3 & Si verifica che i services permettono di interagire correttamente con il back-end. & Front-end::Services & N.E. \\
		\hline TI 4 & Si verifica che il DeveloperProject avvii correttamente il server, fornendo in particolare i file statici del front-end. & Back-end::DeveloperProject & N.E. \\
		\hline TI 5 & Si verifica che la libreria si integri correttamente con il \glossario{Node Package Manager} (npm) E che il suo script di installazione produca un DeveloperProject funzionante. & Back-end::Lib & N.E. \\
		\hline TI 6 & Si verifica che il Middleware si integri correttamente nella gestione delle richieste che arrivano al server. & Back-end::Lib::Middleware & N.E. \\
		\hline TI 7 & Si verifica che i controller si integrino correttamente nella gestione delle richieste che arrivano al server. & Back-end::Lib::Controller & N.E. \\
		\hline TI 8 & Si verifica che l'AuthModel si integri correttamente con il Middleware della gestione dell'autenticazione. & Back-end::Lib::AuthModel & N.E. \\
		\hline TI 9 & Si verifica che la MailView si integri correttamente con il Middleware della gestione dell'invio mail. & Back-end::Lib::MailView & N.E. \\
		\hline TI 10 & Si verifica che le classi che compongono il DSLModel interagiscano correttamente tra loro. & Back-end::Lib::DSLModel & N.E. \\
		\hline TI 11 & Si verifica che il DSLModel si integri correttamente con il funzionamento dell'applicazione. & Back-end::Lib::DSLModel & N.E. \\
		\hline
	\caption{Descrizione test d'Integrazione}
	\end{longtable}
	\egroup

	\subsection{Test di validazione}
	In questa sezione vengono elencati i test di validazione per verificare che il prodotto sia conforme alle attese. I test si svolgono seguendo e verificano tutti passi di cui si compongono.
	
	
	\begin{center}
	\def\arraystretch{1.5}
	\bgroup
		\begin{longtable}{| p{3cm} | p{6cm} | p{1.5cm} | p{2cm} | }
		\hline 
		 \textbf{Test di Validazione} & \textbf{Descrizione} & \textbf{Stato} & \textbf{Requisito} \\ \hline
				TV-RA1O 1 & 
				L'utente non autenticato intende accedere all'applicazione, per farlo deve inserire le proprie credenziali composte da una email ed una password.
All'utente è richiesto di:
\begin{itemize}
\item Raggiungere la pagina di autenticazione;
\item Inserire la mail nel campo apposito;
\item Inserire la password;
\item Procedere con l'autenticazione.
\end{itemize}
 & E & RA1O 1\newline  \\ \hline 
				TV-RA1O 2 & 
				L'utente intende recuperare la password d'accesso all'applicazione.
All'utente è richiesto di:
\begin{itemize}
\item Essere autenticato;
\item Raggiungere la pagina per il reset della password;
\item Richiedere il reset;
\item Raggiungere la casella email collegata all'account del sistema;
\item Seguire il link contento nella mail:
\item Compilare il form richiedente la nuova password;
\item Eseguire il Logout e autenticarsi con la nuova password.
\end{itemize} & E & RA1O 2\newline  \\ \hline 
				TV-RA1D 3 & 
				L'utente autenticato può visualizzare la pagina di Dashboard nella quale potrà aver accesso ad esempio alla lista delle collection presenti e ad altre funzionalità disponibili.
All'utente è richiesto di:
\begin{itemize}
\item Accertarsi di essere autenticato;
\item Accedere alla pagina Dashboard tramite il menu di navigazione;
\end{itemize} & E & RA1D 3\newline  \\ \hline 
				TV-RA1O 4 & 
				L'utente autenticato, selezionata una \glossario{Collection}, ne visualizza in forma tabellare tutti i documenti che contiene. \newline Di questa collection può filtrarne i risultati visualizzabili, può eseguire tramite bottoni predisposti nella pagina azioni personalizzati e per ogni \glossario{Document}, selezionarlo e visualizzarne la show-page corrispondente. \newline L'Admin ha i permessi per modificare un documento o eliminare un \glossario{Document}. \newline
All'utente è richiesto di:
\begin{itemize}
\item Essere autenticato;
\item Aprire la show-page relativa ad un Document;
\item Usare i filtri per filtrare la Collection
\item Eseguire un azione personalizzata, laddove presente;
\item Se admin, modificare un Document;
\item Se admin, eliminare un Document;
\end{itemize} & E & RA1O 4\newline  \\ \hline 
				TV-RA1O 5 & 
				L'utente visualizza la pagina show-page corrispondente ad un \glossario{Document} selezionato visualizzandone gli attributi in forma tabellare. \newline In questa pagina può aprire la show-page o l'index-page dell'array di \glossario{Document} degli attributi innestati se presenti, eseguire un'operazione personalizzata se disponibile. \newline L'Admin può eliminare il \glossario{Document} a cui la show-page corrisponde o modificarlo. 
All'utente è richiesto di:
\begin{item}
\item Essere autenticato;
\item Aprire la show-page degli attributi innestati;
\item Aprire l'index-page dell'arra di Document;
\item Eseguire, se presente, un operazione personalizzata;
\item Se admin, modificare il Document;
\item Se admin, eliminare il Document.
\end{item} & E & RA1O 5\newline  \\ \hline 
				TV-RA1O 6 & 
				L'Admin entra nella sua pagina di amministrazione nella quale visualizza una \glossario{Collection-Index} di tutti gli utenti registrati al sistema.
All'utente è richiesto di:
\begin{itemize}
\item Essere autenticato come admin;
\item Accedere alla pagina di creazione nuovi utenti;
\item Creare un nuovo utente;
\item Accedere alla pagina degli utenti registrati al sistema;
\item Visualizzare la pagina Collection-Show di un utente;
\end{itemize}
 & E & RA1O 6 \newline  \\ \hline 
				TV-RF1O 8 & 
				Lo sviluppatore deve poter creare un nuovo progetto tramite linea di comando.
\newline
Allo sviluppatore è richiesto di:
\begin{itemize}
\item Richiamare il comando di creazione di un nuovo progetto;
\item Passare come parametro il nome della directory che conterrà il progetto;
\item Verificare che siano state importate le librerie necessarie al corretto funzionamento del sistema;
\item Verificare che sia stato creato il file di configurazione di default dell’applicazione generata;
\item Verificare che sia stato creato il sistema di autenticazione per l’applicazione generata;
\item Verificare che siano state create le directory di descrizione delle pagine web;
\item Verificare che sia stato creato un account admin di default.
\end{itemize} & E & RF1O 8 \newline  \\ \hline 
				TV-RF1O 9 & 
				Lo sviluppatore deve poter configurare le Collection tramite il DSL di Maap Framework.
All'utente è richiesto di:
\begin{itemize}
\item creare una Collection-index tramite DSL;
\item creare una Collection-show tramite DSL;
\item modificare il nome della Collection;
\item modificare l'ordine di visualizzazione della Collection.
\end{itemize} & E & RF1O 9 \newline  \\ \hline 
				TV-RS1F 10 & 
				L'utente autenticato verifica che il \glossario{framework} MaaP sia messo a disposizione dal sistema \glossario{MaaS} come servizio Web.
\newline
All'utente è richiesto di:
\begin{itemize}
\item Accedere alla pagina di modifica del proprio profilo;
\item Modificare i dati associati al proprio profilo;
\item Verificare che i dati siano stati aggiornati;
\item Gestire i file di configurazione;
\item Eliminare il proprio account;
\item Verificare l'inaccessibilità al servizio tramite l'autenticazione con le credenziali associate all'account eliminato.
\end{itemize} & N.E & RS1F 10* \newline  \\ \hline 
				TV-RA1D 11 & 
				L'utente non autenticato deve potersi registrare all'applicazione MaaP.
\newline
All'utente è richiesto di:
\begin{itemize}
\item Inserire la mail nell'apposito campo di testo;
\item Inserire la password nell'apposito campo di testo;
\item Verificare che l'account sia stato registrato tramite l'autenticazione all'applicazione.
\end{itemize} & E & RA1D 11\newline  \\ \hline 
				TV-RA1D 12 & 
				L'utente autenticato deve poter eseguire il logout dall'applicazione.
\newline
All'utente è richiesto di:
\begin{itemize}
\item Selezionare l'apposita opzione di logout;
\item Verificare di non essere più autenticato.
\end{itemize} & E & RA1D 12\newline  \\ \hline 
				TV-RA1D 13 & 
				L'utente autenticato deve poter modificare le proprie credenziali d'accesso all'interno della propria pagina profilo.
\newline
All'utente viene richiesto di:
\begin{itemize}
\item Accedere alla propria pagina profilo;
\item Modificare la propria mail;
\item Modificare la propria password;
\item Eseguire il logout;
\item Autenticarsi con le nuove credenziali.
\end{itemize} & E & RA1D 13\newline  \\ \hline 
				TV-RF1O 14 & 
				Lo sviluppatore deve poter configurare i database che compongono il sistema MaaP.
\newline
Allo sviluppatore è richiesto di:
\begin{itemize}
\item Configurare la connessione al database delle credenziali degli utenti;
\item Configurare il \glossario{namespace} corrispondente, se la funzione di \glossario{namespace} è abilitata;
\item Configurare la connessione al database delle \glossario{Collection};
\item Configurare il \glossario{namespace} corrispondente, se la funzione di \glossario{namespace} è abilitata;
\item Selezionare un \glossario{namespace} per il database da configurare, se la funzione di \glossario{namespace} è abilitata.
\end{itemize} & E & RF1O 14\newline  \\ \hline 
				TV-RA1F 15 & 
				L'admin deve poter gestire gli indici da un'apposita pagina.
\newline
All'admin è richiesto di:
\begin{itemize}
\item Accedere alla pagina di gestione degli indici;
\item Visualizzare i suggerimenti per la creazione degli indici;
\item Creare un indice;
\item Creare un indice da quelli suggeriti;
\item Eliminare un indice;
\item Eliminare un indice da quelli suggeriti.
\end{itemize} & N.E & RA1F 15* \newline  \\ \hline 
				TV-RF1F 16 & 
				Lo sviluppatore deve poter abilitare i \glossario{namespace} per l’applicazione creata.
\newline
Allo sviluppatore è richiesto di:
\begin{itemize}
\item Attivare il \glossario{namespace}.
\end{itemize} & N.E & RF1F 16* \newline  \\ \hline 
		\caption{Tracciamento Test di Validazione - Requisiti}
		\end{longtable}
	 \egroup
\end{center}
	
	
	
	

