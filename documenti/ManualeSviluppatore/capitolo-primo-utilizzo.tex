\section{Primo utilizzo}

Vengono riportati in questa sezione i passi necessari per la creazione di una nuova applicazione MaaP, utilizzando il modello \glossario{``scaffold''} fornito dal framework. I comandi descritti devono essere eseguiti da un terminale da cui sia possibile eseguire il Node Packet Manager ed è richiesta una connessione ad Internet.

\subsection{Installazione}

\begin{enumerate}
 	\item Installare il framework MaaP tramite il Node Packet Manager:
 	
 	\centerline{ \code{ \$> npm install -g maap } }
 	
 	\item Creare un'applicazione utilizzando il modello fornito dal framework MaaP: posizionarsi all'interno della cartella nella quale si vorrà creare l'applicazione ed eseguire il comando:
 	
 	\centerline{ \code{ \$> maap create project <ProjectName> } }
 
 	dove \texttt{<ProjectName>} corrisponde al nome del progetto da creare;
 	
 	\item Installare le dipendenze dell'applicazione: posizionarsi all'interno della cartella ed eseguire il comando di installazione
 
 	\centerline{ \code{ \$> cd <ProjectName> } }
 	\centerline{ \code{ \$> npm install } }
 	
\end{enumerate}

\subsection{Configurazione}

\begin{enumerate}

 	\item Configurare l'applicazione: aprire con un editor di testo il file \code{config.js} generato assieme all'applicazione e configurarlo riferendosi alla sezione \ref{config} di questo manuale.

 	\item Configurare le collections dell'applicazione, riferendosi alla sezione \ref{collection} di questo manuale. All'interno della cartella \code{collections} sono presenti dei file \glossario{DSL} di esempio;

\end{enumerate} 

\subsection{Avvio e accesso}

\begin{enumerate}
 	\item Eseguire il server dell'applicazione: posizionarsi all'interno della cartella dell'applicazione ed eseguire il comando:
 
 	\centerline{ \code{ \$> npm start } }
 	
 	\item Accedere all'applicazione da browser: aprire un browser (consigliati \textit{Chrome} o \textit{Firefox}) e accedere al server dalla pagina:
 	
 	\centerline{ \code{localhost:3000/} }
 	
 	Notare che al primo utilizzo le collections sono vuote, in quanto è necessario popolare il database di prova.
 
\end{enumerate}	

\subsection{Popolamento del database di prova}

\begin{enumerate}

	\item Posizionarsi all'interno della cartella \code{extra}:
	
	\centerline{ \code{\$> cd extra/} }
	
	\item Eseguire lo script di popolamento del database utenti di prova:

	\centerline{\code{\$> ./populate-users-db.sh --host localhost --port 27017 -db users}}
	
	Questo popolamento viene effettuato sul database impostato di default;
	
	\item Eseguire lo script di popolamento del database delle Collection di prova:

	\centerline{\code{\$> ./populate-data-db.sh --host localhost --port 27017 -db users}}
	
	Questo popolamento viene effettuato sul database impostato di default.

\end{enumerate}	