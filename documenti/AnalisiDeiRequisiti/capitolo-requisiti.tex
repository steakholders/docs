\section{Requisiti }

\subsection{Requisiti funzionali }

    %Tabella 
      \begin{center}
      \begin{longtable}{ | p{2cm} | p{2cm} | p{5cm} | p{2cm} |}
    
      \cellcolor[gray]{0.9} \textbf{Requisito} & \cellcolor[gray]{0.9} \textbf{Tipologia} 
      & \cellcolor[gray]{0.9} \textbf{Descrizione} & \cellcolor[gray]{0.9} \textbf{Fonti} \\ \hline
      
        RF1O 7 & Funzionale \newline  Obbligatorio  & Maap Framework deve rendere disponibile allo sviluppatore un linguaggio astratto DSL di tipo testuale necessario per la generazione delle pagine. &  Capitolato \newline  \\ \hline      
        RF1O 8  & Funzionale \newline  Obbligatorio  & Maap Framework deve permettere allo sviluppatore di generare un nuovo progetto tramite linea di comando.
 &  UCS3.3.1.9 \newline  UCS3.3.2 \newline  UCS2.1 \newline  UCS2 \newline  UCS3.3.1.6 \newline  \\ \hline      
        RF1O 8.1  & Funzionale \newline  Obbligatorio  & Maap Framework deve generare automaticamente lo scheletro dell'applicazione creata dallo sviluppatore.
 &  \\ \hline      
        RF1O 8.1.1  & Funzionale \newline  Obbligatorio  & Maap Framework deve automaticamente importare in un'apposita directory del progetto tutte le librerie necessarie al corretto funzionamento del sistema.
 &  \\ \hline      
        RF1O 8.1.2 & Funzionale \newline  Obbligatorio  & Maap Framework deve automaticamente creare in un'apposita directory il file di configurazione di default dell'applicazione generata.
 &  \\ \hline      
        RF1O 8.1.3  & Funzionale \newline  Obbligatorio  & Maap Framework deve automaticamente creare il sistema di autenticazione per l'applicazione generata. &  \\ \hline      
        RF1O 8.1.4  & Funzionale \newline  Obbligatorio  & Maap Framework deve automaticamente creare le directory di descrizione delle pagine web.
 &  \\ \hline      
        RF1O 8.2  & Funzionale \newline  Obbligatorio  & Maap Framework deve automaticamente creare un account admin di default. &  \\ \hline      
        RF1O 9  & Funzionale \newline  Obbligatorio  & Maap Framework deve permettere allo sviluppatore di definire delle pagine web da visualizzare sulla base del DSL fornito.
 &  \\ \hline      
        RF1O 9.1  & Funzionale \newline  Obbligatorio  & Il DSL deve permettere allo sviluppatore di creare una pagina di tipo Collection-index.
 &  \\ \hline      
        RF1O 9.1.1  & Funzionale \newline  Obbligatorio  & Il DSL deve permettere allo sviluppatore di poter definire una serie di attributi da visualizzare all'interno della pagina Collection-index.
 &  \\ \hline      
        RF1O 9.1.2  & Funzionale \newline  Obbligatorio  & Il DSL deve permettere allo sviluppatore di poter definire un ordinamento di default di visualizzazione dei Document all'interno della pagina Collection-index. &  \\ \hline      
        RF1O 9.1.3  & Funzionale \newline  Obbligatorio  & Il DSL deve permettere allo sviluppatore di poter definire un eventuale limite di elementi da visualizzare all'interno della pagina Collection-index.
 &  \\ \hline      
        RF1O 9.1.4  & Funzionale \newline  Obbligatorio  & Il DSL deve permettere allo sviluppatore di poter definire quali attributi sono ordinabili all'interno della pagina Collection-index.
 &  \\ \hline      
        RF1O 9.1.5 & Funzionale \newline  Obbligatorio  & Il DSL deve permettere allo sviluppatore di definire la funzione populate per far si che una chiave riferisca ad un documento esterno.
 &  \\ \hline      
        RF1O 9.1.6 & Funzionale \newline  Obbligatorio  & Il DSL deve permettere allo sviluppatore di definire delle query per creare la pagina Collection-index in base al risultato della loro estrazione.
 &  \\ \hline      
        RF1O 9.2 & Funzionale \newline  Obbligatorio  & Il DSL deve permettere allo sviluppatore di creare una pagina Collection-show. &  \\ \hline      
        RF1O 9.2.1  & Funzionale \newline  Obbligatorio  & Il DSL deve permettere allo sviluppatore di definire una serie di attributi visualizzabili all'interno della pagina Collection-show.
 &  \\ \hline      
        RF1O 9.2.2  & Funzionale \newline  Obbligatorio  & Il DSL deve permettere allo sviluppatore di visualizzare un attributo in base al popolamento di un Document esterno.
 &  \\ \hline      
        RS1F 10 & Funzionale \newline  Facoltativo  & Il sistema MaaS deve mettere a disposizione il framework MaaP come servizio web. &  \\ \hline      
        RS1F 10.1 & Funzionale \newline  Facoltativo  & Il sistema MaaS deve permettere allo sviluppatore di scrivere una Collection tramite editor di testo presente nella pagina web. &  \\ \hline      
        RS1F 10.2 & Funzionale \newline  Facoltativo  & Il sistema MaaS deve permettere all'utente di poter creare una Collection caricando un file prodotto dal framework MaaP. &  \\ \hline      
        RS1F 10.3 & Funzionale \newline  Facoltativo  & Il sistema MaaS deve permettere allo sviluppatore di usufruire delle pagine da lui generate direttamente dal servizio web offerto, tramite un collegamento costruito ad-hoc. &  \\ \hline      
        RA1O 1 & Funzionale \newline  Obbligatorio  & Il sistema permette all'utente non autenticato di autenticarsi tramite la visualizzazione di una pagina web, la quale conterrà al suo interno i campi di testo necessari. &  \\ \hline      
        RA1O 1.1 & Funzionale \newline  Obbligatorio  & Il sistema prevede l'inserimento dell'indirizzo email per la verifica delle credenziali in un apposito campo di testo. &  \\ \hline      
        RA1O 1.2 & Funzionale \newline  Obbligatorio  & Il sistema prevede l'inserimento di password per la verifica delle credenziali in un apposito campo di testo. &  \\ \hline      
        RA1O 1.3  & Funzionale \newline  Obbligatorio  & Il sistema, tramite un database indipendente, provvede a verificare l'autenticità  di un utente tramite la verifica di email e password. &  \\ \hline      
        RA1O 1.3.1 & Funzionale \newline  Obbligatorio  & Il sistema mette a disposizione la visualizzazione di una pagina di errore in caso di fallimento dell'autenticazione da parte dell'utente.
 &  \\ \hline      
        RA1O 1.3.2 & Funzionale \newline  Obbligatorio  & Il sistema, nel caso in cui l'autenticazione da parte dell'utente abbia avuto successo, reindirizza automaticamente l'utente sulla dashboard dell'applicazione.
 &  \\ \hline      
        RA1O 2  & Funzionale \newline  Obbligatorio  & Il sistema mette a disposizione dell'utente non autenticato la possibilità  di recuperare le proprie credenziali. &  \\ \hline      
        RA1O 3  & Funzionale \newline  Obbligatorio  & Il sistema deve mettere a disposizione dell'utente autenticato la visualizzazione della dashboard. &  \\ \hline      
        RA1O 4 & Funzionale \newline  Obbligatorio  & L'applicazione deve permettere all'utente la visualizzazione dell'insieme delle Collection presenti tramite un menu di navigazione e la selezione di una di esse. &  \\ \hline      
        RA1O 4.1  & Funzionale \newline  Obbligatorio  & L'applicazione deve permettere all'utente la visualizzazione di una Collection-index tramite una tabella le cui righe corrispondono ai Document presenti nel database e le colonne ai loro attributi visualizzabili.  &  \\ \hline      
        RA1O 4.1.1  & Funzionale \newline  Obbligatorio  & Ogni riga della tabella corrispondente ad un Document deve avere una chiave selezionabile che rimanda alla corrispondente pagina show. &  \\ \hline      
        RA1O 5  & Funzionale \newline  Obbligatorio  & L'applicazione deve permettere all'utente la visualizzazione della Collection-show relativa ad un Document tramite una pagina web che ne mostra gli attributi visualizzabili strutturati in una tabella.
 &  \\ \hline      
        RA1O 5.1 & Funzionale \newline  Obbligatorio  & L'applicazione deve permettere all'admin di poter editare ogni singolo attributo modificabile del documento dalla pagina show. &  \\ \hline      
        RA1O 5.1.1  & Funzionale \newline  Obbligatorio  & L'applicazione deve permettere all'admin di salvare le modifiche apportare agli attributi del documenti nella pagina show e salvare queste modifiche nel database.
 &  \\ \hline      
        RA1O 6 & Funzionale \newline  Obbligatorio  & L'applicazione deve fornire all'admin una pagina di amministrazione in cui visualizzare la Collection di tutti gli utenti registrati all'applicazione.
 &  \\ \hline      
        RA1O 6.1  & Funzionale \newline  Obbligatorio  & L'applicazione deve fornire all'admin la possibilità  di creare un nuovo utente dalla pagina di amministrazione. &  \\ \hline      
        RA1O 6.1.1  & Funzionale \newline  Obbligatorio  & L'applicazione deve fornire all'admin una pagina di creazione di un nuovo utente. &  \\ \hline      
        RA1O 6.1.1.1  & Funzionale \newline  Obbligatorio  & L'applicazione deve permettere all'admin di inserire l'indirizzo email del nuovo utente in un apposito campo di testo presente all'interno della pagina di creazione di un nuovo utente.
 &  \\ \hline      
        RA1O 6.1.1.2 & Funzionale \newline  Obbligatorio  & L'applicazione deve permettere all'admin di inserire la password del nuovo utente in un apposito campo di testo presente all'interno della pagina di creazione di un nuovo utente. &  \\ \hline      
        RA1O 6.1.1.3  & Funzionale \newline  Obbligatorio  & L'applicazione deve permettere all'admin di inserire il "livello utente" del nuovo utente tramite una combo-box presente all'interno della pagina di creazione di un nuovo utente. &  \\ \hline      
        RA1O 6.1.2  & Funzionale \newline  Obbligatorio  & L'applicazione deve prelevare tutti i dati inseriti dall'admin nella pagina di creazione di un nuovo utente ed inviarli al database, il quale provvederà  all'inserimento del nuovo record.
 &  \\ \hline      
        RA1O 6.1.3  & Funzionale \newline  Obbligatorio  & L'applicazione deve visualizzare un messaggio d'errore nel caso in cui l'admin non abbia compilato correttamente i campi presenti all'interno della pagina di creazione di un nuovo utente.
 &  \\ \hline      
        RA1O 6.2  & Funzionale \newline  Obbligatorio  & L'applicazione deve fornire all'admin la possibilità di selezionare un utente dalla index-page e visualizzare la sua relativa show-page.
 &  \\ \hline      
        RA1O 6.2.1 & Funzionale \newline  Obbligatorio  & L'applicazione deve fornire all'admin la possibilità di elevare l'utente normale selezionato al livello "admin" dalla show-page relativa. &  \\ \hline      
        RA1O 6.2.2 & Funzionale \newline  Obbligatorio  & L'applicazione deve fornire all'admin la possibilità di declassare l'admin selezionato a livello di utente normale dalla show-page relativa. &  \\ \hline      
        RA1O 6.2.3 & Funzionale \newline  Obbligatorio  & L'applicazione deve fornire all'admin la possibilità di modificare l'attributo email dell'utente selezionato dalla relativa show-page. &  \\ \hline      
        RA1O 6.2.4 & Funzionale \newline  Obbligatorio  & L'applicazione deve fornire all'admin la possibilità di modificare l'attributo password dell'utente selezionato dalla relativa show-page. &  \\ \hline      
        RA1D 4.1.2  & Funzionale \newline  Desiderabile  & L'utente può decidere di cambiare l'ordine di visualizzazione dei Document selezionando l'intestazione delle colonne aventi la proprietà  sortable.
 &  \\ \hline      
        RA1D 4.1.3  & Funzionale \newline  Desiderabile  & L'applicazione può mettere a disposizione dell'utente per ogni riga della tabella una checkbox per consentire la selezione multipla.
 &  \\ \hline      
        RA1D 4.1.4  & Funzionale \newline  Desiderabile  & L'applicazione può mettere a disposizione dell'admin un link di selezione rapida per l'eliminazione di un documento.
 &  \\ \hline      
        RA1D 4.2  & Funzionale \newline  Desiderabile  & L'applicazione da  eventualmente la possibilità  di impostare dei filtri personalizzati secondo determinati attributi per visualizzare un sottoinsieme di documenti. &  \\ \hline      
        RA1F 4.3 & Funzionale \newline  Facoltativo  & L'applicazione deve mettere a disposizione dell'amministratore un pulsante che lo reindirizzerà alla pagina di creazione di un nuovo Document.
 &  \\ \hline
      \end{longtable}
      \end{center}  
\clearpage

\subsection{Requisiti prestazionali }

    %Tabella 
      \begin{center}
      \begin{longtable}{ | p{2cm} | p{2cm} | p{5cm} | p{2cm} |}
    
      \cellcolor[gray]{0.9} \textbf{Requisito} & \cellcolor[gray]{0.9} \textbf{Tipologia} 
      & \cellcolor[gray]{0.9} \textbf{Descrizione} & \cellcolor[gray]{0.9} \textbf{Fonti} \\ \hline
      
        RA2O 1 & Prestazionale \newline  Obbligatorio  & L’applicazione deve funzionare con versioni 24.x o superiori di Firefox. &  \\ \hline      
        RA2O 2 & Prestazionale \newline  Obbligatorio  & L’applicazione deve funzionare con versioni 30.0.x o superiori di Chrome. &  \\ \hline
      \end{longtable}
      \end{center}  
\clearpage

\subsection{Requisiti di qualità }

    %Tabella 
      \begin{center}
      \begin{longtable}{ | p{2cm} | p{2cm} | p{5cm} | p{2cm} |}
    
      \cellcolor[gray]{0.9} \textbf{Requisito} & \cellcolor[gray]{0.9} \textbf{Tipologia} 
      & \cellcolor[gray]{0.9} \textbf{Descrizione} & \cellcolor[gray]{0.9} \textbf{Fonti} \\ \hline
      
        R3O 1 & Qualità \newline  Obbligatorio  & Devono essere prodotti e rilasciati manuali d'uso ed ogni altra documentazione tecnica necessaria per l’utilizzo del prodotto. &  \\ \hline      
        R3O 2 & Qualità \newline  Obbligatorio  & Per lo sviluppo del prodotto richiesto verranno rispettate tutte le norme descritte nel documento \NormeDiProgetto{}, le normative sulla progettazione e stesura del codice presenti nel  documento \PianoDiQualifica{} . &  \\ \hline
      \end{longtable}
      \end{center}  
\clearpage

\subsection{Requisiti di vincolo }

    %Tabella 
      \begin{center}
      \begin{longtable}{ | p{2cm} | p{2cm} | p{5cm} | p{2cm} |}
    
      \cellcolor[gray]{0.9} \textbf{Requisito} & \cellcolor[gray]{0.9} \textbf{Tipologia} 
      & \cellcolor[gray]{0.9} \textbf{Descrizione} & \cellcolor[gray]{0.9} \textbf{Fonti} \\ \hline
      
        R4O 1 & Vincolo \newline  Obbligatorio  & L’implementazione della componente server deve essere realizzata utilizzando Node.js. &  \\ \hline      
        R4O 2 & Vincolo \newline  Obbligatorio  & Deve essere utilizzato Express per la realizzazione dell’infrastruttura della web application. &  \\ \hline      
        RF4O 6 & Vincolo \newline  Obbligatorio  & Maap Framework deve essere pubblicato in una repository di GitHub e quest'ultima deve permettere l'utilizzo di Issues per la segnalazione di bug. &  \\ \hline      
        RS4F 6 & Vincolo \newline  Facoltativo  & Il sistema MaaS deve verificare se i documenti creati rispettano i vincoli del database. &  \\ \hline      
        RS4F 7 & Vincolo \newline  Facoltativo  & Il sistema MaaS deve salvare le Collection create dallo sviluppatore direttamente dentro la base di dati. &  \\ \hline      
        RA4O 3 & Vincolo \newline  Obbligatorio  & L’applicazione deve utilizzare Mongoose.js per l’interfacciamento con il database. &  \\ \hline      
        RA4O 4 & Vincolo \newline  Obbligatorio  & L’applicazione deve servirsi di  MongoDB come sistema gestionale dei dati. &  \\ \hline      
        RA4O 5 & Vincolo \newline  Obbligatorio  & Deve essere fatto il deployment su Heroku rendendo disponibile on-line l’applicazione Maap. &  \\ \hline
      \end{longtable}
      \end{center}  
\clearpage
