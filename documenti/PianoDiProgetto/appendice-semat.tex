\section{Report SEMAT}
Vengono qui riportati i report redatti dal \textit{Responsabile} di progetto che mette in relazione lo stato del gruppo con quelli identificati nel modello SEMAT.

\subsection{Progettazione architetturale}

	\subparagraph{Opportunity}
	Una soluzione che rispetti i vincoli è stata individuata, alcuni rischi previsti si sono verificati e sono stati gestiti con successo secondo quanto pianificato nell'analisi dei rischi.

	\subparagraph{\glossario{Stakeholders}}
	I dubbi nati durante lo sviluppo sono stati discussi con gli \glossario{stakeholder} raggiungendo una visione condivisa.

	\subparagraph{Requirements}
	I requisiti descritti costituiscono una soluzione accettabile per gli \glossario{stakeholder}. La frequenza di cambiamento dei requisiti è progressivamente diminuita con la definizione di essi.

	In ogni caso, non si esclude la possibilità che questi debbano essere rivisti in seguito ad una valutazione degli \glossario{stakeholder}.

	\subparagraph{Software System}
	L'architettura ad alto livello del prodotto software è stata individuata. Le tecnologie e gli strumenti necessari per lo sviluppo sono stati individuati. Sono stati scelti i componenti e gli strumenti riusabili, e sviluppati quelli per cui non è stato possibile individuarne di adeguati per il progetto.

	\subparagraph{Team}
	Sono state superate le poche difficoltà iniziali di relazione con i componenti del gruppo. Il team lavora più coordinato e collabora per il conseguimento degli obiettivi che ha definito.

	\subparagraph{Work}
	Il progresso del lavoro è monitorato grazie alle tecniche, agli strumenti utilizzati e alla frammentazione decisa dal \textit{Responsabile} di progetto. Il gruppo opera coerentemente con la suddivisione del lavoro.

	\subparagraph{Way of Working}
	Il gruppo procede con delle variazioni sulle scadenze pianificate. Le pratiche di lavoro stabilite recentemente non sono totalmente assimilate dal gruppo, mentre per quelle discusse in precedenza non si notano evidenti violazioni delle norme. L'utilizzo dei nuovi strumenti necessari per lo sviluppo del prodotto necessitano un periodo di addestramento dei componenti del gruppo.
	
	\newpage