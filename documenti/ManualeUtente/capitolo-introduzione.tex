\section{Introduzione}


	\subsection{Scopo del documento}
	Questo documento è rivolto all'utente, ha lo scopo di illustrare le procedure da seguire per svolgere le operazioni utente di \glossario{MaaP}. All'utilizzatore non è richiesta alcuna conoscenza informatica poiché dovrà interfacciarsi, tramite web browser, alle funzionalità di \glossario{MaaP} che vengono erogate con le stesse modalità di un normale sito internet.

	\subsection{Scopo del prodotto}
	\ScopoDelProdotto{}

	\subsection{Prerequisiti}
	L'utente deve possedere una connessione ad internet, un web browser (Chrome $\geq$ 30.0.x, Firefox $\geq$ 24.x), una piattaforma che fornisca \glossario{MaaP} come servizio raggiungibile tramite un \texttt{\glossario{url}} e le relative credenziali di accesso.

	\subsection{Come accedere al manuale}
	Il \glossario{footer} di ogni pagine di \glossario{MaaP} riporta un link denominato \emph{Need help?} che permette di accedere alla manualistica disponibile del prodotto.
	


	\subsection{Glossario}
	Ogni occorrenza di termini tecnici, di dominio e gli acronimi sono marcati con una ``G'' in pedice. Le relative definizioni sono collocate in appendice.




	% se necessario
	% \subsection{Riferimenti}
	% 	\subsubsection{Informativi}
	% 		\begin{itemize}
	%   			\item
	% 		\end{itemize}