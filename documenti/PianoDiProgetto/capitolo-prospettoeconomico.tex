\section{Prospetto economico}

In questa sezione è presentato il prospetto economico del progetto \ProjectName{}, suddiviso per fasi. Per ogni fase sono indicate le ore preventivate per ogni ruolo impiegato.
Il costo è calcolato utilizzando i dati della tabella al paragrafo \ref{tabellacostiruolo}.

\subsection{Analisi}

A scopo di trasparenza viene redatto il prospetto economico riguardante la fase di Analisi dei requisiti, ma si precisa che le ore spese in questa fase sono a carico del fornitore e non del proponente.

\begin{table}[H]
	\centering
	\begin{tabular}{ l c c }
	\textbf{Ruolo} & \textbf{Ore} & \textbf{Costo} \\
	\hline
	Amministratore & 30 & 600 €\\
	Analista & 58 & 870 €\\
	Progettista & 9 & 198 €\\
	Programmatore & 0 & 0 €\\
	Responsabile & 14 & 420 €\\
	Verificatore & 31 & 465 €\\
	\hline
	\textbf{Totale} & 142 & 2553 €\\
	\hline
	\end{tabular}
	\caption{Ore e costo per ruolo, fase di Analisi}
	\end{table}

I seguenti grafici mostrano il peso orario e di costo di ogni ruolo in questa fase.

\begin{figure}[H]
\centering
\includegraphics[scale=0.35]{5-1-1.png}
\caption{Ore per ruolo, fase di Analisi\label{fig:nome}}
\end{figure}

\begin{figure}[H]
\centering
\includegraphics[scale=0.35]{5-1-2.png}
\caption{Costo per ruolo, fase di Analisi\label{fig:nome}}
\end{figure}

\subsection{Progettazione architetturale}

Nella fase di Progettazione architetturale le ore per ogni ruolo sono state cosi suddivise:

\begin{table}[H]
	\centering
	\begin{tabular}{ l c c }
	\textbf{Ruolo} & \textbf{Ore} & \textbf{Costo} \\
	\hline
	
			Amministratore & 48 & 960 € \\
	Analista & 24 & 600 € \\
	Progettista & 108 & 2376 € \\
	Programmatore & 0 & 0 € \\
	Responsabile & 26 & 780 € \\
	Verificatore & 72 & 1080 € \\
\hline
	Totale & 278 & 5796 € \\
\hline

	
	\end{tabular}
	\caption{Ore e costo per ruolo, fase di Progettazione architetturale}
	\end{table}
	
I seguenti grafici mostrano il peso orario e di costo di ogni ruolo in questa fase.

\begin{tikzpicture}

	\pie[sum=auto, text=legend, color={amministratore, analista, progettista, programmatore, responsabile, verificatore}]{48/Amministratore, 24/Analista, 70/Progettista, ./Programmatore, 19.0/Responsabile, 42/Verificatore}


\end{tikzpicture}

% \begin{figure}[H]
% \centering
% \includegraphics[scale=0.35]{5-2-1.png}
% \caption{Ore per ruolo, fase di Progettazione architetturale\label{fig:nome}}
% \end{figure}

\begin{figure}[H]
\centering
\includegraphics[scale=0.4]{5-2-2.png}
\caption{Costo per ruolo, fase di Progettazione architetturale\label{fig:nome}}
\end{figure}

\subsection{Progettazione di dettaglio e codifica}

Nella fase di Progettazione di dettaglio e codifica le ore per ogni ruolo sono state cosi suddivise:

\begin{table}[H]
	\centering
	\begin{tabular}{ l c c }
	\textbf{Ruolo} & \textbf{Ore} & \textbf{Costo} \\
	\hline
	
			Amministratore & 38 & 760 € \\
	Analista & 8 & 200 € \\
	Progettista & 120 & 2640 € \\
	Programmatore & 132 & 1980 € \\
	Responsabile & 0 & 0 € \\
	Verificatore & 90 & 1350 € \\
\hline
	Totale & 388 & 6930 € \\
\hline

	
	\end{tabular}
	\caption{Ore e costo per ruolo, fase di Progettazione di dettaglio e codifica}
	\end{table}

I seguenti grafici mostrano il peso orario e di costo di ogni ruolo in questa fase.

\begin{tikzpicture}

	\pie[text=legend, color={amministratore, analista, progettista, programmatore, responsabile, verificatore}]{10.3/Amministratore, 2.9/Analista, 25.0/Progettista, 41.2/Programmatore, ./Responsabile, 20.6/Verificatore}


\end{tikzpicture}

% \begin{figure}[H]
% \centering
% \includegraphics[scale=0.35]{5-3-1.png}
% \caption{Ore per ruolo, fase di Progettazione di dettaglio e codifica\label{fig:nome}}
% \end{figure}

\begin{figure}[H]
\centering
\includegraphics[scale=0.35]{5-3-2.png}
\caption{Costo per ruolo, fase di Progettazione di dettaglio e codifica\label{fig:nome}}
\end{figure}

\subsection{Validazione}

Nella fase di Validazione le ore per ogni ruolo sono state cosi suddivise:

\begin{table}[H]
	\centering
	\begin{tabular}{ l c c }
	\textbf{Ruolo} & \textbf{Ore} & \textbf{Costo} \\
	
			Amministratore & 16 & 320 € \\
	Analista & 0 & 0 € \\
	Progettista & 8 & 176 € \\
	Programmatore & 0 & 0 € \\
	Responsabile & 8 & 240 € \\
	Verificatore & 84 & 1260 € \\
\hline
	Totale & 116 & 1996 € \\
\hline

	
	\end{tabular}
	\caption{Ore e costo per ruolo, fase di Validazione}
	\end{table}

I seguenti grafici mostrano il peso orario e di costo di ogni ruolo in questa fase.

\begin{tikzpicture}

	\pie[text=legend, color={amministratore, analista, progettista, programmatore, responsabile, verificatore}]{11.6/Amministratore, ./Analista, 5.8/Progettista, ./Programmatore, 5.8/Responsabile, 76.8/Verificatore}


\end{tikzpicture}

% \begin{figure}[H]
% \centering
% \includegraphics[scale=0.35]{5-4-1.png}
% \caption{Ore per ruolo, fase di Validazione\label{fig:nome}}
% \end{figure}

\begin{figure}[H]
\centering
\includegraphics[scale=0.35]{5-4-2.png}
\caption{Costo per ruolo, fase di Validazione\label{fig:nome}}
\end{figure}

\subsection{Totale}

In totale le ore per ogni ruolo sono state cosi suddivise:

\begin{table}[H]
	\centering
	\begin{tabular}{ l c c }
	\textbf{Ruolo} & \textbf{Ore} & \textbf{Costo} \\
	\hline
	
			Amministratore & 95 & 1900 € \\
	Analista & 32 & 800 € \\
	Progettista & 190 & 4180 € \\
	Programmatore & 126 & 1890 € \\
	Responsabile & 34 & 1020 € \\
	Verificatore & 229 & 3435 € \\
\hline
	Totale & 706 & 13225 € \\
\hline

	
	\end{tabular}
	\caption{Ore e costo per ruolo, riassunto progetto}
	\end{table}

I seguenti grafici mostrano il peso orario e di costo di ogni ruolo durante tutto lo svolgimento del progetto, esclusa la fase di Analisi dei requisiti.

\begin{figure}[H]
\centering
\includegraphics[scale=0.35]{5-5-1.png}
\caption{Ore per ruolo\label{fig:nome}}
\end{figure}

\begin{figure}[H]
\centering
\includegraphics[scale=0.4]{5-5-2.png}
\caption{Costo per ruolo\label{fig:nome}}
\end{figure}
