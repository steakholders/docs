\clearpage
\section{Altri capitolati}
Nella scelta tra i capitolati proposti, sono stati trovati per ciascuno di essi dei punti positivi e negativi che ne hanno determinato lo scarto.

\subsection{C2 - RING}
        \subsubsection{Valutazione generale}
        Trattandosi di un progetto proposto dall'Università stessa, il gruppo ha valutato positivamente i fini legati al mondo della ricerca: è stata considerata la visibilità del prodotto legato al research group \emph{BioComputing UP} ed alla eventuale citazione del gruppo in pubblicazioni internazionali dell'ambito. Tuttavia l'ambito della bioinformatica non coincide con gli interessi personali del gruppo nei cui confronti non ha manifestato particolare interesse. Il discriminante a sfavore di \emph{C2} è la considerazione sull'utilizzo di tecnologie quali  \glossario{Java} e  \glossario{C++} già, in parte, note ai membri del gruppo. La scelta di questo capitolato non rappresenterebbe quindi una grossa sfida nell'apprendimento di metodologie radicalmente differenti rispetto a quelle apprese in ambito accademico.
        %si possono aggiungere considerazioni sul tipo di Licenza Creative Common vedi sezione "Destinazione" in C2
        %\subsubsection{Potenziali criticità}
        
\subsection{C3 - Romeo}
        \subsubsection{Valutazione generale}
        L'innegabile utilità e i suoi possibili risvolti, seppur minimi, nel miglioramento dei servizi di cura al cittadino, rappresentano una forte spinta motivazionale nei confronti di \emph{C3}. Il proponente, in sede di presentazione, non ha chiarito alcune zone d'ombra relative al capitolato; la parte algoritmica viene sì presentata come disponibile ma non è stata fornita una spiegazione esaustiva. In particolare è stata considerata la necessità di capire come integrare gli algoritmi disponibili con quanto previsto dal capitolato, il gruppo ritiene che la presa in carico di tale operazione non costituisce un grosso ostacolo. La richiesta obbligatoria di una  \glossario{GUI} user-friendly presuppone uno studio lungo e approfondito di come generare un flusso di navigazione intuitivo all'interno dell'interfaccia utente. All'interno del gruppo i pareri sul valore di tale lavoro sono stati discordanti, pertanto si ritiene che una possibile mancanza di iniziativa porterebbe ad un mancato contributo decisivo da parte di tutti i membri. La  \glossario{GUI} non va quindi ad incrementare l'attenzione complessiva del gruppo nei confronti di \emph{C3}. Inoltre si ritiene che i linguaggi da usare non siano sufficientemente diversi da quelli appresi nei corsi di studi universitari.
        \subsubsection{Potenziali criticità}
        Viene richiesta una forte flessibilità del software prodotto e il gruppo ritiene che tale aspetto potrebbe rappresentare un impedimento per il progetto, principalmente perché non si reputa di avere sufficiente esperienza e capacità progettuale nel creare sistemi così modulari. La realizzazione di una  \glossario{GUI}  \glossario{user-friendly} potrebbe richiedere molto tempo e inoltre l'aspetto  \glossario{stand-alone} e non minimamente distribuito del software richiesto costituisce motivo di perplessità all'interno del gruppo.
        
\subsection{C4 - Seq}
        \subsubsection{Valutazione generale}
        Il desiderio di concretezza è uno fattore che accomuna ogni membro del gruppo. Pur ritenendo \emph{C4} una buona base su cui sviluppare a lungo termine dei progetti concreti si è deciso di non scegliere questo capitolato principalmente perché viene ritenuto troppo generico.
        Le tecnologie da utilizzare, in particolare  \glossario{HTML/CSS/JS} costituiscono fonte di interesse da parte del gruppo ma nel complesso si ritiene che per la natura del progetto non vi sia la possibilità di utilizzarle a fondo.
        %\subsubsection{Potenziali criticità}
        
\subsection{C5 - SGAD}
        \subsubsection{Valutazione generale}
        Gli aspetti principali considerati sono due:
        \begin{itemize}
            \item \textbf{ \glossario{Scalabilità}:} i singoli componenti del gruppo la valutano in modo molto eterogeneo; nessuno ha competenze in materia di  \glossario{linguaggi funzionali} come  \glossario{Scala} e questo indubbiamente richiede un buon investimento di tempo;
            \item \textbf{Test/simulazione:} alla presentazione è stato chiarito l'aspetto di testing del software che si realizzerebbe inizialmente in ambiente  \glossario{LAN} per poi spostarsi verso \glossario{Amazon AWS} per simulare una situazione di utilizzo reale.
        \end{itemize}
        Il gruppo ritiene che le suddette considerazioni necessitino di troppo tempo in attività di \glossario{autoformazione}, inoltre i test richiederebbero una diretta collaborazione con il proponente  \glossario{FunGoStudios} e relative trasferte nella sede della suddetta \glossario{startup}.
        %\subsubsection{Potenziali criticità}