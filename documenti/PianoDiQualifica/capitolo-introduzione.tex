\section{Introduzione}
 L'obiettivo primario è la \glossario{qualità} del prodotto e dei suoi processi, ottenibile tramite una serie di controlli proattivi stabiliti al tempo zero. L'assenza di tali verifiche abbinata ad un team di più soggetti senza particolari accortezze e competenze, portano al progressivo deterioramento del materiale prodotto, sia esso codice sorgente o documentazione. Questo fenomeno è noto sotto il nome di \glossario{Broken windows theory} ed è intrinseco alla componente sociale dell'uomo. Il concetto chiave è prevenire l'inserimento di materiale non aderente alle \NormeDiProgetto poiché, secondo la teoria succitata, innescherebbe un meccanismo che deteriora la qualità all'interno del \glossario{repository}.

Si vogliono gestire le componenti \glossario{accidentali} dei processi, ossia tutte quelle problematiche non intrinseche alla produzione, ma che ne sono direttamente collegate; si desidera scongiurare il pericolo di operare \glossario{by correction} per evitare modifiche in corso d'opera che posso bloccare la  maturazione del prodotto e richiedere dispendiose correzioni.

\subsection{Scopo del documento}
Il \DocTitle  illustra la strategia di verifica e validazione che il gruppo \GroupName ha deciso di adottare. È necessario dare una dimensione alla qualità dei prodotti e dei processi, operazioni che non rientrano nei normali ruoli di progetto, bensì rappresentano una \emph{funzione aziendale}. Vi sono molteplici punti di vista della qualità, il \glossario{committente} sarà in grado di valutare su basi oggettive quanto prodotto. Inoltre, la direzione del progetto potrà fare affidamento sulla consistenza dello stato dello stesso e il proponente avrà una solida base di verifica ideata e funzionante.

Questo documento si colloca nella parte relativa al \emph{Project} delle \emph{quattro P del Software Engineering} (People, Product, Project, Process) perché tratta una parte fondamentale sulle attività a supporto della produzione di \ProjectName.

\subsection{Scopo del prodotto}
Lo scopo del progetto è la realizzazione di un \glossario{framework} per generare interfacce web di amministrazione dei dati di \glossario{business} basato su \glossario{stack} \glossario{Node.js} e \glossario{MondoDB}. L'obbiettivo è quello di semplificare il processo di implementazione di tali interfacce che lo sviluppatore, appoggiandosi alla produttività del framework MaaP, potrà generare in maniera semplice e veloce ottenendo quindi un considerevole risparmio di tempo e di sforzo. Il fruitore finale delle pagine generate sarà l'esperto di business che potrà visualizzare, gestire e modificare le varie entità e dati residenti in \glossario{MongoDB}.
Il prodotto atteso si chiama \glossario{MaaP} ossia \emph{MongoDB as an admin Platform}. 

\subsection{Glossario}
Ogni occorrenza di termini tecnici, di dominio e gli acronimi sono marcati con una "G" in pedice e riportati nel documento \Glossario{}.

\subsection{Riferimenti}
Vengono elencanti i riferimenti su cui si è basata l'organizzazione dell'attività di qualifica.
	\subsubsection{Normativi}
		\begin{itemize}
  			\item \textbf{Norme di Progetto:}  \emph{Norme di Progetto};
			\item \textbf{Capitolato d'appalto C1:} \url{http://www.math.unipd.it/~tullio/IS-1/2013/Progetto/C1.pdf}.
		\end{itemize}
	\subsubsection{Informativi}
		\begin{itemize}
  			\item \textbf{Piano di Progetto:} \emph{Piano di Progetto.pdf};
  			\item \textbf{Slide di Ingegneria del Software mod. A:} \url{http://www.math.unipd.it/~tullio/IS-1/2013/};
  			\item \textbf{SWEBOK 2004 Version - capitolo 11:} \url{http://www.computer.org/portal/web/swebok/htmlformat}; 
  			\item \textbf{Software Engineering 9th - I. Sommerville (Pearson, 2011) - capitoli: 8, 24, 26:} \url{http://www.pearsoned.co.uk/bookshop/detail.asp?item=100000000377819};
  			\item \textbf{Software Engineering Method and Theory (SEMAT)} \url{http://www.ivarjacobson.com/semat/}.
		\end{itemize}
		

