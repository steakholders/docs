\section{Gestione amministrativa della revisione}
% qui bisognerebbe prendere ampiamente spunto Sommerville e SWEBOK, viene spiegate le tipologie di anomalie
	\subsection{Comunicazione e risoluzione anomalie}
	Un anomalia corrisponde a:
	\begin{itemize}
		\item Violazione delle norme tipografiche da parte di un documento.
		\item Uscita dal range di accettazione degli indici di misurazione. %descritti in ...
		\item Incongruenza del prodotto con funzionalità indicate nell'analisi dei requisiti.
		\item Incongruenza del codice con il design del prodotto.
	\end{itemize}
	Le anomalie riscontrate dai \emph{Verificatori} vanno segnalate tramite ticket nelle modalità specificate nelle \emph{Norme di Progetto}.
	%\subsection{Trattamento delle discrepanze}
	%\subsection{Procedure di controllo qualità di processo}
	
	
