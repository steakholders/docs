\section{Requisiti }
I requisiti funzionali, prestazionali, di qualità e di vincolo individuati sono riportati nelle seguenti tabelle. Ogni requisito è identificato da un codice univoco.
Viene inoltre indicato se si tratta di un requisito fondamentale, desiderabile o facoltativo, una sua descrizione e il caso d'uso da cui è stato individuato. 

Ogni requisito è identificato da un codice, che segue il seguente formalismo:
\begin{center}
    \code{R\{X\}\{Y\}\{Z\} \{Gerarchia\}}
\end{center}

Dove:
\begin{itemize}
 \item \textbf{X} corrisponde al sistema di riferimento e può assumere i seguenti valori:
    \begin{itemize}
     \item[] \textbf{A} = Applicazione \glossario{Maap};
     \item[] \textbf{F} = \glossario{Framework MaaP};
     \item[] \textbf{S} = \glossario{MaaS}.
    \end{itemize}

 \item \textbf{Y} corrisponde alla tipologia del requisito e può assumere i seguenti valori:
    \begin{itemize}
     \item[] \textbf{1} = Funzionale;
     \item[] \textbf{2} = Prestazionale;
     \item[] \textbf{3} = Di Qualità;
     \item[] \textbf{4} = Vincolo.
    \end{itemize}

 \item \textbf{Z} corrisponde alla priorità del requisito e può assumere i seguenti valori:
    \begin{itemize}
     \item[] \textbf{O} = Obbligatorio
     \item[] \textbf{D} = Desiderabile
     \item[] \textbf{F} = Facoltativo o Opzionale
    \end{itemize}

 \item \textbf{Gerarchia} identifica la relazione gerarchica che c'è tra i requisiti di uno stesso tipo. C'è quindi una struttura gerarchica per ogni tipologia di requisito.
\end{itemize}

\subsection{Requisiti funzionali }

    %Tabella 
      \begin{center}
      \bgroup
      \def\arraystretch{1.8}
      \begin{longtable}{ | l | p{2cm} | p{5cm} | p{1.7cm} |}
    
      \cellcolor[gray]{0.9} \textbf{Requisito} & \cellcolor[gray]{0.9} \textbf{Tipologia} 
      & \cellcolor[gray]{0.9} \textbf{Descrizione} & \cellcolor[gray]{0.9} \textbf{Fonti} \\ \hline
      
        RA1O 1 & Funzionale \newline  Obbligatorio  & Il sistema permette all'utente non autenticato di autenticarsi tramite la visualizzazione di una pagina web, la quale conterrà al suo interno i campi di testo necessari. &  Capitolato \newline  UCU 1 \newline  \\ \hline      
        RA1O 1.1 & Funzionale \newline  Obbligatorio  & Il sistema prevede l'inserimento dell'indirizzo email per la verifica delle credenziali in un apposito campo di testo. &  Capitolato \newline  UCU 1 \newline  UCU 1.1 \newline  \\ \hline      
        RA1O 1.2 & Funzionale \newline  Obbligatorio  & Il sistema prevede l'inserimento di password per la verifica delle credenziali in un apposito campo di testo. &  Capitolato \newline  UCU 1 \newline  UCU 1.2 \newline  \\ \hline      
        RA1O 1.3 & Funzionale \newline  Obbligatorio  & Il sistema, tramite un database indipendente, ovvero separato da quello che contiene la Collection, provvede a verificare l'autenticità  di un utente tramite la verifica di email e password. &  Capitolato \newline  \\ \hline      
        RA1O 1.3.1 & Funzionale \newline  Obbligatorio  & Il sistema mette a disposizione la visualizzazione di una pagina di errore in caso di fallimento dell'autenticazione da parte dell'utente. &  Interno \newline  UCU 2 \newline  \\ \hline      
        RA1O 1.3.2 & Funzionale \newline  Obbligatorio  & Il sistema, nel caso in cui l'autenticazione da parte dell'utente abbia avuto successo, reindirizza automaticamente l'utente sulla dashboard dell'applicazione.
 &  Interno \newline  UCU 8 \newline  \\ \hline      
        RA1O 2 & Funzionale \newline  Obbligatorio  & Il sistema mette a disposizione dell'utente non autenticato la possibilità  di recuperare la propria password. &  Capitolato \newline  UCU 4 \newline  UCU 4.1 \newline  UCU 4.1.1 \newline  UCU 4.2 \newline  UCU 4.2.1 \newline  \\ \hline      
        RA1O 2.1 & Funzionale \newline  Obbligatorio  & Il sistema permette il recupero password attraverso l'inserimento dell'email. &  Interno \newline  UCU 4 \newline  UCU 4.1 \newline  UCU 4.1.1 \newline  \\ \hline      
        RA1O 2.2 & Funzionale \newline  Obbligatorio  & Il sistema deve inviare un'email contenente un link attraverso il quale l'utente non autenticato può effettuare il reset della propria password. &  Interno \newline  UCU 4 \newline  UCU 4.2 \newline  \\ \hline      
        RA1O 2.3 & Funzionale \newline  Obbligatorio  & L'applicazione \glossario{MaaP} deve permettere il reset della password ad un utente non autenticato che abbia fatto la richiesta attraverso il sistema di recupero password. L'utente deve così poter inserire una nuova password che sostituirà la precedente per l'accesso all'applicazione. &  Interno \newline  UCU 4 \newline  UCU 4.2 \newline  UCU 4.2.1 \newline  \\ \hline      
        RA1D 3 & Funzionale \newline  Desiderabile  & Il sistema deve mettere a disposizione dell'utente autenticato la visualizzazione della dashboard. &  Interno \newline  UCU 8 \newline  \\ \hline      
        RA1O 4 & Funzionale \newline  Obbligatorio  & L'applicazione deve permettere all'utente la visualizzazione dell'insieme delle Collection presenti tramite un menu di navigazione e la selezione di una di esse. &  Capitolato \newline  UCU 9 \newline  \\ \hline      
        RA1O 4.1 & Funzionale \newline  Obbligatorio  & L'applicazione deve permettere all'utente la visualizzazione di una Collection-index tramite una tabella le cui righe corrispondono ai document presenti nel database e le colonne ai loro attributi visualizzabili. &  Capitolato \newline  UCU 9 \newline  \\ \hline      
        RA1O 4.1.1 & Funzionale \newline  Obbligatorio  & Ogni riga della tabella corrispondente ad un Document deve avere una chiave selezionabile che rimanda alla corrispondente pagina show. &  Capitolato \newline  UCS 3.3.1.4 \newline  \\ \hline      
        RA1D 4.1.2 & Funzionale \newline  Desiderabile  & L’applicazione può mettere a disposizione dell’admin un link di selezione rapida per l’eliminazione di un documento. &  Interno \newline  UCU 9 \newline  UCU 9.5 \newline  \\ \hline      
        RA1D 4.1.3 & Funzionale \newline  Desiderabile  & L’applicazione deve permette all’admin la modifica di un document presente nella collection-index. &  Interno \newline  UCU 9.4 \newline  \\ \hline      
        RA1D 4.2 & Funzionale \newline  Desiderabile  & L’applicazione dà la possibilità di impostare dei filtri personalizzati secondo determinati attributi per visualizzare un sottoinsieme di Document. &  Interno \newline  UCU 9.2 \newline  UCS 3.3.1.7 \newline  \\ \hline      
        RA1F 4.3 & Funzionale \newline  Facoltativo  & L’applicazione deve permettere all’amministratore di creare un nuovo Document all’interno della base di dati. &  Capitolato \newline  UCU 9.6 \newline  \\ \hline      
        RA1F 4.4 & Funzionale \newline  Facoltativo  & L’applicazione deve permettere all’utente di poter eseguire un’azione personalizzata tramite l’esecuzione di un pulsante. &  Capitolato \newline  UCU 9.3 \newline  \\ \hline      
        RA1O 5 & Funzionale \newline  Obbligatorio  & L'applicazione deve permettere all'utente la visualizzazione della Collection-show relativa ad un document tramite una pagina web che ne mostra gli attributi visualizzabili strutturati in una tabella. &  Capitolato \newline  UCU 9.1 \newline  UCU 9.1.1 \newline  UCU 9.1.2 \newline  \\ \hline      
        RA1O 5.1 & Funzionale \newline  Obbligatorio  & L'applicazione deve permettere all'admin di poter editare ogni singolo attributo modificabile del documento dalla pagina show. &  Capitolato \newline  UCU 9.7 \newline  UCU 9.1 \newline  UCU 9.1.3 \newline  UCU 9.1.5 \newline  \\ \hline      
        RA1F 5.2 & Funzionale \newline  Facoltativo  & L’applicazione deve permettere all’utente di poter eseguire un’azione personalizzata tramite l’esecuzione di un pulsante. &  Capitolato \newline  UCU 9.3 \newline  \\ \hline      
        RA1O 5.3 & Funzionale \newline  Obbligatorio  & L'applicazione deve permettere all'utente di poter eliminare dalla show-page il Document selezionato.
 &  Capitolato \newline  UCU 9.1.4 \newline  \\ \hline      
        RA1O 6 & Funzionale \newline  Obbligatorio  & L'applicazione deve fornire all'admin una pagina di amministrazione in cui visualizzare la Collection di tutti gli utenti registrati all'applicazione. &  Interno \newline  UCU 11 \newline  \\ \hline      
        RA1O 6.1 & Funzionale \newline  Obbligatorio  & L'applicazione deve fornire all'admin la possibilità  di creare un nuovo utente dalla pagina di amministrazione. &  Capitolato \newline  UCU 11.1 \newline  \\ \hline      
        RA1O 6.1.1 & Funzionale \newline  Obbligatorio  & L'applicazione deve fornire all'admin una pagina di creazione di un nuovo utente. &  Capitolato \newline  UCU 11.1 \newline  \\ \hline      
        RA1O 6.1.1.1 & Funzionale \newline  Obbligatorio  & L'applicazione deve permettere all'admin di inserire l'indirizzo email del nuovo utente in un apposito campo di testo presente all'interno della pagina di creazione di un nuovo utente.
 &  Interno \newline  UCU 11.1 \newline  UCU 11.1.1 \newline  \\ \hline      
        RA1O 6.1.1.2 & Funzionale \newline  Obbligatorio  & L'applicazione deve permettere all'admin di inserire la password del nuovo utente in un apposito campo di testo presente all'interno della pagina di creazione di un nuovo utente. &  Interno \newline  UCU 11.1 \newline  UCU 11.1.2 \newline  \\ \hline      
        RA1O 6.1.1.3 & Funzionale \newline  Obbligatorio  & L'applicazione deve permettere all'admin di inserire il "livello utente" del nuovo utente tramite una combo-box presente all'interno della pagina di creazione di un nuovo utente. &  Capitolato \newline  UCU 11.1 \newline  UCU 11.1.3 \newline  \\ \hline      
        RA1O 6.1.2 & Funzionale \newline  Obbligatorio  & L'applicazione deve prelevare tutti i dati inseriti dall'admin nella pagina di creazione di un nuovo utente ed inviarli al database delle credenziali, il quale provvederà  all'inserimento del nuovo record. &  Interno \newline  UCU 11.1 \newline  \\ \hline      
        RA1O 6.1.3 & Funzionale \newline  Obbligatorio  & L'applicazione deve visualizzare un messaggio d'errore nel caso in cui l'admin non abbia compilato correttamente i campi presenti all'interno della pagina di creazione di un nuovo utente. &  Interno \newline  UCU 11.2 \newline  \\ \hline      
        RA1O 6.2 & Funzionale \newline  Obbligatorio  & L'applicazione deve fornire all'admin la possibilità di selezionare un utente dalla index-page e visualizzare la sua relativa show-page. &  Interno \newline  UCU 11 \newline  UCU 11.1 \newline  UCU 11.1.3 \newline  \\ \hline      
        RA1O 6.2.1 & Funzionale \newline  Obbligatorio  & L'applicazione deve fornire all'admin la possibilità di elevare l'utente normale selezionato al livello "admin" dalla show-page relativa. &  Capitolato \newline  UCU 11.3 \newline  UCU 11.3.1 \newline  \\ \hline      
        RA1O 6.2.2 & Funzionale \newline  Obbligatorio  & L'applicazione deve fornire all'admin la possibilità di declassare l'admin selezionato a livello di utente normale dalla show-page relativa. &  Capitolato \newline  UCU 11.3 \newline  UCU 11.3.2 \newline  \\ \hline      
        RA1O 6.2.3 & Funzionale \newline  Obbligatorio  & L'applicazione deve fornire all'admin la possibilità di modificare l'attributo email dell'utente selezionato dalla relativa show-page. &  Verbale-2013-12-18 \newline  UCU 11.3 \newline  \\ \hline      
        RA1O 6.2.4 & Funzionale \newline  Obbligatorio  & L'applicazione deve fornire all'admin la possibilità di modificare l'attributo password dell'utente selezionato dalla relativa show-page.
 &  Verbale-2013-12-18 \newline  UCU 11.3 \newline  \\ \hline      
        RA1O 6.2.5 & Funzionale \newline  Obbligatorio  & L'applicazione deve fornire all'admin la possibilità di eliminare l'utente visualizzato nella \glossario{show-page}. &  Interno \newline  UCU 11.3.3 \newline  \\ \hline      
        RF1O 7 & Funzionale \newline  Obbligatorio  & MaaP Framework deve rendere disponibile allo sviluppatore un linguaggio astratto DSL di tipo testuale necessario per la generazione delle pagine. &  Capitolato \newline  \\ \hline      
        RF1O 8 & Funzionale \newline  Obbligatorio  & Maap Framework deve permettere allo sviluppatore di generare un nuovo progetto tramite linea di comando. &  Capitolato \newline  UCS 1 \newline  \\ \hline      
        RF1O 8.1  & Funzionale \newline  Obbligatorio  & Maap Framework deve generare automaticamente lo scheletro dell’applicazione creata dallo sviluppatore. &  Capitolato \newline  UCS 1 \newline  \\ \hline      
        RF1O 8.1.1 & Funzionale \newline  Obbligatorio  & Maap Framework deve automaticamente importare in un'apposita directory del progetto tutte le librerie necessarie al corretto funzionamento del sistema. Librerie necessarie: \begin{itemize} \item Express v-3.4.8 \item MongoDB v-1.3.23 \item Mongoose v-3.8.4 \item Nel caso fossero necessarie ulteriori librerie è consigliata una versione uguale o maggiore rispetto a quella disponibile al 2014-01-01 \end{itemize} &  Capitolato \newline  UCS 1 \newline  \\ \hline      
        RF1O 8.1.2 & Funzionale \newline  Obbligatorio  & Maap Framework deve automaticamente creare in un’apposita directory il file di configurazione di default dell’applicazione generata. &  Capitolato \newline  UCS 1 \newline  \\ \hline      
        RF1O 8.1.3 & Funzionale \newline  Obbligatorio  & Maap Framework deve automaticamente creare il sistema di autenticazione per l’applicazione generata. &  Capitolato \newline  UCS 1 \newline  \\ \hline      
        RF1O 8.1.4 & Funzionale \newline  Obbligatorio  & Maap Framework deve automaticamente creare le directory di descrizione delle pagine web. &  Capitolato \newline  UCS 1 \newline  \\ \hline      
        RF1O 8.2 & Funzionale \newline  Obbligatorio  & Maap Framework deve automaticamente creare un account admin di default. &  Verbale-2013-12-05 \newline  UCS 1 \newline  \\ \hline      
        RF1F 8.3 & Funzionale \newline  Facoltativo  & Il framework MaaP deve permettere allo sviluppatore di definire un namespace per l’applicazione generata. &  Interno \newline  UCS 3.3.5 \newline  \\ \hline      
        RF1O 9 & Funzionale \newline  Obbligatorio  & Maap Framework deve permettere allo sviluppatore di configurare le Collection tramite DSL fornito. &  Capitolato \newline  UCS 3 \newline  \\ \hline      
        RF1O 9.1 & Funzionale \newline  Obbligatorio  & Il DSL deve permettere allo sviluppatore di creare una pagina Collection-index. &  Capitolato \newline  UCS 3 \newline  UCS 3.2 \newline  UCS 3.4 \newline  \\ \hline      
        RF1O 9.1.1 & Funzionale \newline  Obbligatorio  & Il DSL deve permettere allo sviluppatore di poter definire una serie di attributi da visualizzare all’interno della pagina Collection-index. &  Capitolato \newline  UCS 3.3 \newline  UCS 3.3.1.1 \newline  \\ \hline      
        RF1O 9.1.2 & Funzionale \newline  Obbligatorio  & Il DSL deve permettere allo sviluppatore di poter definire un ordinamento di default (ordine alfanumerico) di visualizzazione dei document all'interno della pagina Collection-index. &  Capitolato \newline  UCS 3.3 \newline  UCS 3.3.1.2 \newline  \\ \hline      
        RF1O 9.1.3 & Funzionale \newline  Obbligatorio  & Il DSL deve permettere allo sviluppatore di poter definire un eventuale limite di elementi da visualizzare all’interno della pagina Collection-index. &  Capitolato \newline  UCS 3.3 \newline  UCS 3.3.1.3 \newline  \\ \hline      
        RF1O 9.1.4 & Funzionale \newline  Obbligatorio  & Il DSL deve permettere allo sviluppatore di poter definire quali attributi sono ordinabili all’interno della pagina Collection-index. &  Capitolato \newline  UCS 3.3.1.2 \newline  \\ \hline      
        RF1O 9.1.5 & Funzionale \newline  Obbligatorio  & Il DSL deve permettere allo sviluppatore di definire la funzione populate per far si che una chiave riferisca ad un documento esterno. &  Capitolato \newline  UCS 3.3.1.6 \newline  UCS 3.3.2.3 \newline  \\ \hline      
        RF1O 9.1.6 & Funzionale \newline  Obbligatorio  & Il DSL deve permettere allo sviluppatore di definire delle query per creare la pagina Collection-index in base al risultato della loro estrazione. &  Capitolato \newline  UCS 3.3.1 \newline  UCS 3.3.1.5 \newline  \\ \hline      
        RF1O 9.1.7 & Funzionale \newline  Obbligatorio  & Il DSL deve permettere allo sviluppatore di definire delle trasformazioni sugli attributi da visualizzare. &  Capitolato \newline  UCS 3.3.1.8 \newline  \\ \hline      
        RF1O 9.2 & Funzionale \newline  Obbligatorio  & Il DSL deve permettere allo sviluppatore di creare una pagina Collection-show. &  Capitolato \newline  UCS 3.2 \newline  UCS 3.1 \newline  UCS 3.4 \newline  \\ \hline      
        RF1O 9.2.1 & Funzionale \newline  Obbligatorio  & Il DSL deve permettere allo sviluppatore di definire una serie di attributi visualizzabili all’interno della pagina Collection-show. &  Capitolato \newline  UCS 3.3.2 \newline  UCS 3.3.2.1 \newline  \\ \hline      
        RF1O 9.2.2 & Funzionale \newline  Obbligatorio  & Il DSL deve permettere allo sviluppatore definire gli attributi del Document come attributi innestati o array di Document tramite la funzione populate. &  Capitolato \newline  Verbale-2013-12-05 \newline  UCS 3.3.2.3 \newline  \\ \hline      
        RF1O 9.2.3 & Funzionale \newline  Obbligatorio  & Lo sviluppatore deve aver la possibilità di personalizzare la show page definendone l’ordinamento degli attributi. &  Capitolato \newline  UCS 3.3.2 \newline  UCS 3.3.2.2 \newline  \\ \hline      
        RF1O 9.2.4 & Funzionale \newline  Obbligatorio  & Lo sviluppatore deve poter definire trasformazioni agli attributi per poi visualizzarli nella show-page. &  Capitolato \newline  UCS 3.3.2 \newline  UCS 3.3.2.4 \newline  \\ \hline      
        RF1F 9.2.5 & Funzionale \newline  Facoltativo  & Lo sviluppatore deve poter personalizzare la show-page definendo delle operazioni personalizzate che l’utente potrà utilizzare tramite appositi pulsanti. &  Capitolato \newline  UCS 3.3.1.9 \newline  UCS 3.3.2 \newline  UCS 3.3.2.5 \newline  \\ \hline      
        RF1O 9.3 & Funzionale \newline  Obbligatorio  & Il framework MaaP deve permettere allo sviluppatore di cambiare il nome della Collection da visualizzare nel menu di navigazione. &  Capitolato \newline  UCS 3.3.3 \newline  \\ \hline      
        RF1O 9.4 & Funzionale \newline  Obbligatorio  & Il framework MaaP deve permettere allo sviluppatore di modificare l’ordine di visualizzazione della Collection nel menu di navigazione. &  Capitolato \newline  UCS 3.3.4 \newline  \\ \hline      
        RS1F 10 & Funzionale \newline  Facoltativo  & Il sistema MaaS deve mettere a disposizione il framework MaaP come servizio web. &  Capitolato \newline  UCM 0 \newline  \\ \hline      
        RS1F 10.1 & Funzionale \newline  Facoltativo  & Il sistema MaaS deve permettere allo sviluppatore di scrivere una Collection tramite editor di testo presente nella pagina web. &  Capitolato \newline  UCM 8 \newline  UCM 8.1 \newline  UCM 8.2 \newline  UCM 8.4 \newline  \\ \hline      
        RS1F 10.2 & Funzionale \newline  Facoltativo  & Il sistema MaaS deve permettere all’utente di poter scrivere una Collection caricando un file prodotto dal framework MaaP. &  Capitolato \newline  UCM 8 \newline  UCM 8.1 \newline  UCM 8.3 \newline  UCM 8.4 \newline  \\ \hline      
        RS1F 10.3 & Funzionale \newline  Facoltativo  & Il sistema MaaS deve permettere ad un utente non registrato di registrarsi al suo servizio. &  Interno \newline  UCM 1 \newline  \\ \hline      
        RS1F 10.4 & Funzionale \newline  Facoltativo  & Il sistema MaaS deve assegnare automaticamente un namespace sul sistema al nuovo utente registrato. &  Interno \newline  UCM 1 \newline  \\ \hline      
        RS1F 10.5 & Funzionale \newline  Facoltativo  & Il servizio MaaS deve visualizzare un messaggio d’errore nel caso in cui la registrazione fallisca a causa di credenziali già esistenti. &  Interno \newline  UCM 3 \newline  \\ \hline      
        RS1F 10.6 & Funzionale \newline  Facoltativo  & Il servizio MaaS deve mettere a disposizione di un utente non autenticato la possibilità di effettuare il login al sistema. &  Interno \newline  UCM 1.1 \newline  UCM 1.2 \newline  UCM 4 \newline  \\ \hline      
        RS1F 10.7 & Funzionale \newline  Facoltativo  & Il servizio MaaS deve visualizzare un messaggio d’errore nel caso in cui l’utente non autenticato abbia inserito credenziali errate nel sistema di login. &  Interno \newline  UCM 5 \newline  \\ \hline      
        RS1F 10.8 & Funzionale \newline  Facoltativo  & Il sistema MaaS deve permettere ad un utente non autenticato di modificare il proprio profilo. &  Interno \newline  UCM 6 \newline  UCM 6.1 \newline  \\ \hline      
        RS1F 10.9 & Funzionale \newline  Facoltativo  & Il sistema MaaS deve permettere ad un utente non autenticato di eliminare il proprio account dal sistema. &  Interno \newline  UCM 7 \newline  \\ \hline      
        RS1F 10.9.1 & Funzionale \newline  Facoltativo  & Il sistema MaaS deve provvedere all’eliminazione dei file di configurazione associati all’utente rimosso dal sistema. &  UCM 7 \newline  UCM 9 \newline  \\ \hline      
        RS1F 10.10 & Funzionale \newline  Facoltativo  & Il sistema MaaS deve permettere allo sviluppatore di eliminare una Collection esistente. &  Interno \newline  UCM 8.5 \newline  \\ \hline      
        RA1D 11 & Funzionale \newline  Desiderabile  & Deve essere possibile da parte di un utente la registrazione all’Applicazione MaaP. &  Verbale-2013-12-05 \newline  UCU 5 \newline  UCU 6 \newline  \\ \hline      
        RA1D 12 & Funzionale \newline  Desiderabile  & L’utente autenticato nell’applicazione deve poter eseguire il logout. &  Interno \newline  UCU 3 \newline  \\ \hline      
        RA1D 13 & Funzionale \newline  Desiderabile  & L'utente deve poter modificare le proprie credenziali d'accesso all'interno della propria pagina profilo. &  Interno \newline  UCU 10 \newline  \\ \hline      
        RA1D 13.1 & Funzionale \newline  Desiderabile  & L’utente deve poter modificare la password di accesso all’applicazione. &  Interno \newline  UCU 10 \newline  UCU 10.1 \newline  \\ \hline      
        RF1O 14 & Funzionale \newline  Obbligatorio  & Il framework MaaP deve rendere possibile la configurazione dei database di cui dispone. &  Interno \newline  UCS 2 \newline  \\ \hline      
        RF1O 14.1 & Funzionale \newline  Obbligatorio  & Il framework MaaP deve rendere possibile la configurazione dei database delle credenziali. &  Interno \newline  UCS 2 \newline  UCS 2.1 \newline  \\ \hline      
        RF1O 14.2 & Funzionale \newline  Obbligatorio  & Il framework MaaP deve rendere possibile la configurazione dei database delle Collection. &  Interno \newline  UCS 2 \newline  UCS 2.2 \newline  \\ \hline      
        RF1F 14.3 & Funzionale \newline  Facoltativo  & Il framework MaaP deve rendere possibile la selezione di un name-space per un database se la funzione di namespace è abilitata. &  Interno \newline  UCS 2.3 \newline  \\ \hline      
        RA1F 15 & Funzionale \newline  Facoltativo  & L’applicazione MaaP deve mettere a disposizione dell’admin una pagina di gestione degli indici. &  Capitolato \newline  UCU 7 \newline  \\ \hline      
        RA1F 15.1 & Funzionale \newline  Facoltativo  & L’applicazione MaaP deve mettere a disposizione dell’admin la visualizzazione degli indici in base alle query più richieste dall’applicazione. &  Capitolato \newline  UCU 7.1 \newline  \\ \hline      
        RA1F 15.2 & Funzionale \newline  Facoltativo  & L’applicazione MaaP deve permettere all’admin di aggiungere gli indici in base ai suggerimenti forniti. &  Capitolato \newline  UCU 7.2 \newline  \\ \hline      
        RA1F 15.3 & Funzionale \newline  Facoltativo  & L’applicazione MaaP deve permettere all’admin di rimuovere gli indici in base ai suggerimenti forniti. &  Interno \newline  UCU 7.3 \newline  \\ \hline      
        RF1F 16 & Funzionale \newline  Facoltativo  & Il framework MaaP deve permettere allo sviluppatore di abilitare i namespace per l’applicazione creata. &  Interno \newline  UCS 4 \newline  \\ \hline      
        RS1F 17 & Funzionale \newline  Facoltativo  & Il sistema MaaS deve verificare se i documenti creati rispettano i vincoli del database. &  Interno \newline  \\ \hline      
        RA1O 18 & Funzionale \newline  Obbligatorio  & Il sistema deve mettere a disposizione un validatore del codice DSL e visualizzare gli eventuali errori logici o di sintassi in un'apposita pagina. &  Interno \newline  UCU 12 \newline  \\ \hline      
        RS1F 19 & Funzionale \newline  Facoltativo  & Il sistema MaaS deve salvare le pagine definite dagli utenti nel database e non su disco. &  Capitolato \newline  \\ \hline
      \caption{Requisiti funzionali}
      \end{longtable}
      \egroup
      \end{center}  
\clearpage

\subsection{Requisiti di qualità }

    %Tabella 
      \begin{center}
      \bgroup
      \def\arraystretch{1.8}
      \begin{longtable}{ | l | p{2cm} | p{5cm} | p{1.7cm} |}
    
      \cellcolor[gray]{0.9} \textbf{Requisito} & \cellcolor[gray]{0.9} \textbf{Tipologia} 
      & \cellcolor[gray]{0.9} \textbf{Descrizione} & \cellcolor[gray]{0.9} \textbf{Fonti} \\ \hline
      
        R3O 1 & Qualità \newline  Obbligatorio  & Devono essere prodotti e rilasciati manuali d'uso ed ogni altra documentazione tecnica necessaria per l’utilizzo del prodotto. &  Capitolato \newline  \\ \hline      
        R3O 2 & Qualità \newline  Obbligatorio  & Per lo sviluppo del prodotto richiesto verranno rispettate tutte le norme descritte nel documento \NormeDiProgetto{}. &  Interno \newline  \\ \hline
      \caption{Requisiti di qualità}
      \end{longtable}
      \egroup
      \end{center}  
\clearpage

\subsection{Requisiti di vincolo }

    %Tabella 
      \begin{center}
      \bgroup
      \def\arraystretch{1.8}
      \begin{longtable}{ | l | p{2cm} | p{5cm} | p{1.7cm} |}
    
      \cellcolor[gray]{0.9} \textbf{Requisito} & \cellcolor[gray]{0.9} \textbf{Tipologia} 
      & \cellcolor[gray]{0.9} \textbf{Descrizione} & \cellcolor[gray]{0.9} \textbf{Fonti} \\ \hline
      
        RF4O 1 & Vincolo \newline  Obbligatorio  & L’implementazione della componente server deve essere realizzata utilizzando Node.js. &  Capitolato \newline  \\ \hline      
        RF4O 2 & Vincolo \newline  Obbligatorio  & Deve essere utilizzato Express per la realizzazione dell’infrastruttura per la componente server. &  Capitolato \newline  \\ \hline      
        RF4F 3 & Vincolo \newline  Facoltativo  & Per l'interfacciamento al database delle \glossario{Collection} deve venire utilizzato Mongoose.js. &  Capitolato \newline  \\ \hline      
        RF4O 3 & Vincolo \newline  Obbligatorio  & \glossario{MaaP} \glossario{Framework} deve essere pubblicato in una repository di GitHub e quest'ultima deve permettere l'utilizzo di Issues per la segnalazione di bug. &  Capitolato \newline  \\ \hline      
        RF4O 4 & Vincolo \newline  Obbligatorio  & \glossario{MaaP} \glossario{Framework} deve servirsi di MongoDB come sistema gestionale dei dati. &  Capitolato \newline  \\ \hline      
        RA4O 5 & Vincolo \newline  Obbligatorio  & Deve essere fatto il \glossario{deployment} su \glossario{Heroku} rendendo disponibile on-line l’applicazione \glossario{MaaP}. &  Capitolato \newline  \\ \hline      
        RA4O 6 & Vincolo \newline  Obbligatorio  & L’applicazione deve funzionare con versioni 24.x o superiori di Firefox. &  Capitolato \newline  \\ \hline      
        RA4O 7 & Vincolo \newline  Obbligatorio  & L’applicazione deve funzionare con versioni 30.0.x o superiori di Chrome. &  Capitolato \newline  \\ \hline
      \caption{Requisiti di vincolo}
      \end{longtable}
      \egroup
      \end{center}  
\clearpage
\section{Tracciamento Requisiti}
\subsection{Tracciamento requisiti-fonti}
%Tabella 
      \begin{center}
      \bgroup
      \def\arraystretch{1.8}
      \begin{longtable}{ | p{5cm} | p{5cm} |}
    
      \cellcolor[gray]{0.9} \textbf{Requisiti} & \cellcolor[gray]{0.9} \textbf{Fonti} \\ \hline       
        RA1O 1 &  Capitolato \newline  UCU 1 \newline  \\ \hline      
        RA1O 1.1 &  Capitolato \newline  UCU 1 \newline  UCU 1.1 \newline  \\ \hline      
        RA1O 1.2 &  Capitolato \newline  UCU 1 \newline  UCU 1.2 \newline  \\ \hline      
        RA1O 1.3 &  Capitolato \newline  \\ \hline      
        RA1O 1.3.1 &  Interno \newline  UCU 2 \newline  \\ \hline      
        RA1O 1.3.2 &  Interno \newline  UCU 8 \newline  \\ \hline      
        RA1O 2 &  Capitolato \newline  UCU 4 \newline  UCU 4.1 \newline  UCU 4.1.1 \newline  UCU 4.2 \newline  UCU 4.2.1 \newline  \\ \hline      
        RA1O 2.1 &  Interno \newline  UCU 4 \newline  UCU 4.1 \newline  UCU 4.1.1 \newline  \\ \hline      
        RA1O 2.2 &  Interno \newline  UCU 4 \newline  UCU 4.2 \newline  \\ \hline      
        RA1O 2.3 &  Interno \newline  UCU 4 \newline  UCU 4.2 \newline  UCU 4.2.1 \newline  \\ \hline      
        RA1D 3 &  Interno \newline  UCU 8 \newline  \\ \hline      
        RA1O 4 &  Capitolato \newline  UCU 9 \newline  \\ \hline      
        RA1O 4.1 &  Capitolato \newline  UCU 9 \newline  \\ \hline      
        RA1O 4.1.1 &  Capitolato \newline  UCS 3.3.1.4 \newline  \\ \hline      
        RA1D 4.1.2 &  Interno \newline  UCU 9 \newline  UCU 9.5 \newline  \\ \hline      
        RA1D 4.1.3 &  Interno \newline  UCU 9.4 \newline  \\ \hline      
        RA1D 4.2 &  Interno \newline  UCU 9.2 \newline  UCS 3.3.1.7 \newline  \\ \hline      
        RA1F 4.3 &  Capitolato \newline  UCU 9.6 \newline  \\ \hline      
        RA1F 4.4 &  Capitolato \newline  UCU 9.3 \newline  \\ \hline      
        RA1O 5 &  Capitolato \newline  UCU 9.1 \newline  UCU 9.1.1 \newline  UCU 9.1.2 \newline  \\ \hline      
        RA1O 5.1 &  Capitolato \newline  UCU 9.7 \newline  UCU 9.1 \newline  UCU 9.1.3 \newline  UCU 9.1.5 \newline  \\ \hline      
        RA1F 5.2 &  Capitolato \newline  UCU 9.3 \newline  \\ \hline      
        RA1O 5.3 &  Capitolato \newline  UCU 9.1.4 \newline  \\ \hline      
        RA1O 6 &  Interno \newline  UCU 11 \newline  \\ \hline      
        RA1O 6.1 &  Capitolato \newline  UCU 11.1 \newline  \\ \hline      
        RA1O 6.1.1 &  Capitolato \newline  UCU 11.1 \newline  \\ \hline      
        RA1O 6.1.1.1 &  Interno \newline  UCU 11.1 \newline  UCU 11.1.1 \newline  \\ \hline      
        RA1O 6.1.1.2 &  Interno \newline  UCU 11.1 \newline  UCU 11.1.2 \newline  \\ \hline      
        RA1O 6.1.1.3 &  Capitolato \newline  UCU 11.1 \newline  UCU 11.1.3 \newline  \\ \hline      
        RA1O 6.1.2 &  Interno \newline  UCU 11.1 \newline  \\ \hline      
        RA1O 6.1.3 &  Interno \newline  UCU 11.2 \newline  \\ \hline      
        RA1O 6.2 &  Interno \newline  UCU 11 \newline  UCU 11.1 \newline  UCU 11.1.3 \newline  \\ \hline      
        RA1O 6.2.1 &  Capitolato \newline  UCU 11.3 \newline  UCU 11.3.1 \newline  \\ \hline      
        RA1O 6.2.2 &  Capitolato \newline  UCU 11.3 \newline  UCU 11.3.2 \newline  \\ \hline      
        RA1O 6.2.3 &  Verbale-2013-12-18 \newline  UCU 11.3 \newline  \\ \hline      
        RA1O 6.2.4 &  Verbale-2013-12-18 \newline  UCU 11.3 \newline  \\ \hline      
        RA1O 6.2.5 &  Interno \newline  UCU 11.3.3 \newline  \\ \hline      
        RF1O 7 &  Capitolato \newline  \\ \hline      
        RF1O 8 &  Capitolato \newline  UCS 1 \newline  \\ \hline      
        RF1O 8.1  &  Capitolato \newline  UCS 1 \newline  \\ \hline      
        RF1O 8.1.1 &  Capitolato \newline  UCS 1 \newline  \\ \hline      
        RF1O 8.1.2 &  Capitolato \newline  UCS 1 \newline  \\ \hline      
        RF1O 8.1.3 &  Capitolato \newline  UCS 1 \newline  \\ \hline      
        RF1O 8.1.4 &  Capitolato \newline  UCS 1 \newline  \\ \hline      
        RF1O 8.2 &  Verbale-2013-12-05 \newline  UCS 1 \newline  \\ \hline      
        RF1F 8.3 &  Interno \newline  UCS 3.3.5 \newline  \\ \hline      
        RF1O 9 &  Capitolato \newline  UCS 3 \newline  \\ \hline      
        RF1O 9.1 &  Capitolato \newline  UCS 3 \newline  UCS 3.2 \newline  UCS 3.4 \newline  \\ \hline      
        RF1O 9.1.1 &  Capitolato \newline  UCS 3.3 \newline  UCS 3.3.1.1 \newline  \\ \hline      
        RF1O 9.1.2 &  Capitolato \newline  UCS 3.3 \newline  UCS 3.3.1.2 \newline  \\ \hline      
        RF1O 9.1.3 &  Capitolato \newline  UCS 3.3 \newline  UCS 3.3.1.3 \newline  \\ \hline      
        RF1O 9.1.4 &  Capitolato \newline  UCS 3.3.1.2 \newline  \\ \hline      
        RF1O 9.1.5 &  Capitolato \newline  UCS 3.3.1.6 \newline  UCS 3.3.2.3 \newline  \\ \hline      
        RF1O 9.1.6 &  Capitolato \newline  UCS 3.3.1 \newline  UCS 3.3.1.5 \newline  \\ \hline      
        RF1O 9.1.7 &  Capitolato \newline  UCS 3.3.1.8 \newline  \\ \hline      
        RF1O 9.2 &  Capitolato \newline  UCS 3.2 \newline  UCS 3.1 \newline  UCS 3.4 \newline  \\ \hline      
        RF1O 9.2.1 &  Capitolato \newline  UCS 3.3.2 \newline  UCS 3.3.2.1 \newline  \\ \hline      
        RF1O 9.2.2 &  Capitolato \newline  Verbale-2013-12-05 \newline  UCS 3.3.2.3 \newline  \\ \hline      
        RF1O 9.2.3 &  Capitolato \newline  UCS 3.3.2 \newline  UCS 3.3.2.2 \newline  \\ \hline      
        RF1O 9.2.4 &  Capitolato \newline  UCS 3.3.2 \newline  UCS 3.3.2.4 \newline  \\ \hline      
        RF1F 9.2.5 &  Capitolato \newline  UCS 3.3.1.9 \newline  UCS 3.3.2 \newline  UCS 3.3.2.5 \newline  \\ \hline      
        RF1O 9.3 &  Capitolato \newline  UCS 3.3.3 \newline  \\ \hline      
        RF1O 9.4 &  Capitolato \newline  UCS 3.3.4 \newline  \\ \hline      
        RS1F 10 &  Capitolato \newline  UCM 0 \newline  \\ \hline      
        RS1F 10.1 &  Capitolato \newline  UCM 8 \newline  UCM 8.1 \newline  UCM 8.2 \newline  UCM 8.4 \newline  \\ \hline      
        RS1F 10.2 &  Capitolato \newline  UCM 8 \newline  UCM 8.1 \newline  UCM 8.3 \newline  UCM 8.4 \newline  \\ \hline      
        RS1F 10.3 &  Interno \newline  UCM 1 \newline  \\ \hline      
        RS1F 10.4 &  Interno \newline  UCM 1 \newline  \\ \hline      
        RS1F 10.5 &  Interno \newline  UCM 3 \newline  \\ \hline      
        RS1F 10.6 &  Interno \newline  UCM 1.1 \newline  UCM 1.2 \newline  UCM 4 \newline  \\ \hline      
        RS1F 10.7 &  Interno \newline  UCM 5 \newline  \\ \hline      
        RS1F 10.8 &  Interno \newline  UCM 6 \newline  UCM 6.1 \newline  \\ \hline      
        RS1F 10.9 &  Interno \newline  UCM 7 \newline  \\ \hline      
        RS1F 10.9.1 &  UCM 7 \newline  UCM 9 \newline  \\ \hline      
        RS1F 10.10 &  Interno \newline  UCM 8.5 \newline  \\ \hline      
        RA1D 11 &  Verbale-2013-12-05 \newline  UCU 5 \newline  UCU 6 \newline  \\ \hline      
        RA1D 12 &  Interno \newline  UCU 3 \newline  \\ \hline      
        RA1D 13 &  Interno \newline  UCU 10 \newline  \\ \hline      
        RA1D 13.1 &  Interno \newline  UCU 10 \newline  UCU 10.1 \newline  \\ \hline      
        RF1O 14 &  Interno \newline  UCS 2 \newline  \\ \hline      
        RF1O 14.1 &  Interno \newline  UCS 2 \newline  UCS 2.1 \newline  \\ \hline      
        RF1O 14.2 &  Interno \newline  UCS 2 \newline  UCS 2.2 \newline  \\ \hline      
        RF1F 14.3 &  Interno \newline  UCS 2.3 \newline  \\ \hline      
        RA1F 15 &  Capitolato \newline  UCU 7 \newline  \\ \hline      
        RA1F 15.1 &  Capitolato \newline  UCU 7.1 \newline  \\ \hline      
        RA1F 15.2 &  Capitolato \newline  UCU 7.2 \newline  \\ \hline      
        RA1F 15.3 &  Interno \newline  UCU 7.3 \newline  \\ \hline      
        RF1F 16 &  Interno \newline  UCS 4 \newline  \\ \hline      
        RS1F 17 &  Interno \newline  \\ \hline      
        RA1O 18 &  Interno \newline  UCU 12 \newline  \\ \hline      
        RS1F 19 &  Capitolato \newline  \\ \hline      
        R3O 1 &  Capitolato \newline  \\ \hline      
        R3O 2 &  Interno \newline  \\ \hline      
        RF4O 1 &  Capitolato \newline  \\ \hline      
        RF4O 2 &  Capitolato \newline  \\ \hline      
        RF4F 3 &  Capitolato \newline  \\ \hline      
        RF4O 3 &  Capitolato \newline  \\ \hline      
        RF4O 4 &  Capitolato \newline  \\ \hline      
        RA4O 5 &  Capitolato \newline  \\ \hline      
        RA4O 6 &  Capitolato \newline  \\ \hline      
        RA4O 7 &  Capitolato \newline  \\ \hline  
      \caption{Tracciamento requisiti-fonti}    
      \end{longtable}
      \egroup
      \end{center}  
\clearpage

\subsection{Tracciamento fonti-requisiti}
%Tabella 
      \begin{center}
      \bgroup
      \def\arraystretch{1.8}
      \begin{longtable}{ | p{5cm} | p{5cm} |}
    
      \cellcolor[gray]{0.9} \textbf{Fonti} & \cellcolor[gray]{0.9} \textbf{Requisiti} \\ \hline       
            UCU 1 - Login &  RA1O 1 \newline  RA1O 1.1 \newline  RA1O 1.2 \newline  \\ \hline      
            UCU 1.1 - Inserimento email &  RA1O 1.1 \newline  \\ \hline      
            UCU 1.2 - Inserimento Password &  RA1O 1.2 \newline  \\ \hline      
            UCU 2 - Visualizzazione messaggio errore per credenziali errate &  RA1O 1.3.1 \newline  \\ \hline      
            UCU 3 - Logout &  RA1D 12 \newline  \\ \hline      
            UCU 4 - Recupero password &  RA1O 2.2 \newline  RA1O 2.1 \newline  RA1O 2 \newline  RA1O 2.3 \newline  \\ \hline      
            UCU 4.1 - Richiesta reset password &  RA1O 2.1 \newline  RA1O 2 \newline  \\ \hline      
            UCU 4.1.1 - Inserimento email &  RA1O 2.1 \newline  RA1O 2 \newline  \\ \hline      
            UCU 4.2 - Effettuazione reset password &  RA1O 2.2 \newline  RA1O 2 \newline  RA1O 2.3 \newline  \\ \hline      
            UCU 4.2.1 - Inserimento nuova password &  RA1O 2 \newline  RA1O 2.3 \newline  \\ \hline      
            UCU 5 - Registrazione &  RA1D 11 \newline  \\ \hline      
            UCU 6 - Visualizzazione messaggio errore per Registrazione fallita &  RA1D 11 \newline  \\ \hline      
            UCU 7 - Gestione indici &  RA1F 15 \newline  \\ \hline      
            UCU 7.1 - Visualizzazione suggerimenti sugli indici &  RA1F 15.1 \newline  \\ \hline      
            UCU 7.2 - Creazione indici &  RA1F 15.2 \newline  \\ \hline      
            UCU 7.3  - Rimozione indici &  RA1F 15.3 \newline  \\ \hline      
            UCU 8 - Visualizzazione Dashboard &  RA1O 1.3.2 \newline  RA1D 3 \newline  \\ \hline      
            UCU 9 - Apertura Collection Index &  RA1O 4 \newline  RA1O 4.1 \newline  RA1D 4.1.2 \newline  \\ \hline      
            UCU 9.1 - Apertura show-page Document &  RA1O 5 \newline  RA1O 5.1 \newline  \\ \hline      
            UCU 9.1.1 - Visualizzazione show page attributi innestati &  RA1O 5 \newline  \\ \hline      
            UCU 9.1.2 - Visualizzazione index page dell'array di document &  RA1O 5 \newline  \\ \hline      
            UCU 9.1.3 - Modifica Document &  RA1O 5.1 \newline  \\ \hline      
            UCU 9.1.4 - Elimina Document &  RA1O 5.3 \newline  \\ \hline      
            UCU 9.1.5 - Annulla modifica Document &  RA1O 5.1 \newline  \\ \hline      
            UCU 9.2 - Filtra risultati &  RA1D 4.2 \newline  \\ \hline      
            UCU 9.3 - Esegui azione personalizzata &  RA1F 4.4 \newline  RA1F 5.2 \newline  \\ \hline      
            UCU 9.4 - Modifica Document &  RA1D 4.1.3 \newline  \\ \hline      
            UCU 9.5 - Elimina Document &  RA1D 4.1.2 \newline  \\ \hline      
            UCU 9.6 - Creazione Document &  RA1F 4.3 \newline  \\ \hline      
            UCU 9.7 - Annulla modifica Document &  RA1O 5.1 \newline  \\ \hline      
            UCU 10 - Modifica profilo &  RA1D 13 \newline  RA1D 13.1 \newline  \\ \hline      
            UCU 10.1 - Modifica password &  RA1D 13.1 \newline  \\ \hline      
            UCU 11 - Gestione utenti &  RA1O 6 \newline  RA1O 6.2 \newline  \\ \hline      
            UCU 11.1 - Creazione utente &  RA1O 6.1 \newline  RA1O 6.1.1 \newline  RA1O 6.1.1.1 \newline  RA1O 6.1.1.2 \newline  RA1O 6.1.1.3 \newline  RA1O 6.1.2 \newline  RA1O 6.2 \newline  \\ \hline      
            UCU 11.1.1 - Inserimento email &  RA1O 6.1.1.1 \newline  \\ \hline      
            UCU 11.1.2 - Inserimento password &  RA1O 6.1.1.2 \newline  \\ \hline      
            UCU 11.1.3 - Definizione livello utente &  RA1O 6.1.1.3 \newline  RA1O 6.2 \newline  \\ \hline      
            UCU 11.2 - Creazione fallita &  RA1O 6.1.3 \newline  \\ \hline      
            UCU 11.3 - Apertura show-page Utente &  RA1O 6.2.1 \newline  RA1O 6.2.2 \newline  RA1O 6.2.3 \newline  RA1O 6.2.4 \newline  \\ \hline      
            UCU 11.3.1 - Eleva utente &  RA1O 6.2.1 \newline  \\ \hline      
            UCU 11.3.2 - Declassa admin &  RA1O 6.2.2 \newline  \\ \hline      
            UCU 11.3.3 - Eliminazione Utente &  RA1O 6.2.5 \newline  \\ \hline      
            UCU 12 - Visualizzazione messaggio di errore DSL &  RA1O 18 \newline  \\ \hline      
            UCS 1 - Creazione nuovo progetto &  RF1O 8.1.1 \newline  RF1O 8.1  \newline  RF1O 8 \newline  RF1O 8.1.2 \newline  RF1O 8.1.3 \newline  RF1O 8.1.4 \newline  RF1O 8.2 \newline  \\ \hline      
            UCS 2 - Configurazione database &  RF1O 14 \newline  RF1O 14.1 \newline  RF1O 14.2 \newline  \\ \hline      
            UCS 2.1  - Configurazione database credenziali &  RF1O 14.1 \newline  \\ \hline      
            UCS 2.2  - Configurazione database Collection &  RF1O 14.2 \newline  \\ \hline      
            UCS 2.3  - Selezione namespace &  RF1F 14.3 \newline  \\ \hline      
            UCS 3 - Gestione Collection &  RF1O 9.1 \newline  RF1O 9 \newline  \\ \hline      
            UCS 3.1  - Registrazione manuale Collection &  RF1O 9.2 \newline  \\ \hline      
            UCS 3.2  - Creazione file di configurazione &  RF1O 9.2 \newline  RF1O 9.1 \newline  \\ \hline      
            UCS 3.3  - Configurazione Collection &  RF1O 9.1.3 \newline  RF1O 9.1.1 \newline  RF1O 9.1.2 \newline  \\ \hline      
            UCS 3.3.1 - Personalizzazione index page &  RF1O 9.1.6 \newline  \\ \hline      
            UCS 3.3.1.1  - Definizione attributi da visualizzare &  RF1O 9.1.1 \newline  \\ \hline      
            UCS 3.3.1.2  -  Definizione ordinamento attributi &  RF1O 9.1.2 \newline  RF1O 9.1.4 \newline  \\ \hline      
            UCS 3.3.1.3  - Definizione limite elementi da visualizzare &  RF1O 9.1.3 \newline  \\ \hline      
            UCS 3.3.1.4 - Definizione attributo selezionabile &  RA1O 4.1.1 \newline  \\ \hline      
            UCS 3.3.1.5 - Definizione della index-query &  RF1O 9.1.6 \newline  \\ \hline      
            UCS 3.3.1.6 - Definizione populate da eseguire &  RF1O 9.1.5 \newline  \\ \hline      
            UCS 3.3.1.7 - Personalizzazione filtri &  RA1D 4.2 \newline  \\ \hline      
            UCS 3.3.1.8 - Definizione delle trasformazioni &  RF1O 9.1.7 \newline  \\ \hline      
            UCS 3.3.1.9 - Definizione azioni personalizzate &  RF1F 9.2.5 \newline  \\ \hline      
            UCS 3.3.2  -  Personalizzazione show page &  RF1O 9.2.1 \newline  RF1O 9.2.4 \newline  RF1F 9.2.5 \newline  RF1O 9.2.3 \newline  \\ \hline      
            UCS 3.3.2 - Personalizzazione show page &  \\ \hline      
            UCS 3.3.2.1 -  Definizione attributi da visualizzare &  RF1O 9.2.1 \newline  \\ \hline      
            UCS 3.3.2.2 - Definizione ordinamento attributi &  RF1O 9.2.3 \newline  \\ \hline      
            UCS 3.3.2.3 - Definizione populate da eseguire &  RF1O 9.1.5 \newline  RF1O 9.2.2 \newline  \\ \hline      
            UCS 3.3.2.4 - Definizione delle trasformazioni &  RF1O 9.2.4 \newline  \\ \hline      
            UCS 3.3.2.5 - Definizione azioni personalizzate &  RF1F 9.2.5 \newline  \\ \hline      
            UCS 3.3.3  - Rinominazione Collection &  RF1O 9.3 \newline  \\ \hline      
            UCS 3.3.4  - Definizione ordinamento Collection &  RF1O 9.4 \newline  \\ \hline      
            UCS 3.3.5 - Definizione namespace &  RF1F 8.3 \newline  \\ \hline      
            UCS 3.4  - Registrazione automatica Collection &  RF1O 9.2 \newline  RF1O 9.1 \newline  \\ \hline      
            UCS 4 - Abilitazione namespace &  RF1F 16 \newline  \\ \hline      
            UCM 1 -  Inserimento nuova email &  RS1F 10.3 \newline  RS1F 10.4 \newline  \\ \hline      
            UCM 1.1 -  Inserimento nuova email &  RS1F 10.6 \newline  \\ \hline      
            UCM 1.2 - Inserimento nuova password &  RS1F 10.6 \newline  \\ \hline      
            UCM 3 - Credenziali già esistenti &  RS1F 10.5 \newline  \\ \hline      
            UCM 4 - Login &  RS1F 10.6 \newline  \\ \hline      
            UCM 5 - Credenziali errate &  RS1F 10.7 \newline  \\ \hline      
            UCM 6 - Modifica profilo &  RS1F 10.8 \newline  \\ \hline      
            UCM 6.1 - Modifica password &  RS1F 10.8 \newline  \\ \hline      
            UCM 7 - Eliminazione account &  RS1F 10.9 \newline  RS1F 10.9.1 \newline  \\ \hline      
            UCM 8 - Gestione file di configurazione &  RS1F 10.2 \newline  RS1F 10.1 \newline  \\ \hline      
            UCM 8.1 - Creazione file di configurazione &  RS1F 10.2 \newline  RS1F 10.1 \newline  \\ \hline      
            UCM 8.2 - Creazione file di configurazione tramite editor di testo &  RS1F 10.1 \newline  \\ \hline      
            UCM 8.3 - Creazione file di configurazione tramite caricamento &  RS1F 10.2 \newline  \\ \hline      
            UCM 8.4 - Modifica file di configurazione &  RS1F 10.2 \newline  RS1F 10.1 \newline  \\ \hline      
            UCM 8.5 - Eliminazione file di configurazione &  RS1F 10.10 \newline  \\ \hline      
            UCM 9 - Eliminazione file di configurazione &  RS1F 10.9.1 \newline  \\ \hline  
      \caption{Tracciamento fonti-requisiti}   
      \end{longtable}
      \egroup
      \end{center}  
\clearpage
\subsection{Tracciamento Requisiti - Test di Sistema e Validazione}

  \begin{center}
  \def\arraystretch{1.5}
  \bgroup
    \begin{longtable}{| p{2cm} | p{6cm} | p{2.5cm} | p{2.5cm} | }
    \hline 
     \textbf{Requisito} & \textbf{Descrizione} & \textbf{Test di Sistema} & \textbf{Test di Validazione} \\ \hline
        RA1O 1 & 
        Il sistema permette all'utente non autenticato di autenticarsi tramite la visualizzazione di una pagina web, la quale conterrà al suo interno i campi di testo necessari. &  & TV-RA1O 1 \\ \hline 
        RA1O 1.1 & 
        Il sistema prevede l'inserimento dell'indirizzo email per la verifica delle credenziali in un apposito campo di testo. & TS-RA1O 1.1 & \\ \hline 
        RA1O 1.2 & 
        Il sistema prevede l'inserimento di password per la verifica delle credenziali in un apposito campo di testo. & TS-RA1O 1.2 & \\ \hline 
        RA1O 1.3 & 
        Il sistema, tramite un database indipendente, ovvero separato da quello che contiene la Collection, provvede a verificare l'autenticità  di un utente tramite la verifica di email e password. & TS-RA1O 1.3 & \\ \hline 
        RA1O 1.3.1 & 
        Il sistema mette a disposizione la visualizzazione di una pagina di errore in caso di fallimento dell'autenticazione da parte dell'utente. & TS-RA1O 1.3.1 & \\ \hline 
        RA1O 1.3.2 & 
        Il sistema, nel caso in cui l'autenticazione da parte dell'utente abbia avuto successo, reindirizza automaticamente l'utente sulla dashboard dell'applicazione.
 & TS-RA1O 1.3.2 & \\ \hline 
        RA1O 2 & 
        Il sistema mette a disposizione dell'utente non autenticato la possibilità  di recuperare la propria password. &  & TV-RA1O 2 \\ \hline 
        RA1O 2.1 & 
        Il sistema permette il recupero password attraverso l'inserimento dell'email. & TS-RA1O 2.1 & \\ \hline 
        RA1O 2.2 & 
        Il sistema deve inviare un'email contenente un link attraverso il quale l'utente non autenticato può effettuare il reset della propria password. & TS-RA1O 2.2 & \\ \hline 
        RA1O 2.3 & 
        L'applicazione \glossario{MaaP} deve permettere il reset della password ad un utente non autenticato che abbia fatto la richiesta attraverso il sistema di recupero password. L'utente deve così poter inserire una nuova password che sostituirà la precedente per l'accesso all'applicazione. & TS-RA1O 2.3 & \\ \hline 
        RA1D 3 & 
        Il sistema deve mettere a disposizione dell'utente autenticato la visualizzazione della dashboard. &  & TV-RA1D 3 \\ \hline 
        RA1O 4 & 
        L'applicazione deve permettere all'utente la visualizzazione dell'insieme delle Collection presenti tramite un menu di navigazione e la selezione di una di esse. &  & TV-RA1O 4 \\ \hline 
        RA1O 4.1 & 
        L'applicazione deve permettere all'utente la visualizzazione di una Collection-index tramite una tabella le cui righe corrispondono ai document presenti nel database e le colonne ai loro attributi visualizzabili. & TS-RA1O 4.1 & \\ \hline 
        RA1O 4.1.1 & 
        Ogni riga della tabella corrispondente ad un Document deve avere una chiave selezionabile che rimanda alla corrispondente pagina show. & TS-RA1O 4.1.1 & \\ \hline 
        RA1D 4.1.2 & 
        L’applicazione può mettere a disposizione dell’admin un link di selezione rapida per l’eliminazione di un documento. & TS-RA1D 4.1.2 & \\ \hline 
        RA1D 4.1.3 & 
        L’applicazione deve permette all’admin la modifica di un document presente nella collection-index. & TS-RA1D 4.1.3 & \\ \hline 
        RA1D 4.2 & 
        L’applicazione dà la possibilità di impostare dei filtri personalizzati secondo determinati attributi per visualizzare un sottoinsieme di Document. & TS-RA1D 4.2 & \\ \hline 
        RA1F 4.3 & 
        L’applicazione deve permettere all’amministratore di creare un nuovo Document all’interno della base di dati. & TS-RA1F 4.3 & \\ \hline 
        RA1F 4.4 & 
        L’applicazione deve permettere all’utente di poter eseguire un’azione personalizzata tramite l’esecuzione di un pulsante. & TS-RA1F 4.4 & \\ \hline 
        RA1O 5 & 
        L'applicazione deve permettere all'utente la visualizzazione della Collection-show relativa ad un document tramite una pagina web che ne mostra gli attributi visualizzabili strutturati in una tabella. &  & TV-RA1O 5 \\ \hline 
        RA1O 5.1 & 
        L'applicazione deve permettere all'admin di poter editare ogni singolo attributo modificabile del documento dalla pagina show. & TS-RA1O 5.1 & \\ \hline 
        RA1F 5.2 & 
        L’applicazione deve permettere all’utente di poter eseguire un’azione personalizzata tramite l’esecuzione di un pulsante. & TS-RA1F 5.2 & \\ \hline 
        RA1O 5.3 & 
        L'applicazione deve permettere all'utente di poter eliminare dalla show-page il Document selezionato.
 & TS-RA1O 5.3 & \\ \hline 
        RA1O 6 & 
        L'applicazione deve fornire all'admin una pagina di amministrazione in cui visualizzare la Collection di tutti gli utenti registrati all'applicazione. &  & TV-RA1O 6 \\ \hline 
        RA1O 6.1 & 
        L'applicazione deve fornire all'admin la possibilità  di creare un nuovo utente dalla pagina di amministrazione. & TS-RA1O 6.1 & \\ \hline 
        RA1O 6.1.1 & 
        L'applicazione deve fornire all'admin una pagina di creazione di un nuovo utente. & TS-RA1O 6.1.1 & \\ \hline 
        RA1O 6.1.1.1 & 
        L'applicazione deve permettere all'admin di inserire l'indirizzo email del nuovo utente in un apposito campo di testo presente all'interno della pagina di creazione di un nuovo utente.
 & TS-RA1O 6.1.1.1 & \\ \hline 
        RA1O 6.1.1.2 & 
        L'applicazione deve permettere all'admin di inserire la password del nuovo utente in un apposito campo di testo presente all'interno della pagina di creazione di un nuovo utente. & TS-RA1O 6.1.1.2 & \\ \hline 
        RA1O 6.1.1.3 & 
        L'applicazione deve permettere all'admin di inserire il "livello utente" del nuovo utente tramite una combo-box presente all'interno della pagina di creazione di un nuovo utente. & TS-RA1O 6.1.1.3 & \\ \hline 
        RA1O 6.1.2 & 
        L'applicazione deve prelevare tutti i dati inseriti dall'admin nella pagina di creazione di un nuovo utente ed inviarli al database delle credenziali, il quale provvederà  all'inserimento del nuovo record. & TS-RA1O 6.1.2 & \\ \hline 
        RA1O 6.1.3 & 
        L'applicazione deve visualizzare un messaggio d'errore nel caso in cui l'admin non abbia compilato correttamente i campi presenti all'interno della pagina di creazione di un nuovo utente. & TS-RA1O 6.1.3 & \\ \hline 
        RA1O 6.2 & 
        L'applicazione deve fornire all'admin la possibilità di selezionare un utente dalla index-page e visualizzare la sua relativa show-page. & TS-RA1O 6.2 & \\ \hline 
        RA1O 6.2.1 & 
        L'applicazione deve fornire all'admin la possibilità di elevare l'utente normale selezionato al livello "admin" dalla show-page relativa. & TS-RA1O 6.2.1 & \\ \hline 
        RA1O 6.2.2 & 
        L'applicazione deve fornire all'admin la possibilità di declassare l'admin selezionato a livello di utente normale dalla show-page relativa. & TS-RA1O 6.2.2 & \\ \hline 
        RA1O 6.2.3 & 
        L'applicazione deve fornire all'admin la possibilità di modificare l'attributo email dell'utente selezionato dalla relativa show-page. & TS-RA1O 6.2.3 & \\ \hline 
        RA1O 6.2.4 & 
        L'applicazione deve fornire all'admin la possibilità di modificare l'attributo password dell'utente selezionato dalla relativa show-page.
 & TS-RA1O 6.2.4 & \\ \hline 
        RA1O 6.2.5 & 
        L'applicazione deve fornire all'admin la possibilità di eliminare l'utente visualizzato nella \glossario{show-page}. & TS-RA1O 6.2.5 & \\ \hline 
        RF1O 7 & 
        MaaP Framework deve rendere disponibile allo sviluppatore un linguaggio astratto DSL di tipo testuale necessario per la generazione delle pagine. & TS-RF1O 7 & \\ \hline 
        RF1O 8 & 
        Maap Framework deve permettere allo sviluppatore di generare un nuovo progetto tramite linea di comando. &  & TV-RF1O 8 \\ \hline 
        RF1O 8.1  & 
        Maap Framework deve generare automaticamente lo scheletro dell’applicazione creata dallo sviluppatore. & TS-RF1O 8.1  & \\ \hline 
        RF1O 8.1.1 & 
        Maap Framework deve automaticamente importare in un'apposita directory del progetto tutte le librerie necessarie al corretto funzionamento del sistema. Librerie necessarie: \begin{itemize} \item Express v-3.4.8 \item MongoDB v-1.3.23 \item Mongoose v-3.8.4 \item Nel caso fossero necessarie ulteriori librerie è consigliata una versione uguale o maggiore rispetto a quella disponibile al 2014-01-01 \end{itemize} & TS-RF1O 8.1.1 & \\ \hline 
        RF1O 8.1.2 & 
        Maap Framework deve automaticamente creare in un’apposita directory il file di configurazione di default dell’applicazione generata. & TS-RF1O 8.1.2 & \\ \hline 
        RF1O 8.1.3 & 
        Maap Framework deve automaticamente creare il sistema di autenticazione per l’applicazione generata. & TS-RF1O 8.1.3 & \\ \hline 
        RF1O 8.1.4 & 
        Maap Framework deve automaticamente creare le directory di descrizione delle pagine web. & TS-RF1O 8.1.4 & \\ \hline 
        RF1O 8.2 & 
        Maap Framework deve automaticamente creare un account admin di default. & TS-RF1O 8.2 & \\ \hline 
        RF1F 8.3 & 
        Il framework MaaP deve permettere allo sviluppatore di definire un namespace per l’applicazione generata. & TS-RF1F 8.3 & \\ \hline 
        RF1O 9 & 
        Maap Framework deve permettere allo sviluppatore di configurare le Collection tramite DSL fornito. &  & TV-RF1O 9 \\ \hline 
        RF1O 9.1 & 
        Il DSL deve permettere allo sviluppatore di creare una pagina Collection-index. & TS-RF1O 9.1 & \\ \hline 
        RF1O 9.1.1 & 
        Il DSL deve permettere allo sviluppatore di poter definire una serie di attributi da visualizzare all’interno della pagina Collection-index. & TS-RF1O 9.1.1 & \\ \hline 
        RF1O 9.1.2 & 
        Il DSL deve permettere allo sviluppatore di poter definire un ordinamento di default (ordine alfanumerico) di visualizzazione dei document all'interno della pagina Collection-index. & TS-RF1O 9.1.2 & \\ \hline 
        RF1O 9.1.3 & 
        Il DSL deve permettere allo sviluppatore di poter definire un eventuale limite di elementi da visualizzare all’interno della pagina Collection-index. & TS-RF1O 9.1.3 & \\ \hline 
        RF1O 9.1.4 & 
        Il DSL deve permettere allo sviluppatore di poter definire quali attributi sono ordinabili all’interno della pagina Collection-index. & TS-RF1O 9.1.4 & \\ \hline 
        RF1O 9.1.5 & 
        Il DSL deve permettere allo sviluppatore di definire la funzione populate per far si che una chiave riferisca ad un documento esterno. & TS-RF1O 9.1.5 & \\ \hline 
        RF1O 9.1.6 & 
        Il DSL deve permettere allo sviluppatore di definire delle query per creare la pagina Collection-index in base al risultato della loro estrazione. & TS-RF1O 9.1.6 & \\ \hline 
        RF1O 9.1.7 & 
        Il DSL deve permettere allo sviluppatore di definire delle trasformazioni sugli attributi da visualizzare. & TS-RF1O 9.1.7 & \\ \hline 
        RF1O 9.2 & 
        Il DSL deve permettere allo sviluppatore di creare una pagina Collection-show. & TS-RF1O 9.2 & \\ \hline 
        RF1O 9.2.1 & 
        Il DSL deve permettere allo sviluppatore di definire una serie di attributi visualizzabili all’interno della pagina Collection-show. & TS-RF1O 9.2.1 & \\ \hline 
        RF1O 9.2.2 & 
        Il DSL deve permettere allo sviluppatore definire gli attributi del Document come attributi innestati o array di Document tramite la funzione populate. & TS-RF1O 9.2.2 & \\ \hline 
        RF1O 9.2.3 & 
        Lo sviluppatore deve aver la possibilità di personalizzare la show page definendone l’ordinamento degli attributi. & TS-RF1O 9.2.3 & \\ \hline 
        RF1O 9.2.4 & 
        Lo sviluppatore deve poter definire trasformazioni agli attributi per poi visualizzarli nella show-page. & TS-RF1O 9.2.4 & \\ \hline 
        RF1F 9.2.5 & 
        Lo sviluppatore deve poter personalizzare la show-page definendo delle operazioni personalizzate che l’utente potrà utilizzare tramite appositi pulsanti. & TS-RF1F 9.2.5 & \\ \hline 
        RF1O 9.3 & 
        Il framework MaaP deve permettere allo sviluppatore di cambiare il nome della Collection da visualizzare nel menu di navigazione. & TS-RF1O 9.3 & \\ \hline 
        RF1O 9.4 & 
        Il framework MaaP deve permettere allo sviluppatore di modificare l’ordine di visualizzazione della Collection nel menu di navigazione. & TS-RF1O 9.4 & \\ \hline 
        RS1F 10 & 
        Il sistema MaaS deve mettere a disposizione il framework MaaP come servizio web. &  & TV-RS1F 10 \\ \hline 
        RS1F 10.1 & 
        Il sistema MaaS deve permettere allo sviluppatore di scrivere una Collection tramite editor di testo presente nella pagina web. & TS-RS1F 10.1 & \\ \hline 
        RS1F 10.2 & 
        Il sistema MaaS deve permettere all’utente di poter scrivere una Collection caricando un file prodotto dal framework MaaP. & TS-RS1F 10.2 & \\ \hline 
        RS1F 10.3 & 
        Il sistema MaaS deve permettere ad un utente non registrato di registrarsi al suo servizio. & TS-RS1F 10.3 & \\ \hline 
        RS1F 10.4 & 
        Il sistema MaaS deve assegnare automaticamente un namespace sul sistema al nuovo utente registrato. & TS-RS1F 10.4 & \\ \hline 
        RS1F 10.5 & 
        Il servizio MaaS deve visualizzare un messaggio d’errore nel caso in cui la registrazione fallisca a causa di credenziali già esistenti. & TS-RS1F 10.5 & \\ \hline 
        RS1F 10.6 & 
        Il servizio MaaS deve mettere a disposizione di un utente non autenticato la possibilità di effettuare il login al sistema. & TS-RS1F 10.6 & \\ \hline 
        RS1F 10.7 & 
        Il servizio MaaS deve visualizzare un messaggio d’errore nel caso in cui l’utente non autenticato abbia inserito credenziali errate nel sistema di login. & TS-RS1F 10.7 & \\ \hline 
        RS1F 10.8 & 
        Il sistema MaaS deve permettere ad un utente non autenticato di modificare il proprio profilo. & TS-RS1F 10.8 & \\ \hline 
        RS1F 10.9 & 
        Il sistema MaaS deve permettere ad un utente non autenticato di eliminare il proprio account dal sistema. & TS-RS1F 10.9 & \\ \hline 
        RS1F 10.9.1 & 
        Il sistema MaaS deve provvedere all’eliminazione dei file di configurazione associati all’utente rimosso dal sistema. & TS-RS1F 10.9.1 & \\ \hline 
        RS1F 10.10 & 
        Il sistema MaaS deve permettere allo sviluppatore di eliminare una Collection esistente. & TS-RS1F 10.10 & \\ \hline 
        RA1D 11 & 
        Deve essere possibile da parte di un utente la registrazione all’Applicazione MaaP. &  & TV-RA1D 11 \\ \hline 
        RA1D 12 & 
        L’utente autenticato nell’applicazione deve poter eseguire il logout. &  & TV-RA1D 12 \\ \hline 
        RA1D 13 & 
        L'utente deve poter modificare le proprie credenziali d'accesso all'interno della propria pagina profilo. &  & TV-RA1D 13 \\ \hline 
        RA1D 13.1 & 
        L’utente deve poter modificare la password di accesso all’applicazione. & TS-RA1D 13.1 & \\ \hline 
        RF1O 14 & 
        Il framework MaaP deve rendere possibile la configurazione dei database di cui dispone. &  & TV-RF1O 14 \\ \hline 
        RF1O 14.1 & 
        Il framework MaaP deve rendere possibile la configurazione dei database delle credenziali. & TS-RF1O 14.1 & \\ \hline 
        RF1O 14.2 & 
        Il framework MaaP deve rendere possibile la configurazione dei database delle Collection. & TS-RF1O 14.2 & \\ \hline 
        RF1F 14.3 & 
        Il framework MaaP deve rendere possibile la selezione di un name-space per un database se la funzione di namespace è abilitata. & TS-RF1F 14.3 & \\ \hline 
        RA1F 15 & 
        L’applicazione MaaP deve mettere a disposizione dell’admin una pagina di gestione degli indici. &  & TV-RA1F 15 \\ \hline 
        RA1F 15.1 & 
        L’applicazione MaaP deve mettere a disposizione dell’admin la visualizzazione degli indici in base alle query più richieste dall’applicazione. & TS-RA1F 15.1 & \\ \hline 
        RA1F 15.2 & 
        L’applicazione MaaP deve permettere all’admin di aggiungere gli indici in base ai suggerimenti forniti. & TS-RA1F 15.2 & \\ \hline 
        RA1F 15.3 & 
        L’applicazione MaaP deve permettere all’admin di rimuovere gli indici in base ai suggerimenti forniti. & TS-RA1F 15.3 & \\ \hline 
        RF1F 16 & 
        Il framework MaaP deve permettere allo sviluppatore di abilitare i namespace per l’applicazione creata. &  & TV-RF1F 16 \\ \hline 
        RS1F 17 & 
        Il sistema MaaS deve verificare se i documenti creati rispettano i vincoli del database. & TS-RS1F 17 & \\ \hline 
        RA1O 18 & 
        Il sistema deve mettere a disposizione un validatore del codice DSL e visualizzare gli eventuali errori logici o di sintassi in un'apposita pagina. & TS-RA1O 18 & \\ \hline 
        RS1F 19 & 
        Il sistema MaaS deve salvare le pagine definite dagli utenti nel database e non su disco. & TS-RS1F 19 & \\ \hline 
    \caption{Tracciamento Requisiti - Test di Sistema e Validazione}
    \end{longtable}
   \egroup
\end{center}
\clearpage
