\section{Configurazione di un file DSL}
\label{collection}

All'interno della cartella \code{collections/} sono presenti alcuni file \code{DSL} di esempio da cui trarre spunto. Come descritto in precedenza è possibile specificare manualmente su quale cartella andare a prelevare i propri file, intervenendo sulla configurazione dell'applicazione tramite il file \code{config.js}. Di default viene specificata questa cartella.

La configurazione di una Collection avviene tramite la configurazione di un file DSL. Questa attività consiste nella semplice creazione o modifica di un file, quindi è possibile procedere alla configurazione tramite un qualsiasi editor di testo disponibile. La configurazione di base deve avere la seguente sintassi:
\medskip

\begin{lstlisting}
collection(
	name: "NomeCollection", 
	label: "CollectionLabel", 
	id: "CollectionId", 
	weight: "CollectionWeight" 
) {
	index( 
		perpage: DocumentsPerPage, 
		populate: AttributePopulate, 
		sortby: "DefaultSort", 
		order: "DefaultSortOrder", 
		query: CollectionQuery
	) {
		column(
			name: "AttributeName", 
			label: "ColumnLabel", 
			sortable: IndexSortable, 
			selectable: IndexSelectable, 
			transformation: TransformationFunction
		)
		...
	}
	show(
		populate: AttributePopulate
	) {
		row(
			name: "AttributeName", 
			label: "RowLabel", 
			transformation: TransformationFunction
		)
		...
	}
} 
\end{lstlisting}


L'applicazione, all'avvio del server, andrà a leggere il contenuto della directory impostata e preleverà sequenzialmente tutti i file con estensione \code{.dsl} contenuti al suo interno. Su questi file avverrà dunque un processo di \glossario{parsing} da parte dell'interprete DSL, il quale si occuperà di interfacciarsi con le \glossario{API} di MaaP e generare tutte le classi e i modelli necessari alla configurazione delle varie Collection indicate. 

Se durante questo processo dovessero verificarsi degli errori relativi all'interpretazione dei file e all'esecuzione del codice prodotto essi verranno registrati e segnalati nell'applicazione.

All'interno del file è possibile definire funzioni e campi dati all'esterno del codice DSL, per poi riferirle all'interno di esso.

Ciascuna \textit{espressione} prende in input una lista di parametri in stile javascript, ovvero: \code{nomeParametro: valoreParametro}. Alcuni parametri sono obbligatori, mentre altri sono facoltativi, e in caso vengano omessi vengono sostituiti con appropriati valori di default.

\subsection{Configurazione di collection}

Questa espressione indica come la Collection dev'essere configurata. Da essa deriveranno le configurazioni della \textit{index-page} e \textit{show-page}.

\begin{itemize}

	\item \textbf{\code{name}} (required): questo parametro accetta una stringa e coincide con il nome della Collection di riferimento sul database \glossario{MongoDB};
	\item \textbf{\code{label}} (optional): questo parametro accetta una stringa e coincide con il nome che verrà visualizzato nella \textit{index-page}. Se non specificato viene automaticamente inizializzato con il valore del parametro \code{name};
	\item \textbf{\code{id}} (optional): questo parametro accetta una stringa e coincide con l'identificativo della \glossario{URI} della Collection. Di default questo parametro assume il valore del parametro \code{name}. Viene utilizzato se si vuole far puntare più configurazioni alla stessa collection su MongoDB;
	\item \textbf{\code{weight}} (optional): questo parametro accetta un valore intero e coincide con l'ordine di visualizzazione della Collection nella barra di navigazione dell'applicazione e nella lista delle collection della \glossario{Dashboard}. Se non specificato assume 0 come valore di default.

\end{itemize}

\subsection{Configurazione di index}

Questa espressione indica coma la index-page dovrà essere configurata. Da questa configurazione deriverà una precisa strutturazione e visualizzazione della pagina.

\begin{itemize}

	\item \textbf{\code{perpage}} (optional): questo parametro accetta una valore intero maggiore di zero e coincide con il numero di Document che verranno visualizzate per ogni pagina. Se il numero totale di Document è maggiore del valore di questo parametro la index-page viene \textit{paginata}. Se non specificato questo parametro prende come valore di default 50;
	\item \textbf{\code{populate}} (optional): questo parametro accetta una stringa e coincide con l'attributo esterno sul quale effettuare la funzione \glossario{populate};
	\item \textbf{\code{sortby}} (optional): questo parametro accetta una stringa e coincide con l'attributo sul quale poter effettuare l'ordinamento di default nel caso in cui non sia specificata nessuna colonna con il parametro \code{sortable: true};
	\item \textbf{\code{order}} (optional): questo parametro accetta una stringa con valore ``\code{asc}'' o ``\code{desc}'' che coincide con la tipologia di ordinamento del parametro \code{sortby}:
	\begin{itemize}

		\item ``\code{asc}'' indica che l'ordinamento verrà fatto in modo \textit{ascendente};
		\item ``\code{desc}'' indica che l'ordinamento verrà fatto in modo \textit{discendente};	
	
	\end{itemize}
	\item \textbf{\code{query}} (optional): questo parametro accetta un oggetto JSON con al suo interno i parametri e i valori sui quali effettuare la \glossario{query} al database sulla Collection in questione.
\end{itemize}

\subsection{Configurazione di column}

Questa espressione indica la configurazione di una colonna della tabella della \textit{index-page}.

\begin{itemize}

	\item \textbf{\code{name}} (required): questo parametro accetta una stringa e coincide con il nome dell'attributo della collection di riferimento su MongoDB;
	\item \textbf{\code{label}} (optional): questo parametro accetta una stringa e coincide con il nome dell'intestazione della colonna nella tabella della \textit{index-page}. Di default, se non specificato, assume il valore del parametro \code{name};
	\item \textbf{\code{sortable}} (optional): questo parametro accetta un valore booleano che, se uguale a \code{true}, indica che la \textit{index-page} può essere ordinata secondo questa colonna. Se non specificato assume \code{false} come valore di default;
	\item \textbf{\code{selectable}} (optional): questo parametro accetta un valore booleano che, se uguale a \code{true}, indica che l'elemento può essere selezionato tramite un link, il quale rimanderà alla relativa \textit{show-page} del Document selezionato. Se non specificato assume \code{false} come valore di default;
	\item \textbf{\code{transformation}} (optional): questo parametro coincide con una funzione di trasformazione sull'elemento. La funzione deve restituire un valore, il quale sovrascriverà il valore dell'elemento estratto dalla query.
\end{itemize}

\subsection{Configurazione di show}

Questa espressione indica come la \textit{show-page} dovrà essere configurata. Da questa configurazione deriverà una precisa strutturazione e visualizzazione della pagina.

\begin{itemize}

	\item \textbf{\code{populate}} (optional): questo parametro accetta una stringa e coincide con l'attributo esterno sul quale effettuare la funzione \glossario{populate}.

\end{itemize}

\subsection{Configurazione di row}

Questa espressione indica la configurazione di una riga della tabella della \textit{show-page}.

\begin{itemize}

	\item \textbf{\code{name}} (required): questo parametro accetta una stringa e coincide con il nome dell'attributo della collection di riferimento su MongoDB;
	\item \textbf{\code{label}} (optional): questo parametro accetta una stringa e coincide con il nome dell'intestazione della riga nella tabella della \textit{show-page}. Di default, se non specificato, assume lo stesso valore del parametro \code{name};
	\item \textbf{\code{transformation}} (optional): questo parametro coincide con una funzione di trasformazione sull'elemento. La funzione deve restituire un valore, il quale sovrascriverà il valore dell'elemento estratto dalla query.

\end{itemize}



