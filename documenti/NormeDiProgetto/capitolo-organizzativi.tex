\section{Processi organizzativi}

\subsection{Processo di Pianificazione}

	\subsubsection{Scopo del processo}
	Lo scopo di questo processo è la produzione e la comunicazione del piano di progetto ai membri del gruppo. Questo processo determina l'ambito della gestione del progetto e le attività tecniche, identifica gli output del processo e stabilisce un calendario.

	\subsubsection{Risultati osservabili del processo}
	Questo processo produce:
	\begin{itemize}
		\item l'identificazione dell'ambito del progetto;
		\item l'analisi di fattibilità degli obiettivi prefissati;
		\item i task necessari per completare il lavoro sono stati stimati;
		\item un piano per l'esecuzione del lavoro programmato.
	\end{itemize}

	\subsubsection{Descrizione}
Come convenzione il gruppo ha scelto una soglia oraria giornaliera in cui lavorare al progetto. In particolare l'orario di lavoro giornaliero è da considerarsi dalle 9:00 alle 12:00 e dalle 13:00 alle 17:00, per un massimo di 7 ore. Le ore impiegate oltre queste soglie sono considerate ore di \textbf{straordinario}. In particolare al Responsabile è raccomandato pianificare due ore di lavoro al giorno, per un totale di 10 ore settimanali.
		
Il Responsabile, nell'assegnare, task terrà conto di quanto inserito nel calendario, come indicato nel capitolo \ref{Calendario condiviso}, e di quanto indicato nel capitolo \ref{rotazioneruoli} riguardo la rotazione dei ruoli. 

\subsubsection{Rotazione dei ruoli}
\label{rotazioneruoli}
Durante lo sviluppo del progetto vi sono diversi ruoli che ogni membro del gruppo \GroupName{} è tenuto a ricoprire almeno una volta. Per evitare possibili conflitti causati dalla rotazione dei ruoli, le attività principali assegnabili a specifici ruoli sono pianificate nel \PianoDiProgetto. Ogni componente del gruppo è tenuto a svolgere le attività assegnategli e a rispettare il ruolo che ne consegue. Il \textit{Responsabile} di progetto ha il compito di fare rispettare i ruoli assegnati durante le attività, mentre il \textit{Verificatore} deve individuare le possibili incongruenze tra i ruoli e le modifiche registrate nei diari delle modifiche.
Le incongruenze possono essere di due tipi:
\begin{itemize}
\item Il compito svolto non fa parte dei compiti propri di un dato ruolo;
\item La stessa persona verifica ciò che ha precedentemente prodotto.
\end{itemize}

Inoltre, si raccomanda che:
\begin{itemize}
\item Una persona non impieghi più del 50\% delle ore di lavoro in un unico ruolo;
\item Sia assegnato un solo task al giorno per persona.
\item I ruoli vengano ruotati ogni settimana, ma si lascia libertà al \textit{Responsabile} di ruotare in base alle esigenze.
\end{itemize}

Si predilige dunque una maggior frequenza di rotazione a discapito dell'efficienza.

Il responsabile quando assegna un ticket, come indicato nella sezione \ref{teamwork} dovrà indicare il ruolo che il destinatario ricopre nell'adempiere tale task. Questo è necessario per il diario delle modifiche dei singoli documenti e per il rendiconto economico. I ruoli sono assegnati basandosi sul \PianoDiProgetto. 

Deve inoltre tener conto delle non disponibilità a lavorare come indicato nella sezione \ref{Calendario condiviso}. \\
A supporto di tali raccomandazioni si automatizzano le verifiche utilizzando alcune funzionalità predisposte dallo script descritto nella sezione \ref{Gantt}.
	\paragraph{Amministratore}
	
	L'\textit{Amministratore} equipaggia, organizza e gestisce l'ambiente di lavoro e di produzione. Collabora con il \textit{Responsabile} alla stesura del \textit{Piano di Progetto} e redige le \textit{Norme di Progetto}.
	Le responsabilità assunte dall'\textit{Amministratore} sono:

	\begin{itemize}

		\item Attuare le scelte tecnologiche concordate con il \textit{Responsabile} di progetto;
		\item Gestire le liste di distribuzione e assicurarne il rispetto;
		\item Controllare versioni e configurazioni del prodotto;
		\item Risolvere i problemi legati alla gestione dei processi.

	\end{itemize}

	\paragraph{Analista}

	L'\textit{Analista} è responsabile dell'attività di analisi, pertanto deve comprendere appieno il dominio applicativo. Redige lo \textit{Studio di Fattibilità} e l'\textit{Analisi dei Requisiti} ovvero una specifica di progetto ad alto livello con vincoli e rischi tecnologici, affrontabile dal proponente, dal committente e dal \textit{Progettista}.


	\paragraph{Progettista}

	Il \textit{Progettista} è responsabile dell'attività di progettazione, ha una profonda conoscenza dello stack tecnologico utilizzato e competenze tecniche aggiornate. Grazie a tali caratteristiche, ha una forte influenza sugli aspetti tecnici e tecnologici del progetto, e spesso ne assume responsabilità di scelta e gestione.


	\paragraph{Programmatore}

	Il \textit{Programmatore} ha responsabilità circoscritte, si occupa dell'attività di codifica, nel rispetto delle \textit{Norme di Progetto}, miranti alla realizzazione del prodotto e delle componenti di ausilio necessarie per l'esecuzione delle prove di verifica e validazione.


	\paragraph{Responsabile}
	
	Il \textit{Responsabile} ha l'ultima voce in capitolo per quanto concerne le decisioni sul progetto, è il responsabile ultimo dei risultati, infatti approva l'emissione dei documenti. Inoltre, redige il \textit{Piano di Progetto} assieme all'\textit{Amministratore}. 
	Le responsabilità assunte dal \textit{Responsabile} sono:

	\begin{itemize}

		\item Pianificazione e organizzazione dello sviluppo del progetto, stima tempi e costi, e assegnazione delle attività ai componenti del gruppo;
		\item Riportare lo stato del progetto ai committenti;
		\item Analizzare i rischi che possono incorrere, monitorarli e prendere provvedimenti a riguardo;
		\item Stabilire una \textit{ways of working} per ogni componente del gruppo, ai fini di un'influenza positiva delle performance del gruppo.
	
	\end{itemize}

	
	\paragraph{Verificatore}

	Il \textit{Verificatore} organizza ed attua le attività di verifica e controlla che le attività siano conformi alle norme. Redige la parte del \textit{Piano di Qualifica} che documenta le attività svolte e i risultati ottenuti, confrontandoli con le metriche espresse nel \PianoDiQualifica. \\ 
	Le ore assegnate al verificatore devono corrispondere almeno al 30$\%$ delle ore totali suddivise tra ruoli.
	
	


\subsubsection{Revisioni di progetto}
La vita di questo progetto didattico attraversa le revisioni di progetto specificate in \PianoDiProgetto{}, esse consistono in \emph{review}(informali) e \emph{audit} (formali) ma le modalità con cui vengono affrontate sono, da parte del gruppo, le medesime, verranno quindi in seguito denominate presentazioni o revisioni. Vengono qui descritte le procedure per le revisioni secondo quanto descritto in IEEE 1028:
\begin{enumerate}
	\setcounter{enumi}{-1}
	\item Entry evaluation: il Responsabile di Progetto prepara una checklist e si assicura che esistano le condizioni che permettano il successo della presentazione;
	\item Management preparation: il Responsabile si assicura che tutti i membri del gruppo possano partecipare e l'ambiente di lavoro rispetti i vincoli imposti o suggeriti dal committente. A tal fine è stato elaborata una lista di indicazioni (\ref{presentazione});
	\item Planning the review: il Responsabile deve identificare e confermare gli obiettivi della revisione, inoltre si assicura che tutto il team sia equipaggiato con le risorse necessarie per affrontare la revisione;
	\item Overview of review procedures: ci si assicura che tutti i membri del gruppo conoscano gli obiettivi, le procedure e il materiale portato in revisione;
	\item Individual preparation: ogni membro del gruppo si prepara individualmente per la presentazione;
	\item Group examination: il gruppo si incontra al completo e prova la presentazione confermando il rispetto dei tempi e degli obiettivi;
	\item Rework/follow-up: vengono corretti, laddove possibile,  gli errori e/o i difetti;
	\item Exit evaluation: il Responsabile si assicura che tutti gli output prodotti siano pronti per la revisione.
\end{enumerate}

\label{presentazione}
Le indicazioni su come costruire una presentazione sono le seguenti:
\begin{itemize}
	\item le diapositive devono avere la stessa densità: il rapporto tra testo, immagini e spazio vuoto deve essere uniforme;
	\item evitare testi troppo lunghi, immagini piccole o troppo dettagliate;
	\item non va spiegata la sola teoria: argomentare come è stata implementata nel progetto;
	\item la lettura delle slide deve avvenire rivolgendosi al pubblico e non rivolti verso la diapositiva; si può usare con parsimonia lo schermo del portatile per avere un riferimento;
	\item non ripetere quanto scritto nelle diapositive, è inutile;
	\item il passaggio da una diapositiva all'altra deve avvenire  tra un tempo minimo e uno massimo in modo da consentirne la lettura e non annoiare chi ascolta;
	\item le mani non devono stare nelle tasche;
	\item ogni slide deve riportare il proprio numero sul totale delle diapositive.
\end{itemize}




% 6.3.1 Project Planning Process