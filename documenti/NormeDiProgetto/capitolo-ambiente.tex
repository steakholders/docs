\section{Ambiente di lavoro}
	
	\subsection{Ambiente Generale}
		
		\subsubsection{Sistema operativo}
		
		Il progetto verrà sviluppato su sistemi \textbf{Linux}, per la facilità con cui è possibile preparare l'ambiente di lavoro adatto allo sviluppo dell'applicazione descritta dal capitolato. In particolare è raccomandato il sistema operativo \textbf{Ubuntu} $\geq 12.04$.
%		\subsubsection{Installazione componenti aggiuntivi}
		
		\subsubsection{Codifica dei caratteri}
		
		Per assicurarsi la corretta visualizzazione dei caratteri accentati tutti i file testuali presenti nel \glossario{repository} devono essere memorizzati con la codifica \textbf{\glossario{UTF-8}}.
		
		\subsubsection{Versionamento}
		
		Per il versionamento dei documenti e del codice viene usato \textbf{Git} (\url{http://git-scm.com/}).
		Si utilizzerà il servizio \textbf{GitHub} per creare e gestire i repository privati utilizzati dal gruppo.
		
	\subsection{Coordinamento}
	\label{teamworkpm}
	
		Per coordinare le attività, gli eventi, per le comunicazioni e per il conteggio delle ore viene utilizzata la piattaforma \textbf{Teamwork Project Manager} (\url{http://teamworkpm.net})
		\begin{itemize}
			\item Offre un'elevata portabilità ed accessibilità essendo una piattaforma online;
			\item Offre gratuitamente i servizi necessari;
			\item Mette a disposizione delle \glossario{API} per comunicare con la piattaforma ed estenderne eventualmente le funzionalità.
		\end{itemize}
		
		L'indirizzo per accedere all'ambiente di lavoro riservato al gruppo è
		\begin{center}
			\url{https://steakholders.teamworkpm.net}
		\end{center}
		
		\subsubsection{Calendario condiviso}
		\label{Calendario condiviso}
		
		Il gruppo si avvale del calendario condiviso offerto da \glossario{TeamworkPM} relativo al progetto \emph{Ingegneria del software}. È possibile sottoscrivere in modalità \emph{sola lettura} il \emph{Calendario condiviso} con i software di calendario preferito, in particolare Google Calendar e Apple Calendar; per la prima soluzione è disponibile una guida sulle \emph{FAQ} di TeamworkPM; per la seconda il calendario è direttamente sottoscrivibile aprendo via browser l'\emph{URL} \url{webcal://steakholders.teamworkpm.net/feeds/ical/calendar/everybody/?uid=EB9106B928A9810A294CEAA17B406B802E7349A2603B3F461ECB076A2DA57850}.
		
		Ogni membro del gruppo ha la possibilità di visualizzare solo gli eventi per i quali risulta \emph{partecipante}, inoltre può esportare un calendario personale con modalità simili a quelle descritte in precedenza.
		
		\textit{Nota:} per comodità, prima di effettuare qualsiasi operazione sul calendario bisogna assicurarsi di aver selezionato \emph{All projects} tramite il menù a tendina collocato in alto a destra. Questa norma è applicabile perché siamo tutti membri della stessa azienda e lavoriamo ad un solo progetto; in questo modo si velocizzano le operazioni di inserimento nel calendario degli eventi.
		
		\subsubsection{Gestione dei ticket}
		
		Per la gestione dei ticket si utilizzerà il sistema di task offerto da \glossario{TeamworkPM}, utilizzabile unicamente attraverso le procedure descritte nella sezione \ref{procedurediprogetto}.
		
		\subsubsection{Gestione del piano di lavoro}
		
		Al fine di pianificare le attività da svolgere per lo sviluppo del progetto, il gruppo si è affidato alla piattaforma \glossario{TeamworkPM} per i seguenti motivi:
		\begin{itemize}
			\item Genera automaticamente un grafico \glossario{Gantt} che visualizza e permette di modificare i task pianificati, gestendo anche le relative dipendenze;
			\item Permette di esportare il grafico \glossario{Gantt} in un formato compatibile con \glossario{GanttProject} e \glossario{Microsoft Project};
			\item Permette di gestire le \glossario{Milestone};
			\item Permette di registrare il tempo di lavoro trascorso su ogni task.
		\end{itemize}

		\subsubsection{Gestione degli eventi}
		
		Gli eventi di interesse collettivo vengono inseriti nel calendario dall'Amministratore sempre e solo nel \emph{Calendario condiviso} descritto in \ref{Calendario condiviso}.
		
		Le tipologie di eventi sono:
		\begin{itemize}
	  		\item \textbf{Riunioni}: vanno programmate almeno con 48 ore di anticipo tenendo conto della disponibilità dei membri. Ogni evento riunione avrà un ora di inizio e fine, un luogo e la lista dei membri che hanno confermato la partecipazione. L'\emph{ordine del giorno} dovrà essere compilato nella descrizione dell'evento;
	  		%In futuro potremo decidere di stilare l' \emph{ordine del giorno} nel template del verbale delle riunioni.
	  		\item Le \textbf{Revisioni} previste dal committente;
	  		\item \textbf{Non disponibilità}: un membro del gruppo dichiara di non essere disponibile a svolgere attività legate al progetto. Per creare un evento di \emph{Non disponibilità} bisogna, oltre a compilare i soliti campi (titolo, descrizione, data e ora), impostare la visibilità a \emph{tutti i membri della propria azienda}, selezionare la categoria \emph{Non disponibilità} e segnare la persona interessata. In questo modo tutti i membri del gruppo \GroupName{} potranno visualizzare tale evento. 
	  		Il titolo sarà nel formato \textbf{[ND] \{nome membro\}} ossia le iniziali di \emph{Non disponibilità} racchiuse tra parentesi quadre, seguite da il nome del membro del gruppo. Questo formato permetterà di individuare con immediatezza i giorni nei quali non è possibile fissare delle riunioni.
		\end{itemize}
	
	\subsection{Ambiente di produzione dei documenti}
	
		\subsubsection{Scrittura}
		
		Per la stesura dei documenti verrà utilizzato il linguaggio \LaTeX{} (\url{http://www.latex-project.org}).
		Per la stesura sono raccomandati i seguenti editor:
		\begin{itemize}
		 \item \textbf{TexMaker} $\geq 3.2$ (\url{http://www.xm1math.net/texmaker})
		 \item \textbf{Kile} $\geq 2.1.3$ (\url{http://kile.sourceforge.net})
		\end{itemize}
		
		L'output dei documenti sarà in formato \glossario{PDF} e verrà prodotto attraverso il comando \code{pdflatex} (versione $\geq$ 3.1415926-1.40.10-2.2). Per velocizzare questa operazione è stato predisposto il comando \code{make documents}, il cui utilizzo è descritto nella sezione \ref{makefile}.
		
		\subsubsection{Controllo ortografico}
		
		Per aiutare il controllo ortografico verrà utilizzato il software \emph{Aspell} (\url{http://aspell.net}, versione $\geq 0.60.7$) con il dizionario italiano.
		
		Per controllare un file \LaTeX{} con \emph{Aspell} l'utilizzo da terminale è il seguente:
\begin{lstlisting}
aspell --lang it --mode tex --encoding utf-8 --personal root/script/aspell_personal.txt --repl root/script/aspell_replacements.txt check {nome del file da controllare}
\end{lstlisting}
		
		Per comodità è stato predisposto il comando \code{make check}, come descritto nella sezione \ref{makefile}.
		
		Quando Aspell segnala un errore su una parola:
		\begin{itemize}
		 \item Se la parola ha un errore ortografico, correggerla scegliendo una delle parole proposte da Aspell oppure usando il comando \textbf{``rimpiazza''};
		 \item Se la parola è scorretta e deve essere modificata tutta la frase, selezionare il comando \textbf{``abbandona''} e fare la modifica utilizzando una \glossario{IDE};
		 \item Se l'errore segnalato da Aspell è un falso positivo, selezionare il comando \textbf{``aggiungi''}.
		\end{itemize}

		
		\subsubsection{UML}
		
		Per la produzione dei diagrammi \emph{\glossary{UML}} verrà utilizzata la piattaforma \emph{Lucidchart} (\url{https://www.lucidchart.com}).
		
		È stato valutato anche il software \emph{Astah Professional} (\url{http://astah.net/editions/professional}), utilizzabile con una licenza accademica accademica gratuita, ma è stato preferito Lucidchart per la facilità con cui è possibile collaborare online.
		
%		\subsubsection{Validazione}
%		\subsubsection{???Database???}

		\subsubsection{Script di Makefile}
		\label{makefile}

Per agevolare alcune operazioni è stato predisposto uno script \glossario{Makefile}. Al momento sono previsti due utilizzi:
\begin{itemize}

\item \textbf{\code{make test}} \\
Per ogni file \file{*.tex} contenuto nella cartella e nelle sue sotto-cartelle:
\begin{itemize}
	\item Verifica che i file siano memorizzati con la codifica \glossario{UTF-8};
	\item Verifica con \emph{Aspell} che le parole utilizzate nei documenti siano comprese nel dizionario italiano di \emph{Aspell} o nel dizionario personalizzato \file{root/script/aspell\_personal.txt}.
\end{itemize}
	
\item \textbf{\code{make documents}} \\
Compila tutti i documenti presenti nelle sotto-cartelle.

\item \textbf{\code{make test-glossary}} \\
Controlla che tutti i termini marcati con il comando \code{\\glossario\{\dots\}} siano definiti da un corrispondente comando \code{\\definizione\{\dots\}}.

\item \textbf{\code{make test-regexp}} \\
Per ogni regola della forma \code{grep\_test \{RegExp\} \{Messaggio\} \{File\}} definita internamente allo script controlla se ci sono occorrenze dell'espressione regolare \code{\{RegExp\}} nei file \code{\{File\}}. Per ciascuna occorrenza visualizza il messaggio \code{\{Messaggio\}} seguito dal nome del file e la linea in cui è stata trovata l'occorrenza. È raccomandato usarlo per la verifica dei documenti.

\end{itemize}

È possibile eseguire questo script dalle cartelle dei documenti e dalle loro cartelle superiori, fino alla cartella principale del repository.

	\subsection{Ambiente di sviluppo}
		
		\subsubsection{Stesura del codice}
		
		Per la stesura del codice è consigliabile usare i seguenti editor:
		\begin{itemize}
			\item \textbf{gedit} $\geq 3.4.1$ (\url{http://projects.gnome.org/gedit})
			\item \textbf{Sublime text} $\geq 2.0.2$ (\url{http://www.sublimetext.com})
		\end{itemize}
		
		\subsubsection{Framework}
		
		Per lo sviluppo del progetto è previsto l'utilizzo del server \textbf{Node.js} $\geq 0.10.22$ e del relativo \glossario{packet manager} \textbf{npm} $\geq 1.3.6$.

	\subsection{Ambiente di verifica e validazione}
		\subsubsection{Analisi statica}
		
		Per l'analisi statica del codice Javascript e per aiutare a identificare automaticamente gli errori verranno usati i programmi \textbf{JSHint} (\url{http://www.jshint.com}), \textbf{JSLint} (\url{http://www.jslint.com}) e il \textbf{Closure Compiler} (\url{http://developers.google.com/closure/compiler/}).
		
		%\subsubsection{Analisi dinamica}
		
		\subsubsection{Test}
		
		Per i test di unità verrà utilizzata la libreria \textbf{Mocha} (\url{https://github.com/visionmedia/mocha})
		
		%\subsubsection{Validazione}
		
		%Per maggiori dettagli si rimanda al documento Piano di Qualifica v1.0.
