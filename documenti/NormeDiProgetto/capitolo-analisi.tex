\section{Analisi dei requisiti}

Dal capitolato e dagli incontri con il proponente gli Analisti dovranno estrarre i requisiti del progetto, producendo l'Analisi dei Requisiti.

Tutti i requisiti devono essere riconducibile al capitolato o ad un incontro con il proponente. I requisiti obbligatori identificati, se rispettati, dovranno essere necessari e sufficienti per l'approvazione del prodotto.

\subsection{Requisiti}
% TODO controlla che corrispoda a quello usato, in treno non potevo guardarli

Ogni requisito è identificato da un nome, che segue il seguente formalismo:
\begin{center}
	\code{R\{Ambito\}\{Gerarchia\}}
\end{center}

Dove
\begin{itemize}
 \item \textbf{Ambito} può essere:
	\begin{itemize}
	 \item \textbf{U} se è un \emph{requisito utente};
	 \item \textbf{S} se è un \emph{requisito di sistema}.
	\end{itemize}

 \item \textbf{Gerarchia} identifica in modo univoco la posizione che il requisito ha nella gerarchia in cui sono posti tutti i requisiti di uno stesso ambito.
\end{itemize}

\subsection{UML}

Per i diagrammi deve essere utilizzata il linguaggio \emph{UML versione 2.0}.