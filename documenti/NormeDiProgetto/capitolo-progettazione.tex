\subsection{Progettazione}
%\subsubsection{Specifica Tecnica}
La progettazione deve dimostrabilmente rispettare tutti i requisiti che il gruppo ha concordato con il committente. In particolare i componenti progettati devono essere tracciabili rispetto al requisito che soddisfano.
Di seguito vengono elencate le norme a carico dei \emph{Progettisti}.

	\subsubsection{Diagrammi UML}
	Si dovrà usare il linguaggio \glossario{UML} \emph{versione 2.0} per i seguenti diagrammi:
\begin{itemize}
 \item \textbf{Diagrammi dei package:} dovranno essere presenti sia per l'architettura generale che di dettaglio, sarà fondamentale per definire i moduli all'interno del \glossario{framework} \glossario{Node.js} richiesto dal capitolato;
 \item \textbf{Diagrammi delle classi:} qualora il progetto utilizzasse delle classi, i diagrammi delle classi dovranno essere presenti sia per l'architettura generale che di dettaglio. Nell'ambiente \glossario{Node.js} a prima vista sembra che siano poco utilizzate, a favore dei \glossario{package};
 \item \textbf{Diagrammi di flusso:} qualora la codifica di un'unità del progetto sia particolarmente complessa, dovrà essere presente il relativo diagramma di flusso che il programmatore dovrà seguire;
\end{itemize}

	\subsubsection{Stile di progettazione}
	\begin{itemize}
		\item La progettazione dovrà usare quanto più possibile \glossario{design pattern} globalmente affermati, la loro scelta dovrà essere giustificata;
		\item Suddividere il progetto in \glossario{moduli}, in accordo con lo stile di progettazione dell'ambiente \glossario{Node.js};
		\item Non utilizzare codice sincrono per operazioni di \glossario{I/O};
		\item Limitare quanto più possibile le \glossario{callback} annidate;
	\end{itemize}


	%\paragraph{Test di integrazione}
	%I test di integrazione avvengono mediante la definizione e l'utilizzo di opportuni strumenti di test per verificare che i componenti del sistema funzionino nella maniera prevista.


%\subsubsection{Definizione di Prodotto}
%La \emph{Definizione di Prodotto} è la descrizione dettagliata della progettazione descritta nella \emph{Specifica Tecnica}. Questo documento rappresenta lo stadio avanzato della progettazione e indicherà ogni singolo componente del sistema permettendo ai \emph{Programmatori} di procedere con lo sviluppo. Parallelamente alla progettazione di dettaglio dovranno essere progettati i relativi test di unità che verranno descritti nel \PianoDiQualifica .

%	\paragraph{Definizione di classe}
%	Le classi progettate devono comparire nella \emph{Definizione di Prodotto} che dovrà descriverne l'elenco dei metodi e degli attributi, lo scopo e la funzionalità che rappresenta.
%	completare...
%	
%	% \subsubsection{Formalismo di specifica dei metodi}
%	 \subsubsection{Tracciamento classi}
%	
%	\paragraph{Test di unità}
%	I \emph{Progettisti} modellano i test da effettuare sulle unità per le opportune verifiche.



%\subsubsection{Verifica della progettazione architetturale}
%\subsubsection{Verifica della progettazione di dettaglio}
%\subsubsection{Verifica del codice}
%\paragraph{Analisi Statica}
%\paragraph{Analisi Dinamica}
%\subsection{Resoconto bug}
%\subsection{Validazione output}

%\subsection{Codifica}
%	\subsubsection{Codifica e convenzioni}
%		\paragraph{Nomi}
%		\paragraph{Ricorsione}
%	\subsubsection{Documentazione}
	
	
