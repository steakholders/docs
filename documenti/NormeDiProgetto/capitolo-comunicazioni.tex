\section{Comunicazioni e Riunioni}

\subsection{Interne}
\label{Comunicazioniinterne}

Tutte le comunicazioni ad uso interno del gruppo avvengono tramite il servizio di messaggistica offerto da \emph{TeamworkPM}, altre vie di comunicazione sono sconsigliate.

Nel caso sia particolarmente necessario comunicare in modo rapido possono essere utilizzati strumenti di messaggistica istantanea come Hangouts o Skype. L'utilizzo di \glossario{SMS} e di chiamate telefoniche è riservato alle situazioni di urgenza.

\subsubsection{Messaggi}
\label{Comunicazioniinternemessaggi}

\begin{itemize}
\item L'\textbf{oggetto} deve essere sintetico e coerente con il contenuto del messaggio;
\item Il \textbf{contenuto} deve includere i dettagli necessari per una corretta comprensione del messaggio e non deve essere prolisso. Il mittente può servirsi del supporto al linguaggio di \glossario{Markdown} disponibile sulla piattaforma per rendere più chiaro il suo messaggio;
\item La \textbf{categoria} deve essere coerente con l'argomento trattato. Può essere creata una categoria nuova se ritenuto necessario;
\item La \textbf{notifica} via mail deve includere gli interessati al messaggio;
\item Il livello di \textbf{privacy} deve sempre essere ``Everybody on project'' in modo da permettere a tutti i componenti del gruppo di intervenire;
\item Data l'esigua dimensione dello \glossario{storage} che offre la piattaforma nella versione free utilizzata dal gruppo e l'assenza di visualizzatori online integrati, il numero e la dimensione degli \textbf{allegati} devono essere ridotti quanto più possibile.
\end{itemize}

\subsubsection{Commenti a tasks o sub-tasks}

Per tali comunicazioni valgono le regole del paragrafo \ref{Comunicazioniinternemessaggi}. Il contenuto del commento deve riguardare la task che riferisce. Se la discussione si sviluppa in argomenti non più coerenti, i componenti del gruppo \GroupName{} devono terminare la discussione, aprire un messaggio secondo le norme descritte al paragrafo \ref{Comunicazioniinternemessaggi} e inserire come nuovo commento alla discussione interrotta il link al messaggio creato con l'aggiunta della segnatura ``[OT]''. Questo non vincola il proseguimento della discussione sulla task secondo le norme.

\subsection{Esterne}
\label{email}

Le comunicazioni esterne sono gestite esclusivamente dal Responsabile di Progetto. A tal fine è stato creato l'indirizzo di posta elettronica
\begin{center}steakholders.group@gmail.com \end{center}

Il Responsabile di Progetto si fa dunque carico di notificare ai restanti membri del gruppo eventuali corrispondenze
intrattenute con committenti e proponenti, applicando le norme stabilite al paragrafo \ref{Comunicazioniinterne}.

\subsection{Riunioni}

Qualora fosse necessaria una riunione di tutti o alcuni membri del gruppo sarà compito del Responsabile di Progetto avvisare gli interessati, tramite le procedure stabilite al paragrafo \ref{Comunicazioniinterne}.
Il Responsabile di Progetto decide inoltre luogo, date e ora della riunione in base al calendario a sua disposizione. Nel caso in cui qualche membro non risponda entro 24 ore il Responsabile di Progetto dovrà accertarsi con mezzi di comunicazione adeguati che tutti siano stati informati.

Ad ogni riunione verrà prodotto un verbale redatto da un segretario, ruolo svolto a turno da ogni membro del gruppo e deciso di volta in volta dal Responsabile di Progetto. Tale verbale, descritto nel paragrafo \ref{verbale} dovrà essere reso disponibile per la consultazione a tutti i membri del gruppo.

È inoltre presente un facilitatore, ruolo svolto a turno da ogni membro del gruppo e deciso di volta in volta dal Responsabile di Progetto.
% Il compito del facilitatore è di garantire che tutti gli argomenti stabiliti vengano trattati e che tutti abbiano la possibilità di prendere parola.
% TODO il facilitatore assicura che l'ordine del giorno venga seguito?
