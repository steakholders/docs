\section{Introduzione}


	\subsection{Scopo del documento}
	Questo documento è rivolto all'utente, ha lo scopo di illustrare le procedure da seguire per svolgere le operazioni utente di \glossario{MaaP}. All'utilizzatore non è richiesta alcuna conoscenza informatica poiché dovrà interfacciarsi, tramite web browser, alle funzionalità di \glossario{MaaP} che vengono erogate con le stesse modalità di un normale sito internet.

	\subsection{Scopo del prodotto}
	\ScopoDelProdotto{}

	\subsection{Prerequisiti}
	L'utente deve possedere una connessione ad internet, un web browser (Chrome $\geq$ 30.0.x, Firefox $\geq$ 24.x), una piattaforma che fornisca \glossario{MaaP} come servizio raggiungibile tramite un \texttt{url} e le relative credenziali di accesso.

	% \subsection{Come leggere il manuale}
	
	\subsection{Come riportare problemi e malfunzionamenti}
	La segnalazione di problemi e malfunzionamenti avviene creando una \glossario{issue} di \glossario{GitHub} tramite l'apposito pulsante collocato nel \glossario{fotter} di ogni pagina. L'oggetto della \glossario{issue} dovrà essere conciso e descrittivo del problema riscontrato. La descrizione del problema dovrà contenere i seguenti punti:

	\begin{enumerate}
		\item Contatto della persona che ha riscontrato il problema;
		\item Riferimento temporale;
		\item Azioni e pagine in cui si è riscontrato il problema;
		\item Codice dell'errore se visualizzato;
		\item Note aggiuntive utili alla comprensione del problema.
	\end{enumerate}

	\subsection{Glossario}
	Ogni occorrenza di termini tecnici, di dominio e gli acronimi sono marcati con una ``G'' in pedice. Le relative definizioni sono collocate in appendice.




	% se necessario
	% \subsection{Riferimenti}
	% 	\subsubsection{Informativi}
	% 		\begin{itemize}
	%   			\item
	% 		\end{itemize}