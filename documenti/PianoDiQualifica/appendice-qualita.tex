\pagebreak
\section{Qualità}
La qualità perseguita nel presente documento si basa sugli standard ISO/IEC 15504  e ISO/IEC 9126 con l'obiettivo di approfondirne incrementalmente la copertura. Tutti i processi seguono il metodo \glossario{PDCA}

	\subsection{Qualità dei processi}
	Definita in ISO/IEC 15504 come \glossario{SPICE}, specifica come la qualità è collegata alla maturazione dei processi, vengono individuati dei livelli di maturità al quale il fornitore può fare riferimento per determinare le proprie capacità organizzative. Vengono definiti:
	\begin{itemize}
		\item dei \textbf{Modelli di riferimento} su:
			\begin{itemize}
				\item \emph{dimensione del processo};
				\item \emph{livelli di capacità dei processi}:
					\begin{etaremune}
						\item ottimizzato
						\item predicibile
						\item stabilito
						\item gestito
						\item eseguito
						\item incompleto
					\end{etaremune}
					La capacità di un processo viene misurata tramite degli attributi che sono assimilabili alle metriche dei processi individuate in \ref{MetricheProcessi}, in particolare la \emph{Schedule Variance} permette di capire se un processo è incompleto o gestito; il gruppo giungerà a maturazione quando i processi diventeranno predicibili ossia quando la \emph{Schedule Variance} subirà al più lievi oscillazioni.
			\end{itemize}
		\item delle \textbf{Stime} che si concretizzano in una struttura per la misurazione composta da:
			\begin{itemize}
				\item i \emph{processi} di misurazione, indicati nel \PianoDiProgetto ;
				\item un \emph{modello} per la misurazione, identificabile in questo documento;
				\item gli \emph{strumenti} utilizzati, specificati nelle \NormeDiProgetto .
			\end{itemize}
		\item le \textbf{Competenze e Qualifiche} di chi controlla; lo standard redige in modo rigoroso una serie di attività volte a formare chi opera l'attività di stesura del Piano di Qualifica e \emph{Verifica}. Tali competenze sono assenti all'interno del gruppo e considerato che effettuare una formazione in linea con quanto specificato dallo standard sarebbe impossibile, tutti i membri si impegnano a studiare ed applicare al meglio quanto descritto in questo documento.
	\end{itemize}
	
	\subsection{Qualità del prodotto software}
	Specificata in ISO/IEC 9126 si suddivide in:
	\begin{itemize}
		\item \textbf{Quality model}: classifica la qualità del software in un set di caratteristiche che verranno approfondite nel corso del progetto:
			\begin{itemize}
				\item Functionality;
				\item Reliability;
				\item Usability;
				\item Efficiency;
				\item Maintainability;
				\item Portability.
			\end{itemize}
			Allo stato attuale, il gruppo non è in grado individuare un modello specifico per lo \glossario{stack} tecnologico da utilizzare.
		\item \textbf{External metrics}: sono le metriche rilevate tramite analisi dinamica, verranno specificate con il concretizzarsi della \emph{Specifica Tecnica};
		\item \textbf{Internal metrics}: sono le metriche rilevate in analisi statica specificate in \ref{MisureMetriche};
		\item \textbf{Quality in use metrics}: si tratta di metriche rilevabili allo stato di prodotto \emph{usabile} in condizioni reali, si rimanda la definizione di tale aspetto a quando verranno trattate le considerazioni sull'usabilità del prodotto in uno scenario di utilizzo reale, questo deve avvenire non oltre la \emph{Progettazione di Dettaglio e Codifica}.
	\end{itemize}

