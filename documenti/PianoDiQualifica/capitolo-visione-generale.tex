\clearpage
\section{Visione generale della strategia di verifica}
Il gruppo intende applicare le strategia di verifica ai processi descritte in questo documento. L'obiettivo è avere un riscontro affidabile e numericamente trattabile che permetta di assicurare il grado di qualità predeterminato. La strategia generale adottata è quella di automatizzare il più possibile il lavoro di verifica; questo richiede scelta e uso di \emph{tools} adeguatamente configurati. Il lavoro manuale verrà così ridotto al minimo e confinato all'opera di validazione.
La speranza è che dei buoni processi portino ad un buon software.
	
	\subsection{Organizzazione}
	Viene verificata la qualità di ogni processo e di ogni output da esso prodotto. Ogni fase  descritta nel \emph{Piano di Progetto} produce output di diverso tipo, per questo è necessario programmare l'attività di verifica in modo mirato:

	\begin{itemize}
		\item \textbf{Analisi:} in questa fase si controllano che i processi e la documentazione prodotta rispettino le \emph{Norme di Progetto}.
		\item \textbf{Progettazione Architetturale:} in questa fase vanno verificati i processi incrementali relativi all'analisi e ai nuovi documenti di progettazione.
		\item \textbf{Progettazione dettaglio e codifica:} in questa fase vanno verificati i processi incrementali alla progettazione.
	\end{itemize}
	
	Il \emph{Diario delle modifiche} viene incluso in ogni documento e mantiene lo storico dell'evoluzione del documento.
	
	\subsection{Pianificazione strategica e temporale}
	Il \emph{Piano di Progetto} fissa una serie di scadenze improrogabili, pertanto è necessario definire con chiarezza una strategia di qualifica efficace. Gli incrementi sulla documentazione o sul codice possono essere di natura programmata, quindi pre-fissati nel calendario, oppure posso insorgere come inaspettati, in questo caso sarà necessario programmare le dovute modifiche, è questo il caso di \glossario{bug } o \emph{errori}. La qualità di ogni incremento si basa sul fatto che la struttura di qualifica garantisce il rispetto delle \emph{Norme di Progetto}, questo lavoro verrà svolto da automatismi che segnaleranno le problematiche rilevate in modo da permettere una rapida correzione. Questo assicura la qualità dei ogni processo. \\
	L'efficienza dei processi si basa sugli automatismi, sarà così possibile destinare le risorse umane a lavori più mirati. L'utilizzo di software apposito permette di eseguire controlli \emph{Walkthrough} con precisione assoluta e l'esclusivo consumo di risorse tecnologiche, nonché un notevole risparmio di tempo.
	
	\subsection{Responsabilità}
	La responsabilità della verifica viene attribuita al \emph{Responsabile di Progetto} e ai \emph{Verificatori}. I compiti e le modalità di attuazione sono definiti nel \emph{Piano di Progetto}.
	
	\subsection{Risorse}
	La qualifica dei processi essendo un processo consuma delle risorse che si dividono in due categorie:
		\begin{itemize}
  			\item \textbf{Umane:} le figure coinvolte sono il \emph{Responsabile di Progetto} e il \emph{Verificatore}. I processi da loro effettuati consumano \emph{ore di produttività}  contabilizzate e schedulate secondo il \emph{Piano di Progetto}. Le \emph{ore di produttività} sono fissate da \url{http://www.math.unipd.it/~tullio/IS-1/2013/Progetto/PD01b.html} in un minimo di 85 e un massimo di 105 ore individuali. Il \emph{Piano di Progetto} determina la distribuzione di tali quote orarie con la relativa retribuzione. Ai fini della qualifica si potrà parlare di \emph{ore di produttività} tralasciandone l'aspetto economico, in quanto rientra nello \emph{scope} del documento succitato. 
  			
  			\item \textbf{Tecnologiche:} riguardano i \emph{mezzi} utilizzati per gli automatismi per la qualità e la loro gestione. Trattandosi esclusivamente di \emph{mezzi} informatici vengono consumate unità di calcolo che vengono considerate a costo nullo. Questo perché tutti i tipi di elaborazioni informatiche vengono svolte su mezzi per i quali non è richiesto ne un contributo economico ne un quantitativo temporale abbastanza consistente da poter essere considerato degno di nota.
		\end{itemize}
		
	Le modalità del loro impiego sono descritte dettagliatamente nelle \emph{Norme di Progetto}.
	
%		\subsubsection{Risorse disponibili}
%		Le risorse disponibili sono tutte le \emph{risorse umane} e parte delle tecnologiche, in particolare:
%		\begin{itemize}
%			\item spazi web vari
%			\item pc portatili individuali
%		\end{itemize}
		
%		\subsubsection{Risorse necessarie}		% AWS, server per continous integration..
		
	\subsection{Strumenti}
		I \emph{tools} non disponibile ma necessari sono stati inseriti nelle \emph{Norme di Progetto}. % basta questo?
	
%\iffalse     %% è una sorta di commento
	\subsection{Tecniche}
		\subsubsection{Analisi statica}
			\paragraph{Walkthrough}
			\paragraph{Inspection}
		\subsubsection{Analisi dinamica}
			\paragraph{Test di unità}
			\paragraph{Test di integrazione}
			\paragraph{Test di sistema}
			\paragraph{Test di regressione}
			\paragraph{Test di accettazione}
%\fi
	
	\subsection{Misure e Metriche}
		\subsubsection{Metriche per i processi}
			\paragraph{Schedule Variance}		
			...da completare	
		\subsubsection{Metriche per i documenti}
		
			\paragraph{nomemetrica}
			
		\subsubsection{Metriche per il software}
			\paragraph{nomemetrica}
	\subsection{Metodi}

