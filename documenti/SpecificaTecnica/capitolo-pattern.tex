\section{Design pattern}

Un \glossario{Design Pattern} descrive problemi che si ripetono molteplici volte nel nostro ambiente. Oltre al problema descrive anche la soluzione ad esso e i risultati che si ottengono nell'applicarlo. È fondamentale per qualsiasi progettista conoscere a fondo i \glossario{Design Pattern}, in quanto facilita l'attività di progettazione, favorisce la riusabilità e dà benefici enormi in termini di manutenibilità. Fondamentalmente possiamo suddividere i \glossario{Design Pattern} in quattro categorie:

\begin{itemize}

	\item \textbf{\glossario{Design Pattern} architetturali}, che esprimono schemi di base per impostare l'organizzazione strutturale di un sistema software;
	\item \textbf{\glossario{Design Pattern} creazionali}, che forniscono un'astrazione del processo di istanziazione degli oggetti;
	\item \textbf{\glossario{Design Pattern} strutturali}, che si occupano delle modalità di composizione di classi e oggetti per formare strutture complesse; 
	\item \textbf{\glossario{Design Pattern} comportamentali}, che si occupano di algoritmi e dell'assegnamento di responsabilità tra oggetti collaboranti.

\end{itemize}

Per un approfondimento e un richiamo teorico dei \glossario{Design Pattern} utilizzati nel progetto \glossario{MaaP} si rimanda all'Appendice \ref{appendice-pattern}. In seguito verranno descritti i \glossario{Design Pattern} implementati.

\subsection{Design Pattern Architetturali}

\subsubsection{MVC}

\begin{itemize}

	\item \textbf{Scopo}: Questo pattern è utilizzato per separare le responsabilità dell'applicazione a diversi componenti e permettere di fare una chiara divisione tra componenti di presentazione, struttura dei dati e operazioni su di essi.
	\item \textbf{Utilizzo}: Viene utilizzato dall'applicazione principalmente per delegare il ruolo di presentazione dei dati al \glossario{front-end}, lasciando al \glossario{back-end} la gestione della logica dell'applicazione (autenticazione, corrispondenza tra \glossario{API} e operazioni sui dati) e la logica di business.
	
\end{itemize}

\subsubsection{MVW}

\begin{itemize}

	\item \textbf{Scopo}: È un \glossario{Design Pattern} simile a \glossario{MVC} che permette di avere una corrispondenza più diretta e automatica tra la \textit{view} e il \textit{model}. L'acronimo \glossario{MVW} sta per \textit{Model-View-Whatever}, dove \textit{Whatever} secondo i progettisti di \glossario{Angular.js} indica "\textit{whatever works for you}".
	\item \textbf{Utilizzo}: È il \glossario{Design Pattern} utilizzato dal \glossario{framework} \glossario{Angular.js} con il quale viene sviluppata la parte \glossario{front-end} dell'applicazione \glossario{MaaP}.

\end{itemize}

\subsubsection{Middleware} 

\begin{itemize}

	\item \textbf{Scopo}: Si è scelto di utilizzare questo \glossario{Design Pattern} per fornire un \textit{intermediario} tra i vari componenti software dell'applicazione.
	\item \textbf{Utilizzo}: Viene utilizzato da \glossario{express}, è fortemente legato a \ref{chain-of-responsability}. In particolare è utilizzato all'interno del \glossario{package} \texttt{Back-End::Middleware}.

\end{itemize}

\subsection{Design Pattern Creazionali}

\subsubsection{Registry}

\begin{itemize}

	\item \textbf{Scopo}: Viene utilizzato per ottenere oggetti a partire da altri oggetti che hanno un'associazione con esso. Questa ricerca viene effettuata tramite una \textit{classe registro}.
	\item \textbf{Utilizzo}: Le diverse \glossario{Collection} presenti nei file \glossario{DSL} verranno memorizzate per nome in un registro, in modo tale che quando arriva una richiesta venga caricato il file \glossario{DSL} giusto. In particolare è utilizzato all'interno del \glossario{package} \texttt{Back-End::DSLModel}.

\end{itemize}

\subsubsection{Factory method}

\begin{itemize}

	\item \textbf{Scopo}: Nel contesto di \glossario{Node.js} questo pattern viene usato creare una classe attraverso una \textit{funzione factory} esportata dal modulo. In questo modo si potrà costruire e ottenere qualsiasi classe definita in un modulo.
	\item \textbf{Utilizzo}: È utilizzato all'interno del \glossario{package} \texttt{Back-End::Controller}.

\end{itemize}

\subsubsection{Singleton}

\begin{itemize}

	\item \textbf{Scopo}: Viene utilizzato per le classi che devono avere un'unica istanza durante l'esecuzione dell'applicazione;
	\item \textbf{Utilizzo}: Ogni modulo di \glossario{Node.js} è nativamente un singleton, perché viene caricato al primo \code{require} e poi tutti si riferiscono allo stesso. In architettura è utilizzato per la classe \texttt{Back-End::Controller::ControllerFactory}.

\end{itemize}

\subsection{Design Pattern Strutturali}

\subsubsection{Facade}

\begin{itemize}

	\item \textbf{Scopo}: Viene utilizzato per rendere visibili solamente alcune cose agli altri oggetti.
	\item \textbf{Utilizzo}: Viene utilizzato nei seguenti casi:
	\begin{itemize}
	
		\item La classe \texttt{Back-End::Middleware::MiddlewareLoader} utilizza \textit{facade} per nascondere l'esistenza di tutti i \glossario{middleware} alla \texttt{ServerApp};
		\item La classe \texttt{Back-End::Controller::ControllerFactory} utilizza \textit{facade} per nascondere le classi interne.
		
	\end{itemize}
\end{itemize}

\subsection{Design Pattern Comportamentali}

\subsubsection{Chain of responsability}
\label{chain-of-responsability}

\begin{itemize}

	\item \textbf{Scopo}: Viene utilizzato per far sì che un oggetto a cui viene effettuata una richiesta possa esaudire le richieste di più oggetti. In questo modo si evita l'accoppiamento fra il mittente di una richiesta e il destinatario.
	\item \textbf{Utilizzo}: \glossario{Express} usa chain-of-responsability per la gestione dei \glossario{middleware} e del \glossario{routing}. In particolare viene utilizzato nell'architettura nel \glossario{package} \texttt{Back-End::Middleware}.

\end{itemize}

\subsubsection{Strategy}

\begin{itemize}

	\item \textbf{Scopo}: Viene utilizzato perché in futuro potrebbe essere necessario cambiare l'algoritmo di interpretazione del \glossario{DSL}. Utilizzando questo pattern l'algoritmo potrà essere cambiato in modo indipendente da chi ne fa uso.
	\item \textbf{Utilizzo}: Viene utilizzato all'interno del \glossario{package} \texttt{Back-End::DSLModel}.

\end{itemize}