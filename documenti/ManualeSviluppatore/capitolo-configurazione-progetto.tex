\section{Configurazione nuovo progetto}
\label{config}

\subsection{Configurazione del back-end}

Per configurare il back-end dell'applicazione generata dal framework bisogna modificare il file \code{config.js} presente nella cartella principale dell'applicazione. All'interno del file sono presenti due sezioni: ``development'' e ``production'', l'applicazione sceglie quale utilizzare in funzione del valore della variabile di ambiente \code{NODE\_ENV}. Essi sono due configurazioni diverse: la prima riferisce a una configurazione di sviluppo, in cui è possibile impostare dei parametri in fase di sviluppo dell'applicazione; la seconda è invece la configurazione dell'applicazione che verrà resa pubblica. In ciascuna di esse è possibile impostare i seguenti parametri:

\begin{itemize}
\item \textbf{webServer}: un oggetto contenente
		\begin{itemize}
		\item \textbf{port}: la porta su cui il server si metterà in ascolto (Integer);
		\item \textbf{static}: la path assoluta della cartella contenente i file statici che il server dovrà servire al client (String);
		\end{itemize}
		
\item \textbf{userDB}: un oggetto contenente
		\begin{itemize}
		\item \textbf{url}: l'indirizzo del database che l'applicazione deve utilizzare come database degli utenti (String);
		\end{itemize}

\item \textbf{usersPerPage}: di default settato a 5, indica il numero di utenti che si vogliono visualizzare per ogni pagina nella tabella degli utenti;
		
\item \textbf{dataDB}: un oggetto contenente
		\begin{itemize}
		\item \textbf{url}: l'indirizzo del database che l'applicazione deve usare come database dei dati (String);
		\end{itemize}

\item \textbf{smtp}: l' oggetto contenente la configurazione del servizio \glossario{smtp} che verrà utilizzato dalla libreria \texttt{Nodemailer}. Per l'elenco completo di parametri configurabili riferirsi a \url{https://github.com/andris9/Nodemailer};
		
\item \textbf{resetPassword}: contiene la configurazione per il reset della password:
	\begin{itemize}
		\item \textbf{tokenLife}: rappresenta il tempo di espirazione in millisecondi del reset password (Integer);
		\item \textbf{link}: rappresenta il link da inviare per il reset della password (String);
	\end{itemize}

\item \textbf{collectionPath}: la path assoluta della cartella contenente i file di configurazione dsl che il server dovrà utilizzare come configurazione (String);
		
\item \textbf{allowSignup}: contiene un valore booleano e indica se è possibile o meno utilizzare la funzionalità di registrazione (Boolean);

\item \textbf{superAdmins}: contiene un array di utenti che di default dovranno essere settati come \textit{superadmin}. Questo perché l'applicazione deve prevedere almeno un \textit{superAdmin}. Questo array dovrà contenere oggetti con al loro interno l'email dell'utente e opzionalmente la password (String).

\end{itemize}

\subsection{Configurazione del front-end}

Per configurare il front-end dell'applicazione MaaP generata bisogna modificare il file:

\begin{center}
\texttt{./app/scripts/config.js}
\end{center} 

Al suo interno è possibile configurare le costanti di Angular. In particolare è possibile settare:

\begin{itemize}

	\item \textbf{debug}: indica se mostrare i messaggi di errore in modo dettagliato oppure no (Boolean);

	\item \textbf{navBarCollections}: indica il numero di Collections da visualizzare nella barra di navigazione (Integer);

	\item \textbf{showSignup}: indica se mostrare o meno il pulsante di registrazione. È importante ricordarsi di disabilitare o riabilitare la funzionalità di registrazione anche nel back-end (Boolean);

	\item \textbf{reportLink}: indica il link al quale l'utente può segnalare bugs o problematiche relative all'applicazione (String).

\end{itemize}


\subsection{Struttura applicazione}

L'applicazione generata dal framework presenta i seguenti file e cartelle: 
\begin{itemize}
 \item \textbf{\file{\texttt{ProjectName}/}} \\
	Contiene i file che verranno utilizzati come base di partenza (scaffold) per il progetto dello sviluppatore che utilizzerà il prodotto;

 \item \textbf{\file{\texttt{ProjectName}/collections/}} \\
	Contiene i file di configurazione delle collection di esempio utilizzate dall'applicazione scaffold. Di default è la cartella in cui devono essere salvati i file DSL;

 \item \textbf{\file{\texttt{ProjectName}/app/}} \\
	Contiene tutti gli script relativi al Frontend dell'applicazione;

 \item \textbf{\file{\texttt{ProjectName}/app/scripts/}} \\
	Contiene i file dell'applicazione AngularJS usata nel Frontend;

 \item \textbf{\file{\texttt{ProjectName}/app/scripts/services/}} \\
	Contiene i file dei service di AngularJS usata nel Frontend;

 \item \textbf{\file{\texttt{ProjectName}/app/scripts/controllers/}} \\
	Contiene i file dei controller di AngularJS usata nel Frontend;

 \item \textbf{\file{\texttt{ProjectName}/app/view/}} \\
	Contiene i file statici html usati dal Frontend dell'applicazione;

 \item \textbf{\file{\texttt{ProjectName}/app/style/}} \\
	Contiene i file statici CSS usati dal Frontend dell'applicazione;

 \item \textbf{\file{\texttt{ProjectName}/app/bower\_components/}} \\
	Contiene i file delle librerie utilizzate dal Frontend dell'applicazione;
	
 \item \textbf{\file{\texttt{ProjectName}/node\_modules/}} \\
	Contiene i file delle librerie utilizzate dal Backend dell'applicazione.
\end{itemize}
