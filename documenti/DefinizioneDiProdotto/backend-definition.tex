


\subsection{Componente Back-end::Lib}

\subsubsection{Classe ServerApp}

\begin{table}[H]
\begin{center}
\bgroup
\setlength{\arrayrulewidth}{0.6mm}
\def\arraystretch{1}
\begin{tabular}{ | p{12cm} | }
\hline
\centerline{\textbf{ServerApp}}
\\ \hline
 \\ 
\hline
\code{+start()} \\
\code{+\underline{ServerApp}(config:Config)} \\
\hline
\end{tabular}
\egroup
\caption{Classe ServerApp}
\end{center}
\end{table}

\paragraph*{Descrizione}
\begin{itemize}
\item[] Classe che si occupa di avviare il server e di invocare il \glossario{middleware}. È il componente client del \glossario{Design Pattern} \glossario{Chain of responsibility}. Utilizza i pacchetti Mongoose ed Express.
\end{itemize}

\paragraph*{Utilizzo}
\begin{itemize}
\item[] Viene utilizzato per avviare l'applicazione. Internamente inizializza la catena gestione delle chiamate utilizzando la classe \texttt{Back-end::Lib::Middleware::MiddlewareLoader}.
\end{itemize}

\paragraph*{Relazioni con altre classi}
\begin{itemize}


\item[] Utilizza le classi:
\begin{itemize}
\item[$\bullet$] \class{Back-end::Lib::Controller::Middleware::MiddlewareLoader}
\item[$\bullet$] \class{Back-end::Lib::Config}
\end{itemize}
\end{itemize}

\paragraph*{Attributi}
\begin{itemize}
\item[] Assenti
\end{itemize}

\paragraph*{Metodi}
\begin{itemize}
\item[] \method{+start()} \\ Questo metodo fa partire il server. Non ritorna il controllo finché il server è in funzione.
\item[] \method{+\underline{ServerApp}(config:Config)} \\ Questo metodo statico accetta come parametro l'oggetto di configurazione dell'applicazione e costruisce l'oggetto ServerApp che rappresenta il server dell'applicazione.
\begin{itemize}\addtolength{\itemsep}{-0.5\baselineskip}
\item[$\circ$] \parameter{config:Config} \\ È l'oggetto di configurazione dell'applicazione.
\end{itemize}
\end{itemize}

\subsubsection{Classe Config}

\begin{table}[H]
\begin{center}
\bgroup
\setlength{\arrayrulewidth}{0.6mm}
\def\arraystretch{1}
\begin{tabular}{ | p{12cm} | }
\hline
\centerline{\textbf{Config}}
\\ \hline
 \\ 
\hline
\code{+getEnvironment():String} \\
\code{+getServerPort():Integer} \\
\code{+getServerStaticPath():String} \\
\code{+getUserDbUri():String} \\
\code{+getDataDbUri():String} \\
\code{+getSmtpService():String} \\
\code{+getSmtpAuth():String} \\
\code{+getDSLPath():String} \\
\hline
\end{tabular}
\egroup
\caption{Classe Config}
\end{center}
\end{table}

\paragraph*{Descrizione}
\begin{itemize}
\item[] Classe che rappresenta l'interfaccia della classe di configurazione dell'applicazione.
\end{itemize}

\paragraph*{Utilizzo}
\begin{itemize}
\item[] Viene utilizzata per descrivere tutti i parametri dell'applicazione. Quando viene creata una ServerApp le viene passato un oggetto di questo tipo ed essa avvierà l'applicazione a partire da questa configurazione.
\end{itemize}

\paragraph*{Relazioni con altre classi}
\begin{itemize}

\item[] È estesa dalle classi:
\begin{itemize}
\item \class{Back-end::DeveloperProject::Config::ProjectConfig}
\end{itemize}

\end{itemize}

\paragraph*{Attributi}
\begin{itemize}
\item[] Assenti
\end{itemize}

\paragraph*{Metodi}
\begin{itemize}
\item[] \method{+getEnvironment():String} \\ Restituisce la variabile d'ambiente che informa se l'applicazione deve essere eseguita in modalità ``developing'' o ``production''.
\item[] \method{+getServerPort():Integer} \\ Restituisce la porta su cui il server deve mettersi in ascolto.
\item[] \method{+getServerStaticPath():String} \\ Restituisce il percorso della cartella che il server deve utilizzare per fornire file statici.
\item[] \method{+getUserDbUri():String} \\ Restituisce l'uri del database che il server deve utilizzare come database degli utenti
\item[] \method{+getDataDbUri():String} \\ Restituisce l'uri del database che il server deve utilizzare come database di analisi, cioè quello contenente le collection di cui l'applicazione deve permettere la visualizzazione.
\item[] \method{+getSmtpService():String} \\ Restituisce il nome del servizio che potrà essere usato dall'applicazione per inviare email.
\item[] \method{+getSmtpAuth():String} \\ Restituisce le credenziali con cui è possibile utilizzare il servizio smtp per inviare email.
\item[] \method{+getDSLPath():String} \\ Restituisce la path da cui caricare i file DSL definiti dallo sviluppatore.
\end{itemize}

\subsection{Componente Back-end::Lib::View}

\subsubsection{Classe ForgotMailView}

\begin{table}[H]
\begin{center}
\bgroup
\setlength{\arrayrulewidth}{0.6mm}
\def\arraystretch{1}
\begin{tabular}{ | p{12cm} | }
\hline
\centerline{\textbf{ForgotMailView}}
\\ \hline
 \\ 
\hline
\code{+buildForgotMail(userMail:String, senderMail:String, tokenlink:String):Email} \\
\hline
\end{tabular}
\egroup
\caption{Classe ForgotMailView}
\end{center}
\end{table}

\paragraph*{Descrizione}
\begin{itemize}
\item[] Classe che fornisce una rappresentazione della mail.
\end{itemize}

\paragraph*{Utilizzo}
\begin{itemize}
\item[] Viene utilizzata come template di email da inviare nel caso in cui l'utente richieda il recupero password.
\end{itemize}

\paragraph*{Relazioni con altre classi}
Assenti
% TODO: deve esserci almeno una relazione con questa classe!!!

\paragraph*{Attributi}
\begin{itemize}
\item[] Assenti
\end{itemize}

\paragraph*{Metodi}
\begin{itemize}
\item[] \method{+buildForgotMail(userMail:String, senderMail:String, tokenlink:String):Email} \\ Metodo che definisce e restituisce la struttura dell'email da inviare per il reset della password nel formato Email della libreria NodeMailer.
\begin{itemize}\addtolength{\itemsep}{-0.5\baselineskip}
\item[$\circ$] \parameter{userMail:String} \\ Parametro che rappresenta l'email dell'utente a cui inviare l'email.
\item[$\circ$] \parameter{senderMail:String} \\ Parametro che rappresenta l'email del mittente.
\item[$\circ$] \parameter{tokenlink:String} \\ Questo parametro è il link con il token dal quale l'utente può accedere per procedere con il reset della password.
\end{itemize}
\end{itemize}

\subsection{Componente Back-end::Lib::Controller::Middleware}

\subsubsection{Classe MiddlewareLoader}

\begin{table}[H]
\begin{center}
\bgroup
\setlength{\arrayrulewidth}{0.6mm}
\def\arraystretch{1}
\begin{tabular}{ | p{12cm} | }
\hline
\centerline{\textbf{MiddlewareLoader}}
\\ \hline
 \\ 
\hline
\code{+init(app:ServerApp)} \\
\hline
\end{tabular}
\egroup
\caption{Classe MiddlewareLoader}
\end{center}
\end{table}

\paragraph*{Descrizione}
\begin{itemize}
\item[] Classe che definisce un'interfaccia comune per tutte le richieste dell'applicazione. È uno dei componenti ConcreteHandler del \glossario{Design Pattern} \glossario{Chain of responsibility}.
\end{itemize}

\paragraph*{Utilizzo}
\begin{itemize}
\item[] Viene utilizzato per istanziare in modo "nascosto" all'applicazione tutti i \glossario{middleware} presenti nel componente \texttt{Back-end::Lib::Middleware}.
\end{itemize}

\paragraph*{Relazioni con altre classi}
\begin{itemize}


\item[] Utilizza le classi:
\begin{itemize}
\item[$\bullet$] \class{Back-end::Lib::Controller::Middleware::Router}
\item[$\bullet$] \class{Back-end::Lib::Controller::Middleware::NotFoundHandler}
\item[$\bullet$] \class{Back-end::Lib::Controller::Middleware::DSLLoaderHandler}
\item[$\bullet$] \class{Back-end::Lib::Controller::Middleware::ErrorHandler}
\end{itemize}
\end{itemize}

\paragraph*{Attributi}
\begin{itemize}
\item[] Assenti
\end{itemize}

\paragraph*{Metodi}
\begin{itemize}
\item[] \method{+init(app:ServerApp)} \\ Metodo che inserisce in ogni richiesta un riferimento all'applicazione rendendolo accedibile tramite /code{req.app}. \\
Inizializza tutti i middleware richiamando i corrispondenti metodi init.
\begin{itemize}\addtolength{\itemsep}{-0.5\baselineskip}
\item[$\circ$] \parameter{app:ServerApp} \\ È l'istanza del server dell'applicazione.
\end{itemize}
\end{itemize}

\subsubsection{Classe Authentication}

\begin{table}[H]
\begin{center}
\bgroup
\setlength{\arrayrulewidth}{0.6mm}
\def\arraystretch{1}
\begin{tabular}{ | p{12cm} | }
\hline
\centerline{\textbf{Authentication}}
\\ \hline
 \\ 
\hline
\code{+handler(req:Request, res:Response, next:function(MaapError))} \\
\code{+init(app:ServerApp)} \\
\code{+authenticate(req:Request, res:Response, next:function(MaapError))} \\
\code{+requireNotLogged(req:Request, res:Response, next:function(MaapError))} \\
\code{+requireLogged(req:Request, res:Response, next:function(MaapError))} \\
\code{+requireAdmin(req:Request, res:Response, next:function(MaapError))} \\
\code{+requireSuperAdmin(req:Request, res:Response, next:function(MaapError))} \\
\hline
\end{tabular}
\egroup
\caption{Classe Authentication}
\end{center}
\end{table}

\paragraph*{Descrizione}
\begin{itemize}
\item[] Classe che si occupa dell'autenticazione di un'utente. È uno dei componenti ConcreteHandler del \glossario{Design Pattern} \glossario{Chain of responsibility}.
\end{itemize}

\paragraph*{Utilizzo}
\begin{itemize}
\item[] Viene utilizzata per verificare i dati inseriti dall'utente nella pagina di login e controllare l'effettiva corrispondenza delle credenziali nel \glossario{database}.
\end{itemize}

\paragraph*{Relazioni con altre classi}
\begin{itemize}


\item[] Utilizza le classi:
\begin{itemize}
\item[$\bullet$] \class{Back-end::Lib::Model::UserModel}
\end{itemize}
\end{itemize}

\paragraph*{Attributi}
\begin{itemize}
\item[] Assenti
\end{itemize}

\paragraph*{Metodi}
\begin{itemize}
\item[] \method{+handler(req:Request, res:Response, next:function(MaapError))} \\ Metodo che implementa la gestione delle richieste arrivate da Express: effettuata l'elaborazione passa il controllo al successivo middleware, utilizzando il pattern \glossario{Chain of responsibility}.
\begin{itemize}\addtolength{\itemsep}{-0.5\baselineskip}
\item[$\circ$] \parameter{req:Request} \\ Questo oggetto rappresenta la richiesta di tipo Request arrivata al server che il metodo deve gestire.
\item[$\circ$] \parameter{res:Response} \\ Questo oggetto rappresenta la risposta che il server dovrà rispondere al termine dell'elaborazione.
\item[$\circ$] \parameter{next:function(MaapError)} \\ Questo parametro rappresenta la callback che il metodo dovrà chiamare al termine dell'elaborazione per passare il controllo ai successivi middleware. La presenza del parametro facoltativo MaapError attiva la catena di gestione dell'errore in sostituzione della normale catena di gestione delle richieste.
\end{itemize}
\item[] \method{+init(app:ServerApp)} \\ Configura Passport dandogli la strategia che deve utilizzare per l'autenticazione degli utenti e definendo i campi da serializzare e deserializzare per il mantenimento delle informazioni sulla sessione utente.
\begin{itemize}\addtolength{\itemsep}{-0.5\baselineskip}
\item[$\circ$] \parameter{app:ServerApp} \\ È l'istanza del server dell'applicazione.
\end{itemize}
\item[] \method{+authenticate(req:Request, res:Response, next:function(MaapError))} \\ Utilizza il metodo \code{authenticate()} di Passport per effettuare l'autenticazione dell' utente.
\begin{itemize}\addtolength{\itemsep}{-0.5\baselineskip}
\item[$\circ$] \parameter{req:Request} \\ Questo oggetto rappresenta la richiesta di tipo Request arrivata al server che il metodo deve gestire.
\item[$\circ$] \parameter{res:Response} \\ Questo oggetto rappresenta la risposta che il server dovrà rispondere al termine dell'elaborazione.
\item[$\circ$] \parameter{next:function(MaapError)} \\ Questo parametro rappresenta la callback che il metodo dovrà chiamare al termine dell'elaborazione per passare il controllo ai successivi middleware. La presenza del parametro facoltativo MaapError attiva la catena di gestione dell'errore in sostituzione della normale catena di gestione delle richieste.
\end{itemize}
\item[] \method{+requireNotLogged(req:Request, res:Response, next:function(MaapError))} \\ Metodo che verifica se l'utente è autenticato, nel caso lo sia risponde con errore mentre nel caso l'utente non sia autenticato chiama il successivo middleware.
\begin{itemize}\addtolength{\itemsep}{-0.5\baselineskip}
\item[$\circ$] \parameter{req:Request} \\ Questo oggetto rappresenta la richiesta di tipo Request arrivata al server che il metodo deve gestire.
\item[$\circ$] \parameter{res:Response} \\ Questo oggetto rappresenta la risposta che il server dovrà rispondere al termine dell'elaborazione.
\item[$\circ$] \parameter{next:function(MaapError)} \\ Questo parametro rappresenta la callback che il metodo dovrà chiamare al termine dell'elaborazione per passare il controllo ai successivi middleware. La presenza del parametro facoltativo MaapError attiva la catena di gestione dell'errore in sostituzione della normale catena di gestione delle richieste.
\end{itemize}
\item[] \method{+requireLogged(req:Request, res:Response, next:function(MaapError))} \\ Metodo che deve verificare se l'utente è autenticato, richiamando il middleware successivo in caso lo sia mentre deve ritornare un errore in caso contrario.
\begin{itemize}\addtolength{\itemsep}{-0.5\baselineskip}
\item[$\circ$] \parameter{req:Request} \\ Questo oggetto rappresenta la richiesta di tipo Request arrivata al server che il metodo deve gestire.
\item[$\circ$] \parameter{res:Response} \\ Questo oggetto rappresenta la risposta che il server dovrà rispondere al termine dell'elaborazione.
\item[$\circ$] \parameter{next:function(MaapError)} \\ Questo parametro rappresenta la callback che il metodo dovrà chiamare al termine dell'elaborazione per passare il controllo ai successivi middleware. La presenza del parametro facoltativo MaapError attiva la catena di gestione dell'errore in sostituzione della normale catena di gestione delle richieste.
\end{itemize}
\item[] \method{+requireAdmin(req:Request, res:Response, next:function(MaapError))} \\ Metodo che verifica se l'utente autenticato ha un livello admin richiamando il successivo middleware in caso affermativo altrimenti rispondendo con un errore.
\begin{itemize}\addtolength{\itemsep}{-0.5\baselineskip}
\item[$\circ$] \parameter{req:Request} \\ Questo oggetto rappresenta la richiesta di tipo Request arrivata al server che il metodo deve gestire.
\item[$\circ$] \parameter{res:Response} \\ Questo oggetto rappresenta la risposta che il server dovrà rispondere al termine dell'elaborazione.
\item[$\circ$] \parameter{next:function(MaapError)} \\ Questo parametro rappresenta la callback che il metodo dovrà chiamare al termine dell'elaborazione per passare il controllo ai successivi middleware. La presenza del parametro facoltativo MaapError attiva la catena di gestione dell'errore in sostituzione della normale catena di gestione delle richieste.
\end{itemize}
\item[] \method{+requireSuperAdmin(req:Request, res:Response, next:function(MaapError))} \\ Metodo che deve verificare se l'utente autenticato ha livello di super admin richiamando in caso positivo il successivo middleware ed in caso negativo rispondere con errore.
\begin{itemize}\addtolength{\itemsep}{-0.5\baselineskip}
\item[$\circ$] \parameter{req:Request} \\ Questo oggetto rappresenta la richiesta di tipo Request arrivata al server che il metodo deve gestire.
\item[$\circ$] \parameter{res:Response} \\ Questo oggetto rappresenta la risposta che il server dovrà rispondere al termine dell'elaborazione.
\item[$\circ$] \parameter{next:function(MaapError)} \\ Questo parametro rappresenta la callback che il metodo dovrà chiamare al termine dell'elaborazione per passare il controllo ai successivi middleware. La presenza del parametro facoltativo MaapError attiva la catena di gestione dell'errore in sostituzione della normale catena di gestione delle richieste.
\end{itemize}
\end{itemize}

\subsubsection{Classe Router}

\begin{table}[H]
\begin{center}
\bgroup
\setlength{\arrayrulewidth}{0.6mm}
\def\arraystretch{1}
\begin{tabular}{ | p{12cm} | }
\hline
\centerline{\textbf{Router}}
\\ \hline
 \\ 
\hline
\code{+handler(req:Request, res:Response, next:function(MaapError))} \\
\code{+init(app:ServerApp)} \\
\hline
\end{tabular}
\egroup
\caption{Classe Router}
\end{center}
\end{table}

\paragraph*{Descrizione}
\begin{itemize}
\item[] Classe che si occupa della richiesta di risorse. È uno dei componenti Handler del \glossario{Design Pattern} \glossario{Chain of responsibility}. Ha una relazione con la classe Authentication, poiché fa uso di alcuni metodi per controllare l'autenticazione.
\end{itemize}

\paragraph*{Utilizzo}
\begin{itemize}
\item[] Si occupa di smistare la richiesta in base all'\glossario{URI} ricevuto e ad invocare l'opportuno metodo di creazione sulla classe \texttt{Back-end::Lib::Controller::ControllerFactory}.
\end{itemize}

\paragraph*{Relazioni con altre classi}
\begin{itemize}


\item[] Utilizza le classi:
\begin{itemize}
\item[$\bullet$] \class{Back-end::Lib::Controller::Middleware::Authentication}
\item[$\bullet$] \class{Back-end::Lib::Controller::Service::ServiceFactory}
\end{itemize}
\end{itemize}

\paragraph*{Attributi}
\begin{itemize}
\item[] Assenti
\end{itemize}

\paragraph*{Metodi}
\begin{itemize}
\item[] \method{+handler(req:Request, res:Response, next:function(MaapError))} \\ Metodo che implementa la gestione delle richieste arrivate da Express: effettuata l'elaborazione passa il controllo al successivo middleware, utilizzando il pattern \glossario{Chain of responsibility}.
\begin{itemize}\addtolength{\itemsep}{-0.5\baselineskip}
\item[$\circ$] \parameter{req:Request} \\ Questo oggetto rappresenta la richiesta di tipo Request arrivata al server che il metodo deve gestire.
\item[$\circ$] \parameter{res:Response} \\ Questo oggetto rappresenta la risposta che il server dovrà rispondere al termine dell'elaborazione.
\item[$\circ$] \parameter{next:function(MaapError)} \\ Questo parametro rappresenta la callback che il metodo dovrà chiamare al termine dell'elaborazione per passare il controllo ai successivi middleware.
La presenza del parametro facoltativo MaapError attiva la catena di gestione dell'errore in sostituzione della normale catena di gestione delle richieste.
\end{itemize}
\item[] \method{+init(app:ServerApp)} \\ Metodo che definisce per ogni richiesta il corrispondente controller che dovrà gestirla, verificando i permessi dell'utente che la richiede utilizzando i metodi del modulo \code{Authenticate}.

\begin{itemize}\addtolength{\itemsep}{-0.5\baselineskip}
\item[$\circ$] \parameter{app:ServerApp} \\ È l'istanza del server dell'applicazione.
\end{itemize}
\end{itemize}

\subsubsection{Classe NotFoundHandler}

\begin{table}[H]
\begin{center}
\bgroup
\setlength{\arrayrulewidth}{0.6mm}
\def\arraystretch{1}
\begin{tabular}{ | p{12cm} | }
\hline
\centerline{\textbf{NotFoundHandler}}
\\ \hline
 \\ 
\hline
\code{+handler(req:Request, res:Response, next:function(MaapError))} \\
\hline
\end{tabular}
\egroup
\caption{Classe NotFoundHandler}
\end{center}
\end{table}

\paragraph*{Descrizione}
\begin{itemize}
\item[] Classe che si occupa la gestione dell'errore di pagina non trovata. È uno dei componenti ConcreteHandler del \glossario{Design Pattern} \glossario{Chain of responsibility}.
\end{itemize}

\paragraph*{Utilizzo}
\begin{itemize}
\item[] Viene utilizzata per generare una pagina 404 di errore nel caso in cui l'\glossario{URI} passato non corrisponda ad una risorsa presente nell'applicazione.
\end{itemize}

\paragraph*{Relazioni con altre classi}
Assenti
% TODO: deve esserci almeno una relazione con questa classe!!!

\paragraph*{Attributi}
\begin{itemize}
\item[] Assenti
\end{itemize}

\paragraph*{Metodi}
\begin{itemize}
\item[] \method{+handler(req:Request, res:Response, next:function(MaapError))} \\ Metodo che risponde con un errore.
\begin{itemize}\addtolength{\itemsep}{-0.5\baselineskip}
\item[$\circ$] \parameter{req:Request} \\ Questo oggetto rappresenta la richiesta di tipo Request arrivata al server che il metodo deve gestire.
\item[$\circ$] \parameter{res:Response} \\ Questo oggetto rappresenta la risposta che il server dovrà rispondere al termine dell'elaborazione.
\item[$\circ$] \parameter{next:function(MaapError)} \\ Questo parametro rappresenta la callback che il metodo dovrà chiamare al termine dell'elaborazione per passare il controllo ai successivi middleware. La presenza del parametro facoltativo MaapError attiva la catena di gestione dell'errore in sostituzione della normale catena di gestione delle richieste.
\end{itemize}
\end{itemize}

\subsubsection{Classe DSLLoaderHandler}

\begin{table}[H]
\begin{center}
\bgroup
\setlength{\arrayrulewidth}{0.6mm}
\def\arraystretch{1}
\begin{tabular}{ | p{12cm} | }
\hline
\centerline{\textbf{DSLLoaderHandler}}
\\ \hline
\code{- error:MaapError} \\
\code{- dslDomain:DslDomain} \\
\hline
\code{+init(app:ServerApp)} \\
\code{+DSLLoaderHandler(app:ServerApp)} \\
\code{+loadDsl(collectionPath:String)} \\
\code{+handle(req:Request, res:Response, next:function(MaapError))} \\
\hline
\end{tabular}
\egroup
\caption{Classe DSLLoaderHandler}
\end{center}
\end{table}

\paragraph*{Descrizione}
\begin{itemize}
\item[] Classe che si occupa di caricare i \glossario{DSL} presenti nel sistema. È uno dei componenti ConcreteHandler del \glossario{Design Pattern} \glossario{Chain of responsibility}.
\end{itemize}

\paragraph*{Utilizzo}
\begin{itemize}
\item[] Viene utilizzata per caricare i \glossario{DSL} delle \glossario{Collection} all'interno del \glossario{database}.
\end{itemize}

\paragraph*{Relazioni con altre classi}
\begin{itemize}


\item[] Utilizza le classi:
\begin{itemize}
\item[$\bullet$] \class{Back-end::Lib::Model::DSLModel::DSLDomain}
\end{itemize}
\end{itemize}

\paragraph*{Attributi}
\begin{itemize}
\item[] \attribute{- error:MaapError} \\ Mantiene l'ultimo errore di apertura della cartella contenente i DSL.
\item[] \attribute{- dslDomain:DslDomain} \\ Questo campo dati contiene il riferimento al \texttt{DslDomain} con il quale comunicare.
\end{itemize}

\paragraph*{Metodi}
\begin{itemize}
\item[] \method{+init(app:ServerApp)} \\ Metodo che carica i file \glossario{DSL} delle \glossario{Collection} utilizzando \code{browseFileSystem()} facendosi restituire un'array di nomi di file. Per ognuno di questi si occupa di caricarlo correttamente utilizzando il metodo \code{loadDSLFile()} del modulo \code{DslDomain}. Nel caso ci siano errori nei \glossario{DSL} viene richiamato il successivo middleware, attivando la catena di gestione errore.
\begin{itemize}\addtolength{\itemsep}{-0.5\baselineskip}
\item[$\circ$] \parameter{app:ServerApp} \\ È l'istanza del server dell'applicazione.
\end{itemize}
\item[] \method{+DSLLoaderHandler(app:ServerApp)} \\ Questo metodo è il costruttore della classe.
\begin{itemize}\addtolength{\itemsep}{-0.5\baselineskip}
\item[$\circ$] \parameter{app:ServerApp} \\ Questo parametro rappresenta il riferimento alla \texttt{ServerApp} da utilizzare nel costruttore.
\end{itemize}
\item[] \method{+loadDsl(collectionPath:String)} \\ Questo metodo si occupa di effettuare una scansione della cartella dei DSL e di comunicare con il \texttt{DslDomain} passandogli i file caricati.
\begin{itemize}\addtolength{\itemsep}{-0.5\baselineskip}
\item[$\circ$] \parameter{collectionPath:String} \\ Questo parametro rappresenta la path della directory che contiene i file /code{.dsl} da caricare.
\end{itemize}
\item[] \method{+handle(req:Request, res:Response, next:function(MaapError))} \\ 
\begin{itemize}\addtolength{\itemsep}{-0.5\baselineskip}
\item[$\circ$] \parameter{req:Request} \\ Questo oggetto rappresenta la richiesta di tipo Request arrivata al server che il metodo deve gestire.
\item[$\circ$] \parameter{res:Response} \\ Questo oggetto rappresenta la risposta che il server dovrà rispondere al termine dell'elaborazione.
\item[$\circ$] \parameter{next:function(MaapError)} \\ Questo parametro rappresenta la callback che il metodo dovrà chiamare al termine dell'elaborazione per passare il controllo ai successivi middleware. La presenza del parametro facoltativo MaapError attiva la catena di gestione dell'errore in sostituzione della normale catena di gestione delle richieste.
\end{itemize}
\end{itemize}

\subsubsection{Classe ErrorHandler}

\begin{table}[H]
\begin{center}
\bgroup
\setlength{\arrayrulewidth}{0.6mm}
\def\arraystretch{1}
\begin{tabular}{ | p{12cm} | }
\hline
\centerline{\textbf{ErrorHandler}}
\\ \hline
 \\ 
\hline
\code{+handler(err:MaapError, req:Request, res:Response, next:function(MaapError))} \\
\hline
\end{tabular}
\egroup
\caption{Classe ErrorHandler}
\end{center}
\end{table}

\paragraph*{Descrizione}
\begin{itemize}
\item[] Questa classe gestisce gli errori generati nei precedenti middleware o controller. Invia al client una risposta con stato HTTP 500 (server error) con una descrizione dell'errore nel formato JSON.
È uno dei componenti ConcreteHandler del \glossario{Design Pattern} \glossario{Chain of responsibility}.
\end{itemize}

\paragraph*{Utilizzo}
\begin{itemize}
\item[] Questo middleware viene utilizzato per ultimo nella catena di gestione delle richieste di Express, in modo da gestire tutti gli errori generati precedentemente.
\end{itemize}

\paragraph*{Relazioni con altre classi}
Assenti
% TODO: deve esserci almeno una relazione con questa classe!!!

\paragraph*{Attributi}
\begin{itemize}
\item[] Assenti
\end{itemize}

\paragraph*{Metodi}
\begin{itemize}
\item[] \method{+handler(err:MaapError, req:Request, res:Response, next:function(MaapError))} \\ Gestisce la richiesta rispondendo con un json contenente le informazioni dell'errore.
\begin{itemize}\addtolength{\itemsep}{-0.5\baselineskip}
\item[$\circ$] \parameter{req:Request} \\ Questo oggetto rappresenta la richiesta di tipo Request arrivata al server che il metodo deve gestire.
\item[$\circ$] \parameter{res:Response} \\ Questo oggetto rappresenta la risposta che il server dovrà rispondere al termine dell'elaborazione.
\item[$\circ$] \parameter{next:function(MaapError)} \\ Questo parametro rappresenta la callback che il metodo dovrà chiamare al termine dell'elaborazione per passare il controllo ai successivi middleware. La presenza del parametro facoltativo MaapError attiva la catena di gestione dell'errore in sostituzione della normale catena di gestione delle richieste.
\item[$\circ$] \parameter{err:MaapError} \\ Questo oggetto rappresenta l'errore di tipo MaapError arrivato al server che il metodo deve gestire.
\end{itemize}
\end{itemize}

\subsection{Componente Back-end::Lib::Controller::Service}

\subsubsection{Classe ServiceFactory}

\begin{table}[H]
\begin{center}
\bgroup
\setlength{\arrayrulewidth}{0.6mm}
\def\arraystretch{1}
\begin{tabular}{ | p{12cm} | }
\hline
\centerline{\textbf{ServiceFactory}}
\\ \hline
 \\ 
\hline
\code{+getCollectionSevice(app:ServerApp)} \\
\code{+getProfileService(app:ServerApp)} \\
\code{+getForgotService(app:ServerApp)} \\
\code{+getUserService(app:ServerApp)} \\
\code{+getShowService(app:ServerApp)} \\
\code{+getIndexService(app:ServerApp)} \\
\hline
\end{tabular}
\egroup
\caption{Classe ServiceFactory}
\end{center}
\end{table}

\paragraph*{Descrizione}
\begin{itemize}
\item[] Classe che si occupa di istanziare e restituire una classe \textit{Service}. Rappresenta il componente creator del \glossario{Design Pattern} \glossario{Factory method} ed è un \glossario{Design Pattern} \glossario{Singleton}.
\end{itemize}

\paragraph*{Utilizzo}
\begin{itemize}
\item[] Viene costruita una sola volta dalla classe \textit{Back-end::Lib::Middleware::Router} e si occupa di creare e restituire l'oggetto \textit{Service} richiesto.
\end{itemize}

\paragraph*{Relazioni con altre classi}
\begin{itemize}


\item[] Utilizza le classi:
\begin{itemize}
\item[$\bullet$] \class{Back-end::Lib::Controller::Service::ForgotService}
\item[$\bullet$] \class{Back-end::Lib::Controller::Service::ProfileService}
\item[$\bullet$] \class{Back-end::Lib::Controller::Service::UserService}
\item[$\bullet$] \class{Back-end::Lib::Controller::Service::ShowService}
\item[$\bullet$] \class{Back-end::Lib::Controller::Service::IndexService}
\item[$\bullet$] \class{Back-end::Lib::Controller::Service::CollectionService}
\end{itemize}
\end{itemize}

\paragraph*{Attributi}
\begin{itemize}
\item[] Assenti
\end{itemize}

\paragraph*{Metodi}
\begin{itemize}
\item[] \method{+getCollectionSevice(app:ServerApp)} \\ Ritorna la classe \code{collectionService}.
\begin{itemize}\addtolength{\itemsep}{-0.5\baselineskip}
\item[$\circ$] \parameter{app:ServerApp} \\ È l'istanza del server dell'applicazione.
\end{itemize}
\item[] \method{+getProfileService(app:ServerApp)} \\ Metodo che ritorna la classe \code{profileService}.
\begin{itemize}\addtolength{\itemsep}{-0.5\baselineskip}
\item[$\circ$] \parameter{app:ServerApp} \\ È l'istanza del server dell'applicazione.
\end{itemize}
\item[] \method{+getForgotService(app:ServerApp)} \\ Metodo che restituisce la classe \code{forgotService}.
\begin{itemize}\addtolength{\itemsep}{-0.5\baselineskip}
\item[$\circ$] \parameter{app:ServerApp} \\ È l'istanza del server dell'applicazione.
\end{itemize}
\item[] \method{+getUserService(app:ServerApp)} \\ Metodo che deve restituire la classe \code{userService}.
\begin{itemize}\addtolength{\itemsep}{-0.5\baselineskip}
\item[$\circ$] \parameter{app:ServerApp} \\ È l'istanza del server dell'applicazione.
\end{itemize}
\item[] \method{+getShowService(app:ServerApp)} \\ Metodo che ritorna la classe \code{showService}.
\begin{itemize}\addtolength{\itemsep}{-0.5\baselineskip}
\item[$\circ$] \parameter{app:ServerApp} \\ È l'istanza del server dell'applicazione.
\end{itemize}
\item[] \method{+getIndexService(app:ServerApp)} \\ Questo metodo restituisce la classe \code{indexService}.
\begin{itemize}\addtolength{\itemsep}{-0.5\baselineskip}
\item[$\circ$] \parameter{app:ServerApp} \\ È l'istanza del server dell'applicazione.
\end{itemize}
\end{itemize}

\subsubsection{Classe ForgotService}

\begin{table}[H]
\begin{center}
\bgroup
\setlength{\arrayrulewidth}{0.6mm}
\def\arraystretch{1}
\begin{tabular}{ | p{12cm} | }
\hline
\centerline{\textbf{ForgotService}}
\\ \hline
 \\ 
\hline
\code{+passwordResetRequest(req:Request, res:Response, next:function(MaapError))} \\
\code{+passwordReset(req:Request, res:Response, next:function(MaapError))} \\
\hline
\end{tabular}
\egroup
\caption{Classe ForgotService}
\end{center}
\end{table}

\paragraph*{Descrizione}
\begin{itemize}
\item[] Classe che rappresenta il sistema di recupero e ripristino password. È uno dei componenti product del \glossario{Design Pattern} \glossario{Factory method}.
\end{itemize}

\paragraph*{Utilizzo}
\begin{itemize}
\item[] La classe fornisce dei metodi per effettuare una richiesta di reset password e, in un secondo momento, procedere al suo ripristino. La richiesta di reset avviene mandando un'email all'indirizzo dell'utente tramite la classe \texttt{Back-end::Lib::Middleware::Mailer}. All'interno di questo messaggio sarà presente un link che procederà ad effettuare il login dell'utente e a reindirizzarlo nella pagina di modifica profilo, dalla quale potrà modificare la password.
\end{itemize}

\paragraph*{Relazioni con altre classi}
\begin{itemize}


\item[] Utilizza le classi:
\begin{itemize}
\item[$\bullet$] \class{Back-end::Lib::View::ForgotMailView}
\end{itemize}
\end{itemize}

\paragraph*{Attributi}
\begin{itemize}
\item[] Assenti
\end{itemize}

\paragraph*{Metodi}
\begin{itemize}
\item[] \method{+passwordResetRequest(req:Request, res:Response, next:function(MaapError))} \\ Metodo che si occupa di impostare l'email e di inviarla per il reset della password utente, costruendone il template e creando il link col token associato alla richiesta.
\begin{itemize}\addtolength{\itemsep}{-0.5\baselineskip}
\item[$\circ$] \parameter{req:Request} \\ Questo oggetto rappresenta la richiesta di tipo Request arrivata al server che il metodo deve gestire.
\item[$\circ$] \parameter{res:Response} \\ Questo oggetto rappresenta la risposta che il server dovrà rispondere al termine dell'elaborazione.
\item[$\circ$] \parameter{next:function(MaapError)} \\ Questo parametro rappresenta la callback che il metodo dovrà chiamare al termine dell'elaborazione per passare il controllo ai successivi middleware. La presenza del parametro facoltativo MaapError attiva la catena di gestione dell'errore in sostituzione della normale catena di gestione delle richieste.
\end{itemize}
\item[] \method{+passwordReset(req:Request, res:Response, next:function(MaapError))} \\ Questo metodo si occupa di andare ad eseguire il reset della password utente a cui il token appartiene, con la nuova password inserita. 
\begin{itemize}\addtolength{\itemsep}{-0.5\baselineskip}
\item[$\circ$] \parameter{req:Request} \\ Questo oggetto rappresenta la richiesta di tipo Request arrivata al server che il metodo deve gestire.
\item[$\circ$] \parameter{res:Response} \\ Questo oggetto rappresenta la risposta che il server dovrà rispondere al termine dell'elaborazione.
\item[$\circ$] \parameter{next:function(MaapError)} \\ Questo parametro rappresenta la callback che il metodo dovrà chiamare al termine dell'elaborazione per passare il controllo ai successivi middleware. La presenza del parametro facoltativo MaapError attiva la catena di gestione dell'errore in sostituzione della normale catena di gestione delle richieste.
\end{itemize}
\end{itemize}

\subsubsection{Classe ProfileService}

\begin{table}[H]
\begin{center}
\bgroup
\setlength{\arrayrulewidth}{0.6mm}
\def\arraystretch{1}
\begin{tabular}{ | p{12cm} | }
\hline
\centerline{\textbf{ProfileService}}
\\ \hline
 \\ 
\hline
\code{+login(req:Request, res:Response, next:function(MaapError))} \\
\code{+logout(req:Request, res:Response, next:function(MaapError))} \\
\code{+getProfile(req:Request, res:Response, next:function(MaapError))} \\
\code{+updatePassword(req:Request, res:Request, next:function(MaapError))} \\
\hline
\end{tabular}
\egroup
\caption{Classe ProfileService}
\end{center}
\end{table}

\paragraph*{Descrizione}
\begin{itemize}
\item[] Classe che rappresenta la gestione di un profilo utente, il login e il logout. È uno dei componenti product del \glossario{Design Pattern} \glossario{Factory method}.

\end{itemize}

\paragraph*{Utilizzo}
\begin{itemize}
\item[] Viene utilizzata per visualizzare il profilo dell'utente, tramite GET, e per editarlo tramite PUT. Viene anche utilizzata per gestire i dati di e le operazioni relativi all'autenticazione utente e al suo logout dall'applicazione, occupandosi della creazione della sessione utente e della sua distruzione tramite \glossario{cookies}.
\end{itemize}

\paragraph*{Relazioni con altre classi}
Assenti
% TODO: deve esserci almeno una relazione con questa classe!!!

\paragraph*{Attributi}
\begin{itemize}
\item[] Assenti
\end{itemize}

\paragraph*{Metodi}
\begin{itemize}
\item[] \method{+login(req:Request, res:Response, next:function(MaapError))} \\ Metodo che si occupa di reindirizzare l'utente alla pagina \glossario{dashboard}.
\begin{itemize}\addtolength{\itemsep}{-0.5\baselineskip}
\item[$\circ$] \parameter{req:Request} \\ Questo oggetto rappresenta la richiesta di tipo Request arrivata al server che il metodo deve gestire.
\item[$\circ$] \parameter{res:Response} \\ Questo oggetto rappresenta la risposta che il server dovrà rispondere al termine dell'elaborazione.
\item[$\circ$] \parameter{next:function(MaapError)} \\ Questo parametro rappresenta la callback che il metodo dovrà chiamare al termine dell'elaborazione per passare il controllo ai successivi middleware. La presenza del parametro facoltativo MaapError attiva la catena di gestione dell'errore in sostituzione della normale catena di gestione delle richieste.
\end{itemize}
\item[] \method{+logout(req:Request, res:Response, next:function(MaapError))} \\ Questo metodo si occupa di distruggere la sessione utente e di reindirizzarlo alla pagina principale dell'applicazione.
\begin{itemize}\addtolength{\itemsep}{-0.5\baselineskip}
\item[$\circ$] \parameter{req:Request} \\ Questo oggetto rappresenta la richiesta di tipo Request arrivata al server che il metodo deve gestire.
\item[$\circ$] \parameter{res:Response} \\ Questo oggetto rappresenta la risposta che il server dovrà rispondere al termine dell'elaborazione.
\item[$\circ$] \parameter{next:function(MaapError)} \\ Questo parametro rappresenta la callback che il metodo dovrà chiamare al termine dell'elaborazione per passare il controllo ai successivi middleware. La presenza del parametro facoltativo MaapError attiva la catena di gestione dell'errore in sostituzione della normale catena di gestione delle richieste.
\end{itemize}
\item[] \method{+getProfile(req:Request, res:Response, next:function(MaapError))} \\ Metodo che risponde con le informazioni del profilo dell'utente. In caso avvengano errori, il metodo risponde con un json contenente le informazioni relative all'errore.
\begin{itemize}\addtolength{\itemsep}{-0.5\baselineskip}
\item[$\circ$] \parameter{req:Request} \\ Questo oggetto rappresenta la richiesta di tipo Request arrivata al server che il metodo deve gestire.
\item[$\circ$] \parameter{res:Response} \\ Questo oggetto rappresenta la risposta che il server dovrà rispondere al termine dell'elaborazione.
\item[$\circ$] \parameter{next:function(MaapError)} \\ Questo parametro rappresenta la callback che il metodo dovrà chiamare al termine dell'elaborazione per passare il controllo ai successivi middleware. La presenza del parametro facoltativo MaapError attiva la catena di gestione dell'errore in sostituzione della normale catena di gestione delle richieste.
\end{itemize}
\item[] \method{+updatePassword(req:Request, res:Request, next:function(MaapError))} \\ Questo metodo modifica la password utente servendosi del metodo \code{updatePassword} della classe \code{Back-end::Lib::Model::UserModel} e rispondendo con una stringa in caso di successo mentre in caso di fallimento con un json di errore.
\begin{itemize}\addtolength{\itemsep}{-0.5\baselineskip}
\item[$\circ$] \parameter{req:Request} \\ Questo oggetto rappresenta la richiesta di tipo Request arrivata al server che il metodo deve gestire.
\item[$\circ$] \parameter{res:Request} \\ Questo oggetto rappresenta la risposta che il server dovrà rispondere al termine dell'elaborazione.
\item[$\circ$] \parameter{next:function(MaapError)} \\ Questo parametro rappresenta la callback che il metodo dovrà chiamare al termine dell'elaborazione per passare il controllo ai successivi middleware. La presenza del parametro facoltativo MaapError attiva la catena di gestione dell'errore in sostituzione della normale catena di gestione delle richieste.
\end{itemize}
\end{itemize}

\subsubsection{Classe UserService}

\begin{table}[H]
\begin{center}
\bgroup
\setlength{\arrayrulewidth}{0.6mm}
\def\arraystretch{1}
\begin{tabular}{ | p{12cm} | }
\hline
\centerline{\textbf{UserService}}
\\ \hline
 \\ 
\hline
\code{+usersList(req:Request, res:Response, next:function(MaapError))} \\
\code{+insertUser(req:Request, res:Response, next:function(MaapError))} \\
\code{+registerUser(req:Request, res:Response, next:function(MaapError))} \\
\code{+userIdShowPage(next:function(MaapError), req:Request, res:Response)} \\
\code{+deleteUser(req:Request, res:Response, next:function(MaapError))} \\
\code{+updateLevel(req:Request, res:Response, next:function(MaapError))} \\
\code{+disabledRegisterUser(req:Request, res:Response, next:function(MaapError))} \\
\hline
\end{tabular}
\egroup
\caption{Classe UserService}
\end{center}
\end{table}

\paragraph*{Descrizione}
\begin{itemize}
\item[] Classe che si occupa della varie operazioni che l'admin può compiere sugli utenti dell'applicazione. È uno dei componenti product del \glossario{Design Pattern} \glossario{Factory method}.
\end{itemize}

\paragraph*{Utilizzo}
\begin{itemize}
\item[] Viene utilizzata per visualizzare la \glossario{index-page} degli utenti, visualizzare le relative \glossario{show-page}, eliminare un utente e modificare il profilo. Mette a disposizione dei metodi per effettuare tutte queste operazioni.
\end{itemize}

\paragraph*{Relazioni con altre classi}
Assenti
% TODO: deve esserci almeno una relazione con questa classe!!!

\paragraph*{Attributi}
\begin{itemize}
\item[] Assenti
\end{itemize}

\paragraph*{Metodi}
\begin{itemize}
\item[] \method{+usersList(req:Request, res:Response, next:function(MaapError))} \\ Metodo che chiama la funzione \code{getUserList()} dello schema utente in \code{Back-end::Lib::Model::UserModel} facendosi restituire la lista di tutti gli utenti presenti nel sistema e restituendola in risposta. Nel caso si verifichi un errore risponde invece con un errore.
\begin{itemize}\addtolength{\itemsep}{-0.5\baselineskip}
\item[$\circ$] \parameter{req:Request} \\ Questo oggetto rappresenta la richiesta di tipo Request arrivata al server che il metodo deve gestire.
\item[$\circ$] \parameter{res:Response} \\ Questo oggetto viene modificato dal metodo durante l'elaborazione, rappresenta la risposta che il server dovrà rispondere.
\item[$\circ$] \parameter{next:function(MaapError)} \\ Questo parametro rappresenta la callback che il metodo dovrà chiamare al termine dell'elaborazione per passare il controllo ai successivi middleware. La presenza del parametro facoltativo MaapError attiva la catena di gestione dell'errore in sostituzione della normale catena di gestione delle richieste.
\end{itemize}
\item[] \method{+insertUser(req:Request, res:Response, next:function(MaapError))} \\ Metodo che Inserisce un nuovo utente nel database. Nel caso l'elaborazione abbia causato errori risponde con un json di informazioni sull'errore.
\begin{itemize}\addtolength{\itemsep}{-0.5\baselineskip}
\item[$\circ$] \parameter{req:Request} \\ Questo oggetto rappresenta la richiesta di tipo Request arrivata al server che il metodo deve gestire.
\item[$\circ$] \parameter{res:Response} \\ Questo oggetto rappresenta la risposta che il server dovrà rispondere al termine dell'elaborazione.
\item[$\circ$] \parameter{next:function(MaapError)} \\ Questo parametro rappresenta la callback che il metodo dovrà chiamare al termine dell'elaborazione per passare il controllo ai successivi middleware. La presenza del parametro facoltativo MaapError attiva la catena di gestione dell'errore in sostituzione della normale catena di gestione delle richieste.
\end{itemize}
\item[] \method{+registerUser(req:Request, res:Response, next:function(MaapError))} \\ Metodo che registra un nuovo utente nel database tramite la funzione \code{registerUser()} della classe \code{Back-end::Lib::Model::UserModel}, rispondendo con una stringa in caso di successo o con un json di errore in caso di fallimento dell'elaborazione.
\begin{itemize}\addtolength{\itemsep}{-0.5\baselineskip}
\item[$\circ$] \parameter{req:Request} \\ Questo oggetto rappresenta la richiesta di tipo Request arrivata al server che il metodo deve gestire.
\item[$\circ$] \parameter{res:Response} \\ Questo oggetto rappresenta la risposta che il server dovrà rispondere al termine dell'elaborazione.
\item[$\circ$] \parameter{next:function(MaapError)} \\ Questo parametro rappresenta la callback che il metodo dovrà chiamare al termine dell'elaborazione per passare il controllo ai successivi middleware. La presenza del parametro facoltativo MaapError attiva la catena di gestione dell'errore in sostituzione della normale catena di gestione delle richieste.
\end{itemize}
\item[] \method{+userIdShowPage(next:function(MaapError), req:Request, res:Response)} \\ Metodo che risponde con i dati di un utente ottenuti tramite la funzione \code{getUserById()} della classe \code{Back-end::Lib::Model::UserModel}, in caso di errore risponderà con un json di errore.
\begin{itemize}\addtolength{\itemsep}{-0.5\baselineskip}
\item[$\circ$] \parameter{req:Request} \\ Questo oggetto rappresenta la richiesta di tipo Request arrivata al server che il metodo deve gestire.
\item[$\circ$] \parameter{res:Response} \\ Questo oggetto rappresenta la risposta che il server dovrà rispondere al termine dell'elaborazione.
\item[$\circ$] \parameter{next:function(MaapError)} \\ Questo parametro rappresenta la callback che il metodo dovrà chiamare al termine dell'elaborazione per passare il controllo ai successivi middleware. La presenza del parametro facoltativo MaapError attiva la catena di gestione dell'errore in sostituzione della normale catena di gestione delle richieste.
\end{itemize}
\item[] \method{+deleteUser(req:Request, res:Response, next:function(MaapError))} \\ Questo metodo elimina un utente dal database utenti, utilizzando la funzione predisposta dal modello utente \code{deleteUser()}. Nel caso si verifichi un errore durante l'esecuzione, il metodo risponde con un json contenente le informazioni dell'errore.
\begin{itemize}\addtolength{\itemsep}{-0.5\baselineskip}
\item[$\circ$] \parameter{req:Request} \\ Questo oggetto rappresenta la richiesta di tipo Request arrivata al server che il metodo deve gestire.
\item[$\circ$] \parameter{res:Response} \\ Questo oggetto rappresenta la risposta che il server dovrà rispondere al termine dell'elaborazione.
\item[$\circ$] \parameter{next:function(MaapError)} \\ Questo parametro rappresenta la callback che il metodo dovrà chiamare al termine dell'elaborazione per passare il controllo ai successivi middleware. La presenza del parametro facoltativo MaapError attiva la catena di gestione dell'errore in sostituzione della normale catena di gestione delle richieste.
\end{itemize}
\item[] \method{+updateLevel(req:Request, res:Response, next:function(MaapError))} \\ Questo metodo modifica il livello di un utente tramite la funzione \code{updateLevel()} della classe \code{Back-end::Lib::Model::UserModel}.
\begin{itemize}\addtolength{\itemsep}{-0.5\baselineskip}
\item[$\circ$] \parameter{req:Request} \\ Questo oggetto rappresenta la richiesta di tipo Request arrivata al server che il metodo deve gestire.
\item[$\circ$] \parameter{res:Response} \\ Questo oggetto rappresenta la risposta che il server dovrà rispondere al termine dell'elaborazione.
\item[$\circ$] \parameter{next:function(MaapError)} \\ Questo parametro rappresenta la callback che il metodo dovrà chiamare al termine dell'elaborazione per passare il controllo ai successivi middleware. La presenza del parametro facoltativo MaapError attiva la catena di gestione dell'errore in sostituzione della normale catena di gestione delle richieste.
\end{itemize}
\item[] \method{+disabledRegisterUser(req:Request, res:Response, next:function(MaapError))} \\ Metodo che ritorna un messaggio di errore nel caso in cui la registrazione all'applicazione sia stata disabilitata dallo sviluppatore nella configurazione.
\begin{itemize}\addtolength{\itemsep}{-0.5\baselineskip}
\item[$\circ$] \parameter{req:Request} \\ Questo oggetto rappresenta la richiesta di tipo Request arrivata al server che il metodo deve gestire.
\item[$\circ$] \parameter{res:Response} \\ Questo oggetto rappresenta la risposta che il server dovrà rispondere al termine dell'elaborazione.
\item[$\circ$] \parameter{next:function(MaapError)} \\ Questo parametro rappresenta la callback che il metodo dovrà chiamare al termine dell'elaborazione per passare il controllo ai successivi middleware. La presenza del parametro facoltativo MaapError attiva la catena di gestione dell'errore in sostituzione della normale catena di gestione delle richieste.
\end{itemize}
\end{itemize}

\subsubsection{Classe ShowService}

\begin{table}[H]
\begin{center}
\bgroup
\setlength{\arrayrulewidth}{0.6mm}
\def\arraystretch{1}
\begin{tabular}{ | p{12cm} | }
\hline
\centerline{\textbf{ShowService}}
\\ \hline
 \\ 
\hline
\code{+getShowPage(req:Request, res:Response, next:function(MaapError))} \\
\code{+deleteDocument(req:Request, res:Response, next:function(MaapError))} \\
\code{+editDocument(next:function(MaapError), req:Request, res:Response)} \\
\hline
\end{tabular}
\egroup
\caption{Classe ShowService}
\end{center}
\end{table}

\paragraph*{Descrizione}
\begin{itemize}
\item[] Classe che si occupa della gestione della risorsa show-page. È uno dei componenti \textit{product} del \glossario{Design Pattern} \glossario{Factory method}.
\end{itemize}

\paragraph*{Utilizzo}
\begin{itemize}
\item[] Viene utilizzata per gestire una richiesta della risorsa show-page, delegando alla classe \textit{Back-end::Lib::DSLModel::DSLDomain} il compito di eseguire la query e restituire i dati in formato JSON.
\end{itemize}

\paragraph*{Relazioni con altre classi}
Assenti
% TODO: deve esserci almeno una relazione con questa classe!!!

\paragraph*{Attributi}
\begin{itemize}
\item[] Assenti
\end{itemize}

\paragraph*{Metodi}
\begin{itemize}
\item[] \method{+getShowPage(req:Request, res:Response, next:function(MaapError))} \\ Questo metodo si occupa di andare a prelevare il \code{Back-end::Lib::Model::DSLModel::DSLCollectionModel} eseguendo una ricerca all'interno del registro della classe \code{Back-end::Lib::Model::DSLModel::DSLDomain}. Se la ricerca ha successo, viene restituita una \code{DSLCollectionModel}. A questo punto si ottiene lo \code{showModel} dal \code{DSLCollectionModel} e da quest'ultimo viene richiesta la rappresentazione della show-page in formato json, il quale viene utilizzato come risposta del server.
\begin{itemize}\addtolength{\itemsep}{-0.5\baselineskip}
\item[$\circ$] \parameter{req:Request} \\ Questo oggetto rappresenta la richiesta di tipo Request arrivata al server che il metodo deve gestire.
\item[$\circ$] \parameter{res:Response} \\ Questo oggetto rappresenta la risposta che il server dovrà rispondere al termine dell'elaborazione.
\item[$\circ$] \parameter{next:function(MaapError)} \\ Questo parametro rappresenta la callback che il metodo dovrà chiamare al termine dell'elaborazione per passare il controllo ai successivi middleware. La presenza del parametro facoltativo MaapError attiva la catena di gestione dell'errore in sostituzione della normale catena di gestione delle richieste.
\end{itemize}
\item[] \method{+deleteDocument(req:Request, res:Response, next:function(MaapError))} \\ Questo metodo si occupa di eliminare un \glossario{document} dalla \glossario{Collection}.
\begin{itemize}\addtolength{\itemsep}{-0.5\baselineskip}
\item[$\circ$] \parameter{req:Request} \\ Questo oggetto rappresenta la richiesta di tipo Request arrivata al server che il metodo deve gestire.
\item[$\circ$] \parameter{res:Response} \\ Questo oggetto rappresenta la risposta che il server dovrà rispondere al termine dell'elaborazione.
\item[$\circ$] \parameter{next:function(MaapError)} \\ Questo parametro rappresenta la callback che il metodo dovrà chiamare al termine dell'elaborazione per passare il controllo ai successivi middleware. La presenza del parametro facoltativo MaapError attiva la catena di gestione dell'errore in sostituzione della normale catena di gestione delle richieste.
\end{itemize}
\item[] \method{+editDocument(next:function(MaapError), req:Request, res:Response)} \\ Questo metodo si occupa di modificare il document servendosi di metodi della classe \texttt{Back-end::Lib::Model::DSLModel}.
\begin{itemize}\addtolength{\itemsep}{-0.5\baselineskip}
\item[$\circ$] \parameter{req:Request} \\ Questo oggetto rappresenta la richiesta di tipo Request arrivata al server che il metodo deve gestire.
\item[$\circ$] \parameter{res:Response} \\ Questo oggetto rappresenta la risposta che il server dovrà rispondere al termine dell'elaborazione.
\item[$\circ$] \parameter{next:function(MaapError)} \\ Questo parametro rappresenta la callback che il metodo dovrà chiamare al termine dell'elaborazione per passare il controllo ai successivi middleware. La presenza del parametro facoltativo MaapError attiva la catena di gestione dell'errore in sostituzione della normale catena di gestione delle richieste.
\end{itemize}
\end{itemize}

\subsubsection{Classe IndexService}

\begin{table}[H]
\begin{center}
\bgroup
\setlength{\arrayrulewidth}{0.6mm}
\def\arraystretch{1}
\begin{tabular}{ | p{12cm} | }
\hline
\centerline{\textbf{IndexService}}
\\ \hline
 \\ 
\hline
\code{+getIndexPage(req:Request, res:Response, next:function(MaapError))} \\
\hline
\end{tabular}
\egroup
\caption{Classe IndexService}
\end{center}
\end{table}

\paragraph*{Descrizione}
\begin{itemize}
\item[] Classe di gestione per la risorsa index 
È uno dei componenti product del \glossario{Design Pattern} \glossario{Factory method}.

\end{itemize}

\paragraph*{Utilizzo}
\begin{itemize}
\item[] Viene utilizzata per gestire la risorsa corrispondente all'index-page di un \glossario{Document}, offrendo metodi per restituirne gli attributi, effettuarne la modifica o la cancellazione e delega la visualizzazione dell'index-page alla classe \texttt{Back-end::Lib::DSLModel::DSLDomain}.

\end{itemize}

\paragraph*{Relazioni con altre classi}
Assenti
% TODO: deve esserci almeno una relazione con questa classe!!!

\paragraph*{Attributi}
\begin{itemize}
\item[] Assenti
\end{itemize}

\paragraph*{Metodi}
\begin{itemize}
\item[] \method{+getIndexPage(req:Request, res:Response, next:function(MaapError))} \\ Questo metodo si occupa di andare a prelevare il \code{Back-end::Lib::Model::DSLModel::DSLCollectionModel} eseguendo una ricerca all'interno del registro della classe \code{Back-end::Lib::Model::DSLModel::DSLDomain}. Se la ricerca ha successo, viene restituita una \code{DSLCollectionModel}. A questo punto si ottiene l' \code{indexModel} dal \code{DSLCollectionModel} e da quest'ultimo viene richiesta la rappresentazione della index-page in formato json, il quale viene utilizzato come risposta del server.
\begin{itemize}\addtolength{\itemsep}{-0.5\baselineskip}
\item[$\circ$] \parameter{req:Request} \\ Questo oggetto rappresenta la richiesta di tipo Request arrivata al server che il metodo deve gestire.
\item[$\circ$] \parameter{res:Response} \\ Questo oggetto rappresenta la risposta che il server dovrà rispondere al termine dell'elaborazione.
\item[$\circ$] \parameter{next:function(MaapError)} \\ Questo parametro rappresenta la callback che il metodo dovrà chiamare al termine dell'elaborazione per passare il controllo ai successivi middleware. La presenza del parametro facoltativo MaapError attiva la catena di gestione dell'errore in sostituzione della normale catena di gestione delle richieste.
\end{itemize}
\end{itemize}

\subsubsection{Classe CollectionService}

\begin{table}[H]
\begin{center}
\bgroup
\setlength{\arrayrulewidth}{0.6mm}
\def\arraystretch{1}
\begin{tabular}{ | p{12cm} | }
\hline
\centerline{\textbf{CollectionService}}
\\ \hline
 \\ 
\hline
\code{+list(req:Request, res:Response, next:function(MaapError))} \\
\hline
\end{tabular}
\egroup
\caption{Classe CollectionService}
\end{center}
\end{table}

\paragraph*{Descrizione}
\begin{itemize}
\item[] Classe di gestione per la risorsa Collection. È uno dei componenti product del \glossario{Design Pattern} \glossario{Factory method}.
\end{itemize}

\paragraph*{Utilizzo}
\begin{itemize}
\item[] Viene utilizzata per gestire la risorsa corrispondente alle Collection, offrendo metodi per restituire tutte le collection presenti nell'applicazione.
\end{itemize}

\paragraph*{Relazioni con altre classi}
Assenti
% TODO: deve esserci almeno una relazione con questa classe!!!

\paragraph*{Attributi}
\begin{itemize}
\item[] Assenti
\end{itemize}

\paragraph*{Metodi}
\begin{itemize}
\item[] \method{+list(req:Request, res:Response, next:function(MaapError))} \\ Questo metodo risponde con la lista dei nomi delle collection presenti nell'applicazione.
\begin{itemize}\addtolength{\itemsep}{-0.5\baselineskip}
\item[$\circ$] \parameter{req:Request} \\ Questo oggetto rappresenta la richiesta di tipo Request arrivata al server che il metodo deve gestire.
\item[$\circ$] \parameter{res:Response} \\ Questo oggetto rappresenta la risposta che il server dovrà rispondere al termine dell'elaborazione.
\item[$\circ$] \parameter{next:function(MaapError)} \\ Questo parametro rappresenta la callback che il metodo dovrà chiamare al termine dell'elaborazione per passare il controllo ai successivi middleware. La presenza del parametro facoltativo MaapError attiva la catena di gestione dell'errore in sostituzione della normale catena di gestione delle richieste.
\end{itemize}
\end{itemize}

\subsection{Componente Back-end::Lib::Model}

\subsubsection{Classe UserModel}

\begin{table}[H]
\begin{center}
\bgroup
\setlength{\arrayrulewidth}{0.6mm}
\def\arraystretch{1}
\begin{tabular}{ | p{12cm} | }
\hline
\centerline{\textbf{UserModel}}
\\ \hline
\code{- UserSchema:Schema} \\
\hline
\code{+init(app:ServerApp)} \\
\code{+\underline{createUser}(newUser:JSON, callback:function(JSON), errback:function(MaapError))} \\
\code{+updatePassword(userId :String, callback:function(String), errback:function(MaapError))} \\
\code{+updateLevel(newLevel:String, callback:function(String), errback:function(MaapError))} \\
\code{+\underline{findAll}(page:Integer, callback:function(String), errback:function(MaapError))} \\
\code{+\underline{findAllPaginated}(perpage:Integer, page:Integer, callback:function(String), errback:function(MaapError))} \\
\code{+\underline{safeFindById}(id:String, callback:function(String), errback:function(MaapError))} \\
\code{+\underline{safeFindByEmail}(userEmail:String, callback:function(String), errback:function(MaapError))} \\
\code{+\underline{safeFindByIdAndRemove}(id:String, callback:function(String), errback:function(MaapError))} \\
\code{+generateResetPasswordToken(callback:function(String), errback:function(MaapError))} \\
\code{+invalidateResetPasswordToken(callback:function(String), errback:function(MaapError))} \\
\code{+consumeResetPasswordTokenAndUpdatePassword(newPassword:String, callback:function(String), errback:function(MaapError))} \\
\code{+\underline{safeFindByResetPasswordToken}(token:String, callback:function(String), errback:function(MaapError))} \\
\code{+\underline{countAll}(callback:function(Object), errback:function(MaapError))} \\
\hline
\end{tabular}
\egroup
\caption{Classe UserModel}
\end{center}
\end{table}

\paragraph*{Descrizione}
\begin{itemize}
\item[] Classe che si occupa dei metodi per la gestione dei dati utente. 
\end{itemize}

\paragraph*{Utilizzo}
\begin{itemize}
\item[] Viene utilizzata per l'interfacciamento con la libreria \glossario{Mongoose} per la registrazione dello schema dei dati, e con la libreria passport-local-mongoose per il popolamento automatico dello schema con campi dati e metodi predefiniti.
Il costruttore del modello dello schema dei dati viene registrato nella \glossario{Factory} di \glossario{Mongoose} ed ogni istanza condividerà la stessa connessione al server. \\
L' istanza della classe UserModel descrive un documento della collection, i metodi di istanza definiscono le operazioni sul documento mentre i metodi statici della classe descrivono le operazioni e le query che è possibile fare sulla Collection in accordo con i metodi di Mongoose, che utilizzano allo stesso modo la distinzione tra metodi statici e di istanza per operazioni sulla Collection e sul Document (rispettivamente).

\end{itemize}

\paragraph*{Relazioni con altre classi}
Assenti
% TODO: deve esserci almeno una relazione con questa classe!!!

\paragraph*{Attributi}
\begin{itemize}
\item[] \attribute{- UserSchema:Schema} \\ Questo campo dati rappresenta lo schema \glossario{Mongoose} dell'utente \glossario{MaaP}. \\
Lo schema prevede tre attributi:
\begin{itemize}
\item[]  \texttt{email} di tipo \code{String}
\item[]  \texttt{password} di tipo \code{String}
\item[]  \texttt{level} di tipo \code{enum} con tre possibili valori: 
\begin{enumerate}
\item Utente
\item Admin
\item SuperAdmin
\end{enumerate}
\end{itemize}

\end{itemize}

\paragraph*{Metodi}
\begin{itemize}
\item[] \method{+init(app:ServerApp)} \\ Metodo che definisce lo schema \glossario{mongoose} dell'utente rendendo disponibili i metodi da utilizzare per la modifica/creazione/eliminazione di quest'ultimo.
\begin{itemize}\addtolength{\itemsep}{-0.5\baselineskip}
\item[$\circ$] \parameter{app:ServerApp} \\ È l'istanza del server dell'applicazione.
\end{itemize}
\item[] \method{+\underline{createUser}(newUser:JSON, callback:function(JSON), errback:function(MaapError))} \\ Metodo che crea un nuovo utente nel database degli utenti. Al termine dell'operazione senza errori risponde con un json con le informazioni dell'utente appena creato altrimenti risponde con un errore. 
\begin{itemize}\addtolength{\itemsep}{-0.5\baselineskip}
\item[$\circ$] \parameter{newUser:JSON} \\ Questo parametro rappresenta i dati utente da utilizzare nella creazione di un nuovo utente.
\item[$\circ$] \parameter{callback:function(JSON)} \\ Questo parametro rappresenta la callback che il metodo deve chiamare al termine dell'elaborazione senza errori, dandogli il json con le informazioni dell'utente creato.
\item[$\circ$] \parameter{errback:function(MaapError)} \\ Questo parametro rappresenta la callback che il metodo dovrà chiamare se si sono verificati errori durante l'elaborazione passandogli l'errore.
\end{itemize}
\item[] \method{+updatePassword(userId :String, callback:function(String), errback:function(MaapError))} \\ Questo metodo si occupa di modificare il dato corrispondente alla password di un utente presente nel database delle credenziali utente.
\begin{itemize}\addtolength{\itemsep}{-0.5\baselineskip}
\item[$\circ$] \parameter{userId :String} \\ Parametro rappresentante l'id dell'utente di cui modificare i dati.
\item[$\circ$] \parameter{callback:function(String)} \\ Parametro che rappresenta la callback chiamata dal metodo al termine dell'elaborazione senza errori.
\item[$\circ$] \parameter{errback:function(MaapError)} \\ Parametro che rappresenta la callback chiamata dal metodo al verificarsi di errori.
\end{itemize}
\item[] \method{+updateLevel(newLevel:String, callback:function(String), errback:function(MaapError))} \\ Questo metodo modifica il livello di un utente presente nel database degli utenti.
\begin{itemize}\addtolength{\itemsep}{-0.5\baselineskip}
\item[$\circ$] \parameter{newLevel:String} \\ Parametro che rappresenta il dato sul livello utente.
\item[$\circ$] \parameter{callback:function(String)} \\ Questo parametro rappresenta la callback che il metodo deve richiamare al termine dell'elaborazione senza errori.
\item[$\circ$] \parameter{errback:function(MaapError)} \\ Parametro che rappresenta la callback chiamata dal metodo al verificarsi di errori durante la modifica.
\end{itemize}
\item[] \method{+\underline{findAll}(page:Integer, callback:function(String), errback:function(MaapError))} \\ Questo metodo si occupa di ritornare tutti gli utenti nel sistema \glossario{MaaP}.
\begin{itemize}\addtolength{\itemsep}{-0.5\baselineskip}
\item[$\circ$] \parameter{page:Integer} \\ Parametro che rappresenta il numero di pagina.
\item[$\circ$] \parameter{callback:function(String)} \\ Parametro corrispondente alla callback che il metodo richiama al termine dell'elaborazione senza errori.
\item[$\circ$] \parameter{errback:function(MaapError)} \\ Parametro rappresentante la callback che il metodo deve chiamare se durante l'elaborazione avvengono errori.
\end{itemize}
\item[] \method{+\underline{findAllPaginated}(perpage:Integer, page:Integer, callback:function(String), errback:function(MaapError))} \\ Questo metodo si occupa di ottenere la lista di tutti gli utenti, dividendo per pagine e partendo da 1, restituendo la lista degli \code{perpage} utenti della pagina \code{page}.
\begin{itemize}\addtolength{\itemsep}{-0.5\baselineskip}
\item[$\circ$] \parameter{perpage:Integer} \\ Questo parametro rappresenta il numero di utenti da visualizzare in una pagina.
\item[$\circ$] \parameter{page:Integer} \\ Parametro rappresentante il numero di pagina di cui si richiedono gli utenti da mostrare.
\item[$\circ$] \parameter{callback:function(String)} \\ Questo parametro rappresenta la callback che il metodo deve richiamare al termine dell'elaborazione senza errori.
\item[$\circ$] \parameter{errback:function(MaapError)} \\ Parametro che rappresenta la callback chiamata dal metodo al verificarsi di errori durante l'esecuzione.
\end{itemize}
\item[] \method{+\underline{safeFindById}(id:String, callback:function(String), errback:function(MaapError))} \\ Questo metodo ritorna le informazioni dell'utente a cui l'id è associato.
\begin{itemize}\addtolength{\itemsep}{-0.5\baselineskip}
\item[$\circ$] \parameter{id:String} \\ Parametro che rappresenta l'id dell'utente da ritornare.
\item[$\circ$] \parameter{callback:function(String)} \\ Questo parametro rappresenta la callback che il metodo dovrà richiamare al termine dell'esecuzione senza errori.
\item[$\circ$] \parameter{errback:function(MaapError)} \\ Parametro che rappresenta la callback chiamata dal metodo al verificarsi di errori durante l'elaborazione.
\end{itemize}
\item[] \method{+\underline{safeFindByEmail}(userEmail:String, callback:function(String), errback:function(MaapError))} \\ Metodo che restituisce i dati dell'utente a cui l'email passata come parametro corrisponde.
\begin{itemize}\addtolength{\itemsep}{-0.5\baselineskip}
\item[$\circ$] \parameter{userEmail:String} \\ Parametro corrispondente all'email dell'utente da restituire.
\item[$\circ$] \parameter{callback:function(String)} \\ Questo parametro rappresenta la callback che il metodo deve richiamare al termine dell'elaborazione senza errori.
\item[$\circ$] \parameter{errback:function(MaapError)} \\ Parametro che rappresenta la callback chiamata dal metodo al verificarsi di errori durante l'esecuzione.
\end{itemize}
\item[] \method{+\underline{safeFindByIdAndRemove}(id:String, callback:function(String), errback:function(MaapError))} \\ Questo metodo si occupa di ricercare un utente e di eliminarlo.
\begin{itemize}\addtolength{\itemsep}{-0.5\baselineskip}
\item[$\circ$] \parameter{id:String} \\ Parametro corrispondente all'id dell'utente da trovare ed eliminare.
\item[$\circ$] \parameter{callback:function(String)} \\ Questo parametro rappresenta la callback che il metodo deve richiamare al termine dell'elaborazione senza errori.
\item[$\circ$] \parameter{errback:function(MaapError)} \\ Parametro che rappresenta la callback chiamata dal metodo al verificarsi di errori durante l'elaborazione.
\end{itemize}
\item[] \method{+generateResetPasswordToken(callback:function(String), errback:function(MaapError))} \\ Metodo che si occupa di generare un token dandogli un tempo di vita dopo il quale il scade. Token e tempo di vita vengono salvati nelle informazioni dell'utente che ha richiesto il reset della password.
\begin{itemize}\addtolength{\itemsep}{-0.5\baselineskip}
\item[$\circ$] \parameter{callback:function(String)} \\ Questo parametro rappresenta la callback che il metodo deve richiamare al termine dell'elaborazione senza errori.
\item[$\circ$] \parameter{errback:function(MaapError)} \\ Parametro che rappresenta la callback da richiamare se durante l'esecuzione avvengono errori.
\end{itemize}
\item[] \method{+invalidateResetPasswordToken(callback:function(String), errback:function(MaapError))} \\ Questo metodo restituisce il token per il reset della password e poi lo invalida.
\begin{itemize}\addtolength{\itemsep}{-0.5\baselineskip}
\item[$\circ$] \parameter{callback:function(String)} \\ Questo parametro rappresenta la callback che il metodo deve richiamare al termine dell'elaborazione senza errori.
\item[$\circ$] \parameter{errback:function(MaapError)} \\ Parametro che rappresenta la callback chiamata dal metodo al verificarsi di errori durante l'esecuzione.
\end{itemize}
\item[] \method{+consumeResetPasswordTokenAndUpdatePassword(newPassword:String, callback:function(String), errback:function(MaapError))} \\ Metodo che si occupa di verificare se il token è valido, nel caso lo sia procede con il reset della password utente altrimenti risponde con un errore.
\begin{itemize}\addtolength{\itemsep}{-0.5\baselineskip}
\item[$\circ$] \parameter{newPassword:String} \\ Parametro che rappresenta la nuova password utente con cui effettuare il reset.
\item[$\circ$] \parameter{callback:function(String)} \\ Questo parametro rappresenta la callback che il metodo deve richiamare al termine dell'elaborazione senza errori.
\item[$\circ$] \parameter{errback:function(MaapError)} \\ Parametro che rappresenta la callback chiamata dal metodo al verificarsi di errori durante l'elaborazione.
\end{itemize}
\item[] \method{+\underline{safeFindByResetPasswordToken}(token:String, callback:function(String), errback:function(MaapError))} \\ Metodo che ritorna l'utente a cui il token relativo al reset della password corrisponde.
\begin{itemize}\addtolength{\itemsep}{-0.5\baselineskip}
\item[$\circ$] \parameter{token:String} \\ Parametro che corrisponde al token dell'utente da restituire.
\item[$\circ$] \parameter{callback:function(String)} \\ Questo parametro rappresenta la callback che il metodo deve richiamare al termine dell'elaborazione senza errori.
\item[$\circ$] \parameter{errback:function(MaapError)} \\ Parametro che rappresenta la callback chiamata dal metodo al verificarsi di errori durante l'elaborazione.
\end{itemize}
\item[] \method{+\underline{countAll}(callback:function(Object), errback:function(MaapError))} \\ Questo metodo restituisce il numero di utenti presenti nel database.
\begin{itemize}\addtolength{\itemsep}{-0.5\baselineskip}
\item[$\circ$] \parameter{callback:function(Object)} \\ Questo parametro è la callback da richiamare al termine dell'esecuzione passandogli il risultato.
\item[$\circ$] \parameter{errback:function(MaapError)} \\ Parametro rappresentante la callback che il metodo deve richiamare al verificarsi di errori.
\end{itemize}
\end{itemize}

\subsection{Componente Back-end::Lib::Model::DSLModel}

\subsubsection{Classe DSLDomain}

\begin{table}[H]
\begin{center}
\bgroup
\setlength{\arrayrulewidth}{0.6mm}
\def\arraystretch{1}
\begin{tabular}{ | p{12cm} | }
\hline
\centerline{\textbf{DSLDomain}}
\\ \hline
\code{- modelRegistry:Array} \\
\code{- errorRegistry:Array} \\
\code{- db:connection} \\
\code{+ strategy:ConcreteDSLInterpreter} \\
\hline
\code{+loadDSLFile(path:String, callback:function(String))} \\
\code{+registerCollection(model:DslCollectionModel)} \\
\code{+getCollectionModel(collectionId:String)} \\
\code{+getErrors():MaapError [0 ... *]} \\
\code{+DSLDomain(db:connection)} \\
\code{+init(callback:function(), errback:function(MaapError))} \\
\code{+registerError(error:MaapError)} \\
\code{+compareCollectionWeight(modelA:DslCollectionModel, modelB:DslCollectionModel):Integer} \\
\code{+getCollectionModels():DslCollectionModel [0 ... *]} \\
\hline
\end{tabular}
\egroup
\caption{Classe DSLDomain}
\end{center}
\end{table}

\paragraph*{Descrizione}
\begin{itemize}
\item[] Classe che si occupa di caricare i file \glossario{DSL}. Implementa il \glossario{Design Pattern} \glossario{registry}.
\end{itemize}

\paragraph*{Utilizzo}
\begin{itemize}
\item[] Viene utilizzata per caricare dinamicamente tutti i \glossario{DSL} a partire dal \glossario{database} che le viene passato.
\end{itemize}

\paragraph*{Relazioni con altre classi}
\begin{itemize}


\item[] Utilizza le classi:
\begin{itemize}
\item[$\bullet$] \class{Back-end::Lib::Model::DSLModel::DSLInterpreterStrategy}
\item[$\bullet$] \class{Back-end::Lib::Model::DSLModel::DSLCollectionModel}
\end{itemize}
\end{itemize}

\paragraph*{Attributi}
\begin{itemize}
\item[] \attribute{- modelRegistry:Array} \\ Questo campo dati rappresenta un registro all'interno del quale sono contenuti tutti i \texttt{DSLCollectionModel} caricati all'avvio del server.
\item[] \attribute{- errorRegistry:Array} \\ Questo campo dati contiene un registro di \texttt{MaapError} generati durante il caricamento e l'interpretazione dei file DSL.
\item[] \attribute{- db:connection} \\ Questo campo dati rappresenta un riferimento alla connessione con il database delle Collection.
\item[] \attribute{+ strategy:ConcreteDSLInterpreter} \\ Questo campo dati rappresenta un riferimento all'interprete dei file DSL.
\end{itemize}

\paragraph*{Metodi}
\begin{itemize}
\item[] \method{+loadDSLFile(path:String, callback:function(String))} \\ Questo metodo prende in input il \textit{path} di un file DSL da andare ad interpretare. Per fare ciò legge il contenuto testuale del file e lo converte in stringa. Questa stringa viene poi passata all'interprete del DSL tramite una chiamata. Questa chiamata restituirà tramite una callback un array di \texttt{DSLCollectionModel} che andranno inserite nel registro. Se avviene un errore nella lettura del file o nell'interpretazione del DSL viene sollevato un \texttt{MaapError} che viene aggiunto al registro degli errori.
\begin{itemize}\addtolength{\itemsep}{-0.5\baselineskip}
\item[$\circ$] \parameter{path:String} \\ Rappresenta il percorso del file da leggere.
\item[$\circ$] \parameter{callback:function(String)} \\ Rappresenta la funzione callback da chiamare una volta che il metodo è stato eseguito con successo. Prende come parametro un messaggio.
\end{itemize}
\item[] \method{+registerCollection(model:DslCollectionModel)} \\ Questo metodo si occupa di inserire nel registro delle \glossario{DSLCollectionModel} il \texttt{DSLCollectionModel} indicato.
\begin{itemize}\addtolength{\itemsep}{-0.5\baselineskip}
\item[$\circ$] \parameter{model:DslCollectionModel} \\ Questo parametro rappresenta il DslCollectionModel da inserire nel registro.
\end{itemize}
\item[] \method{+getCollectionModel(collectionId:String)} \\ Questo metodo effettua una ricerca all'interno del registro delle Collection in base all'id indicata. Se lo trova restituisce la Collection richiesta, altrimenti restituisce \texttt{undefined}.
\begin{itemize}\addtolength{\itemsep}{-0.5\baselineskip}
\item[$\circ$] \parameter{collectionId:String} \\ Questo parametro rappresenta l'identificativo della Collection da cercare all'interno del registro.
\end{itemize}
\item[] \method{+getErrors():MaapError [0 ... *]} \\ Questo metodo restituisce il registro degli errori generati.
\item[] \method{+DSLDomain(db:connection)} \\ È il metodo costruttore della classe.
\begin{itemize}\addtolength{\itemsep}{-0.5\baselineskip}
\item[$\circ$] \parameter{db:connection} \\ Questo parametro rappresenta il riferimento alla connessione di MongoDB del database delle Collections.
\end{itemize}
\item[] \method{+init(callback:function(), errback:function(MaapError))} \\ Questo metodo si occupa di inizializzare l'interprete dei file DSL.
\begin{itemize}\addtolength{\itemsep}{-0.5\baselineskip}
\item[$\circ$] \parameter{callback:function()} \\ Questo parametro è una funzione callback che viene chiamata nel caso in cui l'interprete venga inizializzato correttamente.
\item[$\circ$] \parameter{errback:function(MaapError)} \\ Questo parametro rappresenta la funzione callback che viene invocata nel caso in cui l'inizializzazione dell'interprete non vada a buon fine e venga generato dunque un errore.
\end{itemize}
\item[] \method{+registerError(error:MaapError)} \\ Questo metodo si occupa di inserire l'errore indicato nel registro degli errori.
\begin{itemize}\addtolength{\itemsep}{-0.5\baselineskip}
\item[$\circ$] \parameter{error:MaapError} \\ Questo parametro rappresenta un riferimento all'errore da inserire.
\end{itemize}
\item[] \method{+compareCollectionWeight(modelA:DslCollectionModel, modelB:DslCollectionModel):Integer} \\ Questo metodo riceve in input due \texttt{DSLCollectionModel} e ne compara l'attributo \texttt{weight}. In base all'operazione di confronto restituisce -1 se la prima pesa meno della seconda, 0 se hanno lo stesso peso e 1 se la seconda pesa meno della prima.
\begin{itemize}\addtolength{\itemsep}{-0.5\baselineskip}
\item[$\circ$] \parameter{modelA:DslCollectionModel} \\ Questo parametro rappresenta un riferimento a un \texttt{DSLCollectionModel}.
\item[$\circ$] \parameter{modelB:DslCollectionModel} \\ Questo parametro rappresenta un riferimento a un \texttt{DSLCollectionModel}.
\end{itemize}
\item[] \method{+getCollectionModels():DslCollectionModel [0 ... *]} \\ Questo metodo restituisce l'array di \texttt{DslCollectionModel} ordinate in base al peso.
\end{itemize}

\subsubsection{Classe DSLInterpreterStrategy}

\begin{table}[H]
\begin{center}
\bgroup
\setlength{\arrayrulewidth}{0.6mm}
\def\arraystretch{1}
\begin{tabular}{ | p{12cm} | }
\hline
\centerline{\textbf{\textit{DSLInterpreterStrategy}}}
\\ \hline
 \\ 
\hline
\code{+\textit{loadDSLFile}()} \\
\hline
\end{tabular}
\egroup
\caption{Classe DSLInterpreterStrategy}
\end{center}
\end{table}

\paragraph*{Descrizione}
\begin{itemize}
\item[] Classe astratta che definisce l'interfaccia dell'algoritmo di interpretazione del linguaggio \glossario{DSL} utilizzato. È il componente strategy del \glossario{Design Pattern} \glossario{strategy}.
\end{itemize}

\paragraph*{Utilizzo}
\begin{itemize}
\item[] Viene utilizzata per incapsulare e rendere intercambiabile l'algoritmo di interpretazione del linguaggio \glossario{DSL}. In questo modo, se in futuro vi fosse necessità di cambiare l'algoritmo di interpretazione l'algoritmo può variare indipendentemente dal client che ne farà uso.
\end{itemize}

\paragraph*{Relazioni con altre classi}
\begin{itemize}

\item[] È estesa dalle classi:
\begin{itemize}
\item \class{Back-end::Lib::Model::DSLModel::DSLInterpreterStrategy::ConcreteDSLInterpreter}
\end{itemize}

\end{itemize}

\paragraph*{Attributi}
\begin{itemize}
\item[] Assenti
\end{itemize}

\paragraph*{Metodi}
\begin{itemize}
\item[] \method{+\textit{loadDSLFile}()} \\ Questo metodo si occupa di interpretare il contenuto del file DSL passatogli, per poi generare ed eseguire il codice derivato dalla trasformazione.
\end{itemize}

\subsubsection{Classe ConcreteDSLInterpreter}

\begin{table}[H]
\begin{center}
\bgroup
\setlength{\arrayrulewidth}{0.6mm}
\def\arraystretch{1}
\begin{tabular}{ | p{12cm} | }
\hline
\centerline{\textbf{ConcreteDSLInterpreter}}
\\ \hline
\code{- macro:SweetMacro} \\
\hline
\code{+init(errback:function(MaapError), callback:function())} \\
\code{+loadDSLFile(content:String, errback:MaapError, callback:function(DslCollectionModel [0 ... *]), domain:DslDomain)} \\
\code{+DSLConcreteStrategy()} \\
\hline
\end{tabular}
\egroup
\caption{Classe ConcreteDSLInterpreter}
\end{center}
\end{table}

\paragraph*{Descrizione}
\begin{itemize}
\item[] Classe che concretizza l'interprete del \glossario{DSL}. È uno dei componenti ConcreteStrategy del \glossario{Design Pattern} \glossario{Strategy}.
\end{itemize}

\paragraph*{Utilizzo}
\begin{itemize}
\item[] Viene utilizzata per implementare l'algoritmo utilizzato nell'interfaccia \texttt{Back-end::Lib::DSLModel::DSLInterpreterStrategy} per l'interpretazione del linguaggio \glossario{DSL}. Conterrà al suo interno un metodo che genererà il \glossario{parser} a partire da una grammatica regolare.
\end{itemize}

\paragraph*{Relazioni con altre classi}
\begin{itemize}
\item[] Estende la classe:
\begin{itemize}
\item \class{Back-end::Lib::Model::DSLModel::DSLInterpreterStrategy}
\end{itemize}


\end{itemize}

\paragraph*{Attributi}
\begin{itemize}
\item[] \attribute{- macro:SweetMacro} \\ Si tratta dell'oggetto che descrive le macro di interpretazione del file DSL. Vengono utilizzate dal modulo \textit{sweet.js}.
\end{itemize}

\paragraph*{Metodi}
\begin{itemize}
\item[] \method{+init(errback:function(MaapError), callback:function())} \\ Questo metodo viene invocato per impostare l'interprete nel modo corretto. Viene settata il modulo sweet.js a partire dal codice di definizione delle macro. Nel caso in cui la classe venga settata correttamente viene invocata una callback che segnalerà la corretta impostazione, altrimenti viene sollevata una callback di errore.
\begin{itemize}\addtolength{\itemsep}{-0.5\baselineskip}
\item[$\circ$] \parameter{callback:function()} \\ Questa callback viene eseguita al termine del settaggio della classe.
\item[$\circ$] \parameter{errback:function(MaapError)} \\ Questa callback viene eseguita nel caso in cui la classe non venga settata in modo corretto a causa di un fallimento nella lettura del codice di macro. In questo caso viene restituito l'errore tramite questa callback.
\end{itemize}
\item[] \method{+loadDSLFile(content:String, errback:MaapError, callback:function(DslCollectionModel [0 ... *]), domain:DslDomain)} \\ Questo metodo si occupa di interpretare il contenuto del file DSL passatogli, per poi generare ed eseguire il codice derivato dalla trasformazione tramite le macro di \textit{sweet.js}. Se il codice viene generato ed eseguito correttamente allora avviene una chiamata alla callback di successo, altrimenti avviene una chiamata alla callback di errore.
\begin{itemize}\addtolength{\itemsep}{-0.5\baselineskip}
\item[$\circ$] \parameter{content:String} \\ Questo parametro rappresenta il contenuto testuale del file DSL da interpretare.
\item[$\circ$] \parameter{callback:function(DslCollectionModel [0 ... *])} \\ Questa callback viene chiamata alla terminazione positiva del metodo e riceve in input l'array di DslCollectionModel da restituire al chiamante.
\item[$\circ$] \parameter{errback:MaapError} \\ Questa callback viene eseguita nel caso in cui ci sia un fallimento nell'interpretazione del DSL o nell'esecuzione del codice generato a partire da esso. Prende in input un MaapError da restituire alla funzione chiamante.
\item[$\circ$] \parameter{domain:DslDomain} \\ Questo parametro rappresenta un riferimento al \texttt{DslDomain} dell'applicazione.
\end{itemize}
\item[] \method{+DSLConcreteStrategy()} \\ Questo metodo è il costruttore della classe.
\end{itemize}

\subsubsection{Classe DSLCollectionModel}

\begin{table}[H]
\begin{center}
\bgroup
\setlength{\arrayrulewidth}{0.6mm}
\def\arraystretch{1}
\begin{tabular}{ | p{12cm} | }
\hline
\centerline{\textbf{DSLCollectionModel}}
\\ \hline
\code{- showModel:ShowModel} \\
\code{- indexModel:IndexModel} \\
\code{- collectionName:String} \\
\code{- id:String} \\
\code{- weight:Integer} \\
\code{- label:String} \\
\hline
\code{+DSLCollectionModel(domain:DslDomain, params:JSON)} \\
\code{+getCollectionName():String} \\
\code{+getIndexModel():IndexModel} \\
\code{+getShowModel():ShowModel} \\
\code{+setIndexModel(indexModel:IndexModel)} \\
\code{+setShowModel(showModel:ShowModel)} \\
\code{+getId():String} \\
\code{+getLabel():String} \\
\code{+getWeight():Integer} \\
\code{+toString():String} \\
\hline
\end{tabular}
\egroup
\caption{Classe DSLCollectionModel}
\end{center}
\end{table}

\paragraph*{Descrizione}
\begin{itemize}
\item[] Classe che si occupa di definire il model della \glossario{Collection} a partire dal \glossario{DSL}. Si ispira all'\glossario{Abstract Syntax Tree}.
\end{itemize}

\paragraph*{Utilizzo}
\begin{itemize}
\item[] È l'oggetto risultante dell'interpretazione del \glossario{DSL}. Definisce una rappresentazione interna di una \glossario{Collection}.
\end{itemize}

\paragraph*{Relazioni con altre classi}
\begin{itemize}


\item[] Utilizza le classi:
\begin{itemize}
\item[$\bullet$] \class{Back-end::Lib::Model::DSLModel::ShowModel}
\item[$\bullet$] \class{Back-end::Lib::Model::DSLModel::IndexModel}
\end{itemize}
\end{itemize}

\paragraph*{Attributi}
\begin{itemize}
\item[] \attribute{- showModel:ShowModel} \\ Questo campo dati rappresenta lo ShowModel della Collection.
\item[] \attribute{- indexModel:IndexModel} \\ Questo campo dati rappresenta l'IndexModel della classe.
\item[] \attribute{- collectionName:String} \\ Questo campo dati rappresenta il nome della Collection.
\item[] \attribute{- id:String} \\ Questo campo dati rappresenta una stringa identificativa della Collection, in modo da poter far sì che più modelli puntino alla stessa Collection ma vengano distinte nella URI.
\item[] \attribute{- weight:Integer} \\ Questo parametro rappresenta l'ordine di visualizzazione della Collection all'interno della dashboard e nella barra di navigazione.
\item[] \attribute{- label:String} \\ Questo parametro rappresenta l'etichetta della Collection, ovvero il nome con il quale apparirà nel front-end.
\end{itemize}

\paragraph*{Metodi}
\begin{itemize}
\item[] \method{+DSLCollectionModel(domain:DslDomain, params:JSON)} \\ Questo metodo è il costruttore pubblico della classe.
\begin{itemize}\addtolength{\itemsep}{-0.5\baselineskip}
\item[$\circ$] \parameter{domain:DslDomain} \\ Questo parametro rappresenta il riferimento al \texttt{DslDomain} del modello.
\item[$\circ$] \parameter{params:JSON} \\ Questo parametro rappresenta l'insieme di parametri rappresentati in formato JSON che verranno utilizzati per inizializzare la classe.
\end{itemize}
\item[] \method{+getCollectionName():String} \\ Questo metodo restituisce il campo collectionName della classe.
\item[] \method{+getIndexModel():IndexModel} \\ Questo metodo restituisce il campo indexModel della classe.
\item[] \method{+getShowModel():ShowModel} \\ Questo metodo restituisce il campo showModel della classe.
\item[] \method{+setIndexModel(indexModel:IndexModel)} \\ Questo metodo imposta il campo dati showModel della classe con il parametro ricevuto in input.
\begin{itemize}\addtolength{\itemsep}{-0.5\baselineskip}
\item[$\circ$] \parameter{indexModel:IndexModel} \\ Questo parametro rappresenta l'IndexModel da settare.
\end{itemize}
\item[] \method{+setShowModel(showModel:ShowModel)} \\ Questo metodo si occupa di settare il campo showModel con il parametro ricevuto in input.
\begin{itemize}\addtolength{\itemsep}{-0.5\baselineskip}
\item[$\circ$] \parameter{showModel:ShowModel} \\ Questo parametro rappresenta lo ShowModel da settare.
\end{itemize}
\item[] \method{+getId():String} \\ Questo metodo restituisce il campo \texttt{id} della classe.
\item[] \method{+getLabel():String} \\ Questo metodo restituisce il campo \texttt{label} della classe.
\item[] \method{+getWeight():Integer} \\ Questo metodo restituisce il campo \texttt{weight} della classe.
\item[] \method{+toString():String} \\ Questo metodo richiama il metodo \texttt{getCollectionName} della classe e restituisce il campo \texttt{collectionName} della classe.
\end{itemize}

\subsubsection{Classe ShowModel}

\begin{table}[H]
\begin{center}
\bgroup
\setlength{\arrayrulewidth}{0.6mm}
\def\arraystretch{1}
\begin{tabular}{ | p{12cm} | }
\hline
\centerline{\textbf{ShowModel}}
\\ \hline
\code{- collectionModel:DSLCollectionModel} \\
\code{- populate:String} \\
\code{- attributes:Row [0...*]} \\
\hline
\code{+ShowModel(collectionModel:DslCollectionModel, params:JSON)} \\
\code{+addRow(attribute:Row)} \\
\code{+getRows():Row [0...*]} \\
\code{+getData(documentId:String, callback:function(JSON), errback:function(MaapError))} \\
\code{+getRowsForDocument(document:Document):JSON} \\
\code{+formatHeader(document:Document, attributes:Row [0...*]):JSON} \\
\code{+deleteDocument(documentId:String, callback:function(), callback:function(), errback:function(MaapError))} \\
\code{+updateDocument(documentId:String, documentUpdated:Document, callback:function(JSON), errback:function(MaapError))} \\
\hline
\end{tabular}
\egroup
\caption{Classe ShowModel}
\end{center}
\end{table}

\paragraph*{Descrizione}
\begin{itemize}
\item[] Classe che racchiude tutte le informazioni relative ad una show-page. Tali informazioni vengono dichiarate dal developer nel DSL. È composta da un numero variabile di attributi, definiti dalla classe \texttt{Back-end::Lib::DSLModel::Row}.
\end{itemize}

\paragraph*{Utilizzo}
\begin{itemize}
\item[] Questa classe viene creata dalla componente che si occupa di caricare il DSL (interpretandolo o facendone il parsing).
\end{itemize}

\paragraph*{Relazioni con altre classi}
\begin{itemize}


\item[] Utilizza le classi:
\begin{itemize}
\item[$\bullet$] \class{Back-end::Lib::Model::DSLModel::Row}
\end{itemize}
\end{itemize}

\paragraph*{Attributi}
\begin{itemize}
\item[] \attribute{- collectionModel:DSLCollectionModel} \\ Questo campo dati rappresenta il riferimento alla \texttt{DSLCollectionModel} della classe.
\item[] \attribute{- populate:String} \\ Questo campo dati rappresenta il nome dell'attributo su cui effettuare la funzione \texttt{populate} di \textit{mongoose}
\item[] \attribute{- attributes:Row [0...*]} \\ Questo campo dati rappresenta l'array di \texttt{Row} del modello.
\end{itemize}

\paragraph*{Metodi}
\begin{itemize}
\item[] \method{+ShowModel(collectionModel:DslCollectionModel, params:JSON)} \\ Questo metodo è il costruttore della classe.
\begin{itemize}\addtolength{\itemsep}{-0.5\baselineskip}
\item[$\circ$] \parameter{collectionModel:DslCollectionModel} \\ Questo parametro rappresenta il riferimento al \texttt{DslCollectionModel} da settare.
\item[$\circ$] \parameter{params:JSON} \\ Questo parametro è costituito da un oggetto JSON che contiene tutti i parametri necessari alla configurazione di una show-page.
\end{itemize}
\item[] \method{+addRow(attribute:Row)} \\ Questo metodo si occupa di aggiungere la \textit{Row} ricevuta in input nell'array \textit{attributes} della classe.
\begin{itemize}\addtolength{\itemsep}{-0.5\baselineskip}
\item[$\circ$] \parameter{attribute:Row} \\ Questo parametro rappresenta la riga da aggiungere all'array.
\end{itemize}
\item[] \method{+getRows():Row [0...*]} \\ Questo metodo si occupa di restituire l'array di righe del modello.
\item[] \method{+getData(documentId:String, callback:function(JSON), errback:function(MaapError))} \\ Questo metodo si occupa di collegarsi a \glossario{MongoDB} tramite \glossario{Mongoose} e di restituire il JSON con tutti i dati tramite una callback, in modo da poter generare correttamente la show-page. In caso di fallimento il metodo restituisce un errore tramite una callback di errore.
\begin{itemize}\addtolength{\itemsep}{-0.5\baselineskip}
\item[$\circ$] \parameter{documentId:String} \\ Questo parametro rappresenta il codice identificativo del Document sul quale effettuare l'estrazione dei dati.
\item[$\circ$] \parameter{callback:function(JSON)} \\ Questa callback viene chiamata al termine dell'estrazione dei dati e riceve in input il JSON da restituire alla funzione chiamante.
\item[$\circ$] \parameter{errback:function(MaapError)} \\ Questa callback viene chiamata nel caso in cui avvenga un errore nell'estrazione dei dati da \glossario{MongoDB}. Prende in input l'errore da restituire alla funzione chiamante.
\end{itemize}
\item[] \method{+getRowsForDocument(document:Document):JSON} \\ Questo metodo si occupa di verificare se l'array di attributi è vuoto. In tal caso inserisce nell'array tante righe quanti sono gli attributi del Document in questione.
\begin{itemize}\addtolength{\itemsep}{-0.5\baselineskip}
\item[$\circ$] \parameter{document:Document} \\ Questo parametro rappresenta il Document dal quale effettuare l'inserimento delle righe di default.
\end{itemize}
\item[] \method{+formatHeader(document:Document, attributes:Row [0...*]):JSON} \\ Questa funzione si occupa di trasformare un Document in formato JSON a partire dagli attributi, in modo da poter restituire al front-end un risultato già formattato e pronto all'uso.
\begin{itemize}\addtolength{\itemsep}{-0.5\baselineskip}
\item[$\circ$] \parameter{document:Document} \\ Questo parametro rappresenta il Document estratto da MongoDB.
\item[$\circ$] \parameter{attributes:Row [0...*]} \\ Questo attributo rappresenta l'array di righe dal quale effettuare la trasformazione.
\end{itemize}
\item[] \method{+deleteDocument(documentId:String, callback:function(), callback:function(), errback:function(MaapError))} \\ Questo metodo si occupa di eliminare correttamente il Document indicato.
\begin{itemize}\addtolength{\itemsep}{-0.5\baselineskip}
\item[$\circ$] \parameter{documentId:String} \\ Questo parametro rappresenta il codice identificativo del Document da eliminare.
\item[$\circ$] \parameter{callback:function()} \\ Questo parametro rappresenta la funzione callback da chiamare in caso di successo della funzione.
\item[$\circ$] \parameter{callback:function()} \\ Questo parametro rappresenta la funzione callback da chiamare in caso di successo della funzione.
\item[$\circ$] \parameter{errback:function(MaapError)} \\ Questo parametro rappresenta una funzione callback da chiamare nel caso in cui l'esecuzione del metodo produca un errore.
\end{itemize}
\item[] \method{+updateDocument(documentId:String, documentUpdated:Document, callback:function(JSON), errback:function(MaapError))} \\ Questo metodo si occupa di effettuare l'update del Document indicato.
\begin{itemize}\addtolength{\itemsep}{-0.5\baselineskip}
\item[$\circ$] \parameter{documentId:String} \\ Questo parametro rappresenta il codice identificativo del Document che dev'essere modificato.
\item[$\circ$] \parameter{documentUpdated:Document} \\ Questo parametro rappresenta il Document a partire dal quale avverrà l'aggiornamento.
\item[$\circ$] \parameter{callback:function(JSON)} \\ Questo parametro rappresenta la funzione callback che dovrà essere invocata nel caso in cui il Document venga aggiornato correttamente.
\item[$\circ$] \parameter{errback:function(MaapError)} \\ Questo metodo rappresenta la funzione callback da invocare nel caso in cui il metodo produca un errore durante la sua esecuzione.
\end{itemize}
\end{itemize}

\subsubsection{Classe IndexModel}

\begin{table}[H]
\begin{center}
\bgroup
\setlength{\arrayrulewidth}{0.6mm}
\def\arraystretch{1}
\begin{tabular}{ | p{12cm} | }
\hline
\centerline{\textbf{IndexModel}}
\\ \hline
\code{- collectionModel:DSLCollectionModel} \\
\code{- columns:Column [0...*]} \\
\code{- perpage:Integer} \\
\code{- sortby:String} \\
\code{- order:String} \\
\code{- conditions:JSON} \\
\hline
\code{+IndexModel()} \\
\code{+addAttribute(attribute:Attribute)} \\
\code{+getColumns():Column [0...*]} \\
\code{+getData(errback:function(MaapError), callback:function(JSON), page:Integer, sortBy:String, order:String)} \\
\code{+addColumn(column:Column)} \\
\code{+noMoreColumns()} \\
\code{+setDefaultColumnSelectable()} \\
\code{+getColumnsForDocuments():Column [0...*]} \\
\code{+formatHeader(document:Document, attributes:Column [0...*]):JSON} \\
\hline
\end{tabular}
\egroup
\caption{Classe IndexModel}
\end{center}
\end{table}

\paragraph*{Descrizione}
\begin{itemize}
\item[] Classe che racchiude tutte le informazioni relative ad una index-page. Tali informazioni vengono dichiarate dal developer nel DSL. È composta da un numero variabile di colonne, definite dalla classe \texttt{Back-end::Lib::DSLModel::Column}.
\end{itemize}

\paragraph*{Utilizzo}
\begin{itemize}
\item[] Questa classe viene creata dalla componente che si occupa di caricare il DSL (interpretandolo o facendone il parsing) e utilizzata dalla classe \texttt{DSLCollectionModel}.
\end{itemize}

\paragraph*{Relazioni con altre classi}
\begin{itemize}


\item[] Utilizza le classi:
\begin{itemize}
\item[$\bullet$] \class{Back-end::Lib::Model::DSLModel::Column}
\end{itemize}
\end{itemize}

\paragraph*{Attributi}
\begin{itemize}
\item[] \attribute{- collectionModel:DSLCollectionModel} \\ Questo campo dati rappresenta il riferimento al \texttt{DSLCollectionModel} di cui è componente.
\item[] \attribute{- columns:Column [0...*]} \\ Questo campo dati rappresenta l'array di colonne di cui è composta la classe.
\item[] \attribute{- perpage:Integer} \\ Questo campo dati descrive il numero di Document da visualizzare per ogni index-page.
\item[] \attribute{- sortby:String} \\ Questo campo dati rappresenta l'attributo del database sul quale effettuare l'ordinamento della query.
\item[] \attribute{- order:String} \\ Questo campo dati rappresenta il modo in cui viene effettuato l'ordinamento, che può essere ascendente o discendente.
\item[] \attribute{- conditions:JSON} \\ Questo campo dati rappresenta l'oggetto JSON contenente le condizioni attraverso le quali verrà effettuata la query di estrazione dei dati.
\end{itemize}

\paragraph*{Metodi}
\begin{itemize}
\item[] \method{+IndexModel()} \\ Questo metodo è il costruttore della classe. Si occupa di definire lo schema e di inzializzare l'array di attributi.
\item[] \method{+addAttribute(attribute:Attribute)} \\ Questo metodo si occupa di aggiungere l'\textit{Attribute} indicato nell'array \textit{attributes} della classe.
\begin{itemize}\addtolength{\itemsep}{-0.5\baselineskip}
\item[$\circ$] \parameter{attribute:Attribute} \\ Questo parametro rappresenta l'attributo da aggiungere all'array.
\end{itemize}
\item[] \method{+getColumns():Column [0...*]} \\ Questo metodo si occupa di restituire l'array di colonne del modello.
\item[] \method{+getData(errback:function(MaapError), callback:function(JSON), page:Integer, sortBy:String, order:String)} \\ Questo metodo si occupa di collegarsi a \glossario{MongoDB} tramite \glossario{Mongoose} e di restituire il JSON con tutti i dati tramite una callback, in modo da poter generare correttamente la index-page. In caso di fallimento il metodo restituisce un errore tramite una callback di errore.
\begin{itemize}\addtolength{\itemsep}{-0.5\baselineskip}
\item[$\circ$] \parameter{callback:function(JSON)} \\ Questa callback viene chiamata al termine dell'estrazione dei dati e riceve in input il JSON da restituire alla funzione chiamante.
\item[$\circ$] \parameter{errback:function(MaapError)} \\ Questa callback viene chiamata nel caso in cui avvenga un errore nell'estrazione dei dati da \glossario{MongoDB}. Prende in input l'errore da restituire alla funzione chiamante.
\item[$\circ$] \parameter{page:Integer} \\ Questo parametro rappresenta la pagina sulla quale effettuare la query di estrazione.
\item[$\circ$] \parameter{sortBy:String} \\ Questo parametro rappresenta l'attributo sul quale effettuare l'ordinamento di default della index-page.
\item[$\circ$] \parameter{order:String} \\ Questo parametro rappresenta il modo nel quale dev'essere fatto l'ordinamento di default, che può essere ascendente o discendente.
\end{itemize}
\item[] \method{+addColumn(column:Column)} \\ Questo metodo si occupa di aggiungere la colonna indicata all'array delle colonne.
\begin{itemize}\addtolength{\itemsep}{-0.5\baselineskip}
\item[$\circ$] \parameter{column:Column} \\ Questo parametro rappresenta un riferimento alla colonna da inserire.
\end{itemize}
\item[] \method{+noMoreColumns()} \\ Questo metodo viene chiamato al termine di tutte le aggiunte di colonne e si occupa di invocare il metodo \texttt{setDefaultColumnSelectable}.
\item[] \method{+setDefaultColumnSelectable()} \\ Questo metodo si occupa di verificare che ci sia almeno una colonna con l'attributo \textit{selectable} impostato a true. Se non ne trova imposta di default o la colonna "$\_$id", se presente, o la prima colonna disponibile.
\item[] \method{+getColumnsForDocuments():Column [0...*]} \\ Questo metodo si occupa di verificare che l'array di colonne non sia vuoto. Nel caso in cui sia vuoto si occupa di estrarre tutti gli attributi di un Document della Collection e di creare e aggiungere colonne a partire da essi.
\item[] \method{+formatHeader(document:Document, attributes:Column [0...*]):JSON} \\ Questa funzione si occupa di trasformare un Document in formato JSON a partire dagli attributi, in modo da poter restituire al front-end un risultato già formattato e pronto all'uso.
\begin{itemize}\addtolength{\itemsep}{-0.5\baselineskip}
\item[$\circ$] \parameter{document:Document} \\ Questo parametro rappresenta il Document estratto da MongoDB.
\item[$\circ$] \parameter{attributes:Column [0...*]} \\ Questo attributo rappresenta l'array di colonne dal quale effettuare la trasformazione.
\end{itemize}
\end{itemize}

\subsubsection{Classe Transformation}

\begin{table}[H]
\begin{center}
\bgroup
\setlength{\arrayrulewidth}{0.6mm}
\def\arraystretch{1}
\begin{tabular}{ | p{12cm} | }
\hline
\centerline{\textbf{Transformation}}
\\ \hline
 \\ 
\hline
\code{+transform(element:Object):Object} \\
\hline
\end{tabular}
\egroup
\caption{Classe Transformation}
\end{center}
\end{table}

\paragraph*{Descrizione}
\begin{itemize}
\item[] Classe che racchiude tutte le informazioni relative alla modalità con cui i dati prelevati dal database verranno modificati prima di essere inviati al front-end.
Tali trasformazioni vengono dichiarate dal developer nel DSL. Questa classe rappresenta una funzione da chiamare sul valore degli attributi
\end{itemize}

\paragraph*{Utilizzo}
\begin{itemize}
\item[] Questa classe viene creata dalla componente che si occupa di caricare il DSL (interpretandolo o facendone il parsing).
\end{itemize}

\paragraph*{Relazioni con altre classi}
Assenti
% TODO: deve esserci almeno una relazione con questa classe!!!

\paragraph*{Attributi}
\begin{itemize}
\item[] Assenti
\end{itemize}

\paragraph*{Metodi}
\begin{itemize}
\item[] \method{+transform(element:Object):Object} \\ Questo metodo prende in input un elemento, applica una trasformazione, e restituisce quest'ultima in output tramite il \texttt{return}.
\begin{itemize}\addtolength{\itemsep}{-0.5\baselineskip}
\item[$\circ$] \parameter{element:Object} \\ Questo parametro rappresenta l'elemento da che dovrà essere trasformato.
\end{itemize}
\end{itemize}

\subsubsection{Classe Column}

\begin{table}[H]
\begin{center}
\bgroup
\setlength{\arrayrulewidth}{0.6mm}
\def\arraystretch{1}
\begin{tabular}{ | p{12cm} | }
\hline
\centerline{\textbf{Column}}
\\ \hline
\code{- label:String} \\
\code{- name:String} \\
\code{- transformation:function} \\
\code{- selectable:Boolean} \\
\code{- sortable:Boolean} \\
\hline
\code{+Column(indexModel:IndexModel, params:JSON)} \\
\code{+getLabel():String} \\
\code{+getName():String} \\
\code{+getTransformation():function} \\
\code{+isSelectable():Boolean} \\
\code{+isSortable():Boolean} \\
\code{+toString():String} \\
\code{+setSelectable(selectable:Boolean)} \\
\hline
\end{tabular}
\egroup
\caption{Classe Column}
\end{center}
\end{table}

\paragraph*{Descrizione}
\begin{itemize}
\item[] Classe che racchiude tutte le informazioni relative ad una colonna di una index-page. Tali informazioni vengono dichiarate dal developer nel DSL.
\end{itemize}

\paragraph*{Utilizzo}
\begin{itemize}
\item[] Questa classe viene creata dalla componente che si occupa di caricare il DSL (interpretandolo o facendone il parsing).
\end{itemize}

\paragraph*{Relazioni con altre classi}
\begin{itemize}


\item[] Utilizza le classi:
\begin{itemize}
\item[$\bullet$] \class{Back-end::Lib::Model::DSLModel::Transformation}
\end{itemize}
\end{itemize}

\paragraph*{Attributi}
\begin{itemize}
\item[] \attribute{- label:String} \\ Questo campo dati rappresenta la stringa con la quale verrà visualizzata l'intestazione della colonna o della riga nella tabella finale della index-page.
\item[] \attribute{- name:String} \\ Questo campo dati rappresenta il nome dell'attributo della Collection di riferimento.
\item[] \attribute{- transformation:function} \\ Questo campo dati rappresenta una funzione di trasformazione da applicare sul valore dell'attributo.
\item[] \attribute{- selectable:Boolean} \\ Questo parametro indica se l'attributo è selezionabile, ovvero se nel front-end cliccando su di esso si viene rimandati tramite un link alla show-page del Document riferito.
\item[] \attribute{- sortable:Boolean} \\ Questo parametro indica se l'attributo è ordinabile, ovvero se i Document possono essere ordinati secondo esso.
\end{itemize}

\paragraph*{Metodi}
\begin{itemize}
\item[] \method{+Column(indexModel:IndexModel, params:JSON)} \\ Questo metodo è il costruttore della classe.
\begin{itemize}\addtolength{\itemsep}{-0.5\baselineskip}
\item[$\circ$] \parameter{indexModel:IndexModel} \\ Questo parametro rappresenta il riferimento all'\texttt{IndexModel} da utilizzare nel costruttore.
\item[$\circ$] \parameter{params:JSON} \\ Questo parametro rappresenta i parametri di settaggio della classe in formato JSON.
\end{itemize}
\item[] \method{+getLabel():String} \\ Questo metodo restituisce il campo \textit{label} della classe.
\item[] \method{+getName():String} \\ Questo metodo restituisce il campo name della classe.
\item[] \method{+getTransformation():function} \\ Questo metodo restituisce il campo transformation della classe.
\item[] \method{+isSelectable():Boolean} \\ Questo metodo restituisce il campo selectable della classe.
\item[] \method{+isSortable():Boolean} \\ Questo metodo restituisce il campo sortable della classe.
\item[] \method{+toString():String} \\ Questo metodo richiama il metodo \texttt{getName} e restituisce il campo \texttt{name} della classe.
\item[] \method{+setSelectable(selectable:Boolean)} \\ Questo metodo imposta il campo dati \texttt{selectable} della classe.
\begin{itemize}\addtolength{\itemsep}{-0.5\baselineskip}
\item[$\circ$] \parameter{selectable:Boolean} \\ Questo parametro indica se la colonna può essere ordinata oppure no.
\end{itemize}
\end{itemize}

\subsubsection{Classe DocumentSchema}

\begin{table}[H]
\begin{center}
\bgroup
\setlength{\arrayrulewidth}{0.6mm}
\def\arraystretch{1}
\begin{tabular}{ | p{12cm} | }
\hline
\centerline{\textbf{DocumentSchema}}
\\ \hline
\code{- DocumentSchema:Schema} \\
\hline
\code{+\underline{findAllPaginatedQuery}(query:JSON, perpage:Integer, page:Integer, errback:function(MaapError)):query} \\
\code{+\underline{findByIdAndPopulate}(documentId:String, populate:String, callback:function(JSON), errback:function(MaapError))} \\
\code{+\underline{safeFindById}(documentId:String, callback:function(JSON), errback:function(MaapError))} \\
\code{+\underline{safeFindByIdAndRemove}(documentId:String, callback:function(JSON), errback:function(MaapError))} \\
\code{+\underline{upsert}(data:JSON, callback:function(JSON), errback:MaapError)} \\
\hline
\end{tabular}
\egroup
\caption{Classe DocumentSchema}
\end{center}
\end{table}

\paragraph*{Descrizione}
\begin{itemize}
\item[] Questa classe astratta si occupa di definire uno schema mongoose per le Collection e fornisce alcuni metodi statici per effettuare operazioni sulla base di dati, in particolare l'estrazione, rimozione e aggiornamento di un Document.
\end{itemize}

\paragraph*{Utilizzo}
\begin{itemize}
\item[] Viene utilizzata dalle classi \texttt{IndexModel} e \texttt{ShowModel} per effettuare operazioni sulla base di dati.
\end{itemize}

\paragraph*{Relazioni con altre classi}
Assenti
% TODO: deve esserci almeno una relazione con questa classe!!!

\paragraph*{Attributi}
\begin{itemize}
\item[] \attribute{- DocumentSchema:Schema} \\ Questo campo dati rappresenta lo schema mongoose per le Collection.
\end{itemize}

\paragraph*{Metodi}
\begin{itemize}
\item[] \method{+\underline{findAllPaginatedQuery}(query:JSON, perpage:Integer, page:Integer, errback:function(MaapError)):query} \\ Questo metodo si occupa di formare una funzione mongoose di estrazione paginata dei Document per poi restituire il riferimento alla query che verrà eseguita dal chiamante. 
\begin{itemize}\addtolength{\itemsep}{-0.5\baselineskip}
\item[$\circ$] \parameter{query:JSON} \\ Questo parametro rappresenta un oggetto JSON dal quale effettuare la query sul database.
\item[$\circ$] \parameter{perpage:Integer} \\ Questo parametro rappresenta il numero di Document da estrarre.
\item[$\circ$] \parameter{page:Integer} \\ Questo parametro rappresenta l'indice della pagina a partire dalla quale effettuare la query. Verrà fissato un offset uguale al valore di questo parametro moltiplicato per il parametro (\texttt{page} - 1).
\item[$\circ$] \parameter{errback:function(MaapError)} \\ Questo parametro rappresenta la funzione di callback che verrà invocata in caso di errore dell'estrazione.
\end{itemize}
\item[] \method{+\underline{findByIdAndPopulate}(documentId:String, populate:String, callback:function(JSON), errback:function(MaapError))} \\ Questo metodo si occupa di ricercare un Document tramite il suo id ed applicare la funzione \textit{populate} di mongoose. Il Document viene restituito tramite una callback.
\begin{itemize}\addtolength{\itemsep}{-0.5\baselineskip}
\item[$\circ$] \parameter{documentId:String} \\ Questo parametro rappresenta l'id del Document sul database.
\item[$\circ$] \parameter{populate:String} \\ Questo parametro rappresenta l'attributo sul quale effettuare la funzione \textit{populate} di mongoose.
\item[$\circ$] \parameter{callback:function(JSON)} \\ Questo parametro rappresenta una callback che viene invocata per restituire il risultato della query.
\item[$\circ$] \parameter{errback:function(MaapError)} \\ Questo parametro rappresenta una callback che viene invocata nel caso in cui la query fallisca.
\end{itemize}
\item[] \method{+\underline{safeFindById}(documentId:String, callback:function(JSON), errback:function(MaapError))} \\ Questo metodo si occupa di effettuare una query di ricerca di un Document tramite il suo id.
\begin{itemize}\addtolength{\itemsep}{-0.5\baselineskip}
\item[$\circ$] \parameter{documentId:String} \\ Questo parametro rappresenta l'id del Document da ricercare.
\item[$\circ$] \parameter{callback:function(JSON)} \\ Questo parametro rappresenta una callback da invocare nel caso in cui la ricerca abbia successo e restituisce il risultato.
\item[$\circ$] \parameter{errback:function(MaapError)} \\ Questo parametro rappresenta la funzione callback da invocare nel caso in cui la query fallisca.
\end{itemize}
\item[] \method{+\underline{safeFindByIdAndRemove}(documentId:String, callback:function(JSON), errback:function(MaapError))} \\ Questo metodo si occupa di effettuare una ricerca sul Document indicato ed eliminarlo dal database.
\begin{itemize}\addtolength{\itemsep}{-0.5\baselineskip}
\item[$\circ$] \parameter{documentId:String} \\ Questo parametro rappresenta l'id del Document nel database.
\item[$\circ$] \parameter{callback:function(JSON)} \\ Questo parametro rappresenta la funzione callback da chiamare nel caso in cui il Document sia stato eliminato correttamente.
\item[$\circ$] \parameter{errback:function(MaapError)} \\ Questo parametro rappresenta la funzione callback da invocare nel caso in cui la query generi un errore.
\end{itemize}
\item[] \method{+\underline{upsert}(data:JSON, callback:function(JSON), errback:MaapError)} \\ Questo metodo si occupa di effettuare l'update del Document indicato.
\begin{itemize}\addtolength{\itemsep}{-0.5\baselineskip}
\item[$\circ$] \parameter{data:JSON} \\ Questo parametro rappresenta i dati dai quali effettuare l'update.
\item[$\circ$] \parameter{callback:function(JSON)} \\ Questo parametro rappresenta una callback da invocare nel caso in cui il metodo effettui l'update con successo.
\item[$\circ$] \parameter{errback:MaapError} \\ Questo metodo rappresenta una callback da invocare nel caso in cui il metodo generi un errore nell'effettuare l'update.
\end{itemize}
\end{itemize}

\subsubsection{Classe Row}

\begin{table}[H]
\begin{center}
\bgroup
\setlength{\arrayrulewidth}{0.6mm}
\def\arraystretch{1}
\begin{tabular}{ | p{12cm} | }
\hline
\centerline{\textbf{Row}}
\\ \hline
\code{- showModel:ShowModel} \\
\code{- name:String} \\
\code{- label:String} \\
\code{- transformation:function} \\
\hline
\code{+getLabel():String} \\
\code{+getName():String} \\
\code{+getTransformation():function} \\
\code{+toString():String} \\
\code{+Row(showModel:ShowModel, params:JSON)} \\
\hline
\end{tabular}
\egroup
\caption{Classe Row}
\end{center}
\end{table}

\paragraph*{Descrizione}
\begin{itemize}
\item[] Classe che racchiude tutte le informazioni relative ad una riga di una show-page. Tali informazioni vengono dichiarate dal developer nel DSL.
\end{itemize}

\paragraph*{Utilizzo}
\begin{itemize}
\item[] Questa classe viene creata dalla componente che si occupa di caricare il DSL (interpretandolo o facendone il parsing).
\end{itemize}

\paragraph*{Relazioni con altre classi}
Assenti
% TODO: deve esserci almeno una relazione con questa classe!!!

\paragraph*{Attributi}
\begin{itemize}
\item[] \attribute{- showModel:ShowModel} \\ Questo parametro rappresenta il riferimento allo \texttt{ShowModel} della classe.
\item[] \attribute{- name:String} \\ Questo campo dati rappresenta il nome dell'attributo della Collection di riferimento.
\item[] \attribute{- label:String} \\ Questo campo dati rappresenta la stringa con la quale verrà visualizzata l'intestazione della colonna o della riga nella tabella finale della show-page.
\item[] \attribute{- transformation:function} \\ Questo campo dati rappresenta una funzione di trasformazione da applicare sul valore dell'attributo.
\end{itemize}

\paragraph*{Metodi}
\begin{itemize}
\item[] \method{+getLabel():String} \\ Questo metodo restituisce il campo \textit{label} della classe.
\item[] \method{+getName():String} \\ Questo metodo restituisce il campo \texttt{name} della classe.
\item[] \method{+getTransformation():function} \\ Questo metodo restituisce il campo \texttt{transformation} della classe.
\item[] \method{+toString():String} \\ Questo metodo richiama il metodo \texttt{getName} e restituisce il campo \texttt{name} della classe.
\item[] \method{+Row(showModel:ShowModel, params:JSON)} \\ Questo è il metodo costruttore della classe.
\begin{itemize}\addtolength{\itemsep}{-0.5\baselineskip}
\item[$\circ$] \parameter{showModel:ShowModel} \\ Questo parametro rappresenta il riferimento allo \texttt{ShowModel} da settare.
\item[$\circ$] \parameter{params:JSON} \\ Questo parametro contiene tutti i parametri della classe.
\end{itemize}
\end{itemize}

\subsection{Componente Back-end::Lib::Utils}

\subsubsection{Classe Mailer}

\begin{table}[H]
\begin{center}
\bgroup
\setlength{\arrayrulewidth}{0.6mm}
\def\arraystretch{1}
\begin{tabular}{ | p{12cm} | }
\hline
\centerline{\textbf{Mailer}}
\\ \hline
 \\ 
\hline
\code{+Mailer(app:ServerApp)} \\
\code{+sendEmail(message:Object, callback:function(responseStatus), errback:function(MaapError))} \\
\hline
\end{tabular}
\egroup
\caption{Classe Mailer}
\end{center}
\end{table}

\paragraph*{Descrizione}
\begin{itemize}
\item[] Classe che si occupa dell'invio di email. È uno dei componenti subsystem class del \glossario{Design Pattern} \glossario{Facade} e handler del \glossario{Design Pattern} \glossario{Chain of responsibility}.
\end{itemize}

\paragraph*{Utilizzo}
\begin{itemize}
\item[] Viene utilizzata per inviare un'email ad un utente che ha effettuato la richiesta di recupero password.
\end{itemize}

\paragraph*{Relazioni con altre classi}
Assenti
% TODO: deve esserci almeno una relazione con questa classe!!!

\paragraph*{Attributi}
\begin{itemize}
\item[] Assenti
\end{itemize}

\paragraph*{Metodi}
\begin{itemize}
\item[] \method{+Mailer(app:ServerApp)} \\ Costruttore che crea il servizio email e rende disponibile l'invio di email tramite \code{sendEmail()}.
\begin{itemize}\addtolength{\itemsep}{-0.5\baselineskip}
\item[$\circ$] \parameter{app:ServerApp} \\ È l'istanza del server che dovrà utilizzare questa classe
\end{itemize}
\item[] \method{+sendEmail(message:Object, callback:function(responseStatus), errback:function(MaapError))} \\ Metodo che si occupa di inviare un'email.
\begin{itemize}\addtolength{\itemsep}{-0.5\baselineskip}
\item[$\circ$] \parameter{message:Object} \\ Questo oggetto rappresenta il template dell'email da inviare.
\item[$\circ$] \parameter{callback:function(responseStatus)} \\ Questo parametro rappresenta la callback che il metodo dovrà chiamare al termine dell'elaborazione senza errori, dove l'oggetto di tipo responseStatus (tipo appartenente alla libreria \code{NodeMailer}) contiene informazioni sullo stato di successo.
\item[$\circ$] \parameter{errback:function(MaapError)} \\ Questo parametro rappresenta la callback che il metodo dovrà chiamare al verificarsi di un errore.
\end{itemize}
\end{itemize}

\subsubsection{Classe MaapError}

\begin{table}[H]
\begin{center}
\bgroup
\setlength{\arrayrulewidth}{0.6mm}
\def\arraystretch{1}
\begin{tabular}{ | p{12cm} | }
\hline
\centerline{\textbf{MaapError}}
\\ \hline
\code{- title:String} \\
\code{- code:Integer} \\
\code{- message:String} \\
\hline
\code{+toDict():JSON} \\
\code{+toString():String} \\
\code{+toError():Error} \\
\code{+MaapError(title:String, code:Integer, message:String)} \\
\hline
\end{tabular}
\egroup
\caption{Classe MaapError}
\end{center}
\end{table}

\paragraph*{Descrizione}
\begin{itemize}
\item[] Classe che rappresenta un errore all'interno del package \texttt{Back-end::Lib}.
\end{itemize}

\paragraph*{Utilizzo}
\begin{itemize}
\item[] Viene utilizzata da tutte le classi presente all'interno del package \texttt{Back-end::Lib} per rappresentare un errore generato, identificandolo tramite nome, descrizione e codice.
\end{itemize}

\paragraph*{Relazioni con altre classi}
Assenti
% TODO: deve esserci almeno una relazione con questa classe!!!

\paragraph*{Attributi}
\begin{itemize}
\item[] \attribute{- title:String} \\ Questo campo dati rappresenta il titolo dell'errore generato in formato stinga.
\item[] \attribute{- code:Integer} \\ Campo dato che reppresenta il codice dell'errore.
\item[] \attribute{- message:String} \\ Campo dati che rappresenta il messaggio corrispondente all'errore.
\end{itemize}

\paragraph*{Metodi}
\begin{itemize}
\item[] \method{+toDict():JSON} \\ Metodo che ritorna l'errore in formato json.
\item[] \method{+toString():String} \\ Metodo che effettua una concatenazione dei campi dati dell'errore in formato \code{String} e la ritorna.
\item[] \method{+toError():Error} \\ Questo metodo converte l'errore dal tipo \code{MaapError} al tipo \code{Error} utilizzato da \glossario{Node.js} ritornandolo.
\item[] \method{+MaapError(title:String, code:Integer, message:String)} \\ È il metodo costruttore della classe.
\begin{itemize}\addtolength{\itemsep}{-0.5\baselineskip}
\item[$\circ$] \parameter{title:String} \\ Questo parametro rappresenta il titolo del messaggio d'errore.
\item[$\circ$] \parameter{code:Integer} \\ Questo parametro rappresenta il codice dell'errore.
\item[$\circ$] \parameter{message:String} \\ Questo parametro rappresenta il messaggio dell'errore.
\end{itemize}
\end{itemize}

\subsubsection{Classe AttributeReader}

\begin{table}[H]
\begin{center}
\bgroup
\setlength{\arrayrulewidth}{0.6mm}
\def\arraystretch{1}
\begin{tabular}{ | p{12cm} | }
\hline
\centerline{\textbf{AttributeReader}}
\\ \hline
 \\ 
\hline
\code{+\underline{readRequiredAttributes}(source:JSON, errback:function(MaapError), required:Array, target:Object)} \\
\code{+\underline{readOptionalAttributes}(source:JSON, target:Object, optional:Array)} \\
\code{+\underline{assertEmptyAttributes}(source:JSON, errback:function(MaapError))} \\
\hline
\end{tabular}
\egroup
\caption{Classe AttributeReader}
\end{center}
\end{table}

\paragraph*{Descrizione}
\begin{itemize}
\item[] Questa classe astratta si occupa di effettuare delle verifiche sugli attributi che vengono passati dalla lettura del file DSL.
\end{itemize}

\paragraph*{Utilizzo}
\begin{itemize}
\item[] Viene utilizzata dalle classi \texttt{DslCollectionModel}, \texttt{IndexModel}, \texttt{ShowModel}, \texttt{Row}, \texttt{Column} per effettuare controlli sui parametri.
\end{itemize}

\paragraph*{Relazioni con altre classi}
Assenti
% TODO: deve esserci almeno una relazione con questa classe!!!

\paragraph*{Attributi}
\begin{itemize}
\item[] Assenti
\end{itemize}

\paragraph*{Metodi}
\begin{itemize}
\item[] \method{+\underline{readRequiredAttributes}(source:JSON, errback:function(MaapError), required:Array, target:Object)} \\ Questo metodo si occupa di verificare che gli attributi richiesti non siano indefiniti e aggiungerli all'oggetto \texttt{target}. Nel caso in cui gli attributi richiesti siano indefiniti genera un orrore che viene restituito con una callback.
\begin{itemize}\addtolength{\itemsep}{-0.5\baselineskip}
\item[$\circ$] \parameter{source:JSON} \\ Questo parametro rappresenta il parametri di configurazione.
\item[$\circ$] \parameter{target:Object} \\ Questo parametro rappresenta l'oggetto sul quale viene effettuata l'aggiunta dei parametri.
\item[$\circ$] \parameter{errback:function(MaapError)} \\ Questo parametro rappresenta la funzione callback da invocare nel caso in cui il metodo produca un errore.
\item[$\circ$] \parameter{required:Array} \\ Questo parametro rappresenta la lista di parametri richiesti.
\end{itemize}
\item[] \method{+\underline{readOptionalAttributes}(source:JSON, target:Object, optional:Array)} \\ Questo metodo si occupa di verificare che gli attributi opzionali non siano indefiniti e aggiungerli all'oggetto \texttt{target}.
\begin{itemize}\addtolength{\itemsep}{-0.5\baselineskip}
\item[$\circ$] \parameter{source:JSON} \\ Questo parametro rappresenta il parametri di configurazione.
\item[$\circ$] \parameter{target:Object} \\ Questo parametro rappresenta l'oggetto sul quale viene effettuata l'aggiunta dei parametri.
\item[$\circ$] \parameter{optional:Array} \\ Questo parametro rappresenta la lista di parametri opzionali.
\end{itemize}
\item[] \method{+\underline{assertEmptyAttributes}(source:JSON, errback:function(MaapError))} \\ Questo metodo verifica che non vi siano parametri non richiesti.
\begin{itemize}\addtolength{\itemsep}{-0.5\baselineskip}
\item[$\circ$] \parameter{source:JSON} \\ Questo parametro rappresenta la lista di parametri richiesti.
\item[$\circ$] \parameter{errback:function(MaapError)} \\ Questo parametro rappresenta la funzione callback da invocare nel caso in cui il metodo produca un errore.
\end{itemize}
\end{itemize}

\subsection{Componente Back-end::DeveloperProject}

\subsubsection{Classe ProjectApp}

\begin{table}[H]
\begin{center}
\bgroup
\setlength{\arrayrulewidth}{0.6mm}
\def\arraystretch{1}
\begin{tabular}{ | p{12cm} | }
\hline
\centerline{\textbf{ProjectApp}}
\\ \hline
 \\ 
\hline
\code{+\underline{start}()} \\
\hline
\end{tabular}
\egroup
\caption{Classe ProjectApp}
\end{center}
\end{table}

\paragraph*{Descrizione}
\begin{itemize}
\item[] Classe modificabile dall'utente-developer che si occupa di configurare e avviare il server dell'applicazione.
\end{itemize}

\paragraph*{Utilizzo}
\begin{itemize}
\item[] Internamente avvia il server utilizzando la classe ServerApp, a cui passa i parametri di configurazione del progetto definiti con un oggetto della classe ProjectConfig.
\end{itemize}

\paragraph*{Relazioni con altre classi}
\begin{itemize}


\item[] Utilizza le classi:
\begin{itemize}
\item[$\bullet$] \class{Back-end::Lib::ServerApp}
\item[$\bullet$] \class{Back-end::DeveloperProject::Config::ProjectConfig}
\end{itemize}
\end{itemize}

\paragraph*{Attributi}
\begin{itemize}
\item[] Assenti
\end{itemize}

\paragraph*{Metodi}
\begin{itemize}
\item[] \method{+\underline{start}()} \\ Questo metodo statico carica l'oggetto di configurazione di tipo ProjectConfig e fa partire il server dell'applicazione, utilizzando la classe ServerApp del package Lib.
\end{itemize}

\subsubsection{Classe ProjectConfig}

\begin{table}[H]
\begin{center}
\bgroup
\setlength{\arrayrulewidth}{0.6mm}
\def\arraystretch{1}
\begin{tabular}{ | p{12cm} | }
\hline
\centerline{\textbf{ProjectConfig}}
\\ \hline
 \\ 
\hline
\code{+getServerPort():Integer} \\
\code{+getServerStaticPath():String} \\
\code{+getUserDbUri():String} \\
\code{+getEnvironment():String} \\
\code{+getDataDbUri():String} \\
\code{+getSmtpService():String} \\
\code{+getSmtpAuth():String} \\
\code{+getDSLPath():String} \\
\hline
\end{tabular}
\egroup
\caption{Classe ProjectConfig}
\end{center}
\end{table}

\paragraph*{Descrizione}
\begin{itemize}
\item[] Questa classe rappresenta la configurazione di un'applicazione.
\end{itemize}

\paragraph*{Utilizzo}
\begin{itemize}
\item[] Viene passato come parametro al costruttore della classe ServerApp per configurare l'applicazione.
\end{itemize}

\paragraph*{Relazioni con altre classi}
\begin{itemize}
\item[] Estende la classe:
\begin{itemize}
\item \class{Back-end::Lib::Config}
\end{itemize}


\end{itemize}

\paragraph*{Attributi}
\begin{itemize}
\item[] Assenti
\end{itemize}

\paragraph*{Metodi}
\begin{itemize}
\item[] \method{+getServerPort():Integer} \\ Restituisce la porta su cui il server deve mettersi in ascolto. È un overriding del metodo della classe ereditata.
\item[] \method{+getServerStaticPath():String} \\ Restituisce il percorso della cartella che il server deve utilizzare per fornire file statici. È un overriding del metodo della classe ereditata.

\item[] \method{+getUserDbUri():String} \\ Restituisce l'uri del database che il server deve utilizzare come database degli utenti. È un overriding del metodo della classe ereditata.

\item[] \method{+getEnvironment():String} \\ Restituisce la variabile d'ambiente che informa se l'applicazione deve essere eseguita in modalità ``developing'' o ``production''. È un overriding del metodo della classe ereditata.

\item[] \method{+getDataDbUri():String} \\ Restituisce l'uri del database che il server deve utilizzare come database di analisi, cioè quello contenente le collection di cui l'applicazione deve permettere la visualizzazione. È un overriding del metodo della classe ereditata.

\item[] \method{+getSmtpService():String} \\ Restituisce il nome del servizio che potrà essere usato dall'applicazione per inviare email. È un overriding del metodo della classe ereditata.

\item[] \method{+getSmtpAuth():String} \\ Restituisce le credenziali con cui è possibile utilizzare il servizio smtp per inviare email. È un overriding del metodo della classe ereditata.

\item[] \method{+getDSLPath():String} \\ Restituisce la path da cui caricare i file DSL definiti dallo sviluppatore. È un overriding del metodo della classe ereditata.

\end{itemize}