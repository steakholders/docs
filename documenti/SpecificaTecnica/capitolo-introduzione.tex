\section{Introduzione}

\subsection{Scopo del documento}

Questo documento ha come scopo quello di definire la progettazione ad alto livello per il prodotto.

Verranno presentati l'architettura generale secondo la quale saranno organizzate le varie componenti software e i \glossario{design pattern} utilizzati nella creazione del prodotto.
Verrà dettagliato il tracciamento tra le componenti software individuate ed i requisiti.
Qualora vengano apportate modifiche o aggiunte al presente documento sarà necessario informare tempestivamente ogni membro del gruppo.

\subsection{Scopo del prodotto}

\ScopoDelProdotto{}

\subsection{Glossario}

Al fine di evitare ogni ambiguità relativa al linguaggio impiegato nei documenti viene fornito il \Glossario{}, contenente la definizione dei termini marcati con una G pedice.

\subsection{Riferimenti}
	\label{Riferimenti}
	
		\subsubsection{Normativi}
		
		\begin{itemize}
		\item \NormeDiProgetto{}
		\item Capitolato d'appalto C1: MaaP: MongoDB as an admin Platform:\\
			\url{http://www.math.unipd.it/~tullio/IS-1/2013/Progetto/C1.pdf}
		\item \AnalisiDeiRequisiti{}  \\	
        \end{itemize}
        
		\subsubsection{Informativi}
		
		\begin{itemize}
		\item Presentazione capitolato d'appalto: \url{http://www.math.unipd.it/~tullio/IS-1/2013/Progetto/C1.pdf};
		\item Ingegneria del software - Ian Sommerville - 8 a edizione (2007);
		\item Dall’idea al codice con UML 2 - L. Baresi, L. Lavazza, M. Pianciamore - 1a edizione
(2006);
		\item Design Patterns - Erich Gamma, Richard Helm, Ralph Johnson, John Vlissides - 1a edizione italiana (2008);
		\item Node.js - Marc Wandschneider - 1a edizione (2013).
		
		\end{itemize}
		
	\pagebreak