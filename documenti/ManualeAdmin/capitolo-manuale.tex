\section{Index e login}

Accedendo a \ProjectName{} la prima pagina a cui si è indirizzati è la index page.
Si tratta dell'unica pagina visibile senza aver effettuato l'accesso.

Per effettuare l'accesso è necessario cliccare sul pulsante login in alto a destra.

Come da immagine \ref{loginfig} per potersi autenticare è necessario disporre di un'email e di una password valide.
Per l'amministratore è possibile aggiungere nuovi utenti. Tale procedura sarà illustrata nella sezione \ref{nuovoutente}.

Nel caso in cui i dati inseriti non sono riconosciuti dal sistema verrà restituito un messaggio di errore e l'utente potrà tentare nuovamente la login. Non ci sono limiti sul numero di tentativi possibili.

Se l'autenticazione ha successo si verrà reindirizzati alla pagina Dashboard, illustrata nella sezione \ref{dashboard}
\begin{figure}[h]
	\label{loginfig}
	\centering \includegraphics[width=1\textwidth]{login.png}
	\caption{Screen della pagina login.}
\end{figure}

\section{Dashboard}
 \label{dashboard}
 
Si tratta della pagina centrale una volta eseguito il login. Come si vede in figura \ref{dashboardfig} in questa pagina sarà presente un elenco delle collection a cui si ha accesso. Per visualizzare una collection in particolare è sufficiente cliccarci sopra.

\begin{figure}[h]
	\label{dashboardfig}
	\centering \includegraphics[width=1\textwidth]{dashboard.png}
	\caption{della pagina dashboard. }
\end{figure}

\section{Gestione delle collection}
\subsection{Visualizzare le collection}
\subsection{Gestione dei documenti di una collection}

\section{Gestione utenti}
\label{nuovoutente}
Cliccando sul pulsante Users in alto a sinistra si accede alla pagina UserList che, come da figura \ref{usersfig}, visualizza la lista degli utenti presenti, tramite le loro email.

Tramite il tasto remove (1) è possibile rimuovere un utente dal sistema, cliccando apparirà un pop-up su cui confermare la rimozione o annullarla. Tramite il tasto 2 %TODO

Utilizzando il form a lato (3) è possibile registrare un nuovo utente definendo email, password e livello dei privilegi.

Se i dati inseriti sono validi l'utente apparirà sulla destra, altrimenti verrà visualizzato l'errore come in figura \ref{userserrorfig}

\begin{figure}[h]
\label{userslist}
	\centering \includegraphics[width=1\textwidth]{userList.png}
	\caption{Screen della pagina userList.}
\end{figure}

\begin{figure}[h]
\label{userserrorfig}
	\centering \includegraphics[width=1\textwidth]{userListError.png}
	\caption{Screen della pagina userList con errore. }
\end{figure}

\section{Gestione del profilo personale}





