\section{Diagrammi di attività}

Vengono in seguito illustrati i diagrammi di attività prodotti durante la progettazione architetturale, i quali descrivono le iterazioni con l'utente al sistema \glossario{MaaP}. Inizialmente verrà fornito uno schema ad alto livello, per poi andare sempre più nel dettaglio tramite sotto-diagrammi più specifici. Per comodità di visualizzazione le attività che verranno \textit{esplose} sono marcate in grassetto. 

Al fine di rendere il diagramma leggibile abbiamo considerato implicito il fatto che un utente possa in qualsiasi momento uscire dall'applicazione \glossario{MaaP}, per esempio chiudendo la finestra del browser.

\subsection{MaaP - Attività principali}

\begin{figure}[H]
\centering
\includegraphics[scale=0.12]{uml/MaaP - Attivita Principali.png}
\caption{Diagramma di attività - Attività principali di un'applicazione MaaP}
\end{figure}

Sostanzialmente un'applicazione generata da \glossario{MaaP} è composta da una serie di pagine web all'interno delle quali un utente può navigare. Un utente accede inizialmente all'applicazione web in una pagina statica in cui può effettuare tre cose:

\begin{itemize}

	\item Registrarsi al sistema;
	\item Effettuare il login;
	\item Recuperare la propria password.

\end{itemize}

Una volta che l'utente ha effettuato il login viene direttamente indirizzato alla \glossario{Dashboard}, dalla quale può navigare all'interno dell'applicazione ed effettuare diverse operazioni:

\begin{itemize}

	\item Effettuare il logout;
	\item Visualizzare il proprio profilo e di conseguenza modificarlo;
	\item Selezionare una \glossario{Collection} esistente.

\end{itemize}

Nel caso in cui l'utente avesse i privilegi di admin può inoltre accedere ad una specifica pagina di gestione degli utenti iscritti.

\subsection{MaaP - Effettua registrazione}

\begin{figure}[H]
\centering
\includegraphics[scale=0.2]{uml/MaaP - Effettua registrazione.png}
\caption{Diagramma di attività - Registrazione di un utente}
\end{figure}

L'utente si trova all'interno della pagina di registrazione e sostanzialmente deve inserire la propria email e la propria password all'interno di due campi di testo. Una volta inseriti l'utente deve premere il pulsante di invio dati; il sistema \glossario{MaaP} procederà dunque alla verifica delle credenziali e, se quest'ultima avrà successo, alla registrazione dell'utente.

\subsection{MaaP - Recupera password}

\begin{figure}[H]
\centering
\includegraphics[scale=0.2]{uml/MaaP - Recupera password.png}
\caption{Diagramma di attività - Recupero password}
\end{figure}

L'utente si trova all'interno della pagina di recupero password, la quale presenta un campo di testo nel quale l'utente dovrà inserire il proprio indirizzo email. Una volta inserito preme il pulsante di richiesta di una nuova password; il sistema \glossario{MaaP} procederà dunque alla verifica dell'indirizzo email e, se quest'ultima avrà esito positivo, invierà un'email all'utente con le relative istruzioni per il ripristino della password.

\subsection{MaaP - Esegui reset password}

\begin{figure}[H]
\centering
\includegraphics[scale=0.2]{uml/MaaP - Esegui reset password.png}
\caption{Diagramma di attività - Reset della password dell'utente}
\end{figure}

L'utente avrà ricevuto un'email con al suo interno un link ad una pagina univoca dell'applicazione \glossario{MaaP} e quindi si troverà in una pagina con al suo interno un campo di testo nel quale inserire la nuova password. Una volta inserita la password deve premere il pulsante di reset; il sistema \glossario{MaaP} procederà dunque al cambio password per l'utente corrente nel \glossario{database} delle credenziali.

\subsection{MaaP - Effettua login}

\begin{figure}[H]
\centering
\includegraphics[scale=0.2]{uml/MaaP - Effettua login.png}
\caption{Diagramma di attività - Login dell'utente}
\end{figure}

L'utente, che precedentemente avrà effettuato la registrazione al sistema, accede all'interno dell'applicazione tramite una pagina di login. Al suo interno saranno presenti due campi di testo in cui l'utente dovrà inserire la propria email e la propria password. Una volta inserite dovrà premere il pulsante di login; il sistema \glossario{MaaP} procederà dunque alla verifica delle credenziali e, se l'esito di tale verifica risulterà positivo, effettuerà il login dell'utente all'applicazione, reindirizzandolo alla \glossario{dashboard}.

\subsection{MaaP - Modifica profilo}

\begin{figure}[H]
\centering
\includegraphics[scale=0.2]{uml/MaaP - Modifica profilo.png}
\caption{Diagramma di attività - Modifica profilo utente}
\end{figure}

L'utente autenticato accede all'interno della propria pagina profilo, dalla quale può decidere di modificare la propria password. Sarà dunque presente un campo di testo in cui l'utente inserirà la nuova password e un bottone tramite il quale invierà la richiesta di modifica; il sistema \glossario{MaaP} procederà dunque alla modifica della password dell'utente.

\subsection{MaaP - Index-page Collection}

\begin{figure}[H]
\centering
\includegraphics[scale=0.2]{uml/MaaP - index-page.png}
\caption{Diagramma di attività - Visualizzazione index-page della Collection selezionata}
\end{figure}

L'utente ha selezionato una \glossario{Collection} dal menu e ora si trova all'interno di una pagina che visualizza una tabella contenente tutti i \glossario{Document} della \glossario{Collection} con alcuni attributi visualizzabili. A questo punto è in grado di fare diverse operazioni:

\begin{itemize}

	\item Può aprire la relativa \glossario{show-page} di un \glossario{Document} selezionando il link che la apre;
	\item Può applicare un filtro ai \glossario{Document} visualizzati in modo da visualizzare un sottoinsieme della tabella;
	\item Se la tabella risulta distribuita su più pagine può accedere alle pagine successive;

\end{itemize}

Se l'utente dispone dei privilegi di admin può inoltre:

\begin{itemize}

	\item Modificare un \glossario{Document} cliccando sul link \textit{edit} visualizzato in ciascuna riga della tabella;
	\item Eliminare un \glossario{Document} cliccando sul link \textit{delete} visualizzato in ciascuna riga della tabella;

\end{itemize}

