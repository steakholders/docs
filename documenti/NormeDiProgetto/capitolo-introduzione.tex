\section{Introduzione}

\subsection{Scopo del documento}

Questo documento ha come obiettivo quello di definire le regole, gli strumenti e le procedure che tutti i membri del team dovranno adottare per l'intero svolgimento del progetto. Tutti i componenti del gruppo sono obbligati a visionare tale documento e ad applicare quanto scritto, al fine di mantenere omogeneità e coesione in ogni aspetto del progetto.

Qualora vengano apportate modifiche o aggiunte al presente documento sarà necessario informare tempestivamente ogni membro del gruppo.

\subsection{Ambiguità}

Al fine di evitare ogni ambiguità relativa al linguaggio impiegato nei documenti viene fornito il \Glossario{}, contenente la definizione dei termini marcati con una G pedice.

In questo documento alcuni termini devono essere interpretati in modo analogo ai termini inglesi descritti in RFC 2119\footnote{\url{http://www.ietf.org/rfc/rfc2119.txt}}:
\begin{itemize}
 \item I termini \textbf{``deve'', ``è richiesto''} e sinonimi stretti sono da intendersi con lo stesso significato di ``MUST'';
 \item I termini \textbf{``non deve'', ``è richiesto che non''} e sinonimi stretti sono da intendersi con lo stesso significato di ``MUST NOT'';
 \item I termini \textbf{``dovrebbe'', ``si raccomanda'', ``è preferibile''} e sinonimi stretti sono da intendersi con lo stesso significato di ``SHOULD'';
 \item I termini \textbf{``non dovrebbe'', ``si raccomanda di non'', ``è preferibile che non''} e sinonimi stretti sono da intendersi con lo stesso significato di ``SHOULD NOT'';
 \item I termini \textbf{``può'', ``opzionalmente''} e sinonimi stretti sono da intendersi con lo stesso significato di ``MAY''.
\end{itemize}

\subsection{Riferimenti normativi}
\begin{itemize}
	\item \textbf{ISO/IEC 12207-1995} \\ \url{http://www.math.unipd.it/~tullio/IS-1/2009/Approfondimenti/ISO_12207-1995.pdf}; \\
	\item \textbf{IEEE Std 1028} \\ \url{http://en.wikipedia.org/wiki/Software_review#cite_ref-MFaganPapers_3-0}; \\
	\item \textbf{ISO/IEC 12207-2008} al paragrafo \emph{5.1.9} per la descrizione di un processo.
\end{itemize}




