\section{Glossario}

\textbf{API}: Application programming interface, sono un insieme di procedure che un sistema software rende accessibile a terzi per interfacciarsi ad esso.

\textbf{Business}: 

\textbf{Collection}: %TODO

\textbf{Dashboard}: Pagina web che rappresenta lo stato corrente di un'applicazione con un interfaccia semplice ed immediata.

\textbf{Document}: %TODO

\textbf{Framework}: %TODO

\textbf{MaaP}: %TODO

\textbf{MongoDB}: Sistema gestionale di basi di dati non relazionale, orientato ai documenti, di tipo \glossario{NoSQL}. Il linguaggio utilizzato per la gestione dei dati è JavaScript, del quale sfrutta in particolare la notazione BSON.

\textbf{Node.js}: %TODO

\textbf{NoSQL}: È un movimento che promuove sistemi software dove la persistenza dei dati è caratterizzata dal fatto di non utilizzare il modello relazionale, tipicamente usato dai database tradizionali. L'espressione NoSQL fa riferimento al linguaggio SQL, che è il più comune linguaggio di interrogazione dei dati nei database relazionali, qui preso a simbolo dell'intero paradigma relazionale. Questi archivi di dati tipicamente non richiedono uno schema fisso, evitano spesso le operazioni di unione e puntano a scalare orizzontalmente.

\textbf{Parsing}: Traducibile con analisi sintattica, è il processo atto ad analizzare uno stream continuo in input in modo da determinare la sua struttura grammaticale grazie ad una data grammatica formale.

\textbf{Populate}: %TODO

\textbf{Stack}: %TODO

\textbf{URI}: Acronimo di Uniform Resource Identifier, è una stringa che identifica univocamente una risorsa generica che può essere un indirizzo Web, un documento, un'immagine, un file, un servizio, ecc. e la rende disponibile tramite protocolli quali HTTP, FTP, ecc.