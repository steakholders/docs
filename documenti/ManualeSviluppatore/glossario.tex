\section{Glossario}

\textbf{Angular}: È un \glossario{framework} open source Javascript, mantenuto da Google, utilizzato per creare componenti front-end.

\textbf{API}: Application programming interface, sono un insieme di procedure che un sistema software rende accessibile a terzi per interfacciarsi ad esso.

\textbf{Business}: Inteso come dominio di Business, l'insieme di tutti i dati che riguardano un dato campo.

\textbf{Collection}: In \glossario{MongoDB} è un insieme di \glossario{documents}.

\textbf{Dashboard}: Pagina web che rappresenta lo stato corrente di un'applicazione con un interfaccia semplice ed immediata.

\textbf{Document}: Equivale ad una riga di una tabella. Vedi la definizione di Documents.

\textbf{DSL}: Domain Specific Language è un linguaggio di programmazione o un linguaggio di specifica dedicato a particolari problemi di un dominio o a una particolare tecnica di rappresentazione.

\textbf{Issue}: È un sinonimo di ticket contestualizzato nel sistema GitHub. Con il termine ticket si intende un'\glossario{unità} di lavoro con cui apportare un miglioramento in un sistema. Un ticket può essere un \glossario{bug}, una richiesta di funzionalità, un compito, e così via.

\textbf{Framework}: Struttura di supporto su cui un applicativo può essere progettato. Un framework comprende librerie di codice, convenzioni di sviluppo e una serie di strumenti di supporto allo sviluppo.

\textbf{MaaP}: \glossario{Framework} per generare interfacce web di amministrazione dei \glossario{dati di business} basato su stack \glossario{Node.js} e \glossario{MongoDB}.

\textbf{MongoDB}: Sistema gestionale di basi di dati non relazionale, orientato ai documenti, di tipo \glossario{NoSQL}. Il linguaggio utilizzato per la gestione dei dati è JavaScript, del quale sfrutta in particolare la notazione BSON.

\textbf{Mongoose}: Utilizzato per creare una struttura logica nei documenti di \glossario{MongoDB}.

\textbf{Node.js}: Piattaforma software utilizzata per creare applicazioni distribuite facilmente scalabili. Node.js utilizza JavaScript come linguaggio di scripting e gestisce le attese I/O in modo asincrono.

\textbf{NoSQL}: È un movimento che promuove sistemi software dove la persistenza dei dati è caratterizzata dal fatto di non utilizzare il modello relazionale, tipicamente usato dai database tradizionali. L'espressione NoSQL fa riferimento al linguaggio SQL, che è il più comune linguaggio di interrogazione dei dati nei database relazionali, qui preso a simbolo dell'intero paradigma relazionale. Questi archivi di dati tipicamente non richiedono uno schema fisso, evitano spesso le operazioni di unione e puntano a scalare orizzontalmente.

\letteraGlossario{P}

\textbf{Parsing}: Traducibile con analisi sintattica, è il processo atto ad analizzare uno stream continuo in input in modo da determinare la sua struttura grammaticale grazie ad una data grammatica formale.

\textbf{Populate}: È un processo di \glossario{Mongoose} che permette di rimpiazzare automaticamente il path specificato nel \glossario{Document} con \glossario{Document} da altre \glossario{Collection}. A differenza dei normali database relazionali non ci sono join in MongoDB, e questa funzionalità permette un utilizzo analogo.

\textbf{Scaffolding}: In informatica è una procedura che automatizza la creazione di oggetti ed interfacce a partire da alcune semplici specifiche dettate dal programmatore.

\textbf{Stack}: Il termine stack o pila indica un tipo di dato astratto che viene usato in diversi contesti per riferirsi a strutture dati, le cui modalità d'accesso ai dati in essa contenuti seguono una modalità LIFO (Last In First Out), tale per cui i dati vengono estratti in ordine rigorosamente inverso rispetto a quello in cui sono stati inseriti.

\textbf{URI}: Acronimo di Uniform Resource Identifier, è una stringa che identifica univocamente una risorsa generica che può essere un indirizzo Web, un documento, un'immagine, un file, un servizio, ecc. e la rende disponibile tramite protocolli quali HTTP, FTP, ecc.