\section{Pianificazione}
\label{pianificazione}

	\subsection{Stati di progresso per SEMAT}

Ad ogni ciclo individuato per lo sviluppo del progetto è pianificato il raggiungimento di uno stato di progresso per \glossario{SEMAT}. Di seguito vengono specificati meglio alcuni stati, adattandoli alle circostanze di questo progetto.

\begin{itemize}
	\item Opportunity: nel nostro caso le opportunità sono valutate confrontando i diversi capitolati.
	\begin{enumerate}
		\item \textit{Benefit Accrued}: nel \glossario{SEMAT} viene identificato con i benefici derivanti dall'operatività del prodotto e dal ritorno sull'investimento. Per il nostro gruppo, il beneficio tangibile è la valutazione finale.
	\end{enumerate}
	\item Software System:
	\begin{enumerate}
		\item \textit{Operational}: nel \glossario{SEMAT} viene identificato quando il sistema è in uso, disponibile e supportato in un ambiente operativo. I fini di questo progetto non prevedono questo stato di sviluppo, pertanto il gruppo non prevede il suo raggiungimento;
		\item \textit{Retired}: nel \glossario{SEMAT} viene identificato quando il sistema non è più supportato e non vengono prodotti aggiornamenti. I fini di questo progetto non prevedono questo stato di sviluppo, pertanto il gruppo non prevede il suo raggiungimento.
	\end{enumerate}
	\item Work:
	\begin{enumerate}
		\item \textit{Concluded/Closed}: il committente accetta il prodotto, i risultati vengono acquisiti e il gruppo ne esce arricchito. 
	\end{enumerate}
\end{itemize}

Per la descrizione degli stati non specificati, si rimanda alle schede informative (vedi paragrafo \ref{Riferimenti}).

Gli stati pianificati sono definiti nella seguente tabella.

\begin{table}[H]
	\begin{adjustwidth}{-4cm}{-4cm}
	\centering
	\begin{tabular}{ l l l l l }
	\hline
	&
	\multicolumn{1}{c}{\textbf{AN}} & 
	\multicolumn{1}{c}{\textbf{PA}} & 
	\multicolumn{1}{c}{\textbf{PDC}} & 
	\multicolumn{1}{c}{\textbf{V}} \\
	\hline

	\textbf{Opportunity} & Value Established & Viable & Addressed & Benefit Accrued \\
	\textbf{Stakeholders} & Involved & In Agreement & Satisfied for Deployment & Satisfied in Use \\
	\textbf{Requirements} & Coherent & Acceptable & Addressed/Fulfilled & Fulfilled \\
	\textbf{Software System} & & Architecture Selected & Useable & Ready \\
	\textbf{Team} & Formed & Collaborating & Performing & Adjourned \\
	\textbf{Work} & Prepared & Started & Under Control & Concluded/Closed \\
	\textbf{Way-of-Working} & In Use & Working Well & Working Well & Retired \\

	\hline
	\end{tabular}
	\caption{Stati di progresso per \glossario{SEMAT}}
	\label{StatiSEMAT}
	\end{adjustwidth}
\end{table}

	\subsection{Analisi}

Questo periodo ha inizio il 2013-12-01 e termina il 2014-01-08, ma dato che la scadenza di consegna dei documenti è prevista il 2013-12-20 la reale durata è di 20 giorni. \\
I ruoli attivi sono quello di \textit{Amministratore}, \textit{Analista}, \textit{Progettista}, \textit{Responsabile}, \textit{Verificatore}. \\ \\
La suddivisione dei task è incentrata sull'\textit{Analisi dei requisiti}. Per tale motivo viene redatta e verificata subito la parte del \textit{Piano di progetto} relativo all'analisi.
Aderendo al \textbf{modello incrementale} l'\textit{Analisi dei requisiti} è suddivisa in due, la prima riguardante stesura e verifica dei requisiti fondamentali e desiderabili, la seconda riguardante quelli opzionali.
\subsubsection{Diagramma di Gantt}
\ganttset{
	vgrid,
	time slot format=isodate,
	newline shortcut=true,
	x unit=0.4cm,
	y unit chart=0.45cm,
%	
	title/.append style={inner sep=0pt},
	title label font=\footnotesize,
	title height=.4,
%
	group label font=\scriptsize,
	group peaks tip position=0,
	group left shift=0,
	group right shift=0,
%
	bar/.style={draw=black, fill=cyan!70, rounded corners=1pt},
	bar label font=\tiny,
	bar label node/.append style={align=right}
}

\begin{figure}[H]
\begin{adjustwidth}{-4cm}{-4cm}
\centering

\begin{ganttchart}{2013-12-01}{2013-12-21}
	
	\gantttitle{	Dicembre}{21}\\
	\gantttitlelist{1,...,21}{1}\\
	
	\ganttgroup{\textbf{AN1 - Norme di progetto}}{2013-12-01}{2013-12-04} \\
	\ganttbar[name=an1.1]{AN1.1 - Stesura}{2013-12-01}{2013-12-03} \\
	\ganttbar[name=an1.2]{AN1.2 - Verifica}{2013-12-04}{2013-12-04} \\
	\ganttlink{an1.1}{an1.2}
	
	\ganttgroup{\textbf{AN2 - Studio di fattibilità}}{2013-12-05}{2013-12-07} \\
	\ganttbar[name=an2.1]{AN2.1 - Stesura}{2013-12-05}{2013-12-06} \\
	\ganttbar[name=an2.2]{AN2.2 - Verifica}{2013-12-07}{2013-12-07} \\
	\ganttlink{an2.1}{an2.2}
	
	\ganttgroup{\textbf{AN3 - Analisi dei requisiti}}{2013-12-10}{2013-12-19} \\
	\ganttbar[name=an3.1]{AN3.1 - Stesura fondamentali \\ e desiderali}{2013-12-10}{2013-12-13} \\
	\ganttbar[name=an3.2]{AN3.2 - Verifica fondamentali \\ e desiderali}{2013-12-14}{2013-12-15} \\
	\ganttbar[name=an3.3]{AN3.3 - Stesura opzionali}{2013-12-16}{2013-12-18} \\
	\ganttbar[name=an3.4]{AN3.4 - Verifica opzionali}{2013-12-18}{2013-12-19} \\
	\ganttlink{an3.1}{an3.2}
	\ganttlink{an3.3}{an3.4}
	
	\ganttgroup{\textbf{AN4 - Piano di progetto}}{2013-12-07}{2013-12-17} \\
	\ganttbar[name=an4.1]{AN4.1 - Stesura AN}{2013-12-07}{2013-12-08} \\
	\ganttbar[name=an4.2]{AN4.2 - Verifica AN}{2013-12-09}{2013-12-09} \\
	\ganttbar[name=an4.3]{AN4.3 - Stesura PA, PDC, VV}{2013-12-09}{2013-12-15} \\
	\ganttbar[name=an4.4]{AN4.4 - Verifica PA, PDC, VV}{2013-12-16}{2013-12-17} \\
	\ganttlink{an4.1}{an4.2}
	\ganttlink{an4.3}{an4.4}
	
	\ganttgroup{\textbf{AN5 - Piano di qualifica}}{2013-12-10}{2013-12-17} \\
	\ganttbar[name=an5.1]{AN5.1 - Stesura}{2013-12-10}{2013-12-15} \\
	\ganttbar[name=an5.2]{AN5.2 - Verifica}{2013-12-16}{2013-12-17} \\
	\ganttlink{an5.1}{an5.2}	
	
\end{ganttchart}

\caption{Diagramma di Gantt, periodo di Analisi}
\end{adjustwidth}
\end{figure}

\subsubsection{Ripartizione ore}


\begin{table}[H]
	\centering
	\begin{tabular}{ l l l c  }
	\hline
	\multicolumn{1}{c}{\textbf{Id}} & 
	\multicolumn{1}{c}{\textbf{Nome}} & 
	\multicolumn{1}{c}{\textbf{Ruolo}}& 
	\multicolumn{1}{c}{\textbf{Ore}} \\
	\hline
	
	\textbf{AN1} & \textbf{Norme di progetto} \\
	\cline{3-4}
	AN1.1 & Stesura & Amministratore1 & 6\\ 
    & & Amministratore2 & 4\\
    & & Amministratore3 & 5 \\
    & & Responsabile & 4 \\
    \cline{3-4}
	AN1.2 & Verifica & Verificatore &  5\\
	
	\hline
	\textbf{AN2} & \textbf{Studio di fattibilità} \\
	\cline{3-4}
	AN2.1 & Stesura & Analista1 & 2\\ 
    & & Analista2 & 3\\
    & & Analista3 & 3 \\
    \cline{3-4}
	AN2.2 & Verifica & Verificatore &  2\\
	
	\hline
	\textbf{AN3} & \textbf{Piano di progetto} \\
	\cline{3-4}
	AN3.1 & Stesura AN & Amministratore & 5\\ 
    & & Responsabile & 5\\
    \cline{3-4}
	AN3.2 & Verifica AN & Verificatore &  2\\
	\cline{3-4}
	AN3.3 & Stesura PA, PDC, AR AN & Amministratore & 5\\ 
    & & Responsabile & 4\\
	\cline{3-4}
	AN3.4 & Verifica PA, PDC, AR AN & Verificatore &  2\\
	
	\hline
	\textbf{AN4} & \textbf{Analisi dei requisiti} \\
	\cline{3-4}
	AN4.1 & Stesura fondamentali e desiderabili & Analista1 & 11\\ 
    & & Analista2 & 10\\
    & & Analista3 & 11\\
    \cline{3-4}
	AN4.2 & Verifica fondamentali e desiderabili & Verificatore &  6\\
	\cline{3-4}
	AN4.3 & Stesura opzionali & Analista1 & 9\\ 
    & & Analista2 & 9\\
	\cline{3-4}
	AN4.4 & Verifica opzionali & Verificatore &  5\\
	
	\hline
	\textbf{AN5} & \textbf{Piano di qualifica} \\
	\cline{3-4}
	AN5.1 & Stesura & Progettista& 9\\ 
    & & Verificatore & 3 \\
    \cline{3-4}
	AN5.2 & Verifica & Verificatore &  4\\
	
	\hline
	\end{tabular}
	\caption{Allocazione risorse, periodo di Analisi}
	\end{table}

	
	\subsection{Progettazione Architetturale}
	
Questo periodo ha inizio il 2014-01-09 e termina il 2014-02-05, per un totale di 28 giorni. \\
I ruoli attivi sono quello di \textit{Amministratore}, \textit{Analista}, \textit{Progettista}, \textit{Responsabile}, \textit{Verificatore}.

Si suddivide la progettazione architetturale in progettazione dei requisiti fondamentali e desiderabili.

\subsubsection{Diagramma di Gantt}

\begin{figure}[H]
\centering
\scalebox{0.65}{%
%\begin{sideways}

	\begin{ganttchart}{2014-01-09}{2014-01-24}
	\gantttitlecalendar{year, month=name, day} \\

	\ganttgroup{\textbf{PA1 - Analisi dei requisiti}}{2014-01-09}{2014-01-19} \\
		\ganttbar[name=1334548]{PA1.1.1 - Correzione }{2014-01-09}{2014-01-13} \\
		\ganttbar[name=1342155]{PA1.1.2 - Correzione }{2014-01-14}{2014-01-15} \\
		\ganttbar[name=1337864]{PA1.1.2 - Incremento }{2014-01-09}{2014-01-12} \\
		\ganttbar[name=1334550]{PA1.2 - Verifica di dettaglio analisi dei requisiti }{2014-01-17}{2014-01-19} \\
		\ganttbar[name=1345221]{PA1.3 - Manutenzione requisteak }{2014-01-13}{2014-01-13} \\
		\ganttlink{1334548}{1334550}
		\ganttlink{1337864}{1334550}
		\ganttlink{1342155}{1334550}

	\ganttgroup{\textbf{PA2 - Norme di progetto}}{2014-01-09}{2014-01-16} \\
		\ganttbar[name=1342192]{PA2.1 - Correzione }{2014-01-09}{2014-01-10} \\
		\ganttbar[name=1337935]{PA2.1.1 - Correzione }{2014-01-11}{2014-01-12} \\
		\ganttbar[name=1337609]{PA2.1.2.1 - Incremento }{2014-01-09}{2014-01-12} \\
		\ganttbar[name=1325389]{PA2.1.2.2 - Incremento }{2014-01-09}{2014-01-10} \\
		\ganttbar[name=1337857]{PA2.2 - Verifica }{2014-01-14}{2014-01-16} \\

	\ganttgroup{\textbf{PA3 - Glossario}}{2014-01-13}{2014-01-22} \\
		\ganttbar[name=1337612]{PA3.1.1 - Correzione }{2014-01-13}{2014-01-13} \\
		\ganttbar[name=1342108]{PA3.1.2 - Incremento (extra)}{2014-01-14}{2014-01-20} \\
		\ganttbar[name=1337614]{PA3.2 - Verifica }{2014-01-21}{2014-01-22} \\
		\ganttlink{1342108}{1337614}

	\ganttgroup{\textbf{PA4 - Piano di qualifica}}{2014-01-13}{2014-01-19} \\
		\ganttbar[name=1342111]{PA4.1.1 - Correzione }{2014-01-13}{2014-01-14} \\
		\ganttbar[name=1334603]{PA4.1.2 - Incremento }{2014-01-15}{2014-01-16} \\
		\ganttbar[name=1334605]{PA4.2 - Verifica }{2014-01-17}{2014-01-19} \\
		\ganttlink{1342111}{1334603}
		\ganttlink{1334603}{1334605}

	\ganttgroup{\textbf{PA5 - Piano di progetto}}{2014-01-09}{2014-01-19} \\
		\ganttbar[name=1337863]{PA5.1.1 - Correzione }{2014-01-11}{2014-01-12} \\
		\ganttbar[name=1337922]{PA5.1.1.1 - Correzione }{2014-01-09}{2014-01-10} \\
		\ganttbar[name=1342162]{PA5.1.2 - Correzione }{2014-01-13}{2014-01-14} \\
		\ganttbar[name=1337933]{PA5.2 - Incremento }{2014-01-12}{2014-01-13} \\
		\ganttbar[name=1334596]{PA5.3 - Verifica }{2014-01-17}{2014-01-19} \\
		\ganttlink{1342162}{1334596}

	\ganttgroup{\textbf{PA6 - Specifica tecnica}}{2014-01-14}{2014-01-23} \\
		\ganttbar[name=1342112]{PA6.1 - Tecnologie utilizzate }{2014-01-14}{2014-01-16} \\
		\ganttbar[name=1334678]{PA6.2.1 - Architettura generale }{2014-01-14}{2014-01-16} \\
		\ganttbar[name=1337900]{PA6.2.2 - Architettura generale }{2014-01-14}{2014-01-16} \\
		\ganttbar[name=1342126]{PA6.3 - Descrizione componenti }{2014-01-17}{2014-01-19} \\
		\ganttbar[name=1334683]{PA6.4 - Tracciamento }{2014-01-20}{2014-01-21} \\
		\ganttbar[name=1334681]{PA6.5.1 - Verifica tecnologie e architettura generale }{2014-01-17}{2014-01-19} \\
		\ganttbar[name=1342127]{PA6.5.2 - Verifica descrizione componenti }{2014-01-22}{2014-01-23} \\
		\ganttbar[name=1342158]{PA6.5.3 - Verifica tracciamento }{2014-01-22}{2014-01-23} \\
		\ganttlink{1334678}{1342126}
		\ganttlink{1334678}{1334683}
		\ganttlink{1337900}{1334683}
		\ganttlink{1334550}{1334683}
		\ganttlink{1342112}{1334681}
		\ganttlink{1342126}{1342127}
		\ganttlink{1334683}{1342158}
\end{ganttchart}


%\end{sideways}
}
\caption{Diagramma di Gantt, periodo di Progettazione Architetturale}
\end{figure}
	
\subsubsection{Ripartizione ore}
	\begin{center}
	\begin{longtable}{ l l l c  }%[H]
	%\begin{adjustwidth}{-4cm}{-4cm}
	\hline
	\multicolumn{1}{c}{\textbf{Id}} & 
	\multicolumn{1}{c}{\textbf{Nome}} & 
	\multicolumn{1}{c}{\textbf{Ruolo}}& 
	\multicolumn{1}{c}{\textbf{Ore}} \\
	\hline
	
		
	\textbf{PA1} & \textbf{Analisi dei requisiti} \\
	PA1.1.1 & Correzione & Analista & 10 \\
	PA1.1.2 & Correzione & Analista & 2 \\
	PA1.1.3 & Correzione & Analista & 4 \\
	PA1.2 & Verifica di dettaglio analisi dei requisiti & Verificatore & 4 \\
	PA1.3 & Incremento & Analista & 8 \\
	PA1.4 & Manutenzione requisteak & Amministratore & 2 \\
	\hline

	\textbf{PA2} & \textbf{Norme di progetto} \\
	PA2.1.1 & Correzione & Amministratore & 4 \\
	PA2.1.2 & Correzione & Amministratore & 4 \\
	PA2.1.3 & Correzione & Amministratore & 4 \\
	PA2.2 & Verifica & Verificatore & 4 \\
	PA2.3.1 & Incremento & Amministratore & 8 \\
	PA2.3.2 & Incremento & Amministratore & 6 \\
	\hline

	\textbf{PA3} & \textbf{Glossario} \\
	PA3.1.1 & Correzione & Amministratore & 2 \\
	PA3.1.2 & Incremento (extra) & (non assegnato) & 14 \\
	PA3.2 & Verifica & Verificatore & 4 \\
	\hline

	\textbf{PA4} & \textbf{Piano di qualifica} \\
	PA4.1.1 & Correzione & Amministratore & 8 \\
	PA4.1.2 & Incremento & Verificatore & 4 \\
	PA4.1.3 & Correzione & Amministratore & 6 \\
	PA4.2 & Verifica & Verificatore & 6 \\
	\hline

	\textbf{PA5} & \textbf{Piano di progetto} \\
	PA5.1.1 & Correzione & Amministratore & 4 \\
	PA5.1.2 & Correzione & Responsabile & 5 \\
	PA5.1.3 & Correzione & Responsabile & 4 \\
	PA5.2 & Incremento & Responsabile & 6 \\
	PA5.3 & Verifica & Verificatore & 4 \\
	PA5.4 & Correzione & Responsabile & 4 \\
	\hline

	\textbf{PA6} & \textbf{Specifica tecnica} \\
	PA6.1 & Tecnologie utilizzate & Progettista & 6 \\
	PA6.1.1 & Architettura generale & Progettista & 10 \\
	PA6.1.3 & Architettura generale & Progettista & 8 \\
	PA6.2.2 & Architettura generale & Progettista & 4 \\
	PA6.2.4 & Architettura generale & Progettista & 4 \\
	PA6.2.5 & Architettura generale & Progettista & 6 \\
	PA6.2.6 & Architettura generale & Progettista & 4 \\
	PA6.2.7 & Architettura generale & Progettista & 4 \\
	PA6.3 & Descrizione componenti & Progettista & 4 \\
	PA6.4.1 & Tracciamento & Progettista & 4 \\
	PA6.4.2 & Tracciamento & Progettista & 6 \\
	PA6.5.1 & Verifica tecnologie utilizzate & Verificatore & 6 \\
	PA6.5.2 & Verifica descrizione componenti & Verificatore & 4 \\
	PA6.5.3 & Verifica tracciamento & Verificatore & 2 \\
	PA6.5.4 & Verifica descrizione componenti & Verificatore & 4 \\
	PA6.6.1 & Correzione & Progettista & 6 \\
	PA6.6.2 & Correzione & Progettista & 4 \\
	\hline

	
	\caption{Allocazione risorse, periodo di Progettazione architetturale}
	%\end{adjustwidth}
	\end{longtable}
	\end{center}	
	%\pagebreak
	\subsection{Progettazione di Dettaglio e Codifica}
	 
Questo periodo ha inizio il 2014-02-13 e termina il 2014-03-13, per un totale di 32 giorni. \\
I ruoli attivi sono quello di \textit{Amministratore}, \textit{Analista}, \textit{Progettista}, \textit{Programmatore}, \textit{Responsabile}, \textit{Verificatore}.

Le attività svolte in questo periodo si dividono in tre parti:

\begin{enumerate}

	\item Progettazione architetturale dei requisiti opzionali;
	\item Progettazione dettagliata dei requisiti fondamentali, desiderabili e opzionali;
	\item Codifica e testing;

\end{enumerate}

Tutto ciò seguirà come sempre il modello incrementale e si darà priorità ai requisiti fondamentali e desiderabili.
In tal modo, in caso di ritardi si potrà procedere ugualmente al periodo successivo, consegnando un prodotto completo.

\subsubsection{Diagramma di Gantt}

\begin{figure}[H]
\centering
\scalebox{0.63}{%
%\begin{sideways}

	\begin{ganttchart}{2014-02-05}{2014-03-05}
	\gantttitlecalendar{year, month=name, day} \\

	\ganttgroup{\textbf{PDC1 - Analisi dei requisiti}}{2014-02-12}{2014-02-17} \\
		\ganttbar[name=1355004]{PDC1.1 - Correzione}{2014-02-12}{2014-02-13} \\
		\ganttbar[name=1355200]{PDC1.2 - Incremento}{2014-02-14}{2014-02-15} \\
		\ganttbar[name=1355006]{PDC1.3 - Verifica di dettaglio analisi dei requisiti}{2014-02-16}{2014-02-17} \\

	\ganttgroup{\textbf{PDC2 - Norme di progetto}}{2014-02-12}{2014-02-17} \\
		\ganttbar[name=1355009]{PDC2.1 - Correzione}{2014-02-12}{2014-02-13} \\
		\ganttbar[name=1355011]{PDC2.2 - Incremento}{2014-02-14}{2014-02-15} \\
		\ganttbar[name=1355012]{PDC2.3 - Verifica}{2014-02-16}{2014-02-17} \\

	\ganttgroup{\textbf{PDC3 - Glossario}}{2014-02-12}{2014-03-02} \\
		\ganttbar[name=1355014]{PDC3.1 - Correzione}{2014-02-12}{2014-02-12} \\
		\ganttbar[name=1355015]{PDC3.2 - Incremento (extra)}{2014-02-14}{2014-02-28} \\
		\ganttbar[name=1355016]{PDC3.3 - Verifica}{2014-03-01}{2014-03-02} \\

	\ganttgroup{\textbf{PDC6 - Specifica tecnica}}{2014-02-08}{2014-02-08} \\
		\ganttbar[name=1386769]{PDC6 - Non devono esserci TaskList senza task! (extra)}{2014-02-08}{2014-02-08} \\

	\ganttgroup{\textbf{PDC7 - Preparazione ambiente di lavoro}}{2014-02-05}{2014-02-09} \\
		\ganttbar[name=1334652]{PDC7.1 - Preparazione ambiente di lavoro}{2014-02-05}{2014-02-09} \\

	\ganttgroup{\textbf{PDC8 - Definizione di prodotto}}{2014-02-05}{2014-02-16} \\
		\ganttbar[name=1334684]{PDC8.1.1 - Progettazione di dettaglio}{2014-02-05}{2014-02-09} \\
		\ganttbar[name=1346672]{PDC8.1.2 - Progettazione di dettaglio}{2014-02-05}{2014-02-09} \\
		\ganttbar[name=1355252]{PDC8.1.3 - Progettazione di dettaglio}{2014-02-05}{2014-02-09} \\
		\ganttbar[name=1355333]{PDC8.1.4 - Progettazione di dettaglio}{2014-02-05}{2014-02-09} \\
		\ganttbar[name=1334574]{PDC8.1.5 - Progettazione di dettaglio}{2014-02-05}{2014-02-09} \\
		\ganttbar[name=1355248]{PDC8.1.6 - Progettazione di dettaglio}{2014-02-05}{2014-02-09} \\
		\ganttbar[name=1334576]{PDC8.2.1 - Verifica progettazione}{2014-02-10}{2014-02-11} \\
		\ganttbar[name=1334693]{PDC8.2.2 - Verifica progettazione}{2014-02-10}{2014-02-11} \\
		\ganttbar[name=1334590]{PDC8.3.1 - Stesura}{2014-02-10}{2014-02-14} \\
		\ganttbar[name=1334696]{PDC8.3.2 - Stesura}{2014-02-10}{2014-02-14} \\
		\ganttbar[name=1334591]{PDC8.4 - Verifica}{2014-02-15}{2014-02-16} \\
		\ganttlink{1334574}{1334576}
		\ganttlink{1346672}{1334693}
		\ganttlink{1334684}{1334693}
		\ganttlink{1334590}{1334591}

	\ganttgroup{\textbf{PDC9 - Codifica}}{2014-02-10}{2014-02-25} \\
		\ganttbar[name=1334600]{PDC9.1.1 - Codifica}{2014-02-10}{2014-02-17} \\
		\ganttbar[name=1346716]{PDC9.1.2 - Codifica}{2014-02-15}{2014-02-22} \\
		\ganttbar[name=1346717]{PDC9.1.3 - Codifica}{2014-02-12}{2014-02-20} \\
		\ganttbar[name=1346718]{PDC9.1.4 - Codifica}{2014-02-12}{2014-02-19} \\
		\ganttbar[name=1334608]{PDC9.1.5 - Codifica}{2014-02-10}{2014-02-19} \\
		\ganttbar[name=1346719]{PDC9.1.6 - Codifica}{2014-02-10}{2014-02-19} \\
		\ganttbar[name=1346720]{PDC9.1.7 - Codifica}{2014-02-10}{2014-02-19} \\
		\ganttbar[name=1346721]{PDC9.1.8 - Codifica}{2014-02-17}{2014-02-22} \\
		\ganttbar[name=1334606]{PDC9.3.1 - Test di unità}{2014-02-21}{2014-02-23} \\
		\ganttbar[name=1367539]{PDC9.3.2 - Test di integrazione}{2014-02-24}{2014-02-24} \\
		\ganttbar[name=1367546]{PDC9.3.3 - Test di sistema}{2014-02-25}{2014-02-25} \\
		\ganttbar[name=1334609]{PDC9.3.4 - Test di unità}{2014-02-23}{2014-02-23} \\
		\ganttbar[name=1367570]{PDC9.3.5 - Test di integrazione}{2014-02-24}{2014-02-24} \\
		\ganttbar[name=1367571]{PDC9.3.6 - Test di sistema}{2014-02-25}{2014-02-25} \\
		\ganttlink{1346718}{1334606}
		\ganttlink{1346717}{1334606}
		\ganttlink{1346720}{1334609}
		\ganttlink{1346721}{1334609}

	\ganttgroup{\textbf{PDC10 - manuale utente}}{2014-02-26}{2014-03-04} \\
		\ganttbar[name=1334610]{PDC10.1 - Stesura}{2014-02-26}{2014-03-02} \\
		\ganttbar[name=1334611]{PDC10.2 - Verifica}{2014-03-03}{2014-03-04} \\
		\ganttlink{1334610}{1334611}
\end{ganttchart}


%\end{sideways}
}
\caption{Diagramma di Gantt, periodo di Progettazione di Dettaglio e Codifica}
\end{figure}

\subsubsection{Ripartizione ore}

\begin{center}
\begin{longtable}{ l l l c  }
	%\begin{adjustwidth}{-4cm}{-4cm}
	\hline
	\multicolumn{1}{c}{\textbf{Id}} & 
	\multicolumn{1}{c}{\textbf{Nome}} & 
	\multicolumn{1}{c}{\textbf{Ruolo}}& 
	\multicolumn{1}{c}{\textbf{Ore}} \\
	\hline
	
		
	\textbf{PDC1} & \textbf{Analisi dei requisiti} \\
	PDC1.1 & Correzione & Analista & 4 \\
	PDC1.2 & Incremento & Analista & 4 \\
	PDC1.3 & Verifica di dettaglio analisi dei requisiti & Verificatore & 4 \\
	\hline

	\textbf{PDC2} & \textbf{Norme di progetto} \\
	PDC2.1 & Correzione & Amministratore & 4 \\
	PDC2.2 & Incremento & Amministratore & 4 \\
	PDC2.3 & Verifica & Verificatore & 4 \\
	\hline

	\textbf{PDC3} & \textbf{Glossario} \\
	PDC3.1 & Correzione & Amministratore & 4 \\
	PDC3.2 & Incremento (extra) & (non assegnato) & 30 \\
	PDC3.3 & Verifica & Verificatore & 4 \\
	\hline

	\textbf{PDC6} & \textbf{Specifica tecnica} \\
	PDC6.1.1 & Progettazione requisiti opzionali & Progettista & 10 \\
	PDC6.1.2 & Progettazione requisiti opzionali & Progettista & 10 \\
	PDC6.1.3 & Progettazione requisiti opzionali & Progettista & 10 \\
	PDC6.1.4 & Progettazione requisiti opzionali & Progettista & 10 \\
	PDC6.2.1 & Verifica requisiti opzionali & Verificatore & 10 \\
	PDC6.3.1 & Stesura requisiti opzionali & Amministratore & 8 \\
	PDC6.3.2 & Stesura requisiti opzionali & Amministratore & 6 \\
	PDC6.4.1 & Progettazione dettagliata requisiti fondamentali e desiderabili & Progettista & 10 \\
	PDC6.4.2 & Progettazione dettagliata requisiti fondamentali e desiderabili & Progettista & 10 \\
	PDC6.4.3 & Progettazione dettagliata & Progettista & 10 \\
	PDC6.5.1 & Verifica requisiti fondamentali e desiderabili & Verificatore & 10 \\
	PDC6.5.2 & Verifica requisiti fondamentali e desiderabili & Verificatore & 10 \\
	PDC6.5.3 & Verifica requisiti fondamentali e desiderabili & Verificatore & 8 \\
	PDC6.6.1 & Progettazione dettagliata requisiti opzionali & Progettista & 10 \\
	PDC6.6.2 & Progettazione dettagliata requisiti opzionali & Progettista & 10 \\
	PDC6.6.3 & Progettazione dettagliata requisiti opzionali & Progettista & 10 \\
	PDC6.7 & Verifica requisiti opzionali & Verificatore & 6 \\
	PDC6.8 & Verifica progettazione generale & Verificatore & 4 \\
	PDC6.9.1 & Correzione & Progettista & 6 \\
	PDC6.9.2 & Correzione & Progettista & 6 \\
	\hline

	\textbf{PDC7} & \textbf{Preparazione ambiente di lavoro} \\
	PDC7.1 & Preparazione ambiente di lavoro & Amministratore & 6 \\
	\hline

	\textbf{PDC8} & \textbf{Definizione di prodotto} \\
	PDC8.1 & Stesura & Progettista & 8 \\
	PDC8.2 & Verifica & Verificatore & 10 \\
	\hline

	\textbf{PDC9} & \textbf{Codifica} \\
	PDC9.1.1 & Codifica requisiti fondamentali e desiderabili & Programmatore & 8 \\
	PDC9.1.2 & Codifica requisiti fondamentali e desiderabili & Programmatore & 12 \\
	PDC9.1.3 & Codifica requisiti fondamentali e desiderabili & Programmatore & 24 \\
	PDC9.1.4 & Codifica requisiti fondamentali e desiderabili & Programmatore & 16 \\
	PDC9.2.1 & Test di unità requisiti fondamentali e desiderabili & Verificatore & 6 \\
	PDC9.2.2 & Test di integrazione requisiti fondamentali e desiderabili & Verificatore & 2 \\
	PDC9.2.3 & Test di sistema requisiti fondamentali e desiderabili & Verificatore & 2 \\
	PDC9.3.1 & Codifica requisiti opzionali & Programmatore & 20 \\
	PDC9.3.2 & Codifica requisiti opzionali & Programmatore & 20 \\
	PDC9.3.3 & Codifica requisiti opzionali & Programmatore & 20 \\
	PDC9.3.4 & Codifica requisiti opzionali & Programmatore & 12 \\
	PDC9.4.1 & Test di unità requisiti opzionali & Verificatore & 2 \\
	PDC9.4.2 & Test di integrazione requisiti opzionali & Verificatore & 2 \\
	PDC9.4.3 & Test di sistema requisiti opzionali & Verificatore & 2 \\
	\hline

	\textbf{PDC10} & \textbf{manuale utente} \\
	PDC10.1 & Stesura & Amministratore & 6 \\
	PDC10.2 & Verifica & Verificatore & 4 \\
	\hline

	
	\caption{Allocazione risorse, periodo di Progettazione di dettaglio e codifica}
%\end{adjustwidth}
\end{longtable}	
\end{center}
	
	
	\subsection{Validazione}
	
Questo periodo ha inizio il 2014-02-27 e termina il 2014-03-17, per un totale di 19 giorni. \\
I ruoli attivi sono quello di \textit{Amministratore}, \textit{Progettista}, \textit{Verificatore}. \\ \\

\subsubsection{Diagramma di Gantt}

\begin{figure}[H]
\centering

	\begin{ganttchart}{2014-03-07}{2014-03-17}
	\gantttitlecalendar{year, month=name, day} \\

	\ganttgroup{\textbf{V1 - Piano di qualifica}}{2014-03-11}{2014-03-16} \\
		\ganttbar[name=1334612]{V1.1 - Correzione}{2014-03-11}{2014-03-12} \\
		\ganttbar[name=1355481]{V1.2 - Incremento}{2014-03-13}{2014-03-14} \\
		\ganttbar[name=1334613]{V1.3 - Verifica}{2014-03-15}{2014-03-16} \\
		\ganttlink{1334612}{1334613}

	\ganttgroup{\textbf{V2 - Definizione di prodotto}}{2014-03-11}{2014-03-16} \\
		\ganttbar[name=1334628]{V2.1 - Correzione}{2014-03-11}{2014-03-12} \\
		\ganttbar[name=1355480]{V2.2 - Incremento}{2014-03-13}{2014-03-14} \\
		\ganttbar[name=1334630]{V2.3 - Verifica}{2014-03-15}{2014-03-16} \\
		\ganttlink{1334628}{1334630}

	\ganttgroup{\textbf{V3 - Manuale utente}}{2014-03-11}{2014-03-16} \\
		\ganttbar[name=1334632]{V3.1 - Correzione}{2014-03-11}{2014-03-12} \\
		\ganttbar[name=1334634]{V3.2 - Incremento}{2014-03-13}{2014-03-14} \\
		\ganttbar[name=1355482]{V3.3 - Verifica}{2014-03-15}{2014-03-16} \\

	\ganttgroup{\textbf{V4 - Collaudo}}{2014-03-07}{2014-03-10} \\
		\ganttbar[name=1334638]{V4.1 - Collaudo}{2014-03-07}{2014-03-10} \\
		\ganttbar[name=1347376]{V4.2 - Collaudo}{2014-03-07}{2014-03-10} \\
		\ganttbar[name=1347377]{V4.3 - Collaudo}{2014-03-07}{2014-03-10} \\
		\ganttbar[name=1347378]{V4.4 - Collaudo}{2014-03-07}{2014-03-10} \\
		\ganttbar[name=1355526]{V4.5 - Collaudo}{2014-03-07}{2014-03-10} \\
		\ganttbar[name=1355538]{V4.6 - Collaudo}{2014-03-07}{2014-03-10} \\
		\ganttbar[name=1355539]{V4.7 - Collaudo}{2014-03-07}{2014-03-10} \\

	\ganttgroup{\textbf{V5 - Piano di progetto}}{2014-03-11}{2014-03-16} \\
		\ganttbar[name=1382849]{V5.1 - Correzione}{2014-03-11}{2014-03-12} \\
		\ganttbar[name=1382850]{V5.2 - Incremento}{2014-03-13}{2014-03-14} \\
		\ganttbar[name=1382851]{V5.3 - Verifica}{2014-03-15}{2014-03-16} \\

	\ganttgroup{\textbf{V6 - Specifica tecnica}}{2014-03-11}{2014-03-16} \\
		\ganttbar[name=1382857]{V6.1 - Correzione}{2014-03-11}{2014-03-12} \\
		\ganttbar[name=1382858]{V6.2 - Incremento}{2014-03-13}{2014-03-14} \\
		\ganttbar[name=1382859]{V6.3 - Verifica}{2014-03-15}{2014-03-16} \\

	\ganttgroup{\textbf{V7 - Glossario}}{2014-03-11}{2014-03-16} \\
		\ganttbar[name=1382862]{V7.1 - Correzione (extra)}{2014-03-11}{2014-03-11} \\
		\ganttbar[name=1382863]{V7.2 - Incremento (extra)}{2014-03-11}{2014-03-16} \\
\end{ganttchart}


\caption{Diagramma di Gantt, periodo di Validazione}
\end{figure}

\subsubsection{Ripartizione ore}

\begin{table}[H]
\begin{adjustwidth}{-4cm}{-4cm}
	\centering
	\begin{tabular}{ l l l c  }
	\hline
	\multicolumn{1}{c}{\textbf{Id}} & 
	\multicolumn{1}{c}{\textbf{Nome}} & 
	\multicolumn{1}{c}{\textbf{Ruolo}}& 
	\multicolumn{1}{c}{\textbf{Ore}} \\
	\hline
	
		
	\textbf{VV1} & \textbf{Piano di qualifica} \\
	\cline{3-4}
	VV1.1 & Aggiornamento & Amministratore & 8 \\
	VV1.2 & Verifica & Verificatore & 4 \\
	\hline

	\textbf{VV2} & \textbf{Definizione di prodotto} \\
	\cline{3-4}
	VV2.1 & Aggiornamento & Amministratore & 8 \\
	VV2.2 & Verifica & Verificatore & 6 \\
	\hline

	\textbf{VV3} & \textbf{Manuale utente} \\
	\cline{3-4}
	VV3.1 & Aggiornamento & Amministratore & 8 \\
	VV3.2 & Verifica & Verificatore & 6 \\
	\hline

	\textbf{VV4} & \textbf{Collaudo} \\
	\cline{3-4}
	VV4.1 & Collaudo & Verificatore & 10 \\
	VV4.2 & Collaudo & Verificatore & 10 \\
	VV4.3 & Collaudo & Verificatore & 8 \\
	VV4.4 & Collaudo & Verificatore & 8 \\
	\hline


	\end{tabular}
	\caption{Allocazione risorse, periodo di Validazione}
\end{adjustwidth}
\end{table}	
	
	
