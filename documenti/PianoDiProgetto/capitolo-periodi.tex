\pgfplotsset{
	ybar stacked, 
	enlargelimits =0.05,
	legend style ={at={(0.5,-0.5)},
	anchor=north,
	legend columns=-1},
	ylabel={Ore},
	xtick=data,
	x tick label style={rotate=45,anchor=east},
	symbolic x coords={Enrico Rotundo,Federico Poli,Giacomo Fornari,Gianluca Donato,Luca De Franceschi,Nicolò Tresoldi,Serena Girardi}
}



\section{Periodi di progetto}

Ogni componente del gruppo dovrà ricoprire ogni ruolo almeno una volta nel corso del progetto. Durante lo stesso periodo un componente può ricoprire più ruoli, a condizione che le mansioni previste non vadano in conflitto tra loro, ad esempio non si può verificare il proprio lavoro.

Nell'intestazione utilizzata per le tabelle di questo capitolo sono state impiegate \textbf{abbreviazioni} per i nomi dei ruoli.
Di seguito viene riportato il loro significato, \textbf{nell'ordine in cui sono utilizzate} nell'intestazione:
\begin{itemize}
	\item Amm.: \textit{Amministratore};
	\item Ana.: \textit{Analista};
	\item Pgt.: \textit{Progettista};
	\item Pgr.: \textit{Programmatore};
	\item Res.: \textit{Responsabile};
	\item Ver.: \textit{Verificatore}.
\end{itemize}


\pagebreak
\subsection{Analisi}

	
	\subsubsection{Suddivisione del lavoro}

	Nel periodo di Analisi dei requisiti ciascun componente dovrà rivestire i seguenti ruoli:
	
	\noindent
	\begin{table}[H]
	\begin{tabular}{lccccccc}
	\toprule
	    \textbf{Nome}  & \multicolumn{6}{c}{\textbf{Ore per ruolo}} & \textbf{Ore totali} \\
	     & Amm. & Ana. & Pgt. & Pgr. & Res. & Ver. & \\
	    \midrule
	    	Enrico Rotundo & 0 & 12 & 9 & 0 & 0 & 2 & 23 \\
	Federico Poli & 0 & 9 & 0 & 0 & 4 & 9 & 22 \\
	Giacomo Fornari & 9 & 2 & 0 & 0 & 5 & 4 & 20 \\
	Gianluca Donato & 0 & 10 & 0 & 0 & 0 & 11 & 21 \\
	Luca De Franceschi & 6 & 11 & 0 & 0 & 0 & 0 & 17 \\
	Nicolò Tresoldi & 15 & 0 & 0 & 0 & 5 & 0 & 20 \\
	Serena Girardi & 0 & 14 & 0 & 0 & 0 & 5 & 19 \\

	    \bottomrule
	\end{tabular}
	\caption{Ore per componente, periodo di Analisi}
	\end{table}

	I valori in tabella sono riassunti nel seguente grafico: \\ 

	\begin{figure}[H]
	\begin{tikzpicture}
	\begin{axis}
		\addplot+[color=Amministratore] plotcoordinates{(Enrico Rotundo,0)(Federico Poli,0)(Giacomo Fornari,9)(Gianluca Donato,0)(Luca De Franceschi,6)(Nicolò Tresoldi,15)(Serena Girardi,0)};
\addplot+[color=Analista] plotcoordinates{(Enrico Rotundo,12)(Federico Poli,9)(Giacomo Fornari,2)(Gianluca Donato,10)(Luca De Franceschi,11)(Nicolò Tresoldi,0)(Serena Girardi,14)};
\addplot+[color=Progettista] plotcoordinates{(Enrico Rotundo,9)(Federico Poli,0)(Giacomo Fornari,0)(Gianluca Donato,0)(Luca De Franceschi,0)(Nicolò Tresoldi,0)(Serena Girardi,0)};
\addplot+[color=Programmatore] plotcoordinates{(Enrico Rotundo,0)(Federico Poli,0)(Giacomo Fornari,0)(Gianluca Donato,0)(Luca De Franceschi,0)(Nicolò Tresoldi,0)(Serena Girardi,0)};
\addplot+[color=Responsabile] plotcoordinates{(Enrico Rotundo,0)(Federico Poli,4)(Giacomo Fornari,5)(Gianluca Donato,0)(Luca De Franceschi,0)(Nicolò Tresoldi,5)(Serena Girardi,0)};
\addplot+[color=Verificatore] plotcoordinates{(Enrico Rotundo,2)(Federico Poli,9)(Giacomo Fornari,4)(Gianluca Donato,11)(Luca De Franceschi,0)(Nicolò Tresoldi,0)(Serena Girardi,5)};

		\legend{amministratore, analista, progettista, programmatore, responsabile, verificatore}
	\end{axis}
	\end{tikzpicture}
	\caption{Ore per componente, periodo di analisi}
	\end{figure}

	Si fa notare che le ore sopra indicate non sono incluse nelle 105 ore rappresentanti il tetto massimo di ore somministrabile da ciascun componente.

	Il seguente grafico mostra il peso orario di ogni ruolo in questo periodo.

	\begin{figure}[H]
	\begin{tikzpicture}
		\pie[text=legend, color={amministratore, analista, progettista, programmatore, responsabile, verificatore}]{21.1/Amministratore, 40.8/Analista, 6.3/Progettista, ./Programmatore, 9.9/Responsabile, 21.8/Verificatore}

	\end{tikzpicture}
	\caption{Ore per ruolo, periodo di Analisi}
	\end{figure}


	\subsubsection{Prospetto orario}

	La tabella sottostante riporta le ore preventivate e le ore effettivamente impiegate (riportate tra parentesi) per ciascun componente del gruppo \GroupName{}.

	\begin{center}
	\begin{table}[H]
	\begin{tabular}{lccccccc}
	\toprule
	    \textbf{Nome}  & \multicolumn{6}{c}{\textbf{Ore per ruolo}} & \textbf{Ore totali} \\
	     & Amm. & Ana. & Pgt. & Pgr. & Res. & Ver. & \\
	    \midrule
	    	Enrico Rotundo & 0 (+0) & 12 (-12) & 9 (-9) & 0 (+0) & 0 (+0) & 2 (-2) & 23 (-23) \\
	Federico Poli & 0 (+0) & 9 (-9) & 0 (+0) & 0 (+0) & 4 (-4) & 9 (-9) & 22 (-22) \\
	Giacomo Fornari & 9 (-9) & 2 (-2) & 0 (+0) & 0 (+0) & 5 (-5) & 4 (-4) & 20 (-20) \\
	Gianluca Donato & 0 (+0) & 10 (-10) & 0 (+0) & 0 (+0) & 0 (+0) & 11 (-11) & 21 (-21) \\
	Luca De Franceschi & 6 (-6) & 11 (-11) & 0 (+0) & 0 (+0) & 0 (+0) & 0 (+0) & 17 (-17) \\
	Nicolò Tresoldi & 15 (-15) & 0 (+0) & 0 (+0) & 0 (+0) & 5 (-5) & 0 (+0) & 20 (-20) \\
	Serena Girardi & 0 (+0) & 14 (-14) & 0 (+0) & 0 (+0) & 0 (+0) & 5 (-5) & 19 (-19) \\

	    \bottomrule
	\end{tabular}
	\caption{Differenza preventivo consuntivo per componente, periodo di Analisi}
	\end{table}
	\end{center}


	\subsubsection{Prospetto economico}
	
	A scopo di trasparenza viene redatto il prospetto economico riguardante il periodo di Analisi dei requisiti, ma si precisa che le ore spese in questo periodo sono a carico del fornitore e non del proponente.

	\begin{table}[H]
	\centering
	\begin{tabular}{ l c c }
		\textbf{Ruolo} & \textbf{Ore} & \textbf{Costo} \\
		\hline
			Amministratore & 30 & 600 € \\
	Analista & 58 & 1450 € \\
	Progettista & 9 & 198 € \\
	Programmatore & 0 & 0 € \\
	Responsabile & 14 & 420 € \\
	Verificatore & 31 & 465 € \\
\hline
	Totale & 142 & 3133 € \\
\hline

	\end{tabular}
	\caption{Ore e costo per ruolo, periodo di Analisi}
	\end{table}

	Il seguente grafico mostra il peso del costo di ogni ruolo in questo periodo.

	\begin{figure}[H]
	\begin{tikzpicture}
		\pie[text=legend, color={amministratore, analista, progettista, programmatore, responsabile, verificatore}]{19.2/Amministratore, 46.3/Analista, 6.3/Progettista, ./Programmatore, 13.4/Responsabile, 14.8/Verificatore}

	\end{tikzpicture}
	\caption{Costo per ruolo, periodo di Analisi}
	\end{figure}

	Di seguito viene invece presentato il consuntivo relativo al periodo di \textit{Analisi}. La tabella sottostante riporta le ore preventivate e le ore effettivamente impiegate (riportate tra parentesi) per ciascun ruolo.

	\begin{table}[H]
	\centering
	\begin{tabular}{lccccccc}
	\toprule
	    \textbf{Ruolo}  & \textbf{Ore} & \textbf{Costo} \\
	    \midrule
	    	Amministratore & 30 (+0) & 600 (+0) € \\
	Analista & 58 (-8) & 1450 (-200) € \\
	Progettista & 9 (-1) & 198 (-22) € \\
	Programmatore & 0 (+0) & 0 (+0) € \\
	Responsabile & 14 (+1) & 420 (+30) € \\
	Verificatore & 31 (+9) & 465 (+135) € \\
\hline
\textbf{Totale consuntivo} & +141 & +3190 € \\
\textbf{Totale preventivo} & +142 & +3133 € \\
\textbf{Differenza dei totali} & +1 & -57 € \\

	    \bottomrule
	\end{tabular}
	\caption{Differenza preventivo consuntivo per ruolo, periodo di Analisi}
	\end{table}

	Viene di seguito incluso il grafico che illustra la differenza tra ore preventivate e ore effettivamente impiegate per ciascun ruolo nel periodo di \textit{Analisi}.

	\begin{figure}[H]
	\centering
	\begin{tikzpicture}
	\begin{axis}[stileRuoli]
		\addplot+[color=Pianificate] coordinates {(Amministratore,30)(Analista,58)(Progettista,9)(Programmatore,0)(Responsabile,14)(Verificatore,31)};

		\legend{preventivate,consumate}
	\end{axis}
	\end{tikzpicture}
	\caption{Differenza preventivo consuntivo per ruolo, periodo di analisi}
	\end{figure}


\pagebreak
\subsection{Progettazione architetturale}

	
	\subsubsection{Suddivisione del lavoro}

	Nel periodo di Progettazione architetturale ciascun componente dovrà rivestire i seguenti ruoli:

	\begin{table}[H]
	\centering
	\begin{tabular}{lccccccc}
	\toprule 
	    \textbf{Nome}  & \multicolumn{6}{c}{\textbf{Ore per ruolo}} & \textbf{Ore totali} \\
	     & Amm. & Ana. & Pgt. & Pgr. & Res. & Ver. \\
	    \midrule
	     & 4 & 0 & 6 & 0 & 0 & 8 & 18 \\
 & 4 & 0 & 6 & 0 & 5 & 6 & 21 \\
 & 8 & 0 & 0 & 0 & 4 & 6 & 18 \\
 & 2 & 8 & 6 & 0 & 0 & 4 & 20 \\
 & 4 & 0 & 6 & 0 & 4 & 6 & 20 \\
 & 10 & 4 & 4 & 0 & 0 & 6 & 24 \\
 & 0 & 10 & 0 & 0 & 0 & 10 & 20 \\

	    \bottomrule
	\end{tabular}
	\caption{Ore per componente, periodo di Progettazione architetturale}
	\end{table}

	I valori in tabella sono riassunti nel seguente grafico: \\

	\begin{figure}[H]
	\begin{tikzpicture}
	\begin{axis}
		\addplot+[color=Amministratore] plotcoordinates{(Enrico Rotundo,0)(Federico Poli,4)(Giacomo Fornari,8)(Gianluca Donato,6)(Luca De Franceschi,6)(Nicolò Tresoldi,10)(Serena Girardi,8)};
\addplot+[color=Analista] plotcoordinates{(Enrico Rotundo,4)(Federico Poli,0)(Giacomo Fornari,2)(Gianluca Donato,8)(Luca De Franceschi,0)(Nicolò Tresoldi,0)(Serena Girardi,10)};
\addplot+[color=Progettista] plotcoordinates{(Enrico Rotundo,16)(Federico Poli,28)(Giacomo Fornari,8)(Gianluca Donato,14)(Luca De Franceschi,16)(Nicolò Tresoldi,8)(Serena Girardi,18)};
\addplot+[color=Programmatore] plotcoordinates{(Enrico Rotundo,0)(Federico Poli,0)(Giacomo Fornari,0)(Gianluca Donato,0)(Luca De Franceschi,0)(Nicolò Tresoldi,0)(Serena Girardi,0)};
\addplot+[color=Responsabile] plotcoordinates{(Enrico Rotundo,0)(Federico Poli,4)(Giacomo Fornari,4)(Gianluca Donato,4)(Luca De Franceschi,6)(Nicolò Tresoldi,8)(Serena Girardi,0)};
\addplot+[color=Verificatore] plotcoordinates{(Enrico Rotundo,16)(Federico Poli,4)(Giacomo Fornari,22)(Gianluca Donato,8)(Luca De Franceschi,10)(Nicolò Tresoldi,4)(Serena Girardi,14)};

		\legend{amministratore, analista, progettista, programmatore, responsabile, verificatore}
	\end{axis}
	\end{tikzpicture}
	\caption{Ore per componente, periodo di progettazione architetturale}
	\end{figure}

	Il seguente grafico mostra il peso orario di ogni ruolo in questo periodo.

	\begin{figure}[H]
	\begin{tikzpicture}
		\pie[sum=auto, text=legend, color={amministratore, analista, progettista, programmatore, responsabile, verificatore}]{48/Amministratore, 24/Analista, 70/Progettista, ./Programmatore, 19.0/Responsabile, 42/Verificatore}

	\end{tikzpicture}
	\caption{Ore per ruolo, periodo di Progettazione architetturale}
	\end{figure}


	\subsubsection{Prospetto orario}

	La tabella sottostante riporta le ore preventivate e le ore effettivamente impiegate (riportate tra parentesi) per ciascun componente del gruppo \GroupName{}.

	\begin{center}
	\begin{table}[H]
	\begin{tabular}{lccccccc}
	\toprule
	    \textbf{Nome}  & \multicolumn{6}{c}{\textbf{Ore per ruolo}} & \textbf{Ore totali} \\
	     & Amm. & Ana. & Pgt. & Pgr. & Res. & Ver. & \\
	    \midrule
	    	Enrico Rotundo & 0 (+0) & 4 (+0) & 16 (+2) & 0 (+0) & 0 (+0) & 16 (+0) & 36 (+2) \\
	Federico Poli & 4 (+0) & 0 (+0) & 28 (+2) & 0 (+0) & 4 (+0) & 4 (+0) & 40 (+2) \\
	Giacomo Fornari & 8 (+0) & 2 (-0) & 8 (+2) & 0 (+0) & 4 (-2) & 22 (+1) & 44 (+1) \\
	Gianluca Donato & 6 (+0) & 8 (+0) & 14 (+2) & 0 (+0) & 4 (+0) & 4 (-1) & 36 (+1) \\
	Luca De Franceschi & 6 (-4) & 0 (+0) & 16 (+2) & 0 (+0) & 6 (-1) & 10 (-4) & 38 (-7) \\
	Nicolò Tresoldi & 10 (+0) & 0 (+0) & 8 (+2) & 0 (+0) & 8 (+0) & 5 (+1) & 31 (+3) \\
	Serena Girardi & 8 (+0) & 10 (-2) & 18 (+0) & 0 (+0) & 0 (+0) & 14 (-2) & 50 (-4) \\

	    \bottomrule
	\end{tabular}
	\caption{Differenza preventivo consuntivo per componente, periodo di Progettazione Architetturale}
	\end{table}
	\end{center}


	\subsubsection{Prospetto economico}

	Nel periodo di Progettazione architetturale le ore per ogni ruolo sono state cosi suddivise:

	\begin{table}[H]
	\centering
	\begin{tabular}{ l c c }
		\textbf{Ruolo} & \textbf{Ore} & \textbf{Costo} \\
		\hline
			Amministratore & 48 & 960 € \\
	Analista & 24 & 600 € \\
	Progettista & 108 & 2376 € \\
	Programmatore & 0 & 0 € \\
	Responsabile & 26 & 780 € \\
	Verificatore & 72 & 1080 € \\
\hline
	Totale & 278 & 5796 € \\
\hline

	\end{tabular}
	\caption{Ore e costo per ruolo, periodo di Progettazione architetturale}
	\end{table}

	Il seguente grafico mostra il peso del costo di ogni ruolo in questo periodo.

	\begin{figure}[H]
	\begin{tikzpicture}
		\pie[text=legend, color={amministratore, analista, progettista, programmatore, responsabile, verificatore}]{14.7/Amministratore, 10.5/Analista, 41.5/Progettista, ./Programmatore, 13.6/Responsabile, 19.7/Verificatore}

	\end{tikzpicture}
	\caption{Costo per ruolo, periodo di Progettazione architetturale}
	\end{figure}

	La tabella sottostante riporta le ore preventivate e le ore effettivamente impiegate (riportate tra parentesi) per ciascun ruolo presente nel periodo di \textit{Progettazione architetturale}.

	\begin{table}[H]
	\centering
	\begin{tabular}{lccccccc}
	\toprule
	    \textbf{Ruolo}  & \textbf{Ore} & \textbf{Costo} \\
	    \midrule
	    	Amministratore & 41 (+4) & 820 (+77) € \\
	Analista & 24 (+2) & 600 (+54) € \\
	Progettista & 108 (-12) & 2376 (-268) € \\
	Programmatore & 0 (+0) & 0 (+0) € \\
	Responsabile & 26 (+3) & 780 (+90) € \\
	Verificatore & 78 (+8) & 1170 (+124) € \\
\hline
\textbf{Totale consuntivo} & +272 & +5669 € \\
\textbf{Totale preventivo} & +277 & +5746 € \\
\textbf{Differenza dei totali} & +5 & +77 € \\

	    \bottomrule
	\end{tabular}
	\caption{Differenza preventivo consuntivo per ruolo, periodo di Progettazione Architetturale}
	\end{table}

	Viene di seguito incluso il grafico che illustra la differenza tra ore preventivate e ore effettivamente impiegate per ciascun ruolo nel periodo di \textit{Progettazione architetturale}.

	\begin{figure}[H]
	\centering
	\begin{tikzpicture}
	\begin{axis}[stileRuoli]
		\addplot+[color=Pianificate] plotcoordinates {(Amministratore,42.0)(Analista,24)(Progettista,108)(Programmatore,0)(Responsabile,26)(Verificatore,75.0)};
\addplot+[color=Consumate] plotcoordinates {(Amministratore,38.1666666667)(Analista,21.8333333333)(Progettista,120.166666667)(Programmatore,0)(Responsabile,23.0)(Verificatore,69.75)};

		\legend{preventivate,consumate}
	\end{axis}
	\end{tikzpicture}
	\caption{Differenza preventivo consuntivo per ruolo, periodo di progettazione architetturale}
	\end{figure}


\pagebreak
\subsection{Progettazione di dettaglio e codifica}

	
	\subsubsection{Suddivisione del lavoro}

	Nel periodo di Progettazione di dettaglio e codifica ciascun componente dovrà rivestire i seguenti ruoli:

	\begin{table}[H]
	\centering
	\begin{tabular}{lccccccc}
	\toprule 
	    \textbf{Nome}  & \multicolumn{6}{c}{\textbf{Ore per ruolo}} & \textbf{Ore totali}\\
	     & Amm. & Ana. & Pgt. & Pgr. & Res. & Ver. \\
	    \midrule
	    	Enrico Rotundo & 10 & 0 & 14 & 18 & 0 & 0 & 42 \\
	Federico Poli & 10 & 4 & 0 & 18 & 0 & 8 & 40 \\
	Giacomo Fornari & 0 & 4 & 14 & 18 & 0 & 4 & 40 \\
	Gianluca Donato & 14 & 0 & 10 & 18 & 0 & 2 & 44 \\
	Luca De Franceschi & 0 & 0 & 4 & 18 & 0 & 16 & 38 \\
	Nicolò Tresoldi & 4 & 0 & 16 & 18 & 0 & 14 & 52 \\
	Serena Girardi & 0 & 0 & 10 & 18 & 0 & 2 & 30 \\

	    \bottomrule
	\end{tabular}
	\caption{Ore per componente, periodo di Progettazione di dettaglio e codifica}
	\end{table}

	I valori in tabella sono riassunti nel seguente grafico: \\ \\ \\

	\begin{figure}[H]
	\begin{tikzpicture}
	\begin{axis}
			Enrico Rotundo & 6 & 0 & 26 & 20 & 0 & 0 & 52 \\
	Federico Poli & 10 & 4 & 10 & 16 & 0 & 14 & 54 \\
	Giacomo Fornari & 6 & 4 & 28 & 12 & 0 & 8 & 58 \\
	Gianluca Donato & 4 & 0 & 20 & 20 & 0 & 8 & 52 \\
	Luca De Franceschi & 4 & 0 & 10 & 24 & 0 & 22 & 60 \\
	Nicolò Tresoldi & 8 & 0 & 10 & 20 & 0 & 18 & 56 \\
	Serena Girardi & 0 & 0 & 16 & 20 & 0 & 20 & 56 \\

		\legend{amministratore, analista, progettista, programmatore, responsabile, verificatore}
	\end{axis}
	\end{tikzpicture}
	\caption{Ore per componente, periodo di Progettazione di dettaglio e codifica}
	\end{figure}

	Il seguente grafico mostra il peso orario di ogni ruolo in questo periodo.

	\begin{figure}[H]
	\begin{tikzpicture}
		\pie[text=legend, color={amministratore, analista, progettista, programmatore, responsabile, verificatore}]{10.3/Amministratore, 2.9/Analista, 25.0/Progettista, 41.2/Programmatore, ./Responsabile, 20.6/Verificatore}

	\end{tikzpicture}
	\caption{Ore per ruolo, periodo di Progettazione di dettaglio e codifica}
	\end{figure}

	\subsubsection{Prospetto orario}

	La tabella sottostante riporta le ore preventivate e le ore effettivamente impiegate (riportate tra parentesi) per ciascun componente del gruppo \GroupName{}.

	\begin{center}
	\begin{table}[H]
	\begin{tabular}{lccccccc}
	\toprule
	    \textbf{Nome}  & \multicolumn{6}{c}{\textbf{Ore per ruolo}} & \textbf{Ore totali} \\
	     & Amm. & Ana. & Pgt. & Pgr. & Res. & Ver. & \\
	    \midrule
	  		Enrico Rotundo & 10 (+0) & 0 (+0) & 14 (+0) & 18 (+0) & 2 (+0) & 2 (+0) & 46 (+0) \\
	Federico Poli & 10 (+0) & 4 (+0) & 0 (+0) & 18 (+0) & 0 (+0) & 14 (+0) & 46 (+0) \\
	Giacomo Fornari & 4 (+0) & 0 (+0) & 14 (+0) & 18 (+0) & 0 (+0) & 4 (+0) & 40 (+0) \\
	Gianluca Donato & 14 (+0) & 0 (+0) & 10 (+0) & 18 (+0) & 0 (+0) & 2 (+0) & 44 (+0) \\
	Luca De Franceschi & 0 (+0) & 0 (+0) & 4 (+0) & 18 (+0) & 2 (+0) & 16 (+0) & 40 (+0) \\
	Nicolò Tresoldi & 4 (+0) & 4 (+0) & 14 (+0) & 18 (+0) & 0 (+0) & 14 (+0) & 54 (+0) \\
	Serena Girardi & 0 (+0) & 0 (+0) & 10 (+0) & 18 (+0) & 0 (+0) & 2 (+0) & 30 (+0) \\

	    \bottomrule
	\end{tabular}
	\caption{Differenza preventivo consuntivo per componente, periodo di Progettazione di dettaglio e codifica}
	\end{table}
	\end{center}


	\subsubsection{Prospetto economico}

	Nel periodo di Progettazione di dettaglio e codifica le ore per ogni ruolo sono state cosi suddivise:

	\begin{table}[H]
	\centering
	\begin{tabular}{ l c c }
		\textbf{Ruolo} & \textbf{Ore} & \textbf{Costo} \\
		\hline
			Amministratore & 38 & 760 € \\
	Analista & 8 & 200 € \\
	Progettista & 120 & 2640 € \\
	Programmatore & 132 & 1980 € \\
	Responsabile & 0 & 0 € \\
	Verificatore & 90 & 1350 € \\
\hline
	Totale & 388 & 6930 € \\
\hline

	\end{tabular}
	\caption{Ore e costo per ruolo, periodo di Progettazione di dettaglio e codifica}
	\end{table}

	Il seguente grafico mostra il peso del costo di ogni ruolo in questo periodo.

	\begin{figure}[H]
	\begin{tikzpicture}
		\pie[text=legend, color={amministratore, analista, progettista, programmatore, responsabile, verificatore}]{15.2/Amministratore, 3.8/Analista, 29.2/Progettista, 39.3/Programmatore, ./Responsabile, 12.5/Verificatore}

	\end{tikzpicture}
	\caption{Costo per ruolo, periodo di Progettazione di dettaglio e codifica}
	\end{figure}

	La tabella sottostante riporta le ore preventivate e le ore effettivamente impiegate (riportate tra parentesi) per ciascun ruolo presente nel periodo di \textit{Progettazione di dettaglio e codifica}.

	\begin{table}[H]
	\centering
	\begin{tabular}{lccccccc}
	\toprule
	    \textbf{Ruolo}  & \textbf{Ore} & \textbf{Costo} \\
	    \midrule
	    	Amministratore & 43 (+0) & 860 (+0) € \\
	Analista & 8 (+0) & 200 (+0) € \\
	Progettista & 66 (+0) & 1452 (+0) € \\
	Programmatore & 126 (+0) & 1890 (+0) € \\
	Responsabile & 4 (+0) & 120 (+0) € \\
	Verificatore & 49 (-4) & 735 (-60) € \\
\hline
\textbf{Totale consuntivo} & +300 & +5317 € \\
\textbf{Totale preventivo} & +296 & +5257 € \\
\textbf{Differenza dei totali} & -4 & -60 € \\

	    \bottomrule
	\end{tabular}
	\caption{Differenza preventivo consuntivo per ruolo, periodo di Progettazione di dettaglio e codifica}
	\end{table}

	Viene di seguito incluso il grafico che illustra la differenza tra ore preventivate e ore effettivamente impiegate per ciascun ruolo nel periodo di \textit{Progettazione di dettaglio e codifica}.

	\begin{figure}[H]
	\centering
	\begin{tikzpicture}
	\begin{axis}[stileRuoli]
		\addplot+[color=Pianificate] plotcoordinates {(Amministratore,42)(Analista,8)(Progettista,66)(Programmatore,126)(Responsabile,4)(Verificatore,54)};
\addplot+[color=Consumate] plotcoordinates {(Amministratore,42)(Analista,8)(Progettista,66)(Programmatore,126)(Responsabile,4)(Verificatore,54)};

		\legend{preventivate,consumate}
	\end{axis}
	\end{tikzpicture}
	\caption{Differenza preventivo consuntivo per ruolo, periodo di progettazione di dettaglio e codifica}
	\end{figure}


\pagebreak
\subsection{Validazione}

	
	\subsubsection{Suddivisione del lavoro}

	Nel periodo di Validazione ciascun componente dovrà rivestire i seguenti ruoli:

	\begin{table}[H]
	\centering
	\begin{tabular}{lccccccc}
	\toprule 
	    \textbf{Nome}  & \multicolumn{6}{c}{\textbf{Ore per ruolo}} & \textbf{Ore totali}\\
	    & Amm. & Ana. & Pgt. & Pgr. & Res. & Ver. \\
	    \midrule
	    	Enrico Rotundo & 4 & 0 & 0 & 0 & 0 & 16 & 20 \\
	Federico Poli & 0 & 0 & 0 & 0 & 2 & 16 & 18 \\
	Giacomo Fornari & 4 & 0 & 0 & 0 & 0 & 16 & 20 \\
	Gianluca Donato & 2 & 0 & 8 & 0 & 0 & 10 & 20 \\
	Luca De Franceschi & 0 & 0 & 8 & 8 & 0 & 4 & 20 \\
	Nicolò Tresoldi & 0 & 0 & 0 & 0 & 0 & 19 & 19 \\
	Serena Girardi & 0 & 0 & 0 & 10 & 2 & 8 & 20 \\

	    \bottomrule
	\end{tabular}
	\caption{Ore per componente, periodo di Validazione}
	\end{table}

	I valori in tabella sono riassunti nel seguente grafico: \\ 

	\begin{figure}[H]
	\begin{tikzpicture}
	\begin{axis}
		\addplot+[color=Amministratore] plotcoordinates{(Enrico Rotundo,4)(Federico Poli,0)(Giacomo Fornari,4)(Gianluca Donato,4)(Luca De Franceschi,4)(Nicolò Tresoldi,0)(Serena Girardi,0)};
\addplot+[color=Analista] plotcoordinates{(Enrico Rotundo,0)(Federico Poli,0)(Giacomo Fornari,0)(Gianluca Donato,0)(Luca De Franceschi,0)(Nicolò Tresoldi,0)(Serena Girardi,0)};
\addplot+[color=Progettista] plotcoordinates{(Enrico Rotundo,0)(Federico Poli,0)(Giacomo Fornari,0)(Gianluca Donato,4)(Luca De Franceschi,4)(Nicolò Tresoldi,0)(Serena Girardi,0)};
\addplot+[color=Programmatore] plotcoordinates{(Enrico Rotundo,0)(Federico Poli,0)(Giacomo Fornari,0)(Gianluca Donato,0)(Luca De Franceschi,0)(Nicolò Tresoldi,0)(Serena Girardi,0)};
\addplot+[color=Responsabile] plotcoordinates{(Enrico Rotundo,0)(Federico Poli,4)(Giacomo Fornari,0)(Gianluca Donato,0)(Luca De Franceschi,0)(Nicolò Tresoldi,0)(Serena Girardi,4)};
\addplot+[color=Verificatore] plotcoordinates{(Enrico Rotundo,14)(Federico Poli,14)(Giacomo Fornari,14)(Gianluca Donato,10)(Luca De Franceschi,14)(Nicolò Tresoldi,14)(Serena Girardi,18)};

		\legend{amministratore, analista, progettista, programmatore, responsabile, verificatore}
	\end{axis}
	\end{tikzpicture}
	\caption{Ore per componente, periodo di validazione}
	\end{figure}

	Il seguente grafico mostra il peso orario di ogni ruolo in questo periodo.

	\begin{figure}[H]
	\begin{tikzpicture}
		\pie[text=legend, color={amministratore, analista, progettista, programmatore, responsabile, verificatore}]{11.6/Amministratore, ./Analista, 5.8/Progettista, ./Programmatore, 5.8/Responsabile, 76.8/Verificatore}

	\end{tikzpicture}\caption{Ore per ruolo, periodo di Validazione}
	\end{figure}

	\subsubsection{Prospetto orario}

	La tabella sottostante riporta le ore preventivate e le ore effettivamente impiegate (riportate tra parentesi) per ciascun componente del gruppo \GroupName{}.

	\begin{center}
	\begin{table}[H]
	\begin{tabular}{lccccccc}
	\toprule
	    \textbf{Nome}  & \multicolumn{6}{c}{\textbf{Ore per ruolo}} & \textbf{Ore totali} \\
	     & Amm. & Ana. & Pgt. & Pgr. & Res. & Ver. & \\
	    \midrule
	    	Enrico Rotundo & 4 (+0) & 0 (+0) & 0 (+0) & 0 (+0) & 0 (+0) & 14 (+0) & 18 (+0) \\
	Federico Poli & 0 (+0) & 0 (+0) & 0 (+0) & 0 (+0) & 2 (+0) & 14 (+0) & 16 (+0) \\
	Giacomo Fornari & 4 (+0) & 0 (+0) & 0 (+0) & 0 (+0) & 0 (+0) & 16 (+0) & 20 (+0) \\
	Gianluca Donato & 2 (+0) & 0 (+0) & 8 (+0) & 0 (+0) & 0 (+0) & 10 (+0) & 20 (+0) \\
	Luca De Franceschi & 0 (+0) & 0 (+0) & 8 (+0) & 0 (+0) & 0 (+0) & 14 (+0) & 22 (+0) \\
	Nicolò Tresoldi & 0 (+0) & 0 (+0) & 0 (+0) & 0 (+0) & 0 (+0) & 19 (+0) & 19 (+0) \\
	Serena Girardi & 0 (+0) & 0 (+0) & 0 (+0) & 0 (+0) & 2 (+0) & 18 (+0) & 20 (+0) \\

	    \bottomrule
	\end{tabular}
	\caption{Differenza preventivo consuntivo per componente, periodo di Validazione}
	\end{table}
	\end{center}


	\subsubsection{Prospetto economico}

	Nel periodo di Validazione le ore per ogni ruolo sono state cosi suddivise:

	\begin{table}[H]
	\centering
	\begin{tabular}{ l c c }
		\textbf{Ruolo} & \textbf{Ore} & \textbf{Costo} \\
			Amministratore & 16 & 320 € \\
	Analista & 0 & 0 € \\
	Progettista & 8 & 176 € \\
	Programmatore & 0 & 0 € \\
	Responsabile & 8 & 240 € \\
	Verificatore & 84 & 1260 € \\
\hline
	Totale & 116 & 1996 € \\
\hline

	\end{tabular}
	\caption{Ore e costo per ruolo, periodo di Validazione}
	\end{table}

	Il seguente grafico mostra il peso del costo di ogni ruolo in questo periodo.

	\begin{figure}[H]
	\begin{tikzpicture}
		\pie[text=legend, color={amministratore, analista, progettista, programmatore, responsabile, verificatore}]{7.3/Amministratore, ./Analista, 16.1/Progettista, ./Programmatore, 5.5/Responsabile, 71.2/Verificatore}

	\end{tikzpicture}
	\caption{Costo per ruolo, periodo di Validazione}
	\end{figure}

	La tabella sottostante riporta le ore preventivate e le ore effettivamente impiegate (riportate tra parentesi) per ciascun ruolo presente nel periodo di \textit{Validazione}.

	\begin{table}[H]
	\centering
	\begin{tabular}{lccccccc}
	\toprule
	    \textbf{Ruolo}  & \textbf{Ore} & \textbf{Costo} \\
	    \midrule
	    	Amministratore & 8 (+0) & 160 (+0) € \\
	Analista & 0 (+0) & 0 (+0) € \\
	Progettista & 16 (+0) & 352 (+0) € \\
	Programmatore & 0 (+0) & 0 (+0) € \\
	Responsabile & 4 (+0) & 120 (+0) € \\
	Verificatore & 104 (+0) & 1560 (+0) € \\
\hline
\textbf{Totale consuntivo} & +132 & +2192 € \\
\textbf{Totale preventivo} & +132 & +2192 € \\
\textbf{Differenza dei totali} & +0 & +0 € \\

	    \bottomrule
	\end{tabular}
	\caption{Differenza preventivo consuntivo per ruolo, periodo di Validazione}
	\end{table}

	Viene di seguito incluso il grafico che illustra la differenza tra ore preventivate e ore effettivamente impiegate per ciascun ruolo nel periodo di \textit{Validazione}.

	\begin{figure}[H]
	\centering
	\begin{tikzpicture}
	\begin{axis}[stileRuoli]
		\addplot+[color=Pianificate] plotcoordinates {(Amministratore,10.0)(Analista,0)(Progettista,16)(Programmatore,18)(Responsabile,4)(Verificatore,89.0)};
\addplot+[color=Consumate] plotcoordinates {(Amministratore,10.0)(Analista,0)(Progettista,16)(Programmatore,18)(Responsabile,4)(Verificatore,89.0)};

		\legend{preventivate,consumate}
	\end{axis}
	\end{tikzpicture}
	\caption{Differenza preventivo consuntivo per ruolo, periodo di Validazione}
	\end{figure}