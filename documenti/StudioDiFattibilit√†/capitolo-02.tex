\section{Capitolato C1 - MaaP}

\subsection{Descrizione}
Il capitolato scelto prevede la realizzazione di un \emph{framework} per generare interfacce web di amministrazione dei dati di \emph{business}. In particolare, l'amministrazione dei dati deve essere disponibile a livello di \textbf{database}, nel quale vengono effettuate operazioni direttamente sugli oggetti che lo rappresentano (tabelle, indici, viste) in modo tale da permettere un accesso veloce e consistente ai dati. Questa tipologia di amministrazione non si preoccupa di interpretare le informazioni in dati di \emph{business}, ma si limita a interagire in modo agnostico con le entità del database. Inoltre, deve essere possibile l'amministrazione a livello di \textbf{dati di business}, in cui vengono effettuate operazioni su una o più di tali entità, le quali vengono interpretate nel modello di \emph{business} richiesto. La realizzazione delle pagine web di visualizzazione deve essere svolta in maniera semplice e veloce da parte dello sviluppatore, e le modalità di fruizione delle pagine generate devono essere adeguate ad un esperto di business.

\subsection{Studio del dominio}

	\subsubsection{Dominio tecnologico}
	\begin{itemize}
		\item \textbf{Node.js} per la realizzazione della componente server
		\item \textbf{Express} per la realizzazione dell’infrastruttura della web application generata
		\item \textbf{Mongoose.js} per l’interfacciamento con il database
		\item \textbf{MongoDB} per il recupero dei dati
		\item conoscenza di framework per la componente front-end (i.e. Angular.js, Ember.js)
		\item conoscenze nella definizione di linguaggi astratti DSL per la generazione delle pagine da parte dello sviluppatore.
	\end{itemize}

	\subsubsection{Dominio applicativo}

\subsection{Valutazione}
\label{StudioDiFattibilità}
	Il capitolato presenta aspetti positivi che hanno determinato la sua scelta:
	\begin{itemize}
	\item Apprendimento di tecnologie innovative che portano un bagaglio di conoscenze ritenuto importante dato il grande uso di 	          		quest'ultime nella panoramica delle tecnologie presenti attualmente nel mercato.
	\item Interesse del gruppo a vedere la propria applicazione dare vita ad una community dato che non esiste attualmente, con lo 	       stack tecnologico proposto, un applicativo simile.
	\item Requisiti richiesti dal proponente sono ben delineati.
	\end{itemize}
Similmente, il gruppo ha trovato aspetti negativi:
	\begin{itemize}
	\item Le tecnologie utilizzate nello sviluppo del progetto non sono conosciute da nessun membro del gruppo \GroupName e
	quindi richiederanno un tempo di formazione per il loro apprendimento considerevole.
	\item La mole di lavoro per lo sviluppo del progetto al gruppo sembra notevole.
	\end{itemize}
	\subsubsection{Potenziali criticità}
	
	
