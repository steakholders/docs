\section{Introduzione}


\subsection{Scopo del documento}
Il seguente documento vuole descrivere la progettazione di dettaglio definita per il progetto \ProjectName{}. Il documento si basa sulla \SpecificaTecnica{}. I \textit{programmatori} si serviranno di tale documento per procedere con le attività di codifica.

\subsection{Scopo del prodotto}
\ScopoDelProdotto 

\subsection{Glossario}
Ogni occorrenza di termini tecnici, di dominio e gli acronimi sono marcati con una ``G'' in pedice e riportati nel documento \Glossario{}.

\subsection{Riferimenti}
Vengono elencanti i riferimenti su cui si è basata la definizione della progettazione di dettaglio.
	\subsubsection{Normativi}
		\begin{itemize}
  			\item \textbf{Norme di Progetto:}  \NormeDiProgetto;
			\item \textbf{Capitolato d'appalto C1:} \\ \url{http://www.math.unipd.it/~tullio/IS-1/2013/Progetto/C1.pdf};
			\item \textbf{Specifica Tecnica:} \SpecificaTecnica{}.
		\end{itemize}
	\subsubsection{Informativi}
		\begin{itemize}
			\item Dall’idea al codice con UML 2 - L. Baresi, L. Lavazza, M. Pianciamore - 1a edizione;
			\item Design Patterns - Erich Gamma, Richard Helm, Ralph Johnson, John Vlissides - 1a edizione italiana (2008);
			\item Node.js - Marc Wandschneider - 1a edizione (2013).
		\end{itemize}
	

