\section{Requisiti }
I requisiti funzionali, prestazionali, di qualità e di vincolo individuati sono riportati nelle seguenti tabelle. Ogni requisito è identificato da un codice univoco.
Viene inoltre indicato se si tratta di un requisito fondamentale, desiderabile o facoltativo, una sua descrizione e il caso d'uso da cui è stato individuato. 

Ogni requisito è identificato da un codice, che segue il seguente formalismo:
\begin{center}
    \code{R\{X\}\{Y\}\{Z\} \{Gerarchia\}}
\end{center}

Dove:
\begin{itemize}
 \item \textbf{X} corrisponde al sistema di riferimento e può assumere i seguenti valori:
    \begin{itemize}
     \item[] \textbf{A} = Applicazione \glossario{Maap};
     \item[] \textbf{F} = \glossario{Framework MaaP};
     \item[] \textbf{S} = \glossario{MaaS}.
    \end{itemize}

 \item \textbf{Y} corrisponde alla tipologia del requisito e può assumere i seguenti valori:
    \begin{itemize}
     \item[] \textbf{1} = Funzionale;
     \item[] \textbf{2} = Prestazionale;
     \item[] \textbf{3} = Di Qualità;
     \item[] \textbf{4} = Vincolo.
    \end{itemize}

 \item \textbf{Z} corrisponde alla priorità del requisito e può assumere i seguenti valori:
    \begin{itemize}
     \item[] \textbf{O} = Obbligatorio
     \item[] \textbf{D} = Desiderabile
     \item[] \textbf{F} = Facoltativo o Opzionale
    \end{itemize}

 \item \textbf{Gerarchia} identifica la relazione gerarchica che c'è tra i requisiti di uno stesso tipo. C'è quindi una struttura gerarchica per ogni tipologia di requisito.
\end{itemize}

\subsection{Requisiti funzionali }

    %Tabella 
      \begin{center}
      \bgroup
      \def\arraystretch{1.8}
      \begin{longtable}{ | l | p{2cm} | p{5cm} | p{1.7cm} |}
    
      \cellcolor[gray]{0.9} \textbf{Requisito} & \cellcolor[gray]{0.9} \textbf{Tipologia} 
      & \cellcolor[gray]{0.9} \textbf{Descrizione} & \cellcolor[gray]{0.9} \textbf{Fonti} \\ \hline
      
        RA1O 1 & Funzionale \newline  Obbligatorio  & Il sistema permette all'utente non autenticato di autenticarsi tramite la visualizzazione di una pagina web, la quale conterrà al suo interno i campi di testo necessari. &  UCU1 \newline  Capitolato \newline  \\ \hline      
        RA1O 1.1 & Funzionale \newline  Obbligatorio  & Il sistema prevede l'inserimento dell'indirizzo email per la verifica delle credenziali in un apposito campo di testo. &  UCU1.1 \newline  UCU1 \newline  Capitolato \newline  \\ \hline      
        RA1O 1.2 & Funzionale \newline  Obbligatorio  & Il sistema prevede l'inserimento di password per la verifica delle credenziali in un apposito campo di testo. &  UCU1.2 \newline  UCU1 \newline  Capitolato \newline  \\ \hline      
        RA1O 1.3  & Funzionale \newline  Obbligatorio  & Il sistema, tramite un database indipendente, ovvero separato da quello che contiene la Collection, provvede a verificare l'autenticità  di un utente tramite la verifica di email e password. &  Capitolato \newline  \\ \hline      
        RA1O 1.3.1 & Funzionale \newline  Obbligatorio  & Il sistema mette a disposizione la visualizzazione di una pagina di errore in caso di fallimento dell'autenticazione da parte dell'utente.
 &  UCU2 \newline  Interno \newline  \\ \hline      
        RA1O 1.3.2 & Funzionale \newline  Obbligatorio  & Il sistema, nel caso in cui l'autenticazione da parte dell'utente abbia avuto successo, reindirizza automaticamente l'utente sulla dashboard dell'applicazione.
 &  UCU8 \newline  Interno \newline  \\ \hline      
        RA1O 2  & Funzionale \newline  Obbligatorio  & Il sistema mette a disposizione dell'utente non autenticato la possibilità  di recuperare la propria password. &  UCU4 \newline  Capitolato \newline  UCU4.2 \newline  UCU4.1 \newline  UCU4.2.1 \newline  UCU4.1.1 \newline  \\ \hline      
        RA1O 2.1 & Funzionale \newline  Obbligatorio  & Il sistema permette il recupero password attraverso l'inserimento dell'utente non autenticato dell'email.
 &  UCU4 \newline  UCU4.1 \newline  UCU4.1.1 \newline  \\ \hline      
        RA1O 2.2 & Funzionale \newline  Obbligatorio  & Il sistema deve inviare un'email contenente un link attraverso il quale l'utente non autenticato può effettuare il reset della propria password. &  UCU4 \newline  UCU4.2 \newline  \\ \hline      
        RA1O 2.3 & Funzionale \newline  Obbligatorio  & L'applicazione \glossario{MaaP} deve permettere il reset della password ad un utente non autenticato inserendone una nuova che sostituirà la precedente come password di accesso all'applicazione. &  UCU4 \newline  UCU4.2 \newline  UCU4.2.1 \newline  \\ \hline      
        RA1O 4 & Funzionale \newline  Obbligatorio  & L'applicazione deve permettere all'utente la visualizzazione dell'insieme delle Collection presenti tramite un menu di navigazione e la selezione di una di esse. &  UCU9 \newline  Capitolato \newline  \\ \hline      
        RA1O 4.1  & Funzionale \newline  Obbligatorio  & L'applicazione deve permettere all'utente la visualizzazione di una Collection-index tramite una tabella le cui righe corrispondono ai document presenti nel database e le colonne ai loro attributi visualizzabili.  &  Capitolato \newline  UCU9 \newline  \\ \hline      
        RA1O 4.1.1  & Funzionale \newline  Obbligatorio  & Ogni riga della tabella corrispondente ad un Document deve avere una chiave selezionabile che rimanda alla corrispondente pagina show. &  Capitolato \newline  UCS3.3.1.4 \newline  \\ \hline      
        RA1O 5  & Funzionale \newline  Obbligatorio  & L'applicazione deve permettere all'utente la visualizzazione della Collection-show relativa ad un document tramite una pagina web che ne mostra gli attributi visualizzabili strutturati in una tabella.
 &  UCU9.1 \newline  Capitolato \newline  UCU9.1.2 \newline  UCU9.1.1 \newline  \\ \hline      
        RA1O 5.1 & Funzionale \newline  Obbligatorio  & L'applicazione deve permettere all'admin di poter editare ogni singolo attributo modificabile del documento dalla pagina show. &  UCU9.1.3 \newline  UCU9.1 \newline  Capitolato \newline  UCU9.1.5 \newline  UCU9.7 \newline  \\ \hline      
        RA1O 5.3 & Funzionale \newline  Obbligatorio  & L'applicazione deve permettere all'utente di poter eliminare dalla show-page il Document selezionato. &  UCU9.1.4 \newline  Capitolato \newline  \\ \hline      
        RA1O 6 & Funzionale \newline  Obbligatorio  & L'applicazione deve fornire all'admin una pagina di amministrazione in cui visualizzare la Collection di tutti gli utenti registrati all'applicazione.
 &  UCU11 \newline  Interno \newline  \\ \hline      
        RA1O 6.1  & Funzionale \newline  Obbligatorio  & L'applicazione deve fornire all'admin la possibilità  di creare un nuovo utente dalla pagina di amministrazione. &  UCU11.1 \newline  UCU11 \newline  Capitolato \newline  \\ \hline      
        RA1O 6.1.1  & Funzionale \newline  Obbligatorio  & L'applicazione deve fornire all'admin una pagina di creazione di un nuovo utente. &  UCU11.1 \newline  Capitolato \newline  \\ \hline      
        RA1O 6.1.1.1  & Funzionale \newline  Obbligatorio  & L'applicazione deve permettere all'admin di inserire l'indirizzo email del nuovo utente in un apposito campo di testo presente all'interno della pagina di creazione di un nuovo utente.
 &  UCU11.1.1 \newline  UCU11.1 \newline  \\ \hline      
        RA1O 6.1.1.2 & Funzionale \newline  Obbligatorio  & L'applicazione deve permettere all'admin di inserire la password del nuovo utente in un apposito campo di testo presente all'interno della pagina di creazione di un nuovo utente. &  UCU11.1.2 \newline  UCU11.1 \newline  \\ \hline      
        RA1O 6.1.1.3  & Funzionale \newline  Obbligatorio  & L'applicazione deve permettere all'admin di inserire il "livello utente" del nuovo utente tramite una combo-box presente all'interno della pagina di creazione di un nuovo utente. &  UCU11.1.3 \newline  UCU11.1 \newline  Capitolato \newline  \\ \hline      
        RA1O 6.1.2  & Funzionale \newline  Obbligatorio  & L'applicazione deve prelevare tutti i dati inseriti dall'admin nella pagina di creazione di un nuovo utente ed inviarli al database delle credenziali, il quale provvederà  all'inserimento del nuovo record.
 &  UCU11.1 \newline  Capitolato \newline  Interno \newline  \\ \hline      
        RA1O 6.1.3  & Funzionale \newline  Obbligatorio  & L'applicazione deve visualizzare un messaggio d'errore nel caso in cui l'admin non abbia compilato correttamente i campi presenti all'interno della pagina di creazione di un nuovo utente.
 &  UCU11.2 \newline  \\ \hline      
        RA1O 6.2  & Funzionale \newline  Obbligatorio  & L'applicazione deve fornire all'admin la possibilità di selezionare un utente dalla index-page e visualizzare la sua relativa show-page.
 &  UCU11.1 \newline  UCU11.3 \newline  UCU11 \newline  \\ \hline      
        RA1O 6.2.1 & Funzionale \newline  Obbligatorio  & L'applicazione deve fornire all'admin la possibilità di elevare l'utente normale selezionato al livello "admin" dalla show-page relativa. &  UCU11.3.1 \newline  UCU11.3 \newline  Capitolato \newline  \\ \hline      
        RA1O 6.2.2 & Funzionale \newline  Obbligatorio  & L'applicazione deve fornire all'admin la possibilità di declassare l'admin selezionato a livello di utente normale dalla show-page relativa. &  UCU11.3.2 \newline  UCU11.3 \newline  \\ \hline      
        RA1O 6.2.3 & Funzionale \newline  Obbligatorio  & L'applicazione deve fornire all'admin la possibilità di modificare l'attributo email dell'utente selezionato dalla relativa show-page. &  UCU11.3 \newline  Verbale-2013/12/18 \newline  \\ \hline      
        RA1O 6.2.4 & Funzionale \newline  Obbligatorio  & L'applicazione deve fornire all'admin la possibilità di modificare l'attributo password dell'utente selezionato dalla relativa show-page. &  UCU11.3 \newline  Verbale-2013/12/18 \newline  \\ \hline      
        RA1O 6.2.5 & Funzionale \newline  Obbligatorio  & L'applicazione deve fornire all'admin la possibilità di eliminare l'utente visualizzato nella \glossario{show-page}. &  UCU11.3.3 \newline  \\ \hline      
        RA1D 3  & Funzionale \newline  Desiderabile  & Il sistema deve mettere a disposizione dell'utente autenticato la visualizzazione della dashboard. &  UCU8 \newline  Interno \newline  \\ \hline      
        RA1D 4.1.2  & Funzionale \newline  Desiderabile  & L'utente può decidere di cambiare l'ordine di visualizzazione dei document selezionando l'intestazione delle colonne aventi la \glossario{proprietà  sortable}.
 &  Interno \newline  \\ \hline      
        RA1D 4.1.3  & Funzionale \newline  Desiderabile  & L'applicazione può mettere a disposizione dell'utente per ogni riga della tabella una checkbox per consentire la selezione multipla.
 &  Interno \newline  \\ \hline      
        RA1D 4.1.4  & Funzionale \newline  Desiderabile  & L'applicazione può mettere a disposizione dell'admin un link di selezione rapida per l'eliminazione di un documento.
 &  UCU9.5 \newline  UCU9 \newline  \\ \hline      
        RA1D 4.1.5 & Funzionale \newline  Desiderabile  & L'applicazione deve permette all'admin la modifica di un document presente nella collection-index. &  UCU9.4 \newline  Capitolato \newline  \\ \hline      
        RA1D 4.2  & Funzionale \newline  Desiderabile  & L'applicazione dà la possibilità di impostare dei filtri personalizzati secondo determinati attributi per visualizzare un sottoinsieme di Document. &  UCU9.2 \newline  UCU9 \newline  UCS3.3.1.7 \newline  Interno \newline  \\ \hline      
        RA1D 11 & Funzionale \newline  Desiderabile  & Deve essere possibile da parte di un utente la registrazione all'Applicazione MaaP. &  UCU5 \newline  Verbale-2013/12/05 \newline  UCU6 \newline  \\ \hline      
        RA1D 12 & Funzionale \newline  Desiderabile  & L'utente autenticato nell'applicazione deve poter eseguire il logout.  &  UCU3 \newline  \\ \hline      
        RA1D 13 & Funzionale \newline  Desiderabile  & L'utente con credenziali d'accesso all'applicazione deve poterle modificare all'interno dell'applicazione tramite pagina apposita. &  UCU10 \newline  \\ \hline      
        RA1D 13.1 & Funzionale \newline  Desiderabile  & L'utente deve poter modificare la propria email con cui ha accesso all'applicazione MaaP. &  UCU10 \newline  \\ \hline      
        RA1D 13.2 & Funzionale \newline  Desiderabile  & L'utente deve poter modificare la password di accesso all'applicazione. &  UCU10 \newline  UCU10.1 \newline  \\ \hline      
        RA1F 4.3 & Funzionale \newline  Facoltativo  & L'applicazione deve permettere all'amministratore di creare un nuovo Document all'interno della base di dati. &  UCU9.6 \newline  Capitolato \newline  \\ \hline      
        RA1F 4.4 & Funzionale \newline  Facoltativo  & L'applicazione deve permettere all'utente di poter eseguire un'azione personalizzata tramite l'esecuzione di un pulsante. &  UCU9.3 \newline  Capitolato \newline  \\ \hline      
        RA1F 5.2 & Funzionale \newline  Facoltativo  & L'applicazione deve permettere all'utente di poter eseguire un'azione personalizzata tramite l'esecuzione di un pulsante. &  UCU9.3 \newline  Capitolato \newline  \\ \hline      
        RA1F 15 & Funzionale \newline  Facoltativo  & L'applicazione MaaP deve mettere a disposizione dell'admin una pagina di gestione degli indici. &  UCU7 \newline  Capitolato \newline  \\ \hline      
        RA1F 15.1 & Funzionale \newline  Facoltativo  & L'applicazione MaaP deve mettere a disposizione dell'admin la visualizzazione degli indici in base alle query più richieste dall'applicazione. &  UCU7.1 \newline  Capitolato \newline  \\ \hline      
        RA1F 15.2 & Funzionale \newline  Facoltativo  & L'applicazione MaaP deve permettere all'admin di aggiungere gli indici in base ai suggerimenti forniti. &  UCU7.2 \newline  Capitolato \newline  \\ \hline      
        RA1F 15.3 & Funzionale \newline  Facoltativo  & L'applicazione MaaP deve permettere all'admin di rimuovere gli indici in base ai suggerimenti forniti. &  UCU7.3 \newline  Interno \newline  \\ \hline      
        RF1O 7 & Funzionale \newline  Obbligatorio  & MaaP Framework deve rendere disponibile allo sviluppatore un linguaggio astratto DSL di tipo testuale necessario per la generazione delle pagine. &  Capitolato \newline  \\ \hline      
        RF1O 8  & Funzionale \newline  Obbligatorio  & Maap Framework deve permettere allo sviluppatore di generare un nuovo progetto tramite linea di comando.
 &  UCS1 \newline  Capitolato \newline  \\ \hline      
        RF1O 8.1  & Funzionale \newline  Obbligatorio  & Maap Framework deve generare automaticamente lo scheletro dell'applicazione creata dallo sviluppatore.
 &  UCS1 \newline  Capitolato \newline  \\ \hline      
        RF1O 8.1.1  & Funzionale \newline  Obbligatorio  & Maap Framework deve automaticamente importare in un'apposita directory del progetto tutte le librerie necessarie al corretto funzionamento del sistema.
Librerie necessarie:
\begin{itemize}
\item Express v-3.4.8
\item MongoDB v-1.3.23
\item Mongoose v-3.8.4
\item Nel caso fossero necessarie ulteriori librerie è consigliata una versione uguale o maggiore rispetto a quella disponibile al 2014-01-01
\end{itemize}
 &  UCS1 \newline  Capitolato \newline  \\ \hline      
        RF1O 8.1.2 & Funzionale \newline  Obbligatorio  & Maap Framework deve automaticamente creare in un'apposita directory il file di configurazione di default dell'applicazione generata.
 &  UCS1 \newline  Capitolato \newline  \\ \hline      
        RF1O 8.1.3  & Funzionale \newline  Obbligatorio  & Maap Framework deve automaticamente creare il sistema di autenticazione per l'applicazione generata. &  UCS1 \newline  Capitolato \newline  \\ \hline      
        RF1O 8.1.4  & Funzionale \newline  Obbligatorio  & Maap Framework deve automaticamente creare le directory di descrizione delle pagine web.
 &  UCS1 \newline  Capitolato \newline  \\ \hline      
        RF1O 8.2  & Funzionale \newline  Obbligatorio  & Maap Framework deve automaticamente creare un account admin di default. &  UCS1 \newline  Verbale-2013/12/05 \newline  \\ \hline      
        RF1O 8.3 & Funzionale \newline  Facoltativo  & Il framework MaaP deve permettere allo sviluppatore di definire un namespace per l'applicazione generata. &  UCS3.3.5 \newline  Interno \newline  \\ \hline      
        RF1O 9  & Funzionale \newline  Obbligatorio  & Maap Framework deve permettere allo sviluppatore di configurare le Collection tramite DSL fornito.
 &  Capitolato \newline  UCS3 \newline  \\ \hline      
        RF1O 9.1  & Funzionale \newline  Obbligatorio  & Il DSL deve permettere allo sviluppatore di creare una pagina Collection-index.
 &  UCS3.1 \newline  UCS3.2 \newline  UCS3.4 \newline  UCS3 \newline  Capitolato \newline  \\ \hline      
        RF1O 9.1.1  & Funzionale \newline  Obbligatorio  & Il DSL deve permettere allo sviluppatore di poter definire una serie di attributi da visualizzare all'interno della pagina Collection-index.
 &  UCS3.3 \newline  Capitolato \newline  UCS3.3.1.1 \newline  \\ \hline      
        RF1O 9.1.2  & Funzionale \newline  Obbligatorio  & Il DSL deve permettere allo sviluppatore di poter definire un ordinamento di default (ordine alfanumerico) di visualizzazione dei document all'interno della pagina Collection-index. &  UCS3.3 \newline  UCS3.3.1.2 \newline  Capitolato \newline  \\ \hline      
        RF1O 9.1.3  & Funzionale \newline  Obbligatorio  & Il DSL deve permettere allo sviluppatore di poter definire un eventuale limite di elementi da visualizzare all'interno della pagina Collection-index.
 &  UCS3.3 \newline  UCS3.3.1.3 \newline  Capitolato \newline  \\ \hline      
        RF1O 9.1.4  & Funzionale \newline  Obbligatorio  & Il DSL deve permettere allo sviluppatore di poter definire quali attributi sono ordinabili all'interno della pagina Collection-index.
 &  UCS3.3.1.2 \newline  Capitolato \newline  \\ \hline      
        RF1O 9.1.5 & Funzionale \newline  Obbligatorio  & Il DSL deve permettere allo sviluppatore di definire la funzione populate per far si che una chiave riferisca ad un documento esterno.
 &  UCS3.3.1.6 \newline  UCS3.3.2.3 \newline  Capitolato \newline  \\ \hline      
        RF1O 9.1.6 & Funzionale \newline  Obbligatorio  & Il DSL deve permettere allo sviluppatore di definire delle query per creare la pagina Collection-index in base al risultato della loro estrazione.
 &  UCS3.3.1.5 \newline  Capitolato \newline  UCS3.3.1 \newline  \\ \hline      
        RF1O 9.1.7 & Funzionale \newline  Obbligatorio  & Il DSL deve permettere allo sviluppatore di definire delle trasformazioni sugli attributi da visualizzare. &  UCS3.3.1.8 \newline  Capitolato \newline  \\ \hline      
        RF1O 9.2 & Funzionale \newline  Obbligatorio  & Il DSL deve permettere allo sviluppatore di creare una pagina Collection-show. &  UCS3.1 \newline  UCS3.2 \newline  UCS3.4 \newline  Capitolato \newline  \\ \hline      
        RF1O 9.2.1  & Funzionale \newline  Obbligatorio  & Il DSL deve permettere allo sviluppatore di definire una serie di attributi visualizzabili all'interno della pagina Collection-show.
 &  UCS3.3.2 \newline  UCS3.3.2.1 \newline  Capitolato \newline  \\ \hline      
        RF1O 9.2.2  & Funzionale \newline  Obbligatorio  & Il DSL deve permettere allo sviluppatore definire gli attributi del Document come attributi innestati o array di Document tramite la funzione populate.
 &  UCS3.3.2.3 \newline  Verbale-2013/12/05 \newline  Capitolato \newline  \\ \hline      
        RF1O 9.2.3 & Funzionale \newline  Obbligatorio  & Lo sviluppatore deve aver la possibilità di personalizzare la show page definendone l'ordinamento degli attributi. &  UCS3.3.2 \newline  UCS3.3.2.2 \newline  Capitolato \newline  \\ \hline      
        RF1O 9.2.4 & Funzionale \newline  Obbligatorio  & Lo sviluppatore deve poter definire trasformazioni agli attributi per poi visualizzarli nella show-page. &  UCS3.3.2 \newline  UCS3.3.2.4 \newline  Capitolato \newline  \\ \hline      
        RF1O 9.3 & Funzionale \newline  Facoltativo  & Il framework MaaP deve permettere allo sviluppatore di cambiare il nome della Collection da visualizzare nel menu di navigazione. &  UCS3.3.3 \newline  Capitolato \newline  \\ \hline      
        RF1O 9.4 & Funzionale \newline  Obbligatorio  & Il framework MaaP deve permettere allo sviluppatore di modificare l'ordine di visualizzazione della Collection nel menu di navigazione. &  UCS3.3.4 \newline  Capitolato \newline  \\ \hline      
        RF1O 14 & Funzionale \newline  Obbligatorio  & Il framework MaaP deve rendere possibile la configurazione dei database di cui dispone. &  UCS2 \newline  Interno \newline  \\ \hline      
        RF1O 14.1 & Funzionale \newline  Obbligatorio  & Il framework MaaP deve rendere possibile la configurazione dei database delle credenziali. &  UCS2.1 \newline  UCS2 \newline  Interno \newline  \\ \hline      
        RF1O 14.2 & Funzionale \newline  Obbligatorio  & Il framework MaaP deve rendere possibile la configurazione dei database delle Collection. &  UCS2 \newline  UCS2.2 \newline  Interno \newline  \\ \hline      
        RF1F 9.2.5 & Funzionale \newline  Facoltativo  & Lo sviluppatore deve poter personalizzare la show-page definendo delle operazioni personalizzate che l'utente potrà utilizzare tramite appositi pulsanti. &  UCS3.3.2 \newline  UCS3.3.2.5 \newline  Capitolato \newline  UCS3.3.1.9 \newline  \\ \hline      
        RF1F 14.3 & Funzionale \newline  Facoltativo  & Il framework MaaP deve rendere possibile la selezione di un namespace per un database se la funzione di namespace è abilitata. &  UCS2.3 \newline  Interno \newline  \\ \hline      
        RF1F 16 & Funzionale \newline  Facoltativo  & Il framework MaaP deve permettere allo sviluppatore di abilitare i namespace per l'applicazione creata. &  UCS4 \newline  Interno \newline  \\ \hline      
        RS1F 10 & Funzionale \newline  Facoltativo  & Il sistema MaaS deve mettere a disposizione il framework MaaP come servizio web. &  Capitolato \newline  UCM \newline  \\ \hline      
        RS1F 10.1 & Funzionale \newline  Facoltativo  & Il sistema MaaS deve permettere allo sviluppatore di scrivere una Collection tramite editor di testo presente nella pagina web. &  Capitolato \newline  UCM8 \newline  UCM8.2 \newline  UCM8.4 \newline  UCM8.1 \newline  \\ \hline      
        RS1F 10.2 & Funzionale \newline  Facoltativo  & Il sistema MaaS deve permettere all'utente di poter scrivere una Collection caricando un file prodotto dal framework MaaP. &  UCM8.4 \newline  UCM8.1 \newline  UCM8.3 \newline  UCM8 \newline  \\ \hline      
        RS1F 10.3 & Funzionale \newline  Facoltativo  & Il sistema MaaS deve permettere ad un utente non registrato di registrarsi al suo servizio. &  Interno \newline  UCM1 \newline  \\ \hline      
        RS1F 10.4 & Funzionale \newline  Facoltativo  & Il sistema MaaS deve assegnare automaticamente un namespace sul sistema al nuovo utente registrato. &  UCM1 \newline  Interno \newline  \\ \hline      
        RS1F 10.5 & Funzionale \newline  Facoltativo  & Il servizio MaaS deve visualizzare un messaggio d'errore nel caso in cui la registrazione fallisca a causa di credenziali già esistenti. &  UCM3 \newline  Interno \newline  \\ \hline      
        RS1F 10.6 & Funzionale \newline  Facoltativo  & Il servizio MaaS deve mettere a disposizione di un utente non autenticato la possibilità di effettuare il login al sistema. &  Interno \newline  UCM4 \newline  UCM1.1 \newline  UCM1.2 \newline  \\ \hline      
        RS1F 10.7 & Funzionale \newline  Facoltativo  & Il servizio MaaS deve visualizzare un messaggio d'errore nel caso in cui l'utente non autenticato abbia inserito credenziali errate nel sistema di login. &  Interno \newline  UCM5 \newline  \\ \hline      
        RS1F 10.8 & Funzionale \newline  Facoltativo  & Il sistema MaaS deve permettere ad un utente non autenticato di modificare il proprio profilo. &  UCM6 \newline  UCM6.1 \newline  Interno \newline  \\ \hline      
        RS1F 10.9 & Funzionale \newline  Facoltativo  & Il sistema MaaS deve permettere ad un utente non autenticato di eliminare il proprio account dal sistema. &  UCM7 \newline  Interno \newline  \\ \hline      
        RS1F 10.9.1 & Funzionale \newline  Facoltativo  & Il sistema \glossario{MaaS} deve provvedere all'eliminazione dei file di configurazione associati all'utente rimosso dal sistema. &  UCM9 \newline  UCM7 \newline  \\ \hline      
        RS1F 10.10 & Funzionale \newline  Facoltativo  & Il sistema MaaS deve permettere allo sviluppatore di eliminare una \glossario{Collection} esistente. &  UCM8.5 \newline  \\ \hline      
        RS4F 17 & Funzionale \newline  Desiderabile  & Il sistema MaaS deve verificare se i documenti creati rispettano i vincoli del database. &  Capitolato \newline  \\ \hline      
        RS4F 18 & Funzionale \newline  Facoltativo  & Il sistema MaaS deve salvare le Collection create dallo sviluppatore direttamente dentro la base di dati. &  Capitolato \newline  \\ \hline
      \end{longtable}
      \egroup
      \end{center}  
\clearpage

\subsection{Requisiti di qualità }

    %Tabella 
      \begin{center}
      \bgroup
      \def\arraystretch{1.8}
      \begin{longtable}{ | l | p{2cm} | p{5cm} | p{1.7cm} |}
    
      \cellcolor[gray]{0.9} \textbf{Requisito} & \cellcolor[gray]{0.9} \textbf{Tipologia} 
      & \cellcolor[gray]{0.9} \textbf{Descrizione} & \cellcolor[gray]{0.9} \textbf{Fonti} \\ \hline
      
        R3O 1 & Qualità \newline  Obbligatorio  & Devono essere prodotti e rilasciati manuali d'uso ed ogni altra documentazione tecnica necessaria per l’utilizzo del prodotto. &  Capitolato \newline  \\ \hline      
        R3O 2 & Qualità \newline  Obbligatorio  & Per lo sviluppo del prodotto richiesto verranno rispettate tutte le norme descritte nel documento \NormeDiProgetto{}. &  Interno \newline  \\ \hline      
        R3O 3 & Qualità \newline  Obbligatorio  & Maap Framework deve essere pubblicato in una repository di GitHub e quest'ultima deve permettere l'utilizzo di Issues per la segnalazione di bug. &  Capitolato \newline  Verbale-2013/12/05 \newline  \\ \hline
      \end{longtable}
      \egroup
      \end{center}  
\clearpage

\subsection{Requisiti di vincolo }

    %Tabella 
      \begin{center}
      \bgroup
      \def\arraystretch{1.8}
      \begin{longtable}{ | l | p{2cm} | p{5cm} | p{1.7cm} |}
    
      \cellcolor[gray]{0.9} \textbf{Requisito} & \cellcolor[gray]{0.9} \textbf{Tipologia} 
      & \cellcolor[gray]{0.9} \textbf{Descrizione} & \cellcolor[gray]{0.9} \textbf{Fonti} \\ \hline
      
        R4O 1 & Vincolo \newline  Obbligatorio  & L’implementazione della componente server deve essere realizzata utilizzando Node.js. &  Capitolato \newline  \\ \hline      
        RA4O 2 & Vincolo \newline  Obbligatorio  & Deve essere utilizzato Express per la realizzazione dell’infrastruttura della web application. &  Capitolato \newline  \\ \hline      
        RA4O 3 & Vincolo \newline  Obbligatorio  & L’applicazione deve utilizzare Mongoose.js per l’interfacciamento con il database. &  Capitolato \newline  \\ \hline      
        RA4O 4 & Vincolo \newline  Obbligatorio  & L’applicazione deve servirsi di  MongoDB come sistema gestionale dei dati. &  Capitolato \newline  \\ \hline      
        RA4O 5 & Vincolo \newline  Obbligatorio  & Deve essere fatto il deployment su Heroku rendendo disponibile on-line l’applicazione Maap. &  Capitolato \newline  \\ \hline      
        RA4O 6 & Vincolo \newline  Obbligatorio  & L’applicazione deve funzionare con versioni 24.x o superiori di Firefox. &  Capitolato \newline  \\ \hline      
        RA4O 7 & Vincolo \newline  Obbligatorio  & L’applicazione deve funzionare con versioni 30.0.x o superiori di Chrome. &  Capitolato \newline  \\ \hline      
        RA4O 8 & Vincolo \newline  Obbligatorio  & Il sistema deve mettere a disposizione un validatore del codice DSL e visualizzare gli eventuali errori logici o di sintassi in un'apposita pagina. &  UCU12 \newline  UCS3.6 \newline  \\ \hline
      \end{longtable}
      \egroup
      \end{center}  
\clearpage

\section{Tracciamento Requisiti}
\subsection{Tracciamento requisiti-fonti}
%Tabella 
      \begin{center}
      \bgroup
      \def\arraystretch{1.8}
      \begin{longtable}{ | p{5cm} | p{5cm} |}
    
      \cellcolor[gray]{0.9} \textbf{Requisiti} & \cellcolor[gray]{0.9} \textbf{Fonti} \\ \hline       
        RA1O 1 &  UCU1 \newline  Capitolato \newline  \\ \hline      
        RA1O 1.1 &  UCU1.1 \newline  UCU1 \newline  Capitolato \newline  \\ \hline      
        RA1O 1.2 &  UCU1.2 \newline  UCU1 \newline  Capitolato \newline  \\ \hline      
        RA1O 1.3  &  Capitolato \newline  \\ \hline      
        RA1O 1.3.1 &  UCU2 \newline  Interno \newline  \\ \hline      
        RA1O 1.3.2 &  UCU8 \newline  Interno \newline  \\ \hline      
        RA1O 2  &  UCU4 \newline  Capitolato \newline  UCU4.2 \newline  UCU4.1 \newline  UCU4.2.1 \newline  UCU4.1.1 \newline  \\ \hline      
        RA1O 2.1 &  UCU4 \newline  UCU4.1 \newline  UCU4.1.1 \newline  \\ \hline      

        RA1O 2.2 &  UCU4 \newline  UCU4.2 \newline  UCU4.2.1 \newline  \\ \hline    
        RA1O 4 &  UCU9 \newline  Capitolato \newline  \\ \hline      
        RA1O 4.1  &  Capitolato \newline  UCU9 \newline  \\ \hline      
        RA1O 4.1.1  &  Capitolato \newline  UCS3.3.1.4 \newline  \\ \hline      
        RA1O 5  &  UCU9.1 \newline  Capitolato \newline  UCU9.1.2 \newline  UCU9.1.1 \newline  \\ \hline      
        RA1O 5.1 &  UCU9.1.3 \newline  UCU9.1 \newline  Capitolato \newline  UCU9.1.5 \newline  UCU9.7 \newline  \\ \hline      
        RA1O 5.3 &  UCU9.1.4 \newline  Capitolato \newline  \\ \hline      
        RA1O 6 &  UCU11 \newline  Interno \newline  \\ \hline      
        RA1O 6.1  &  UCU11.1 \newline  UCU11 \newline  Capitolato \newline  \\ \hline      
        RA1O 6.1.1  &  UCU11.1 \newline  Capitolato \newline  \\ \hline      
        RA1O 6.1.1.1  &  UCU11.1.1 \newline  UCU11.1 \newline  \\ \hline      
        RA1O 6.1.1.2 &  UCU11.1.2 \newline  UCU11.1 \newline  \\ \hline      
        RA1O 6.1.1.3  &  UCU11.1.3 \newline  UCU11.1 \newline  Capitolato \newline  \\ \hline      
        RA1O 6.1.2  &  UCU11.1 \newline  Capitolato \newline  Interno \newline  \\ \hline      
        RA1O 6.1.3  &  UCU11.2 \newline  \\ \hline      
        RA1O 6.2  &  UCU11.1 \newline  UCU11.3 \newline  UCU11 \newline  \\ \hline      
        RA1O 6.2.1 &  UCU11.3.1 \newline  UCU11.3 \newline  Capitolato \newline  \\ \hline      
        RA1O 6.2.2 &  UCU11.3.2 \newline  UCU11.3 \newline  \\ \hline      
        RA1O 6.2.3 &  UCU11.3 \newline  Verbale-2013/12/18 \newline  \\ \hline      
        RA1O 6.2.4 &  UCU11.3 \newline  Verbale-2013/12/18 \newline  \\ \hline      
        RA1O 6.2.5 &  UCU11.3.3 \newline  \\ \hline      
        RA1D 3  &  UCU8 \newline  Interno \newline  \\ \hline      
        RA1D 4.1.2  &  Interno \newline  \\ \hline      
        RA1D 4.1.3  &  Interno \newline  \\ \hline      
        RA1D 4.1.4  &  UCU9.5 \newline  UCU9 \newline  \\ \hline      
        RA1D 4.1.5 &  UCU9.4 \newline  Capitolato \newline  \\ \hline      
        RA1D 4.2  &  UCU9.2 \newline  UCU9 \newline  UCS3.3.1.7 \newline  Interno \newline  \\ \hline      
        RA1D 11 &  UCU5 \newline  Verbale-2013/12/05 \newline  UCU6 \newline  \\ \hline      
        RA1D 12 &  UCU3 \newline  \\ \hline      
        RA1D 13 &  UCU10 \newline  \\ \hline      
        RA1D 13.1 &  UCU10 \newline  \\ \hline      
        RA1D 13.2 &  UCU10 \newline  UCU10.1 \newline  \\ \hline      
        RA1F 4.3 &  UCU9.6 \newline  Capitolato \newline  \\ \hline      
        RA1F 4.4 &  UCU9.3 \newline  Capitolato \newline  \\ \hline      
        RA1F 5.2 &  UCU9.3 \newline  Capitolato \newline  \\ \hline      
        RA1F 15 &  UCU7 \newline  Capitolato \newline  \\ \hline      
        RA1F 15.1 &  UCU7.1 \newline  Capitolato \newline  \\ \hline      
        RA1F 15.2 &  UCU7.2 \newline  Capitolato \newline  \\ \hline      
        RA1F 15.3 &  UCU7.3 \newline  Interno \newline  \\ \hline      
        RF1O 7 &  Capitolato \newline  \\ \hline      
        RF1O 8  &  UCS1 \newline  Capitolato \newline  \\ \hline      
        RF1O 8.1  &  UCS1 \newline  Capitolato \newline  \\ \hline      
        RF1O 8.1.1  &  UCS1 \newline  Capitolato \newline  \\ \hline      
        RF1O 8.1.2 &  UCS1 \newline  Capitolato \newline  \\ \hline      
        RF1O 8.1.3  &  UCS1 \newline  Capitolato \newline  \\ \hline      
        RF1O 8.1.4  &  UCS1 \newline  Capitolato \newline  \\ \hline      
        RF1O 8.2  &  UCS1 \newline  Verbale-2013/12/05 \newline  \\ \hline      
        RF1O 8.3 &  UCS3.3.5 \newline  Interno \newline  \\ \hline      
        RF1O 9  &  Capitolato \newline  UCS3 \newline  \\ \hline      
        RF1O 9.1  &  UCS3.1 \newline  UCS3.2 \newline  UCS3.4 \newline  UCS3 \newline  Capitolato \newline  \\ \hline      
        RF1O 9.1.1  &  UCS3.3 \newline  Capitolato \newline  UCS3.3.1.1 \newline  \\ \hline      
        RF1O 9.1.2  &  UCS3.3 \newline  UCS3.3.1.2 \newline  Capitolato \newline  \\ \hline      
        RF1O 9.1.3  &  UCS3.3 \newline  UCS3.3.1.3 \newline  Capitolato \newline  \\ \hline      
        RF1O 9.1.4  &  UCS3.3.1.2 \newline  Capitolato \newline  \\ \hline      
        RF1O 9.1.5 &  UCS3.3.1.6 \newline  UCS3.3.2.3 \newline  Capitolato \newline  \\ \hline      
        RF1O 9.1.6 &  UCS3.3.1.5 \newline  Capitolato \newline  UCS3.3.1 \newline  \\ \hline      
        RF1O 9.1.7 &  UCS3.3.1.8 \newline  Capitolato \newline  \\ \hline      
        RF1O 9.2 &  UCS3.1 \newline  UCS3.2 \newline  UCS3.4 \newline  Capitolato \newline  \\ \hline      
        RF1O 9.2.1  &  UCS3.3.2 \newline  UCS3.3.2.1 \newline  Capitolato \newline  \\ \hline      
        RF1O 9.2.2  &  UCS3.3.2.3 \newline  Verbale-2013/12/05 \newline  Capitolato \newline  \\ \hline      
        RF1O 9.2.3 &  UCS3.3.2 \newline  UCS3.3.2.2 \newline  Capitolato \newline  \\ \hline      
        RF1O 9.2.4 &  UCS3.3.2 \newline  UCS3.3.2.4 \newline  Capitolato \newline  \\ \hline      
        RF1O 9.3 &  UCS3.3.3 \newline  Capitolato \newline  \\ \hline      
        RF1O 9.4 &  UCS3.3.4 \newline  Capitolato \newline  \\ \hline      
        RF1O 14 &  UCS2 \newline  Interno \newline  \\ \hline      
        RF1O 14.1 &  UCS2.1 \newline  UCS2 \newline  Interno \newline  \\ \hline      
        RF1O 14.2 &  UCS2 \newline  UCS2.2 \newline  Interno \newline  \\ \hline      
        RF1F 9.2.5 &  UCS3.3.2 \newline  UCS3.3.2.5 \newline  Capitolato \newline  UCS3.3.1.9 \newline  \\ \hline      
        RF1F 14.3 &  UCS2.3 \newline  Interno \newline  \\ \hline      
        RF1F 16 &  UCS4 \newline  Interno \newline  \\ \hline      
        RS1F 10 &  Capitolato \newline  UCM \newline  \\ \hline      
        RS1F 10.1 &  Capitolato \newline  UCM8 \newline  UCM8.2 \newline  UCM8.4 \newline  UCM8.1 \newline  \\ \hline      
        RS1F 10.2 &  UCM8.4 \newline  UCM8.1 \newline  UCM8.3 \newline  UCM8 \newline  \\ \hline      
        RS1F 10.3 &  Interno \newline  UCM1 \newline  \\ \hline      
        RS1F 10.4 &  UCM1 \newline  Interno \newline  \\ \hline      
        RS1F 10.5 &  UCM3 \newline  Interno \newline  \\ \hline      
        RS1F 10.6 &  Interno \newline  UCM4 \newline  UCM1.1 \newline  UCM1.2 \newline  \\ \hline      
        RS1F 10.7 &  Interno \newline  UCM5 \newline  \\ \hline      
        RS1F 10.8 &  UCM6 \newline  UCM6.1 \newline  Interno \newline  \\ \hline      
        RS1F 10.9 &  UCM7 \newline  Interno \newline  \\ \hline      
        RS1F 10.9.1 &  UCM9 \newline  UCM7 \newline  \\ \hline      
        RS1F 10.10 &  UCM8.5 \newline  \\ \hline      
        RS4F 17 &  Capitolato \newline  \\ \hline      
        RS4F 18 &  Capitolato \newline  \\ \hline      
        R3O 1 &  Capitolato \newline  \\ \hline      
        R3O 2 &  Interno \newline  \\ \hline      
        R3O 3 &  Capitolato \newline  Verbale-2013/12/05 \newline  \\ \hline      
        R4O 1 &  Capitolato \newline  \\ \hline      
        RA4O 2 &  Capitolato \newline  \\ \hline      
        RA4O 3 &  Capitolato \newline  \\ \hline      
        RA4O 4 &  Capitolato \newline  \\ \hline      
        RA4O 5 &  Capitolato \newline  \\ \hline      
        RA4O 6 &  Capitolato \newline  \\ \hline      
        RA4O 7 &  Capitolato \newline  \\ \hline      
        RA4O 8 &  UCU12 \newline  UCS3.6 \newline  \\ \hline     
      \end{longtable}
      \egroup
      \end{center}  
\clearpage


\subsection{Tracciamento fonti-requisiti}
%Tabella 
      \begin{center}
      \bgroup
      \def\arraystretch{1.8}
      \begin{longtable}{ | p{5cm} | p{5cm} |}
    
      \cellcolor[gray]{0.9} \textbf{Fonti} & \cellcolor[gray]{0.9} \textbf{Requisiti} \\ \hline       
            Verbale-2013/12/18 &  RA1O 6.2.3 \newline  RA1O 6.2.4 \newline  \\ \hline      
            Verbale-2013/12/05 &  RA1D 11 \newline  RF1O 9.2.2  \newline  R3O 3 \newline  RF1O 8.2  \newline  \\ \hline      
            Capitolato &  RA1D 4.1.5 \newline  RF1O 9.2.4 \newline  RA1F 15 \newline  RA1F 15.1 \newline  RA1F 15.2 \newline  RA1O 5.3 \newline  RA1F 4.4 \newline  RA1F 5.2 \newline  RF1O 9.2.3 \newline  R4O 1 \newline  RA4O 4 \newline  RA4O 5 \newline  R3O 1 \newline  RF1O 9.2.2  \newline  RF1O 8.1.3  \newline  RA4O 3 \newline  RF1O 9.2 \newline  RF1O 9.2.1  \newline  R3O 3 \newline  RA4O 6 \newline  RF1F 9.2.5 \newline  RF1O 9.1.5 \newline  RF1O 9.1.2  \newline  RF1O 9.1.3  \newline  RF1O 8.1.1  \newline  RF1O 8.1.4  \newline  RF1O 8.1.2 \newline  RF1O 8.1  \newline  RF1O 9.3 \newline  RF1O 8  \newline  RF1O 9.1  \newline  RF1O 9.4 \newline  RA4O 2 \newline  RF1O 9.1.7 \newline  RF1O 9.1.6 \newline  RF1O 9.1.4  \newline  RF1O 9  \newline  RF1O 9.1.1  \newline  RA4O 7 \newline  RA1O 4.1.1  \newline  RF1O 7 \newline  RA1O 4 \newline  RA1O 4.1  \newline  RA1O 1 \newline  RA1F 4.3 \newline  RS1F 10 \newline  RS1F 10.1 \newline  RA1O 6.1.1  \newline  RA1O 6.1.1.3  \newline  RA1O 6.1  \newline  RA1O 6.1.2  \newline  RA1O 1.3  \newline  RA1O 1.2 \newline  RA1O 1.1 \newline  RA1O 5.1 \newline  RA1O 2  \newline  RA1O 5  \newline  RA1O 6.2.1 \newline  RS4F 17 \newline  RS4F 18 \newline  \\ \hline      
            UCU1 - Login &  RA1O 1 \newline  RA1O 1.2 \newline  RA1O 1.1 \newline  \\ \hline      
            UCU1.1 - Inserimento email &  RA1O 1.1 \newline  \\ \hline      
            UCU1.2 - Inserimento Password &  RA1O 1.2 \newline  \\ \hline      
            UCU2 - Visualizzazione messaggio errore per credenziali errate &  RA1O 1.3.1 \newline  \\ \hline      
            UCU3 - Logout &  RA1D 12 \newline  \\ \hline      

            UCU4 - Recupero password &  RA1O 2.1 \newline  RA1O 2.2 \newline  RA1O 2  \newline  \\ \hline      
            UCU4.1 - Richiesta reset password &  RA1O 2.1 \newline  RA1O 2  \newline  \\ \hline      
            UCU4.1.1 - Inserimento email &  RA1O 2.1 \newline  RA1O 2  \newline  \\ \hline      
            UCU4.2 - Effettuazione reset password &  RA1O 2.2 \newline  RA1O 2  \newline  \\ \hline      
            UCU4.2.1 - Inserimento nuova password &  RA1O 2.2 \newline  RA1O 2  \newline  \\ \hline      
            UCU5 - Registrazione &  RA1D 11 \newline  \\ \hline      
            UCU6 - Visualizzazione messaggio errore per Registrazione fallita &  RA1D 11 \newline  \\ \hline      
            UCU7 - Gestione indici &  RA1F 15 \newline  \\ \hline      
            UCU7.1 - Visualizzazione suggerimenti sugli indici &  RA1F 15.1 \newline  \\ \hline      
            UCU7.2 - Creazione indici &  RA1F 15.2 \newline  \\ \hline      
            UCU7.3 - Rimozione indici &  RA1F 15.3 \newline  \\ \hline      
            UCU8 - Visualizzazione Dashboard &  RA1O 1.3.2 \newline  RA1D 3  \newline  \\ \hline      
            UCU9 - Apertura Collection Index &  RA1D 4.2  \newline  RA1O 4 \newline  RA1D 4.1.4  \newline  RA1O 4.1  \newline  \\ \hline      
            UCU9.1 - Apertura show-page Document &  RA1O 5.1 \newline  RA1O 5  \newline  \\ \hline      
            UCU9.1.1 -  Visualizzazione show page attributi innestati &  RA1O 5  \newline  \\ \hline      
            UCU9.1.2 - Visualizzazione index page dell'array di document &  RA1O 5  \newline  \\ \hline      
            UCU9.1.3 - Modifica Document &  RA1O 5.1 \newline  \\ \hline      
            UCU9.1.4 - Elimina Document &  RA1O 5.3 \newline  \\ \hline      
            UCU9.1.5 - Annulla modifica Document &  RA1O 5.1 \newline  \\ \hline      
            UCU9.2 - Filtra risultati &  RA1D 4.2  \newline  \\ \hline      
            UCU9.3 - Esegui azione personalizzata &  RA1F 4.4 \newline  RA1F 5.2 \newline  \\ \hline      
            UCU9.4 - Modifica Document &  RA1D 4.1.5 \newline  \\ \hline      
            UCU9.5 - Elimina Document &  RA1D 4.1.4  \newline  \\ \hline      
            UCU9.6 - Creazione Document &  RA1F 4.3 \newline  \\ \hline      
            UCU9.7 - Annulla modifica Document &  RA1O 5.1 \newline  \\ \hline      
            UCU10 - Modifica profilo &  RA1D 13 \newline  RA1D 13.1 \newline  RA1D 13.2 \newline  \\ \hline      
            UCU10.1 - Modifica password &  RA1D 13.2 \newline  \\ \hline      
            UCU11 - Gestione utenti &  RA1O 6.1  \newline  RA1O 6 \newline  RA1O 6.2  \newline  \\ \hline      
            UCU11.1 - Creazione utente &  RA1O 6.1.1  \newline  RA1O 6.1.1.3  \newline  RA1O 6.1.1.2 \newline  RA1O 6.1  \newline  RA1O 6.1.1.1  \newline  RA1O 6.1.2  \newline  RA1O 6.2  \newline  \\ \hline      
            UCU11.1.1 - Inserimento email &  RA1O 6.1.1.1  \newline  \\ \hline      
            UCU11.1.2 - Inserimento password &  RA1O 6.1.1.2 \newline  \\ \hline      
            UCU11.1.3 - Definizione livello utente &  RA1O 6.1.1.3  \newline  \\ \hline      
            UCU11.2 - Creazione fallita &  RA1O 6.1.3  \newline  \\ \hline      
            UCU11.3 - Apertura show-page Utente &  RA1O 6.2.3 \newline  RA1O 6.2  \newline  RA1O 6.2.1 \newline  RA1O 6.2.2 \newline  RA1O 6.2.4 \newline  \\ \hline      
            UCU11.3.1 - Eleva utente &  RA1O 6.2.1 \newline  \\ \hline      
            UCU11.3.2 - Declassa admin &  RA1O 6.2.2 \newline  \\ \hline      
            UCU11.3.3 - Eliminazione Utente &  RA1O 6.2.5 \newline  \\ \hline      
            UCU12 - Visualizzazione messaggio di errore DSL &  RA4O 8 \newline  \\ \hline      
            UCS - Operazioni ad alto livello &  \\ \hline      
            UCS1 - Creazione nuovo progetto &  RF1O 8.1.3  \newline  RF1O 8.1.1  \newline  RF1O 8.1.4  \newline  RF1O 8.1.2 \newline  RF1O 8.1  \newline  RF1O 8  \newline  RF1O 8.2  \newline  \\ \hline      
            UCS2 - Configurazione database &  RF1O 14.1 \newline  RF1O 14.2 \newline  RF1O 14 \newline  \\ \hline      
            UCS2.1 - Configurazione database credenziali &  RF1O 14.1 \newline  \\ \hline      
            UCS2.2 - Configurazione database Collection &  RF1O 14.2 \newline  \\ \hline      
            UCS2.3 - Selezione namespace &  RF1F 14.3 \newline  \\ \hline      
            UCS3 - Gestione Collection &  RF1O 9.1  \newline  RF1O 9  \newline  \\ \hline      
            UCS3.1 - Registrazione manuale Collection &  RF1O 9.2 \newline  RF1O 9.1  \newline  \\ \hline      
            UCS3.2 - Creazione file di configurazione &  RF1O 9.2 \newline  RF1O 9.1  \newline  \\ \hline      
            UCS3.3 - Configurazione Collection &  RF1O 9.1.2  \newline  RF1O 9.1.3  \newline  RF1O 9.1.1  \newline  \\ \hline      
            UCS3.3.1 - Personalizzazione index page &  RF1O 9.1.6 \newline  \\ \hline      
            UCS3.3.1.1 - Definizione attributi da visualizzare &  RF1O 9.1.1  \newline  \\ \hline      
            UCS3.3.1.2 - Definizione ordinamento attributi &  RF1O 9.1.2  \newline  RF1O 9.1.4  \newline  \\ \hline      
            UCS3.3.1.3 - Definizione limite elementi da visualizzare &  RF1O 9.1.3  \newline  \\ \hline      
            UCS3.3.1.4 - Definizione attributo selezionabile &  RA1O 4.1.1  \newline  \\ \hline      
            UCS3.3.1.5 - Definizione della index-query &  RF1O 9.1.6 \newline  \\ \hline      
            UCS3.3.1.6 - Definizione populate da eseguire &  RF1O 9.1.5 \newline  \\ \hline      
            UCS3.3.1.7 - Personalizzazione filtri &  RA1D 4.2  \newline  \\ \hline      
            UCS3.3.1.8 - Definizione delle trasformazioni &  RF1O 9.1.7 \newline  \\ \hline      
            UCS3.3.1.9 - Definizione azioni personalizzate &  RF1F 9.2.5 \newline  \\ \hline      
            UCS3.3.2 - Personalizzazione show page &  RF1O 9.2.4 \newline  RF1O 9.2.3 \newline  RF1O 9.2.1  \newline  RF1F 9.2.5 \newline  \\ \hline      
            UCS3.3.2.1 - Definizione attributi da visualizzare &  RF1O 9.2.1  \newline  \\ \hline      
            UCS3.3.2.2 - Definizione ordinamento attributi &  RF1O 9.2.3 \newline  \\ \hline      
            UCS3.3.2.3 - Definizione populate da eseguire &  RF1O 9.2.2  \newline  RF1O 9.1.5 \newline  \\ \hline      
            UCS3.3.2.4 - Definizione delle trasformazioni &  RF1O 9.2.4 \newline  \\ \hline      
            UCS3.3.2.5 - Definizione azioni personalizzate &  RF1F 9.2.5 \newline  \\ \hline      
            UCS3.3.3 - Rinominazione Collection &  RF1O 9.3 \newline  \\ \hline      
            UCS3.3.4 - Definizione ordinamento Collection &  RF1O 9.4 \newline  \\ \hline      
            UCS3.3.5 - Definizione namespace &  RF1O 8.3 \newline  \\ \hline      
            UCS3.4 - Registrazione automatica Collection &  RF1O 9.2 \newline  RF1O 9.1  \newline  \\ \hline      
            UCS3.6 - Visualizzazione messaggio di errore DSL &  RA4O 8 \newline  \\ \hline      
            UCS4 - Abilitazione namespace &  RF1F 16 \newline  \\ \hline      
            UCM1 - Registrazione &  RS1F 10.4 \newline  RS1F 10.3 \newline  \\ \hline      
            UCM1.1 - Inserimento nuova email &  RS1F 10.6 \newline  \\ \hline      
            UCM1.2 - Inserimento nuova password &  RS1F 10.6 \newline  \\ \hline      
            UCM3 - Credenziali già esistenti &  RS1F 10.5 \newline  \\ \hline      
            UCM4 - Login &  RS1F 10.6 \newline  \\ \hline      
            UCM5 - Credenziali errate &  RS1F 10.7 \newline  \\ \hline      
            UCM6 - Modifica profilo &  RS1F 10.8 \newline  \\ \hline      
            UCM6.1 - Modifica password &  RS1F 10.8 \newline  \\ \hline      
            UCM7 - Eliminazione account &  RS1F 10.9 \newline  RS1F 10.9.1 \newline  \\ \hline      
            UCM8 - Gestione file di configurazione &  RS1F 10.1 \newline  RS1F 10.2 \newline  \\ \hline      
            UCM8.1 - Creazione file di configurazione &  RS1F 10.1 \newline  RS1F 10.2 \newline  \\ \hline      
            UCM8.2 - Creazione file di configurazione tramite editor di testo &  RS1F 10.1 \newline  \\ \hline      
            UCM8.3 - Creazione file di configurazione tramite caricamento &  RS1F 10.2 \newline  \\ \hline      
            UCM8.4 - Modifica file di configurazione &  RS1F 10.1 \newline  RS1F 10.2 \newline  \\ \hline      
            UCM8.5 - Eliminazione file di configurazione &  RS1F 10.10 \newline  \\ \hline      
            UCM9 - Eliminazione file di configurazione &  RS1F 10.9.1 \newline  \\ \hline     
      \end{longtable}
      \egroup
      \end{center}  
\clearpage

