\pagebreak
\section{Qualità}
\label{appendice-qualita}
La qualità perseguita nel presente documento si basa sugli standard ISO/IEC 15504  e ISO/IEC 9126 con l'obiettivo di approfondirne incrementalmente la copertura.
La qualità va ricercata non sul prodotto bensì sui \textbf{processi alla base del prodotto}, per questo tutti i processi seguono il metodo \glossario{PDCA} che prevede l'iterazione ripetuta tra i quattro stadi, assicurando un \textbf{incremento della qualità} ad ogni ciclo. \\
	\begin{figure}[h]
	\centering \includegraphics[width=1\textwidth]{PDCA.png}
	\caption{Continuous quality improvement with PDCA}
	\end{figure}
	
	\begin{enumerate}
		\item \textbf{PLAN}: vengono stabiliti gli obiettivi e i processi necessari per raggiungere la qualità attesa, nel dettaglio:
		\begin{itemize}
			\item Identificare il problema, o i processi da migliorare; Per descrivere il problema è necessario raccogliere i dati tramite misurazioni;
			\item Analizzare il problema e individuare gli effetti negativi, definendo la loro importanza e le priorità di intervento;
			\item Definire gli obiettivi di massima in modo chiaro e quantitativo, indicando i benefici ottenibili con il suo raggiungimento. Devono essere definiti anche i tempi, gli indicatori e gli strumenti di controllo.
		\end{itemize}				
		\item \textbf{DO}: viene implementato il punto precedente, applicando le soluzioni individuate al problema;
		\item \textbf{CHECK}: Verificare i risultati delle azioni intraprese, un confronto con i risultati attesi sarà il riscontro se quanto operato va nella direzione giusta. Vanno considerate metriche come la \emph{Schedule Variance} (vedi \ref{ScheduleVariance}) e la completezza dei risultati attesi soddisfatti, vanno elaborati grafici e tabelle per avere una visione chiara di quanto rilevato. 
		Se si è raggiunto l'obiettivo definito nello stadio di Plan, si può passare allo stadio di Act, altrimenti è necessario ripetere un nuovo ciclo PDCA sullo stesso problema, analizzando i vari stadi del ciclo precedente individuandone le cause del non raggiungimento dell'obiettivo stabilito;
		\item \textbf{ACT}: La soluzione individuata viene standardizzata  e tutti i membri del gruppo vengono informati e formati. Si potrà eseguire tramite riunioni o strumenti di messaggistica interna al gruppo. Terminato questo stadio si procederà con una nuova iterazione a partire dal punto 1.
	\end{enumerate}

Il ciclo \glossario{PDCA} è stato attuato nell'appendice \ref{appendiceQualità}. 

	\subsection{Qualità dei processi}
	\label{appendice-qualitaDeiProcessi}
	Definita in ISO/IEC 15504 come \glossario{SPICE}, specifica come la qualità è collegata alla maturazione dei processi. Vengono individuati dei livelli di maturità al quale il fornitore può fare riferimento per determinare le proprie capacità organizzative. Vengono definiti:
	\begin{itemize}
		\item Dei \textbf{Modelli di riferimento} su:
		\begin{itemize}
			\item \emph{dimensione del processo};
			\item \emph{livelli di capacità dei processi}:
			\begin{etaremune}
			\setcounter{enumi}{6}
				\item ottimizzato
				\item predicibile
				\item stabilito
				\item gestito
				\item eseguito
				\item incompleto
			\end{etaremune}
			La capacità di un processo viene misurata tramite degli attributi che sono assimilabili alle metriche dei processi individuate in \ref{MetricheProcessi}, in particolare la \emph{Schedule Variance} permette di capire se un processo è incompleto o gestito; il gruppo giungerà a maturazione quando i processi diventeranno predicibili ossia quando la \emph{Schedule Variance} subirà al più lievi oscillazioni;
		\end{itemize}
		\item Delle \textbf{Stime} che si concretizzano in una struttura per la misurazione composta da:
		\begin{itemize}
			\item I \emph{processi} di misurazione, indicati nel \PianoDiProgetto;
			\item Un \emph{modello} per la misurazione identificabile in questo documento;
			\item Gli \emph{strumenti} utilizzati, specificati nelle \NormeDiProgetto.
		\end{itemize}
		\item Le \textbf{Competenze e Qualifiche} di chi controlla; lo standard redige in modo rigoroso una serie di attività volte a formare chi opera l'attività di stesura del \emph{Piano di Qualifica} e \emph{Verifica}. Tali competenze sono assenti all'interno del gruppo e, considerato che effettuare una formazione in linea con quanto specificato dallo standard sarebbe impossibile, tutti i membri si impegnano a studiare ed applicare al meglio quanto descritto in questo documento.
	\end{itemize}
	
	\subsection{Qualità del prodotto software}
	\label{appendice-qualitaDelProdotto}
	Specificata in ISO/IEC 9126 si suddivide in:
	
	\begin{itemize}
		\item \textbf{Quality model}: classifica la qualità del software in un set di caratteristiche che verranno approfondite nel corso del progetto:
			
		\begin{itemize}
			\item Functionality: viene controllata grazie al tracciamento dei requisiti individuati ed analizzati e i componenti;
			\item Reliability: viene dimostrata combinando i test;
			\item Usability: viene controllata con i test di validazione, inoltre la stesura del manuale d'uso aiuterà a verificarne l'usabilità e ad intervenire laddove necessario;
			\item Efficiency: combinando analisi statica e dinamica controlliamo che il prodotto sia efficiente;
			\item Maintainability: viene realizzata con l'utilizzo di design pattern e la stesura di documentazione dettagliata;
			\item Portability: essendo \glossario{MaaP} un applicazione Web non ci sono particolari problemi di portabilità per gli utenti.
		\end{itemize}

		\item \textbf{External metrics}: sono le metriche rilevate tramite analisi dinamica specificate in \ref{MisureMetriche};
		\item \textbf{Internal metrics}: sono le metriche rilevate in analisi statica specificate in \ref{MisureMetriche};
		\item \textbf{Quality in use metrics}: si tratta di metriche rilevabili allo stato di prodotto \emph{usabile} in condizioni reali, si rimanda la definizione di tale aspetto a quando verranno trattate le considerazioni sull'usabilità del prodotto in uno scenario di utilizzo reale, questo deve avvenire non oltre la \emph{Progettazione di Dettaglio e Codifica}.
	\end{itemize}

