\section{Analisi dei requisiti}

Dal capitolato e dagli incontri con il proponente gli Analisti dovranno estrarre i requisiti del progetto, producendo l'Analisi dei Requisiti.

Tutti i requisiti devono essere riconducibile al capitolato o ad un incontro con il proponente. I requisiti obbligatori identificati, se rispettati, dovranno essere necessari e sufficienti per l'approvazione del prodotto.

\subsection{Requisiti}

Ogni requisito è identificato da un nome, che segue il seguente formalismo:
\begin{center}
	\code{R\{X\}\{Y\}\{Z\} \{num\}}
\end{center}

Dove:
\begin{itemize}
 \item \textbf{X} corrisponde al sistema di riferimento e può assumere i seguenti valori:
	\begin{itemize}
	 \item[] \textbf{A} = Applicazione \glossario{Maap};
	 \item[] \textbf{F} = \glossario{Framework} di \glossario{Maap};
	 \item[] \textbf{S} = \glossario{Maas}.
	\end{itemize}

 \item \textbf{Y} corrisponde alla tipologia del requisito e può assumere i seguenti valori:
	\begin{itemize}
	 \item[] \textbf{1} = Funzionale;
	 \item[] \textbf{2} = Di prestazione;
	 \item[] \textbf{3} = Di qualità;
	 \item[] \textbf{4} = Vincolo.
	\end{itemize}

 \item \textbf{Z} corrisponde alla priorità del requisito e può assumere i seguenti valori:
	\begin{itemize}
	 \item[] \textbf{O} = Obbligatorio
	 \item[] \textbf{D} = Desiderabile
	 \item[] \textbf{F} = Facoltativo o opzionale
	\end{itemize}

 \item \textbf{num} identifica in modo univoco la posizione che il requisito ha nella gerarchia. La numerazione va fatta in modo incrementale per i fratelli. % TODO la gerarchia è unica o ce n'è una per ogni sistema/tipologia/priorità?
\end{itemize}

\subsection{Casi d'uso}

Ogni caso d'uso è identificato da un nome, che segue il seguente formalismo:
\begin{center}
	\code{UC\{X\} \{num\}}
\end{center}

Dove: % TODO fare la descrizione del campo X e num
\begin{itemize}
 \item \textbf{X} corrisponde al sistema di riferimento e può assumere i seguenti valori:
	\begin{itemize}
	 \item[] \textbf{A} = Applicazione \glossario{Maap};
	 \item[] \textbf{F} = \glossario{Framework} di \glossario{Maap};
	 \item[] \textbf{S} = \glossario{Maas}.
	\end{itemize}

 \item \textbf{num} identifica in modo univoco la posizione che il requisito ha nella gerarchia. La numerazione va fatta in modo incrementale per i fratelli. % TODO la gerarchia è unica o ce n'è una per ogni sistema/tipologia/priorità?
\end{itemize}

\subsection{UML}

Per i diagrammi deve essere utilizzata il linguaggio \emph{UML versione 2.0}.