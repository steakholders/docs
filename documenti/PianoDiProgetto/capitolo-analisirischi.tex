\section{Analisi dei Rischi}
L'Analisi dei rischi si suddivide in 4 momenti:
\begin{itemize}
\item \textit{Identificazione}: sono identificati i rischi potenziali, e inseriti in sei categorie;
\item \textit{Analisi}: ogni rischio individuato viene considerato singolarmente, e gli viene assegnata una probabilità di occorrenza, utilizzando la seguente scala: \textit{molto bassa, bassa, media, alta, molto alta}. \\
Ad ogni rischio viene inoltre assegnato un livello di gravità, utilizzando la scala \textit{insignificante, tollerabile, serio, catastrofico};
\item \textit{Pianificazione}: per ogni rischio viene individuato un metodo per evitarlo o nel caso non sia possibile per mitigarne gli effetti sul progetto;
\item \textit{Controllo}: viene stabilito un metodo per verificare il successo della pianificazione, e periodicamente si controlla se la probabilità di occorrenza del rischio è variata.
\end{itemize}

I rischi identificati sono suddivisi in 6 categorie:
\begin{itemize}
\item \textit{Rischi tecnologici}: derivati da rotture hardware;
\item \textit{Rischi sulle persone}: associati alle persone che formano il gruppo;
\item \textit{Rischi organizzativi}: derivati dall'ambiente e dalle necessità organizzative;
\item \textit{Rischi sugli strumenti software}: derivati da problemi a carico degli strumenti e dei servizi utilizzati a supporto del progetto;
\item \textit{Rischi sui requisiti}: derivati dall'errata o incompleta comprensione dei requisiti, e dal possibile cambio o aggiunta di requisiti da parte del proponente in fase di progettazione;
\item \textit{Rischi sulle stime}: derivati dalla sottostima dei tempi e dei costi necessari per completare \ProjectName{}.
\end{itemize}

	\subsection{Rischi tecnologici}
	
		\subsubsection{Guasto hardware}
		
Ogni componente del gruppo è provvisto di un computer portatile, il rischio insito è un guasto tecnico ad uno o più computer.
\begin{enumerate}
\item \textit{Probabilità}: Bassa;
\item \textit{Effetti}: Tollerabile;
\item \textit{Pianificazione}: nel caso in cui si verificassero uno o più guasti hardware, verranno utilizzati i laboratori informatici messi a disposizione dall'Università di Padova;
\item \textit{Controllo}: il numero dei computer utilizzati è invariato durante tutto lo sviluppo, quindi il rischio resta invariato.
\end{enumerate}	
	
	
	\subsection{Rischi sul personale}
		\subsubsection{Problemi dei componenti del gruppo}
		
Ogni elemento del gruppo ha impegni personali, oltre alla necessità di dedicare parte della giornata alla preparazione di altri esami. Non ci sono studenti lavoratori all'interno del gruppo. \\
Si prende in considerazione anche il caso in cui un componente del gruppo si ammali. 		
\begin{enumerate}
\item \textit{Probabilità}: Bassa;
\item \textit{Effetti}: Tollerabili;
\item \textit{Pianificazione}: a prescindere dalla motivazione, nel caso in cui un membro del gruppo sia impossibilitato ad eseguire i propri task per un periodo limitato di tempo, il responsabile provvederà a riassegnare i task ad altri in modo da non ritardare le consegne previste;
\item \textit{Controllo}: utilizzo del calendario di gruppo per individuare le fasi critiche.
\end{enumerate}

		\subsubsection{Problemi tra i componenti del gruppo}

Ogni membro del gruppo è alla prima esperienza in un gruppo numeroso, nonostante alla formazione del gruppo non siano state riscontrate incompatibilità il rischio è che all'aumentare del carico di lavoro sorgano problemi.		
\begin{enumerate}
\item \textit{Probabilità}: Media;
\item \textit{Effetti}: Seri;
\item \textit{Pianificazione}: nel caso di forti contrasti, sarà compito del \textit{Responsabile di progetto} fare da mediatore al fine di risolvere la contesa. Se tale contromisura non si rivelasse sufficiente il \textit{Responsabile} ripartirà il lavoro in modo tale da evitare il più possibile il contatto tra i due;
\item \textit{Controllo}: l'\textit{Amministratore} include tra le sue responsabilità il mantenimento di un clima cooperativo nell'ambiente di lavoro.
\end{enumerate}	
		
		\subsubsection{Scarsa conoscenza delle tecnologie}
		
\begin{itemize}
\item \textit{Probabilità}: Alta;
\item \textit{Effetti}: Seri;
\item \textit{Pianificazione}: ogni membro è tenuto a studiare le tecnologie coinvolte nello sviluppo di \ProjectName{} per conto proprio. Inoltre, sono stati pianificati incontri con il proponente per approfondire gli aspetti più complessi;
\item \textit{Controllo}: il \textit{Responsabile} ha il compito di verificare il grado di conoscenze relativo alle tecnologie utilizzate di ogni membro del gruppo.
\end{itemize}	


	\subsection{Rischi organizzativi}
	
		\subsubsection{Rotazione dei ruoli}

La rotazione dei ruoli prevista può creare difficoltà ai componenti del gruppo a causa del cambio di competenze e di responsabilità associati al diverso ruolo da ricoprire.

\begin{enumerate}
\item \textit{Probabilità}: Bassa;
\item \textit{Effetti}: Tollerabili;
\item \textit{Pianificazione}: la rotazione dei ruoli, essendo prestabilita, da la possibilità ai componenti del gruppo di sapere preventivamente il ruolo successivo che dovranno ricoprire e di studiare la documentazione già prodotta;
\item \textit{Controllo}: il \emph{Responsabile} verifica che ogni membro del gruppo ricopra tutti i ruoli previsti dalle \NormeDiProgetto .
\end{enumerate}
	
	\subsection{Rischi sugli strumenti software}
	
		\subsubsection{Piattaforme fuori servizio}	

In particolare, le piattaforme coinvolte sono \glossario{TeamworkPM}, \glossario{Amazon AWS} e \glossario{GitHub}. 

\begin{enumerate}
\item \textit{Probabilità}: Molto bassa;
\item \textit{Effetti}: Catastrofici;
\item \textit{Pianificazione}: i rischi legati alle diverse piattaforme sono così suddivisi.

	\begin{itemize}
	\item \glossario{TeamworkPM} dichiara di appoggiarsi ai servizi di backup offerti da \glossario{Amazon};
	\item \glossario{Amazon AWS} dichiara di disporre di sedi in tutto il mondo riducendo i rischi derivanti da guasti o catastrofi. Fornisce una documentazione riguardante le misure di sicurezza adottate all'indirizzo \url{aws.amazon.com/security};
	\item \glossario{GitHub} dichiara di effettuare backup su tre differenti server,  di cui uno in un'altra sede. Fornisce una documentazione riguardante le misure di sicurezza adottate all'indirizzo \url{help.github.com/articles/github-security}. Inoltre, ogni componente del gruppo ha una copia locale della \glossario{repository} remota, permettendo un recupero parziale o totale del lavoro svolto. 
	\end{itemize}
	
\item \textit{Controllo}: non è possibile effettuare un controllo sulla pianificazione, pertanto ci si affida alle misure di sicurezza adottate dalle piattaforme in uso.
\end{enumerate}
	
	\subsection{Rischi sui requisiti}
	
		\subsubsection{Modifica dei requisiti}
		
Nel capitolato è fatta presente la riserva, da parte del committente, di apportare variazioni ai requisiti sia precedentemente alla consegna delle offerte che durante la realizzazione del sistema.

\begin{enumerate}
\item \textit{Probabilità}: Bassa;
\item \textit{Effetti}: Seri;
\item \textit{Pianificazione}: si cerca di coinvolgere quanto più possibile il proponente mantenendo un contatto diretto con i rappresentanti. Inoltre, la probabilità di occorrenza di una variazione ai requisiti è ridotta grazie alla competenza dei rappresentati la quale determina una visione più chiara del quadro d'insieme;
\item \textit{Controllo}: ad ogni incontro con i rappresentanti del proponente corrisponde un verbale interno al gruppo. Inoltre, ogni comunicazione con i rappresentanti del proponente viene notificato ad ogni membro come descritto nel paragrafo 2.2 delle \NormeDiProgetto.  
\end{enumerate}
		
		\subsubsection{Comprensione errata dei requisiti}

Data l'inesperienza dei componenti del gruppo nell'analisi dei requisiti, è possibile un'errata comprensione dei requisiti comportando un'offerta non conforme alle richieste.

\begin{enumerate}
\item \textit{Probabilità}: Bassa;
\item \textit{Effetti}: Seri;
\item \textit{Pianificazione}: ogni componente del gruppo è tenuto a colmare le lacune concernenti i fondamenti dell'analisi dei requisiti visti in sede di lezione;
\item \textit{Controllo}: nel caso di dubbi sugli aspetti tecnici dell'analisi dei requisiti, è consigliato consultare gli altri membri del gruppo o eventualmente contattare il Prof. Riccardo Cardin.
\end{enumerate}

	\subsection{Rischi sulle stime}
	
		\subsubsection{Sottostima dei tempi necessari}

Data l'inesperienza dei componenti del gruppo nella pianificazione di progetto e l'attuazione della stessa su una arco di tempo medio-lungo, la sottostima dei tempi necessari alla realizzazione del progetto risulta un rischio concreto.

\begin{enumerate}
\item \textit{Probabilità}: Alta;
\item \textit{Effetti}: Tollerabili;
\item \textit{Pianificazione}: i gruppi di attività pianificate relative alle scadenze fissate dal committente non ricoprono tutto l'arco di tempo a disposizione lasciando uno \glossario{slack time} prima di ogni consegna; 
\item \textit{Controllo}: il \textit{Responsabile}, grazie alle piattaforme di gestione delle attività, può verificare lo stato di avanzamento delle stesse.
\end{enumerate}
