\section{Progettazione}
\subsection{Specifica Tecnica}
La fase di progettazione utilizzerà il documento di Analisi dei Requisiti e produrrà un progetto. Il progetto deve dimostrabilmente rispettare tutti i requisiti che il gruppo ha concordato con il committente.
Di seguito vengono elencate le norme a carico dei \emph{Progettisti}.

	\subsubsection{Diagrammi UML}
	Si dovrà usare il linguaggio \glossario{UML} \emph{versione 2.0} per i seguenti diagrammi:
\begin{itemize}
 \item \textbf{Diagrammi delle classi} %TODO
 \item \textbf{Diagrammi di flusso} %TODO
 \item \textbf{Diagrammi di package} %TODO
\end{itemize}

	\subsubsection{Design pattern}
	I \glossario{design pattern} utilizzati nella progettazione vanno giustificati e descritti sia testualmente sia mediante degli schemi che esemplifichino il funzionamento e la struttura.

	\subsubsection{Tracciamento componenti}
	I componenti progettati devono essere tracciabili rispetto al requisito che soddisfano. I requisiti vengono tracciati mediante il software \emph{Requisteak} appositamente creato dal gruppo \GroupName . Tale strumento è raggiungibile all'indirizzo \url{http://steakholders.herokuapp.com}. In questo modo sarà possibile misurare l'andamento della percentuale tra requisiti individuati e progettati.
	 
	\subsubsection{Test di integrazione}
	I test di integrazione avvengono mediante la definizione e l'utilizzo di opportuni strumenti di test per verificare che i componenti del sistema funzionino nella maniera prevista.

\subsection{Definizione di Prodotto}
La \emph{Definizione di Prodotto} è la descrizione dettagliata della progettazione descritta nella \emph{Specifica Tecnica}. Questo documento rappresenta lo stadio avanzato della progettazione e indicherà ogni singolo componente del sistema permettendo ai \emph{Programmatori} di procedere con lo sviluppo. Parallelamente alla progettazione di dettaglio dovranno essere progettati i relativi test di unità che verranno descritti nel \PianoDiQualifica .

	\subsubsection{Diagrammi UML}
	I seguenti diagrammi vanno aggiornati e completati al dettaglio:
	\begin{itemize}
 		\item \textbf{Diagrammi delle classi} %TODO
 		\item \textbf{Diagrammi di flusso} %TODO
		 \item \textbf{Diagrammi di package} %TODO
	\end{itemize}
	
	\subsubsection{Definizione di classe}
	Le classi progettate devono comparire nella \emph{Definizione di Prodotto} che dovrà descriverne l'elenco dei metodi e degli attributi, lo scopo e la funzionalità che rappresenta.
	%TODO completare
	
	% \subsection{Formalismo di specifica dei metodi}
	% \subsection{Tracciamento classi}
	
	\subsubsection{Test di unità}
	I \emph{Progettisti} modellano i test da effettuare sulle unità per le opportune verifiche.

\subsection{Verifica}
L'attività di verifica viene effettuata dai \emph{Verificatori} scelti tra i membri del gruppo secondo il principio di assenza di conflitti d'interesse, ossia colui che viene chiamato a verificare un determinato componente non può aver in alcun modo partecipato alla creazione. La verifica è un attività continua nei processi, nei documenti e nel software. Di seguito ne vengono spiegate le modalità operative.

	\subsubsection{Metriche per le anomalie riscontrate e gestione dei cambiamenti}
	Si definiscono come vanno trattate le metriche indicate nel \PianoDiQualifica  da parte dei \emph{Verificatori}.
	%TODO
	
	
	\subsubsection{Verifica dei documenti}
	La verifica dei documenti avviene quando il \emph{Verificatore} prende in carico il documento e compila opportunamente il diario delle modifiche. La verifica di un documento è composta dai seguenti passi:
	\begin{enumerate}
		\item \textbf{Controllo sintattico e del periodo:} utilizzando Aspell vengono evidenziati e corretti gli errori grammaticali. Un successivo controllo walktrough permetterà di analizzare i periodi ed eventualmente correggerne la forma;
		\item \textbf{Rispetto delle norme di progetto:} i documenti devo seguire le norme tipografiche specificate. Tale verifica non può essere automatizzata poiché il setup dei tools necessari richiederebbe molto tempo e delle competenze specifiche. 
		\item \textbf{Verifica delle proprietà di glossario:} va utilizzato l'apposito script che permette di confrontare le i termini racchiusi dal tag "\glossario{}", presenti nei file con estensione \emph{.tex}, con quelli presenti nel documento \emph{Glossario}.
		\item \textbf{Calcolo dell'indice Gulpease:} per ogni documento redatto, il \emph{Verificatore} deve calcolare l'indice di leggibilità gulpease. Qualora tale indice risultasse troppo basso andranno eseguite le opportune operazioni per semplificare frasi troppo lunghe o complesse;
		\item \textbf{Miglioramento del processo di verifica:} per migliorare ciclicamente il processo di verifica, verranno riportati gli errori frequenti. In questo modo sarà più facile eseguire, negli incrementi successivi, controlli di tipo inspection.
		\item \textbf{Segnalazione delle anomalie riscontrate:} il \emph{Verificatore} deve generare un ticket secondo le modalità descritte in questo documento.
	\end{enumerate}
	
\subsection{Verifica del tracciamento dei requisiti}
Al verificatore è richiesto di controllare che tutti i requisiti individuati siano scritti secondo le norme e facciano riferimento ad un diagramma di caso d'uso.

\subsection{Verifica dei diagrammi UML}
I diagrammi \glossario{UML} prodotti vengono controllati dal \emph{Verificatore}:
	\begin{itemize}
		\item \textbf{Diagrammi di caso d'uso:} il controllo dei diagrammi di caso d'uso deve avvenire manualmente, controllando il rispetto delle specifiche \glossario{UML} e il corretto uso delle relazioni di inclusione ed estensione. Il diagramma di caso d'uso deve rappresentare quanto descritto dal caso d'uso.
		\item \textbf{Diagrammi delle classi:} il \glossario{Verificatore} controlla la corrispondenza tra progettazione e diagrammi delle classi.
	\end{itemize}

%\subsection{Verifica della progettazione architetturale}
%\subsection{Verifica della progettazione di dettaglio}
%\subsection{Verifica del codice}
%\subsubsection{Analisi Statica}
%\subsubsection{Analisi Dinamica}
%\section{Resoconto bug}
%\section{Validazione output}

%\section{Codifica}
%	\subsection{Codifica e convenzioni}
%		\subsubsection{Nomi}
%		\subsubsection{Ricorsione}
%	\subsection{Documentazione}
	
	