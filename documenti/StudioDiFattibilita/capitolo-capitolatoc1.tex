\section{Capitolato C1 - MaaP}

\subsection{Descrizione}
Il capitolato scelto prevede la realizzazione di un \glossario{framework} per generare interfacce web di amministrazione dei dati di \glossario{business}. In particolare, l'amministrazione dei dati deve essere disponibile a livello di \glossario{database}, nel quale vengono effettuate operazioni direttamente sugli oggetti che lo rappresentano (tabelle, indici, viste) in modo tale da permettere un accesso veloce e consistente ai dati. Questa tipologia di amministrazione non si preoccupa di interpretare le informazioni in dati di \glossario{business}, ma si limita a interagire in modo agnostico con le entità del \glossario{database}.

Inoltre, deve essere possibile l'amministrazione a livello di \glossario{dati di business}, in cui vengono effettuate operazioni su una o più di tali entità, le quali vengono interpretate nel modello di \glossario{business} richiesto. La realizzazione delle pagine web di visualizzazione deve essere svolta in maniera semplice e veloce da parte dello \glossario{sviluppatore}, e le modalità di fruizione delle pagine generate devono essere adeguate ad un esperto di \glossario{business}.

\subsection{Studio del dominio}

        \subsubsection{Dominio applicativo}
        Il contesto operativo in cui si inserisce il progetto è strettamente legato alla persistenza dei dati tramite l'utilizzo di \glossario{basi di dati} distribuite di tipo non-relazionale(\glossario{NoSQL}), nel particolare \glossario{MongoDB}.\\
         \glossario{MongoDB} è un database  \glossario{NoSQL} adottato in maniera crescente soprattutto in contesti, come quello  \glossario{Ruby On Rails} e  \glossario{Node.js}, dove l'attenzione maggiore è rivolta ad una modellazione  \glossario{agile}, alla ricerca della possibilità di rimodulare continuamente la definizione degli  \glossario{schema} di  \glossario{database}, e alla produttività.\\
        \ProjectName{} si inserisce in questo contesto venendo in contro all'esigenza sia da parte degli  \glossario{esperti di business} sia dagli  \glossario{sviluppatori} che operano su questa tecnologia, di avere uno strumento che permetta la generazione in modo rapido di pagine gestionali, al fine di amministrare e interagire con le entità e i dati presenti in  \glossario{MongoDB}.
        
        
In particolare, gli utenti interessati nel dominio applicativo saranno lo  \glossario{sviluppatore}, che utilizzerà \ProjectName{} per generare le pagine, e l' \glossario{esperto di business}, che non dev'essere necessariamente un esperto di informatica, il quale usufruisce delle pagine generate per poter amministrare facilmente le entità di  \glossario{business} interagendo con la  \glossario{base di dati}.

Si dovrà quindi implementare un \glossario{framework} che permetta allo sviluppatore di creare e personalizzare, per mezzo di un linguaggio  \glossario{DSL} definito, le suddette interfacce web.
        
        \subsubsection{Dominio tecnologico}
        \begin{itemize}
                \item \textbf{ \glossario{Node.js}} per la realizzazione della componente  \glossario{server};
                \item \textbf{ \glossario{Express}} per la realizzazione dell'infrastruttura della  \glossario{web application} generata;
                \item \textbf{ \glossario{Mongoose.js}} per l'interfacciamento con il database;
                \item \textbf{ \glossario{MongoDB}} per il recupero dei dati;
                \item conoscenza di  \glossario{framework} per la componente  \glossario{front-end} (i.e.  \glossario{Angular.js},  \glossario{Ember.js});
                \item conoscenze nella definizione di linguaggi astratti  \glossario{DSL} per la generazione delle pagine da parte dello  \glossario{sviluppatore}.
        \end{itemize}


\subsection{Valutazione}
        Vengono di seguito elencati gli aspetti positivi che hanno determinato la scelta del capitolato:
        \begin{itemize}
        \item Apprendimento di tecnologie innovative che portano un bagaglio di conoscenze ritenuto importante dato il grande uso di quest'ultime nella panoramica delle tecnologie presenti attualmente nel mercato;
        \item Interesse del gruppo a vedere la propria applicazione dare vita ad una community dato che non esiste attualmente,con 		lo  \glossario{stack tecnologico} proposto, un applicativo simile;
        \item Requisiti richiesti dal proponente sono ben delineati.
        \end{itemize}
		Similmente, il gruppo ha trovato aspetti negativi:
        \begin{itemize}
        \item Le tecnologie utilizzate nello sviluppo del progetto non sono conosciute da nessun membro del gruppo \GroupName{} e
        quindi richiederanno un tempo di formazione per il loro apprendimento considerevole;
        \item La mole di lavoro per lo sviluppo del progetto al gruppo sembra notevole.
        \end{itemize}
        %\subsubsection{Potenziali criticità}
        
