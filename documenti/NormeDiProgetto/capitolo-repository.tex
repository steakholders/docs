\section{Repository e condivisione file}

\subsection{Repository dei documenti}

\subsubsection{Struttura del repository}

La struttura del repository, la cui cartella principale chiameremo \file{root}, è così composta:
\begin{itemize}
 \item \textbf{\file{root/modello/}} \\
	Contiene i file comuni a due o più documenti.

 \item \textbf{\file{root/documenti/\{NomeDelDocumento\}/}} \\
	Ciascuna cartella descritta da questo percorso contiene i file che vengono utilizzati dal documento \file{\{NomeDelDocumento\}}. In particolare conterrà il file \file{\{NomeDelDocumento\}.pdf} e il file \file{diario\_ modifiche.tex}, contenente il diario delle modifiche del documento.
	
	Il nome della cartella \file{\{NomeDelDocumento\}} e il file \file{\{NomeDelDocumento\}.pdf} devono rispettare la convenzione \glossario{CamelCase}.

 \item \textbf{\file{root/script/}} \\
	Contiene gli script di supporto alle operazioni di verifica e compilazione dei documenti.
\end{itemize}

Tutti i file e le cartelle non devono contenere spazi nel loro nome.

\subsubsection{Branch}

Il nome utilizzato per le \glossario{branch} del repository dovrà essere nella forma \textbf{nome-branch}, ossia con tutte le lettere minuscole e le parole separate da un trattino.

\subsubsection{Utilizzo del Makefile}
\label{makefile}

Per agevolare alcune operazioni sul repository è presente uno script \glossario{Makefile}. Al momento sono previsti due utilizzi:
\begin{itemize}
 \item \textbf{\code{make test}} \\
	Per ogni file \file{*.tex} contenuto nella cartella e nelle sue sotto-cartelle:
	\begin{itemize}
		\item Verifica che i file siano memorizzati con la codifica \glossario{UTF-8};
		\item Verifica con \emph{Aspell} che le parole utilizzate nei documenti siano comprese nel dizionario italiano di \emph{Aspell} o nel dizionario personalizzato \file{root/script/aspell\_personal.txt}.
	\end{itemize}
 \item \textbf{\code{make documents}} \\
	Compila tutti i documenti presenti nelle sotto-cartelle.

% TODO modificare gli script in modo che ci sia anche make build, che prepara tutti i file root/build/{RR}/NomeDelDocumento.pdf
\end{itemize}

È possibile eseguire questo script dalle cartelle dei documenti e dalle loro cartelle superiori, fino alla cartella principale del repository.

\subsubsection{Script di pre-commit}

Con l'intenzione di ridurre al minimo la presenza di errori nei documenti caricati sul repository è disponibile uno script di \code{pre-commit}. Lo script, attivato direttamente dal programma \code{git} nel momento in cui viene eseguito un commit, esegue in automatico i comandi descritti nella sezione \ref{makefile}. Se uno dei due comandi fallisce il commit viene annullato e all'utente viene mostrato un messaggio di errore.

Per abilitare tale script di \code{pre-commit} è necessario creare un \glossario{link simbolico} con il seguente comando:
\begin{lstlisting}
ln -s {percorso assoluto del repository}/script/pre-commit {percorso assoluto del repository}/.git/hooks/pre-commit
\end{lstlisting}

\subsection{Repository del codice}

Da definire.

\subsection{Condivisione file}

Qualora ce ne fosse la necessità, per condividere tra i membri del gruppo dei file che non necessitano di versionamento verrà utilizzata la piattaforma \emph{Google Drive} (\url{http://drive.google.com}).

Se servisse, per condividere file con l'esterno del gruppo verrà utilizzato l'account Google del gruppo descritto nella sezione \ref{email}.
