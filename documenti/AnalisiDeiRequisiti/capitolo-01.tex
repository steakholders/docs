\section{Introduzione}


\subsection{Scopo del documento}
OGNI COSA CHE HO SCRITTO E' UNA BOZZA/IDEA/PAROLE A CASO : MODIFICATELA, ELIMINATE ED AGGIUNGETE COME PENSATE SIA GIUSTO.
PIÙ MENTI MEGLIO E'. \\ 
Il presente documento ha come obiettivo descrivere formalmente i requisiti evidenziati dall'analisi del capitolato d'appalto "MaaP: MongoDB as an admin Platform" (C1) secondo le esigenze del proponente CoffeeStrap che il fornitore \GroupName si impegna a soddisfare.


\subsection{Scopo del prodotto}
Lo scopo del prodotto è produrre un framework per generare interfacce web di amministrazione di dati di business basati sullo stack tecnologico Node.js e MongoDB, permettendo allo sviluppatore, utilizzando il software come framework di lavoro, di produrre tipi di pagine web visualizzabili e finalizzate all'utilizzo di figure non esperte di informatica ma del dominio di business.

\subsection{Glossario}
Con l'obiettivo di evitare ridondanze e ambiguità di linguaggio, i termini tecnici e gli acronimi utilizzati nei documenti verranno definiti e descritti riportandoli nel documento "Glossario.pdf". I vocaboli riportati vengono indicati con una [G] a pedice.

\subsection{Riferimenti}
	\subsubsection{Normativi}
	\begin{itemize}
	\item \textbf{Capitolato d'appalto}: MaaP: MongoDB as an admin Platform rilasciato dal proponente\ped{G}
	CoffeeStrap e reperibile all'indirizzo 
	\url{http://www.math.unipd.it/~tullio/IS-1/2013/Progetto/C1.pdf}
	 
		\item \textbf{Norme di Progetto} : documento "Norme di Progetto.pdf". 
	\end{itemize}
	
	
	\subsubsection{Informativi}
		\begin{itemize}
			\item \textbf{Presentazione capitolato d'appalto}: \url{http://www.math.unipd.it/~tullio/IS-1/2013/Progetto/C1p.pdf}.
			\item \textbf{Standard 830-1998}: \url{http://www.math.uaa.alaska.edu/~afkjm/cs401/IEEE830.pdf}. 
		\end{itemize}
		
