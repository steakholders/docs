\section{Domande e Risposte riassunte}
	\begin{itemize} 
		\item 
		{\bfseries Quanto e come devono essere personalizzabili per lo sviluppatore le pagine Collection-index
		e Collection-show tramite l'uso di DSL?} \\
		Il proponente utilizzando come esempio active admin mostra come devono essere composte le pagine.
		Le funzioni principali illustrate sono la possibilità di selezionare un sottoinsieme di documenti tramite query di MongoDB scritte
		dallo sviluppatore e la scelta dei campi visualizzabili.	
		
		\item 
		{\bfseries Ci deve essere una dashboard?} \\
		Si è utile, ma non essenziale. 
		
		\item
		{\bfseries Dobbiamo implementare filtri sui campi restituiti dalle query?} \\
		I filtri sono delle query particolari ed è difficile implementarli in quanto bisogna avere conoscenza di come fare le query su di un database non relazionale.
		Possiamo implementarne di default o lasciare la possibilità allo sviluppatore di creare dei filtri.
		
		\item
		{\bfseries \`E necessario creare un profilo admin di default?} \\
		Si, l'admin va aggiunto al database con inserimento di username e password di default.
		
		\item
		{\bfseries \`E possibile che lo sviluppatore possa decidere di permettere la registrazione da parte degli utenti?} \\
		 Si lo sviluppatore lo può specificare, ma è meglio lasciarlo come opzione di default e di non togliere questa possibilità
		 all'utente.
		
		\item
		{\bfseries Come avviene il recupero password?} \\
		Il recupero password deve avvenire tramite email, quindi \ProjectName{}deve avere un servizio email configurabile dallo sviluppatore.
		Il proponente mostra il pacchetto "Nodemailer".
		
		\item
		{\bfseries Nel capitolato si parla di creazione di indici, cosa si intende per indici?} \\
		Gli indici sono specifici di MongoDB, il quale fa le ricerche scansionando i documenti che talvolta possono essere di grandi dimensioni. Per migliorare le performance di interrogazione, MongoDB crea strutture dati che raccolgono informazioni sui valori dei "campi indice" specificati nei documenti di una collection e queste strutture dati, sono gli indici citati nel capitolato.
		
In questo modo quando effettuerà la ricerca, verrà scansionato quest'ultimo documento creato che è più piccolo della collection originale rendendo la ricerca più veloce.	 
Dopo diversi mesi di utilizzo, il database può avere una mole di informazioni tale da poter identificare quali sono le query
maggiormente richieste dagli utenti e quindi proporre la creazione di uno o più indici.
La creazione di un indice può essere fatta automaticamente dal database o delegata allo sviluppatore via shell.
		
		\item
		{\bfseries Cosa rende complessa la creazione di nuovi documenti o la modifica di un insieme di documenti?} \\
		Tutto ciò che cambia la struttura del documento risulta complesso se non si conoscere la business logic dell'applicazione.
		Per creare o modificare documenti si ha bisogno di utilizzare Mongoose che mantiene i vincoli del modello e quindi non corrompe il funzionamento dell'applicazione che si interfaccia ai documenti.
		
		\item
		{\bfseries Come deve essere implementato il database dei dati utente?} \\
		Può essere implementato con qualsiasi tecnologia, anche diversa da MongoDB ma è importante che sia supportata da Heroku.
		
		Le password devono essere criptate.
		Non bisogna occuparsi dell'efficienza.
		
		\item
		{\bfseries Siamo vincolati all'utilizzo delle issue di GitHub per segnalare bug?} \\
		No, non è necessario l'utilizzo delle issue. 
		Il concetto sta nell'utilizzare durante lo sviluppo una repository, anche diversa, ma la pubblicazione del
		progetto deve avvenire su GitHub.
		
		\item
		{\bfseries Come implementare il DSL? Si possono creare delle apposite API ad alto livello?} \\
		Si, possono essere implementate delle librerie API in quanto l'output ottenuto è lo stesso del DSL, la differenza a cui 	fare attenzione sta nel dove mettere i vincoli in quanto nel DSL i vincoli sono statici e il parser li garantisce mentre nelle librerie API sono dinamici.
		
		\item
		{\bfseries \Proponente{} è disponibile per incontri durante l'intero periodo di sviluppo?} \\
		Il proponente dichiara che non potrà essere disponibile nei giorni 18-28 gennaio e che a marzo prevede il trasferimento in
		altra sede estera, ma resta disponibile ad incontri virtuali su Skype.
				
		\item
		{\bfseries Ci sono delle richieste particolari sul testing di \ProjectName{}} \\
		Almeno il 70\% codice prodotto deve essere testato nelle modalità preferite dal gruppo.
		
	\end{itemize}