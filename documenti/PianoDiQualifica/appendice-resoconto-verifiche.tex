\section{Resoconto delle attività di verifica}

	\subsection{Riassunto delle attività di verifica}
	\label{RiassuntoAttivitaVerifica}
	
	 	\subsubsection{Revisione dei Requisiti}
	 	L'attività di verifica svolta dai \emph{Verificatori} è avvenuta come determinato dal \PianoDiProgetto{} al termine della stesura di ogni documento previsto. La verifica svolta sui documenti è avvenuta seguendo le indicazioni delle \NormeDiProgetto{} e misurando le metriche indicate in \ref{metrichedocumenti}. L'attività di \emph{walkthrough} ha evidenziato una serie di anomalie, in questo modo è stato possibile stilare la lista di anomalie frequenti (vedi \NormeDiProgetto{}) che si potranno controllare tramite \emph{Inspection}. Successivamente si è proceduto con le misurazioni delle metriche relative ai documenti.
		In questa revisione non è stato possibile valutare i processi poiché lo stato embrionale del team e impegni universitari sovrapposti non hanno permesso il rilevamento accurato di tutti i parametri necessari. Il gruppo ha in programma di colmare tale mancanza per la revisione successiva.


		\subsubsection{Revisione di Progettazione}
		L'attività di verifica svolta dai \emph{Verificatori} è avvenuta come determinato dal \PianoDiProgetto{} al termine della stesura di ogni documento previsto. La verifica svolta sui documenti è avvenuta seguendo le indicazioni delle \NormeDiProgetto{} e misurando le metriche indicate in \ref{metrichedocumenti}. Successivamente si è proceduto con le misurazioni delle metriche relative ai documenti.
		Sono quindi state misurate le metriche sui processi per valutarne la bontà e fornire una base per la pianificazione dei cicli \glossario{PDCA}.

	
		\subsubsection{Revisione di Qualifica}
		L'attività di verifica svolta dai \emph{Verificatori} è avvenuta come determinato dal \PianoDiProgetto{} al termine della stesura di ogni documento previsto. La verifica svolta sui documenti è avvenuta seguendo le indicazioni delle \NormeDiProgetto{} e misurando le metriche indicate in \ref{metrichedocumenti}. Le anomalie evidenziate non incidono in modo determinante sulla consistenza del prodotto del processo di documentazione. Le metriche hanno contribuito al controllo sui processi permettendo di monitorare e misurare il loro andamento. 

	
	 \pagebreak
	 \subsection{Miglioramenti post Revisione}
	 
	 	\subsubsection{Miglioramenti post RR}
		A seguito delle attività di verifica e controllo è stato sottoposto un questionario ad ogni membro del gruppo che ha contribuito ad identificare le problematiche relative ai processi e a formulare proposte risolutive. Da queste idee sono nate diverse modifiche e miglioramenti ai documenti e in generale al nostro modo di lavorare. Seguendo questa linea abbiamo applicato coerentemente la politica di \textit{plan-do-check-act}, utilissima per il miglioramento della qualità: \\
			

		\begin{table}[H]
	    \begin{tabular}{ | p{5cm} | p{5cm} | p{2cm} | }
			\hline
			Problema & Possibile soluzione & Stato \\ \hline
		    Il dizionario personale di \glossario{Aspell}, essendo un file collaborativo compilato in automatico da tale \glossario{tool}, impone molto spesso attività manuali extra di gestione del \glossario{repository}, in particolare vanno risolti molti conflitti. & Uno script che ordina il file in questione dovrebbe diminuire i conflitti. & Eseguito. \\ \hline
			 Contrassegnare le parole di glossario con il relativo \glossario{tag} è un attività fortemente propensa a dimenticanze ed errori. & Uno script potrebbe contrassegnare le parole di glossario presenti nei documenti in automatico. & Non eseguito.  \\ \hline
			Scarsa frammentazione dei \glossario{task} & Incremento dell'utilizzo dello strumento di gestione dei processi. & Eseguito.   \\ \hline
			Mantenere la corrispondenza tra casi d'uso e relativi \glossario{url} dei diagrammi è un operazione lunga e manuale. & Uno script può automatizzare tale attività. & Eseguito. \\ \hline
			Difficoltà nell'uso dello strumento di controllo di versione. & I membri del gruppo devono eseguire attività extra di autoformazione sulla base del materiale messo a disposizione da alcuni membri. & Eseguito. \\	
			\hline
	    \end{tabular}
	    	\caption{Problemi individuati in RR e relative soluzioni.}
		\end{table}

		\subsubsection{Miglioramenti post RP}
		A seguito delle attività di verifica e controllo è stato sottoposto un questionario ad ogni membro del gruppo che ha contribuito ad identificare le problematiche relative ai processi e a formulare proposte risolutive. Da queste idee sono nate diverse modifiche e miglioramenti ai documenti e in generale al nostro modo di lavorare. Seguendo questa linea abbiamo applicato coerentemente la politica di \textit{plan-do-check-act}, utilissima per il miglioramento della qualità: \\
	
		\begin{table}[H]
    		\begin{tabular}{ | p{5cm} | p{5cm} | p{2cm} | }
				\hline
				Problema & Possibile soluzione & Stato \\ \hline
				Forte sbilanciamento tra i membri del gruppo in merito alle conoscenze dello \glossario{stack tecnologico} utilizzato. & Autoformazione teorica e pratica. & Eseguito. \\ \hline
				Forte sbilanciamento in merito alla capacità di effettuare scelte di design architetturale, con la conseguente necessità di un'interazione ripetitiva tra i membri del gruppo. & Autoformazione. & Eseguito. \\ \hline
				Difficoltà nell'apportare modifiche allo strumento \glossario{Requisteak} (descritto in \NormeDiProgetto{}) dovute al linguaggio di sviluppo (\glossario{Ruby on Rails}) conosciuto solo da una parte dei membri del gruppo. & Utilizzo sistematico delle issues di \glossario{GitHub} per lo strumento in questione. & Eseguito. \\ \hline
				Sviluppo di \glossario{Requisteak} (descritto in \NormeDiProgetto{}) ritardatario per mancanza di progettazione e tempi ristretti. & Per la progettazione è necessario interagire con i docenti per capire al meglio le associazioni tra quanto progettato ed i relativi test. & Eseguito. \\ \hline
				Difficoltà nel approcciare con un documento al quale non si è lavorato in precedenza. & Va predisposta una maggiore sistematicità nel commentare il materiale prodotto, per la Specifica Tecnica sono necessari commenti per i diagrammi e le scelte architetturali. & Eseguito. \\ \hline
				Segnalazioni delle anomalie nel repository dei documenti troppo informali e non tracciabili. & Utilizzo delle issues per il \glossario{repository} dei documenti e relativa autoformazione per il loro corretto utilizzo. & Eseguito. \\ \hline	 
    		\end{tabular}
    			\caption{Problemi individuati in RP e relative soluzioni.}
		\end{table}


	 
	 \subsection{Dettaglio delle verifiche tramite analisi}
	 \label{DettaglioVerificheAnalisi}
	 
	 	\subsubsection{Analisi}
	 	\paragraph{Documenti}
	 	Vengono qui riportati i valori dell’indice Gulpease per ogni documento durante l'analisi e relativo esito basato sui range stabiliti in \ref{gulpease}.
	
		\begin{table}[H]
		\centering
		\begin{tabular}{ | c | c | c | }
	    \hline
	    Documento & Valore indice & Esito \\ \hline
	     \emph{Analisi dei Requisiti v1.3.1} & 45 &  superato \\ \hline
	     \emph{Glossario v1.3.1} & 63 &  superato \\ \hline
	     \emph{Norme di Progetto v1.3.1} & 47 &  superato \\ \hline
	     \emph{Piano di Progetto v1.3.1} & 51 &  superato \\ \hline
	     \emph{Piano di Qualifica v1.3.1} & 53 &  superato \\ \hline
	     \emph{Studio di Fattibilità v1.3.1} & 43 &  superato \\ \hline
	    \end{tabular}
		\caption{Esiti verifica documenti, Analisi}
		\end{table}
		
	%Il file \AnalisiDeiRequisiti{} contiene numerose tabelle che possono facilmente falsare l'indice rilevato. Il gruppo si riserva di analizzare tale problematica e aggiornare i relativi strumenti entro la prossima revisione.
	
	\subsubsection{Progettazione Architetturale}
	\paragraph{Documenti}
	 Vengono qui riportati i valori dell’indice Gulpease per ogni documento durante la progettazione architetturale e relativo esito basato sui range stabiliti in \ref{gulpease}.
	
	\begin{table}[H]
	\centering
	\begin{tabular}{ | c | c | c | }
    \hline
    Documento & Valore indice & Esito \\ \hline
    \emph{Analisi dei Requisiti v3.0.0} & 51 &  superato \\ \hline
    \emph{Glossario v3.0.0} & 49 &  superato \\ \hline
    \emph{Norme di Progetto v3.0.0} & 47 &  superato \\ \hline
    \emph{Piano di Progetto v3.0.0} & 73 &  superato \\ \hline
    \emph{Piano di Qualifica v3.0.0} & 50 &  superato \\ \hline
    \emph{Studio di Fattibilità v3.0.0} & 44 &  superato \\ \hline
    \emph{Specifica Tecnica v3.0.0} & 58 & superato \\ \hline
    \end{tabular}
	\caption{Esiti verifica documenti, Progettazione Architetturale}
	\end{table}
	

	\subsubsection{Progettazione di dettaglio e codifica}
	\paragraph{Documenti}
	 Vengono qui riportati i valori dell’indice Gulpease per ogni documento durante la progettazione di dettaglio e codifica, e relativo esito basato sui range stabiliti in \ref{gulpease}.
	
	%TODO completare tabella risultati Gulpease
	\begin{table}[H]
	\centering
	\begin{tabular}{ | c | c | c | }
    \hline
    Documento & Valore indice & Esito \\ \hline
    \AnalisiDeiRequisiti{} & 54 & superato  \\ \hline
    \DefinizioneDiProdotto{} & 42 & superato  \\ \hline
    \Glossario{} & 49 & superato  \\ \hline
    \ManualeAdmin{} & 51 & superato  \\ \hline
    \ManualeSviluppatore{} & 43 & superato  \\ \hline
    \ManualeUtente{} & 50 & superato  \\ \hline
    \NormeDiProgetto{} & 53 & superato  \\ \hline
    \PianoDiProgetto{} & 52 & superato  \\ \hline
    \PianoDiQualifica{} & 51 & superato  \\ \hline
    \SpecificaTecnica{} & 76 & superato \\ \hline
    \end{tabular}
	\caption{Esiti verifica documenti, Progettazione di Dettaglio e Codifica}
	\end{table}
	
	\paragraph{Codice}
	%TODO vedere in https://jenkins-steakholders.rhcloud.com se c'è qualcosa di utile
	Vengono qui riportate le misure rilevate con le metriche sull'analisi statica e dinamica del codice. Per ogni metrica si riportano i valore calcolati mantenendo una separazione tra backend e frontend. Per una descrizione delle metriche si rimanda alla sezione \ref{metrichesoftware}.

	% TODO misure sul software da aggiornare
	\subparagraph{Complessità ciclomatica}
	\begin{itemize}
		\item \emph{Backend}: 
		\begin{itemize}
			\item medio: 1.38
			\item massimo: 6
		\end{itemize} 
		\item \emph{Frontend}: 
		\begin{itemize}
			\item medio: 1.34
			\item massimo: 5
		\end{itemize} 
	\end{itemize}


	\subparagraph{Numero di metodi}
	\begin{itemize}
		\item \emph{Backend}: 155
		\item \emph{Frontend}: 62
	\end{itemize}
	 	

	\subparagraph{Numero parametri per metodo}
	\begin{itemize}
		\item \emph{Backend}: 
		\begin{itemize}
			\item medio: 1.68
			\item massimo: 5
		\end{itemize} 
		\item \emph{Frontend}: 
		\begin{itemize}
			\item medio: 1.39
			\item massimo: 2
		\end{itemize} 
	\end{itemize}


	\subparagraph{\glossario{Halstead} difficulty per-function}
	\begin{itemize}
		\item \emph{Backend}: 
		\begin{itemize}
			\item medio: 3.42
			\item massimo: 16.90
		\end{itemize} 
		\item \emph{Frontend}: 
		\begin{itemize}
			\item medio: 3.46
			\item massimo: 21.36
		\end{itemize} 
	\end{itemize}

	
	\subparagraph{\glossario{Halstead} volume per-function}
	\begin{itemize}
		\item \emph{Backend}: 
		\begin{itemize}
			\item medio: 100.76
			\item massimo: 1303.56
		\end{itemize} 
		\item \emph{Frontend}: 
		\begin{itemize}
			\item medio: 102.23
			\item massimo: 671.55
		\end{itemize} 
	\end{itemize}


	\subparagraph{\glossario{Halstead} effort per-function}
	\begin{itemize}
		\item \emph{Backend}: 
		\begin{itemize}
			\item medio: 653.64
			\item massimo: 9778.01
		\end{itemize}
		\item \emph{Frontend}: 
		\begin{itemize}
			\item medio: 572.79
			\item massimo: 14342.28
		\end{itemize} 
	\end{itemize}


	\subparagraph{Maintainability index}
	\begin{itemize}
		\item \emph{Backend}: 
		\begin{itemize}
			\item medio: 77.05
			\item minimo: 56.57
		\end{itemize} 
		\item \emph{Frontend}: 
		\begin{itemize}
			\item medio: 71.98
			\item minimo: 50.2
		\end{itemize} 
	\end{itemize}


	\subparagraph{Statement Coverage}
	\begin{itemize}
		\item \emph{Backend}: 63.45\%
		\item \emph{Frontend}: 66.2\%
	\end{itemize}
	
	I valori non raggiungono i parametri di accettazione. Tale mancanza verrà colmata nel prossimo periodo di sviluppo.

	\subparagraph{Branch Coverage}
	\begin{itemize}
		\item \emph{Backend}: 52.8\%
		\item \emph{Frontend}: 48.69\%
	\end{itemize}

	I valori non raggiungono i parametri di accettazione. Tale mancanza verrà colmata nel prossimo periodo di sviluppo.
	

	\subsection{Dettaglio dell'esito delle revisioni}
	Lo sviluppo di questo progetto didattico si basa sull'attraversamento di quattro revisioni presiedute dal committente. Tre delle quattro revisioni produrranno delle segnalazioni degli errori riscontrati da parte del committente, deve seguire un report di come sono state risolte in ogni documento.
		
		\subsubsection{Revisione dei Requisiti}
		Per la Revisione dei Requisiti le segnalazioni da parte del committente sono state corrette:
		
		\begin{itemize}
			\item \emph{Norme di Progetto}: il documento è stato riorganizzato per processi, attività, procedure, strumenti. Sono state aggiunte indicazioni sugli strumenti per la gestione del \glossario{repository} e le regole per la rotazione dei ruoli sono state definite in modo dettagliato;
			\item \emph{Analisi dei Requisiti}: le \NormeDiProgetto{} descrivono la modalità di consegna che è stata ben definita che include la generazione dei nomi dei documenti con la relativa versione. Inoltre sono stati rivisti tutti i requisiti e casi d'uso segnalati dal committente;
			\item \emph{Piano di Progetto}: l'Organigramma è stato spostato in appendice e sono stati rimossi i costi orari dei ruoli. Sono state ripartizionate le ore considerando attività di analisi successive al 2013-12-20 e la percentuale di ore di verifica è almeno del 30\% del totale;
			\item \emph{Piano di Qualifica}: la trattazione del \glossario{SEMAT} è stata spostata ed approfondita nel piano di progetto e le tecniche adottate sono state spostate nelle norme di progetto;
			\item \emph{Glossario}: è stato creato l'indice del documento e ogni gruppo di lettera inizia su una pagina nuova.
		\end{itemize}
		
		\subsubsection{Revisione di Progettazione}

		Per la Revisione di Progettazione le segnalazioni da parte del committente sono state corrette:
		
		\begin{itemize}
			\item \emph{Norme di Progetto}: il documento è stato riorganizzato secondo quanto definito nello standard ISO/IEC 12207;
			\item \emph{Analisi dei Requisiti}: sono stati rivisti tutti i requisiti e casi d'uso segnalati dal committente;
			\item \emph{Specifica Tecnica}: rivisti e corretti i contenuti segnalati dal committente. I design pattern sono stati contestualizzati come suggerito;
			\item \emph{Piano di Progetto}: sono state aggiunte le dipendenze ai diagrammi di Gantt, l'analisi dei rischi è stata attualizzata allo stato dell'arte, è stato riorganizzato il documento prima individuando i rischi, poi pianificando e infine fissando i costi a preventivo mettendoli in relazione con il consuntivo corrente, è stato pianificato il raggiungimento degli stati del \glossario{SEMAT};
			\item \emph{Piano di Qualifica}: sono state aggiunte metriche relative ai prodotti dell'Analisi e della Progettazione, è stato discusso il significato del ciclo \glossario{PDCA} con il committente.
		\end{itemize}

%		\subsubsection{Revisione di Qualifica}
			


	
