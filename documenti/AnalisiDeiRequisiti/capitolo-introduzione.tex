\section{Introduzione}


\subsection{Scopo del documento}
Il presente documento ha come obiettivo quello di descrivere formalmente i requisiti evidenziati dall'analisi e deduzione del capitolato d'appalto \ProjectName{}  (C1) secondo le esigenze del proponente \Proponente{}. Il gruppo \GroupName{} si impegna formalmente a soddisfare tutti i requisiti qui descritti al fine di fornire al committente il prodotto richiesto.

\subsection{Scopo del prodotto} 
\ScopoDelProdotto

\subsection{Glossario}
Con l'obiettivo di evitare ridondanze e ambiguità di linguaggio, i termini tecnici e gli acronimi utilizzati nei documenti verranno definiti e descritti riportandoli nel documento "Glossario.pdf". I vocaboli riportati vengono indicati con una [G] a pedice.

\subsection{Riferimenti}
	\subsubsection{Normativi}
	\begin{itemize}
	\item \textbf{Capitolato d'appalto}: MaaP: MongoDB as an admin Platform rilasciato dal proponente\ped{G}
	\Proponente{} e reperibile all'indirizzo 
	\url{http://www.math.unipd.it/~tullio/IS-1/2013/Progetto/C1.pdf};
	\item \textbf{Verbale di incontro} con il proponente \Proponente{} in data 2013-12-02 (\textit{Verbale esterno...});
	\item \textbf{Norme di Progetto} : documento "Norme di Progetto.pdf". 
	\end{itemize}
	
	
	\subsubsection{Informativi}
		\begin{itemize}
			\item \textbf{Presentazione capitolato d'appalto}: \url{http://www.math.unipd.it/~tullio/IS-1/2013/Progetto/C1p.pdf};
			\item \textbf{Standard 830-1998}: \url{http://www.math.uaa.alaska.edu/~afkjm/cs401/IEEE830.pdf}; 
			\item \textbf{Ingegneria del software - Ian Sommerville - 8\textsuperscript{a} edizione (2007)};
			\item \textbf{Lucidi didattici del corso di Ingegneria del Software modulo A:}
			\begin{itemize}
				\item \textit{Ingegneria dei requisiti}, \url{http://www.math.unipd.it/~tullio/IS-1/2013/Dispense/L06.pdf}
			\end{itemize}
			\item \textbf{Dall'idea al codice con UML\ped{G}2 - L. Baresi, L. Lavazza, M. Pianciamore - 1\textsuperscript{a} edizione (2006):}
			\begin{itemize}			
				\item \textit{Capitolo 3 - Analisi dei requisiti}.			
			\end{itemize}
		\end{itemize}
		
