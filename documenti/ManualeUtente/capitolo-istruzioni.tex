\section{Istruzioni per l'uso}

	\subsection{Registrazione}
	\label{registrazione}
	Il primo passo per poter utilizzare l'applicazione \glossario{MaaP} è la registrazione. \`E possibile registrasi cliccando sul link \texttt{Sign up} presente nella homepage. Verrà proposta un form da compilare con \texttt{email} e \texttt{password}, dopodiché bisognerà proseguire cliccando il pulsante \texttt{Sign up}.  

	\subsection{Autenticazione}
	\label{autenticazione}
	Per poter effettuare il login è necessario essere registrati (vedi \ref{registrazione}). Il login permette di accedere alla dashboard del sistema e quindi a tutte le funzionalità offerte da \glossario{MaaP}. Per autenticarsi è necessario cliccare sul link \texttt{Sign in} presente nella homepage, verrà proposta una form da compilare con \texttt{email} e \texttt{password}, dopodiché bisognerà proseguire cliccando il pulsante \texttt{Sign in}. Terminata l'autenticazione, l'utente viene reindirizzato nella dashboard (vedi \ref{visualizzazionedashboard}).

	\subsection{Modifica profilo}
	\label{modificaprofilo}
	L'utente ha a disposizione una pagina in cui poter modificare il proprio profilo, editando \texttt{email} e \texttt{password}. La pagina preposta visualizza una form con la quale è possibile modificare la password dell'utente.

	\subsection{Recupero password}
	\label{recuperopassword}
	Il recupero password avviene nel caso in cui l'utente abbia perso la password, per poter procedere con un recupero password è necessario però conoscere l'\texttt{email} di registrazione. Il recupero dalla password avviene partendo dalla schermata di login (vedi \ref{autenticazione}) cliccando sul link \texttt{Forgot your password?}, verrà proposta una semplice form dove bisognerà inserire l'\texttt{email} di registrazione e cliccare il pulsante di reset password. 
	L'utente riceverà le istruzioni da seguire nella casella di posta associata alla \texttt{email} di registrazione.

	\subsection{Visualizzazione dashboard} %UCU 8
	\label{visualizzazionedashboard}
	La dashboard è la pagina principale dalla quale è possibile avere accesso alla lista delle collection presenti nel sistema e alle altre funzionalità.
	di un utente che ha eseguito l'autenticazione (vedi \ref{autenticazione}).
	

	\subsection{Apertura collection index} %UCU 9
	\label{aperturacollectionindex}
	La collection index è la pagina in cui viene visualizzato il contenuto di una collection in forma tabellare. \`E possibile filtrare i contenuti della pagina tramite il pannello situato sulla destra.
	

	\subsection{Apertura della show-page di un document} %UCU 9.1
	La show-page di un document visualizza gli attributi del documento in forma tabellare.


	\subsection{Visualizzazione show page attributi innestati} % 9.1.1

	\subsection{Visualizzazione index page dell'array di document} % 9.1.2



	%\subsection{Filtra risultati}

	




	\subsection{Visualizzazione ..}
	
	%operazioni possibili.........